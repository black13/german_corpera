\documentclass[a4paper,backgrid,frontgrid]{flacards}
\usepackage{array,booktabs,tabularx}
\usepackage[condensed,sfdefault]{universalis}
\usepackage[T1]{fontenc}
\begin{document}
\pagesetup{2}{4}

%===ändern===
\card{\normalfont \Huge ändern}{
\begin{tabular}{lll}
\parbox[t][][t]{2.0 cm}{\normalfont \raggedleft ich\\du\\er/sie/es\\wir\\ihr\\sie} &    
\parbox[t][][t]{2cm}{\normalfont ändreich ändereich änder\\änderst\\ändert\\ändern\\ändert\\ändern} &
\parbox[t][][t]{2cm}{\normalfont änderte\\ändertest\\änderte\\änderten\\ändertet\\änderten}\\
\end{tabular}
\begin{tabular}{l}
\parbox[t][][t]{8cm}{}\\
\parbox[t][][t]{8cm}{\normalfont \small ['(transitive) to change (reflexive) to vary'] }\\
\end{tabular}
}
%===abbiegen===
\card{\normalfont \Huge abbiegen}{
\begin{tabular}{lll}
\parbox[t][][t]{2.0 cm}{\normalfont \raggedleft ich\\du\\er/sie/es\\wir\\ihr\\sie} &    
\parbox[t][][t]{2cm}{\normalfont biege ab\\biegst ab\\biegt ab\\biegen ab\\biegt ab\\biegen ab} &
\parbox[t][][t]{2cm}{\normalfont bog ab\\bogst ab\\bog ab\\bogen ab\\bogt ab\\bogen ab}\\
\end{tabular}
\begin{tabular}{l}
\parbox[t][][t]{8cm}{}\\
\parbox[t][][t]{8cm}{\normalfont \small ['(transitive) to bend (transitive, technical) to', 'fold (metal) (transitive, figuratively) to head', 'off, to stave off (intransitive) to turn, to turn', 'off Biegen Sie jetzt links ab.Now turn left.', '(intransitive) to branch off'] }\\
\end{tabular}
}
%===abfahren===
\card{\normalfont \Huge abfahren}{
\begin{tabular}{lll}
\parbox[t][][t]{2.0 cm}{\normalfont \raggedleft ich\\du\\er/sie/es\\wir\\ihr\\sie} &    
\parbox[t][][t]{2cm}{\normalfont fahre ab\\fährst ab\\fährt ab\\fahren ab\\fahrt ab\\fahren ab} &
\parbox[t][][t]{2cm}{\normalfont fuhr ab\\fuhrst ab\\fuhr ab\\fuhren ab\\fuhrt ab\\fuhren ab}\\
\end{tabular}
\begin{tabular}{l}
\parbox[t][][t]{8cm}{}\\
\parbox[t][][t]{8cm}{\normalfont \small ['to depart, to leave Ihr Zug fhrt um fnf Uhr ab.', "Her train departs at 5 o'clock. Wir werden morgen", 'abfahren. We will leave tomorrow (idiomatic,', 'colloquial) to be crazy about something. Sie fhrt', "voll auf Kultfilme ab. She's really crazy about", 'cult movies.'] }\\
\end{tabular}
}
%===abgeben===
\card{\normalfont \Huge abgeben}{
\begin{tabular}{lll}
\parbox[t][][t]{2.0 cm}{\normalfont \raggedleft ich\\du\\er/sie/es\\wir\\ihr\\sie} &    
\parbox[t][][t]{2cm}{\normalfont gebe ab\\gibst ab\\gibt ab\\geben ab\\gebt ab\\geben ab} &
\parbox[t][][t]{2cm}{\normalfont gab ab\\gabst ab\\gab ab\\gaben ab\\gabt ab\\gaben ab}\\
\end{tabular}
\begin{tabular}{l}
\parbox[t][][t]{8cm}{}\\
\parbox[t][][t]{8cm}{\normalfont \small ['to hand in to deliver to deposit Wo kann ich mein', 'Gepck abgeben? Where can I deposit my luggage? to', 'give away (to die) den Lffel abgeben to sell the', 'farm, to peg out, to pip out to sell Khlschrank', '(gnstig) abzugeben fridge for sale (at a good', 'price) to hand over to pass on to concede to emit,', 'give off Wrme abgeben to give off heat (sports,', 'transitive) to pass a ball (shoot) to fire einen', 'Schuss abgeben (compare schieen) to fire a shot', '(declaration) to give eine Erklrung abgeben to', 'issue a statement (vote) to cast (stuff, plot,', 'background, frame) to provide Stoff fr eine Komdie', 'abgeben to deliver the plot for a comedy (to act', 'like someone) to make den Clown abgeben to make', 'the clown (sports, intransitive) to pass', '(reflexive) to bother oneself with', 'somebody/something, to associate with', 'somebody/something, to mess around with somebody.', "Damit gibt er sich nicht ab. He doesn't bother", 'himself with it.'] }\\
\end{tabular}
}
%===abholen===
\card{\normalfont \Huge abholen}{
\begin{tabular}{lll}
\parbox[t][][t]{2.0 cm}{\normalfont \raggedleft ich\\du\\er/sie/es\\wir\\ihr\\sie} &    
\parbox[t][][t]{2cm}{\normalfont hole ab\\holst ab\\holt ab\\holen ab\\holt ab\\holen ab} &
\parbox[t][][t]{2cm}{\normalfont holte ab\\holtest ab\\holte ab\\holten ab\\holtet ab\\holten ab}\\
\end{tabular}
\begin{tabular}{l}
\parbox[t][][t]{8cm}{}\\
\parbox[t][][t]{8cm}{\normalfont \small ['to collect, pick up or fetch someone or something'] }\\
\end{tabular}
}
%===ablehnen===
\card{\normalfont \Huge ablehnen}{
\begin{tabular}{lll}
\parbox[t][][t]{2.0 cm}{\normalfont \raggedleft ich\\du\\er/sie/es\\wir\\ihr\\sie} &    
\parbox[t][][t]{2cm}{\normalfont lehne ab\\lehnst ab\\lehnt ab\\lehnen ab\\lehnt ab\\lehnen ab} &
\parbox[t][][t]{2cm}{\normalfont lehnte ab\\lehntest ab\\lehnte ab\\lehnten ab\\lehntet ab\\lehnten ab}\\
\end{tabular}
\begin{tabular}{l}
\parbox[t][][t]{8cm}{}\\
\parbox[t][][t]{8cm}{\normalfont \small ['to decline, refuse, reject, turn down Das Angebot', 'wurde abgelehnt.The offer was turned down.'] }\\
\end{tabular}
}
%===abnehmen===
\card{\normalfont \Huge abnehmen}{
\begin{tabular}{lll}
\parbox[t][][t]{2.0 cm}{\normalfont \raggedleft ich\\du\\er/sie/es\\wir\\ihr\\sie} &    
\parbox[t][][t]{2cm}{\normalfont nehme ab\\nimmst ab\\nimmt ab\\nehmen ab\\nehmt ab\\nehmen ab} &
\parbox[t][][t]{2cm}{\normalfont nahm ab\\nahmst ab\\nahm ab\\nahmen ab\\nahmt ab\\nahmen ab}\\
\end{tabular}
\begin{tabular}{l}
\parbox[t][][t]{8cm}{}\\
\parbox[t][][t]{8cm}{\normalfont \small ['to decrease, to diminish to lose weight to accept', 'to receive (of telephones) to pick up'] }\\
\end{tabular}
}
%===achten===
\card{\normalfont \Huge achten}{
\begin{tabular}{lll}
\parbox[t][][t]{2.0 cm}{\normalfont \raggedleft ich\\du\\er/sie/es\\wir\\ihr\\sie} &    
\parbox[t][][t]{2cm}{\normalfont achte\\achtest\\achtet\\achten\\achtet\\achten} &
\parbox[t][][t]{2cm}{\normalfont achtete\\achtetest\\achtete\\achteten\\achtetet\\achteten}\\
\end{tabular}
\begin{tabular}{l}
\parbox[t][][t]{8cm}{}\\
\parbox[t][][t]{8cm}{\normalfont \small ['(intransitive, with auf and accusative) to care', 'about, to pay attention to (intransitive, with auf', 'and accusative) to keep an eye on (transitive) to', 'respect'] }\\
\end{tabular}
}
%===anbieten===
\card{\normalfont \Huge anbieten}{
\begin{tabular}{lll}
\parbox[t][][t]{2.0 cm}{\normalfont \raggedleft ich\\du\\er/sie/es\\wir\\ihr\\sie} &    
\parbox[t][][t]{2cm}{\normalfont biete an\\bietest an\\bietet an\\bieten an\\bietet an\\bieten an} &
\parbox[t][][t]{2cm}{\normalfont bot an\\botest an\\bot an\\boten an\\botet an\\boten an}\\
\end{tabular}
\begin{tabular}{l}
\parbox[t][][t]{8cm}{}\\
\parbox[t][][t]{8cm}{\normalfont \small ['to offer, to provide'] }\\
\end{tabular}
}
%===anfangen===
\card{\normalfont \Huge anfangen}{
\begin{tabular}{lll}
\parbox[t][][t]{2.0 cm}{\normalfont \raggedleft ich\\du\\er/sie/es\\wir\\ihr\\sie} &    
\parbox[t][][t]{2cm}{\normalfont fange an\\fängst an\\fängt an\\fangen an\\fangt an\\fangen an} &
\parbox[t][][t]{2cm}{\normalfont fing an\\fingst an\\fing an\\fingen an\\fingt an\\fingen an}\\
\end{tabular}
\begin{tabular}{l}
\parbox[t][][t]{8cm}{}\\
\parbox[t][][t]{8cm}{\normalfont \small ['(intransitive) to begin; to commence Das Konzert', 'fngt gleich an. The concert is beginning shortly.', '(intransitive, with zu + infinitive) to start', 'doing something Wann hast du angefangen zu', 'rauchen? When did you start smoking?', '(intransitive, with mit) to begin something; to', 'start something Morgen fangen wir mit dem Projekt', 'an. Tomorrow we are starting the project.', '(intransitive, with von) to start a topic; to keep', 'talking about something Er fngt stndig von seinen', "Hunden an. He's always talking about his dogs.", '(transitive, colloquial) to begin something; to', 'start something Wann hast du das Rauchen', 'angefangen? When did you start smoking? Morgen', 'fangen wir das Projekt an. Tomorrow we are', 'starting the project.'] }\\
\end{tabular}
}
%===anfassen===
\card{\normalfont \Huge anfassen}{
\begin{tabular}{lll}
\parbox[t][][t]{2.0 cm}{\normalfont \raggedleft ich\\du\\er/sie/es\\wir\\ihr\\sie} &    
\parbox[t][][t]{2cm}{\normalfont fasse an\\fasst an\\fasst an\\fassen an\\fasst an\\fassen an} &
\parbox[t][][t]{2cm}{\normalfont fasste an\\fasstest an\\fasste an\\fassten an\\fasstet an\\fassten an}\\
\end{tabular}
\begin{tabular}{l}
\parbox[t][][t]{8cm}{}\\
\parbox[t][][t]{8cm}{\normalfont \small ["to touch Nicht anfassen!Don't touch! Fass das", "nicht an!Don't touch that! Er fasst mir die Hand", 'an.He touches my hand.'] }\\
\end{tabular}
}
%===ankommen===
\card{\normalfont \Huge ankommen}{
\begin{tabular}{lll}
\parbox[t][][t]{2.0 cm}{\normalfont \raggedleft ich\\du\\er/sie/es\\wir\\ihr\\sie} &    
\parbox[t][][t]{2cm}{\normalfont komme an\\kommst an\\kommt an\\kommen an\\kommt an\\kommen an} &
\parbox[t][][t]{2cm}{\normalfont kam an\\kamst an\\kam an\\kamen an\\kamt an\\kamen an}\\
\end{tabular}
\begin{tabular}{l}
\parbox[t][][t]{8cm}{}\\
\parbox[t][][t]{8cm}{\normalfont \small ['(intransitive) to arrive (impersonal, with', 'preposition auf) to depend on (impersonal, with', 'preposition auf) to be important, to matter 2012', 'June 12, Die Welt [1], page 22: Bei der', 'Herstellung von Glas kommt es auf die schnelle', 'Abkhlung der mindestens 1400 Grad heien Schmelze', 'an. In the production of glass, the fast cooling', 'of the at least 1400-degree hot melt is important.'] }\\
\end{tabular}
}
%===anmachen===
\card{\normalfont \Huge anmachen}{
\begin{tabular}{lll}
\parbox[t][][t]{2.0 cm}{\normalfont \raggedleft ich\\du\\er/sie/es\\wir\\ihr\\sie} &    
\parbox[t][][t]{2cm}{\normalfont mache an\\machst an\\macht an\\machen an\\macht an\\machen an} &
\parbox[t][][t]{2cm}{\normalfont machte an\\machtest an\\machte an\\machten an\\machtet an\\machten an}\\
\end{tabular}
\begin{tabular}{l}
\parbox[t][][t]{8cm}{}\\
\parbox[t][][t]{8cm}{\normalfont \small ['to switch on, to turn on (a lamp, a stove, an', 'electronic device, etc.) Synonyms: anstellen,', 'anschalten, einschalten, anschmeien Antonyms:', 'ausmachen, ausstellen, abstellen, abschalten,', 'ausschalten to light, to start (a fire, a candle,', 'gas etc.); Synonyms: anbrennen, entznden, anznden,', 'befeuern Antonyms: ausmachen, erlschen, auslschen,', 'lschen to turn on (a person, sexually), to hit on', '(a person) (colloquial) to start something with', '(someone), to provoke (someone), to accost,', '(internet) to flame (approach aggressively) to', 'dress (a salad)'] }\\
\end{tabular}
}
%===anmelden===
\card{\normalfont \Huge anmelden}{
\begin{tabular}{lll}
\parbox[t][][t]{2.0 cm}{\normalfont \raggedleft ich\\du\\er/sie/es\\wir\\ihr\\sie} &    
\parbox[t][][t]{2cm}{\normalfont melde an\\meldest an\\meldet an\\melden an\\meldet an\\melden an} &
\parbox[t][][t]{2cm}{\normalfont meldete an\\meldetest an\\meldete an\\meldeten an\\meldetet an\\meldeten an}\\
\end{tabular}
\begin{tabular}{l}
\parbox[t][][t]{8cm}{}\\
\parbox[t][][t]{8cm}{\normalfont \small ['to announce, declare to enroll to register Er hat', 'seinen Wohnsitz bei der Gemeinde angemeldet.He', 'registered his residence with the community.'] }\\
\end{tabular}
}
%===annehmen===
\card{\normalfont \Huge annehmen}{
\begin{tabular}{lll}
\parbox[t][][t]{2.0 cm}{\normalfont \raggedleft ich\\du\\er/sie/es\\wir\\ihr\\sie} &    
\parbox[t][][t]{2cm}{\normalfont nehme an\\nimmst an\\nimmt an\\nehmen an\\nehmt an\\nehmen an} &
\parbox[t][][t]{2cm}{\normalfont nahm an\\nahmst an\\nahm an\\nahmen an\\nahmt an\\nahmen an}\\
\end{tabular}
\begin{tabular}{l}
\parbox[t][][t]{8cm}{}\\
\parbox[t][][t]{8cm}{\normalfont \small ['to assume, to suppose to receive to accept Ich', 'habe das Angebot angenommen. to adopt (a child)', "1960, Marie Luise Kaschnitz, 'Schneeschmelze': Das", 'ist es, warum man keine Kinder der annehmen soll.', 'Man wei nicht, was in ihnen liegt. "That\'s why you', "shouldn't adopt any children. You just don't know", 'what lies within them." (reflexive, colloquial,', 'regional, northern and central Germany, with an', 'indefinite pronoun and von) to be touched by; to', 'care much about; to have oneself be impressed by;', 'to feel responsible for Da nehm ich mir nix von', 'an.  "I don\'t care much about that."'] }\\
\end{tabular}
}
%===anrufen===
\card{\normalfont \Huge anrufen}{
\begin{tabular}{lll}
\parbox[t][][t]{2.0 cm}{\normalfont \raggedleft ich\\du\\er/sie/es\\wir\\ihr\\sie} &    
\parbox[t][][t]{2cm}{\normalfont rufe an\\rufst an\\ruft an\\rufen an\\ruft an\\rufen an} &
\parbox[t][][t]{2cm}{\normalfont rief an\\riefst an\\rief an\\riefen an\\rieft an\\riefen an}\\
\end{tabular}
\begin{tabular}{l}
\parbox[t][][t]{8cm}{}\\
\parbox[t][][t]{8cm}{\normalfont \small ['(transitive) to call by telephone; to ring', '(someone) (transitive) to call on (someone, e.g. a', 'divine being); to appeal to; to call upon'] }\\
\end{tabular}
}
%===anschauen===
\card{\normalfont \Huge anschauen}{
\begin{tabular}{lll}
\parbox[t][][t]{2.0 cm}{\normalfont \raggedleft ich\\du\\er/sie/es\\wir\\ihr\\sie} &    
\parbox[t][][t]{2cm}{\normalfont schaue an\\schaust an\\schaut an\\schauen an\\schaut an\\schauen an} &
\parbox[t][][t]{2cm}{\normalfont schaute an\\schautest an\\schaute an\\schauten an\\schautet an\\schauten an}\\
\end{tabular}
\begin{tabular}{l}
\parbox[t][][t]{8cm}{}\\
\parbox[t][][t]{8cm}{\normalfont \small ['to look at; to behold; to observe (visually); to', 'view'] }\\
\end{tabular}
}
%===ansehen===
\card{\normalfont \Huge ansehen}{
\begin{tabular}{lll}
\parbox[t][][t]{2.0 cm}{\normalfont \raggedleft ich\\du\\er/sie/es\\wir\\ihr\\sie} &    
\parbox[t][][t]{2cm}{\normalfont sehe an\\siehst an\\sieht an\\sehen an\\seht an\\sehen an} &
\parbox[t][][t]{2cm}{\normalfont sah an\\sahst an\\sah an\\sahen an\\saht an\\sahen an}\\
\end{tabular}
\begin{tabular}{l}
\parbox[t][][t]{8cm}{}\\
\parbox[t][][t]{8cm}{\normalfont \small ['to look at (figuratively) to regard (reflexive) to', 'view'] }\\
\end{tabular}
}
%===antworten===
\card{\normalfont \Huge antworten}{
\begin{tabular}{lll}
\parbox[t][][t]{2.0 cm}{\normalfont \raggedleft ich\\du\\er/sie/es\\wir\\ihr\\sie} &    
\parbox[t][][t]{2cm}{\normalfont antworte\\antwortest\\antwortet\\antworten\\antwortet\\antworten} &
\parbox[t][][t]{2cm}{\normalfont antwortete\\antwortetest\\antwortete\\antworteten\\antwortetet\\antworteten}\\
\end{tabular}
\begin{tabular}{l}
\parbox[t][][t]{8cm}{}\\
\parbox[t][][t]{8cm}{\normalfont \small ['(intransitive) to answer, to reply Der Knig', 'antwortete: "Wenn es einen solchen Menschen gibt,', 'dann muss er verrckt sein."The King answered, "If', 'such a man exists, I think he must be mad."'] }\\
\end{tabular}
}
%===anziehen===
\card{\normalfont \Huge anziehen}{
\begin{tabular}{lll}
\parbox[t][][t]{2.0 cm}{\normalfont \raggedleft ich\\du\\er/sie/es\\wir\\ihr\\sie} &    
\parbox[t][][t]{2cm}{\normalfont ziehe an\\ziehst an\\zieht an\\ziehen an\\zieht an\\ziehen an} &
\parbox[t][][t]{2cm}{\normalfont zog an\\zogst an\\zog an\\zogen an\\zogt an\\zogen an}\\
\end{tabular}
\begin{tabular}{l}
\parbox[t][][t]{8cm}{}\\
\parbox[t][][t]{8cm}{\normalfont \small ['(reflexive) to get dressed Ich ziehe mich an.', '"I\'m getting dressed." 1912, Franz Kafka, Die', 'Verwandlung, in: Die Weien Bltter. Eine', 'Monatsschrift. year 2, issue 10, Verlag der Weien', 'Bcher (1915), page 1180: Zunchst wollte er ruhig', 'und ungestrt aufstehen, sich anziehen und vor', 'allem frhstcken, und dann erst das Weitere', 'berlegen, denn, das merkte er wohl, im Bett wrde', 'er mit dem Nachdenken zu keinem vernnftigen Ende', 'kommen. To begin with, he wanted to get up calmly', 'and undisturbed, get dressed and, above all, have', 'breakfast, and only then think about everything', 'else, because, as he realized very well, in bed he', 'would not come to a sensible conclusion with the', 'thinking. (transitive, often with a reflexive', 'dative) to put on; to dress oneself in Ich ziehe', '(mir) meinen Pulli an.  "I put on my pullover."', '(transitive) to dress someone Ich ziehe ihn an.', '"I\'m dressing him." (transitive, with an', 'additional dative) to dress someone in Ich ziehe', 'ihm seine Jacke an.  "I\'m dressing him in his', 'jacket." (transitive) to attract Die Stille alter', 'Kirchen hat mich immer angezogen. "The quietness', 'of old churches has always attracted me."', '(reflexive) to attract one another Gegenstze', 'ziehen sich an.  "Opposites attract one another."', '(transitive) to fasten (a screw) Die Schraube muss', 'fest angezogen werden.  "The screw must be', 'fastened tight." (transitive) to pull (a lever) Er', 'zog die Handbremse an.  "He put on the hand', 'brake." (transitive) to pull lightly Zieh die', 'Schublade etwas an, ohne den losen Knopf', 'abzubrechen. Pull the drawer a bit without', 'breaking off the loose knob. (intransitive) to', 'speed up; to sprint Sie zog noch einmal an und', 'gewann das Rennen.  "She speeded up again and won', 'the race." (intransitive, of rates, stocks) to', 'climb Nach berwundener Krise ziehen die Kurse', 'wieder an.  "With the crisis overcome, the prices', 'are climbing again." (reflexive, colloquial,', 'regional, western Germany, with an indefinite', 'pronoun and von) to be touched by; to care much', 'about; to have oneself be impressed by; to feel', 'responsible for Da zieh ich mir nix von an.  "I', 'don\'t care much about that."'] }\\
\end{tabular}
}
%===anzünden===
\card{\normalfont \Huge anzünden}{
\begin{tabular}{lll}
\parbox[t][][t]{2.0 cm}{\normalfont \raggedleft ich\\du\\er/sie/es\\wir\\ihr\\sie} &    
\parbox[t][][t]{2cm}{\normalfont zünde an\\zündest an\\zündet an\\zünden an\\zündet an\\zünden an} &
\parbox[t][][t]{2cm}{\normalfont zündete an\\zündetest an\\zündete an\\zündeten an\\zündetet an\\zündeten an}\\
\end{tabular}
\begin{tabular}{l}
\parbox[t][][t]{8cm}{}\\
\parbox[t][][t]{8cm}{\normalfont \small ['(transitive) to light (anything flamable)', '(transitive) to strike (a match) (transitive) to', 'set on fire, to set fire to (transitive) to kindle', '(as in the process of making a camping fire or', 'something comparable) (transitive) to torch (used', 'when referring to arson or the destruction of', 'vehicles by fire)'] }\\
\end{tabular}
}
%===arbeiten===
\card{\normalfont \Huge arbeiten}{
\begin{tabular}{lll}
\parbox[t][][t]{2.0 cm}{\normalfont \raggedleft ich\\du\\er/sie/es\\wir\\ihr\\sie} &    
\parbox[t][][t]{2cm}{\normalfont arbeite\\arbeitest\\arbeitet\\arbeiten\\arbeitet\\arbeiten} &
\parbox[t][][t]{2cm}{\normalfont arbeitete\\arbeitetest\\arbeitete\\arbeiteten\\arbeitetet\\arbeiteten}\\
\end{tabular}
\begin{tabular}{l}
\parbox[t][][t]{8cm}{}\\
\parbox[t][][t]{8cm}{\normalfont \small ['(intransitive) to work (to do a specific task by', 'employing physical or mental powers) 1932, Erich', 'Mhsam, Die Befreiung der Gesellschaft vom Staat,', 'in: Erich Mhsam: Prosaschriften II, Verlag', 'europische ideen Berlin (1978), page 255: Wir', 'verstehen unter Kommunismus die auf', 'Gtergemeinschaft beruhende Gesellschaftsbeziehung,', 'die jedem nach seinen Fhigkeiten zu arbeiten,', 'jedem nach seinen Bedrfnissen zu verbrauchen', 'erlaubt. We understand by communism the', 'relationship of society that is based on public', 'ownership, that allows everyone to work according', 'to his capabilities, everyone to consume according', 'to his needs. (intransitive) to work, function,', 'run, operate (to be operative, in action)', '(intransitive) to ferment (to react, using', 'fermentation) (intransitive) to work, execute (to', 'set into action) (transitive, briefly  artisanal)', 'to make, produce (to create) Die Kommode ist aus', 'Eichenholz gearbeitet.The drawer is made of oak-', 'wood. (transitive, only with pronouns like etwas,', 'nichts) to do, perform (to carry out or execute,', 'especially something involving work) Was arbeitest', "du?  What are you doing? Ich arbeite nichts.  I'm", 'doing nothing. (reflexive) to work oneself (to)', '(to make oneself (a certain state) by working)', 'sich zu Tode arbeiten  to work oneself to death', 'sich erschpft arbeiten  to work oneself to the', "point of exhaustion (reflexive) to work one's way", '(to attain through work, by gradual degrees)', '(reflexive and impersonal) to work (translated by', 'rephrasing to use a general "you" or with the', 'gerund, "working") (to do a specific task by', 'employing physical or mental powers)'] }\\
\end{tabular}
}
%===ärgern===
\card{\normalfont \Huge ärgern}{
\begin{tabular}{lll}
\parbox[t][][t]{2.0 cm}{\normalfont \raggedleft ich\\du\\er/sie/es\\wir\\ihr\\sie} &    
\parbox[t][][t]{2cm}{\normalfont ärgreich ärgereich ärger\\ärgerst\\ärgert\\ärgern\\ärgert\\ärgern} &
\parbox[t][][t]{2cm}{\normalfont ärgerte\\ärgertest\\ärgerte\\ärgerten\\ärgertet\\ärgerten}\\
\end{tabular}
\begin{tabular}{l}
\parbox[t][][t]{8cm}{}\\
\parbox[t][][t]{8cm}{\normalfont \small ['(transitive) to annoy, to make angry (+accusative)', 'Die anderen Kinder rgern ihn immer.The other kids', 'keep annoying him. (reflexive) to be annoyed, to', 'be angry, to get annoyed, to get angry'] }\\
\end{tabular}
}
%===atmen===
\card{\normalfont \Huge atmen}{
\begin{tabular}{lll}
\parbox[t][][t]{2.0 cm}{\normalfont \raggedleft ich\\du\\er/sie/es\\wir\\ihr\\sie} &    
\parbox[t][][t]{2cm}{\normalfont atme\\atmest\\atmet\\atmen\\atmet\\atmen} &
\parbox[t][][t]{2cm}{\normalfont atmete\\atmetest\\atmete\\atmeten\\atmetet\\atmeten}\\
\end{tabular}
\begin{tabular}{l}
\parbox[t][][t]{8cm}{}\\
\parbox[t][][t]{8cm}{\normalfont \small ['to breathe'] }\\
\end{tabular}
}
%===auffordern===
\card{\normalfont \Huge auffordern}{
\begin{tabular}{lll}
\parbox[t][][t]{2.0 cm}{\normalfont \raggedleft ich\\du\\er/sie/es\\wir\\ihr\\sie} &    
\parbox[t][][t]{2cm}{\normalfont fordre aufich fordere aufich forder auf\\forderst auf\\fordert auf\\fordern auf\\fordert auf\\fordern auf} &
\parbox[t][][t]{2cm}{\normalfont forderte auf\\fordertest auf\\forderte auf\\forderten auf\\fordertet auf\\forderten auf}\\
\end{tabular}
\begin{tabular}{l}
\parbox[t][][t]{8cm}{}\\
\parbox[t][][t]{8cm}{\normalfont \small ['to ask, request to invite (someone to dance, etc.)'] }\\
\end{tabular}
}
%===aufgeben===
\card{\normalfont \Huge aufgeben}{
\begin{tabular}{lll}
\parbox[t][][t]{2.0 cm}{\normalfont \raggedleft ich\\du\\er/sie/es\\wir\\ihr\\sie} &    
\parbox[t][][t]{2cm}{\normalfont gebe auf\\gibst auf\\gibt auf\\geben auf\\gebt auf\\geben auf} &
\parbox[t][][t]{2cm}{\normalfont gab auf\\gabst auf\\gab auf\\gaben auf\\gabt auf\\gaben auf}\\
\end{tabular}
\begin{tabular}{l}
\parbox[t][][t]{8cm}{}\\
\parbox[t][][t]{8cm}{\normalfont \small ["(transitive) to give up on (one's efforts) Er gab", 'es auf die schwierige Aufgabe zu lsen.He gave up', 'on solving the difficult task. (reflexive) to give', 'up Sich selbst aufgeben.To give up oneself.', '(transitive) to abandon, to forsake, to relinquish', 'Die Stadt wurde auf Grund des starken', 'Wassermangels aufgegeben.The city has been', 'forsaken due to the severe water shortage.', '(transitive, military) to surrender, to capitulate', '(transitive) to lose hope, to resign, to quit Die', 'Suche nach der vermissten Person wurde von der', 'Polizei aufgegeben.The search for the missing', 'person was quit by the police. Sie hatte bereits', 'jegliche Hoffnung aufgegeben ihn wiederzusehen.She', 'had already lost any hope of seeing him again.', '(transitive, mail) to send, to mail (transitive,', 'homework) to give Heute gab man uns sehr viele', 'Hausaufgaben auf.Today, we were given a lot of', 'homework.'] }\\
\end{tabular}
}
%===aufheben===
\card{\normalfont \Huge aufheben}{
\begin{tabular}{lll}
\parbox[t][][t]{2.0 cm}{\normalfont \raggedleft ich\\du\\er/sie/es\\wir\\ihr\\sie} &    
\parbox[t][][t]{2cm}{\normalfont hebe auf\\hebst auf\\hebt auf\\heben auf\\hebt auf\\heben auf} &
\parbox[t][][t]{2cm}{\normalfont hob auf\\hobst auf\\hob auf\\hoben auf\\hobt auf\\hoben auf}\\
\end{tabular}
\begin{tabular}{l}
\parbox[t][][t]{8cm}{}\\
\parbox[t][][t]{8cm}{\normalfont \small ['to pick up (something lying on the ground) Er hob', 'seinen Hut wieder auf, der ihm vom Kopf geweht', 'wurde. He again picked up his hat that was blown', 'off his head. to abolish (rule, law, etc) Die', 'Prohibition wurde aufgehoben. Prohibition was', "abolished. to keep (can be reflexive if kept 'for", "someone') Er hob das Buch auf. He kept the book.", 'Er hob sich das Buch auf. He kept the book (for', 'himself). Er hob ihr das Buch auf. He kept the', 'book for her.'] }\\
\end{tabular}
}
%===aufhören===
\card{\normalfont \Huge aufhören}{
\begin{tabular}{lll}
\parbox[t][][t]{2.0 cm}{\normalfont \raggedleft ich\\du\\er/sie/es\\wir\\ihr\\sie} &    
\parbox[t][][t]{2cm}{\normalfont höre auf\\hörst auf\\hört auf\\hören auf\\hört auf\\hören auf} &
\parbox[t][][t]{2cm}{\normalfont hörte auf\\hörtest auf\\hörte auf\\hörten auf\\hörtet auf\\hörten auf}\\
\end{tabular}
\begin{tabular}{l}
\parbox[t][][t]{8cm}{}\\
\parbox[t][][t]{8cm}{\normalfont \small ['(with mit + noun or zu + infinitive) to stop; to', 'quit; to cease 1960, Marie Luise Kaschnitz,', "'Gespenster': Aber wir waren beide sehr mde und", 'darum hrten wir bald auf zu sprechen. But we were', 'both very tired, and we soon stopped talking about', 'it. Der Regen hrt bestimmt gleich auf.  Surely the', 'rain will stop soon. Hr damit auf und komm her!', 'Stop that and come here! Sie hat aufgehrt zu', "rauchen. Sie hat mit dem Rauchen aufgehrt.She's", 'stopped smoking. (informal, with von) to stop', 'talking about something Kannst du nicht mal davon', "aufhren? Das ist Jahre her.Can't you stop bringing", 'it up? This was years ago.'] }\\
\end{tabular}
}
%===aufpassen===
\card{\normalfont \Huge aufpassen}{
\begin{tabular}{lll}
\parbox[t][][t]{2.0 cm}{\normalfont \raggedleft ich\\du\\er/sie/es\\wir\\ihr\\sie} &    
\parbox[t][][t]{2cm}{\normalfont passe auf\\passt auf\\passt auf\\passen auf\\passt auf\\passen auf} &
\parbox[t][][t]{2cm}{\normalfont passte auf\\passtest auf\\passte auf\\passten auf\\passtet auf\\passten auf}\\
\end{tabular}
\begin{tabular}{l}
\parbox[t][][t]{8cm}{}\\
\parbox[t][][t]{8cm}{\normalfont \small ['to watch out to look after something Knntest du', 'kurz auf meinen Hund aufpassen?Could you look', 'after my dog for a moment? to be attentive (with', 'something) Er soll mit seinem Geld aufpassenHe', 'should be careful with his money.'] }\\
\end{tabular}
}
%===aufräumen===
\card{\normalfont \Huge aufräumen}{
\begin{tabular}{lll}
\parbox[t][][t]{2.0 cm}{\normalfont \raggedleft ich\\du\\er/sie/es\\wir\\ihr\\sie} &    
\parbox[t][][t]{2cm}{\normalfont räume auf\\räumst auf\\räumt auf\\räumen auf\\räumt auf\\räumen auf} &
\parbox[t][][t]{2cm}{\normalfont räumte auf\\räumtest auf\\räumte auf\\räumten auf\\räumtet auf\\räumten auf}\\
\end{tabular}
\begin{tabular}{l}
\parbox[t][][t]{8cm}{}\\
\parbox[t][][t]{8cm}{\normalfont \small ['to tidy up'] }\\
\end{tabular}
}
%===aufregen===
\card{\normalfont \Huge aufregen}{
\begin{tabular}{lll}
\parbox[t][][t]{2.0 cm}{\normalfont \raggedleft ich\\du\\er/sie/es\\wir\\ihr\\sie} &    
\parbox[t][][t]{2cm}{\normalfont rege auf\\regst auf\\regt auf\\regen auf\\regt auf\\regen auf} &
\parbox[t][][t]{2cm}{\normalfont regte auf\\regtest auf\\regte auf\\regten auf\\regtet auf\\regten auf}\\
\end{tabular}
\begin{tabular}{l}
\parbox[t][][t]{8cm}{}\\
\parbox[t][][t]{8cm}{\normalfont \small ['(transitive) to excite (reflexive) to annoy, to', 'upset'] }\\
\end{tabular}
}
%===aufstehen===
\card{\normalfont \Huge aufstehen}{
\begin{tabular}{lll}
\parbox[t][][t]{2.0 cm}{\normalfont \raggedleft ich\\du\\er/sie/es\\wir\\ihr\\sie} &    
\parbox[t][][t]{2cm}{\normalfont stehe auf\\stehst auf\\steht auf\\stehen auf\\steht auf\\stehen auf} &
\parbox[t][][t]{2cm}{\normalfont stand auf\\standest auf\\stand auf\\standen auf\\standet auf\\standen auf}\\
\end{tabular}
\begin{tabular}{l}
\parbox[t][][t]{8cm}{}\\
\parbox[t][][t]{8cm}{\normalfont \small ['(intransitive) to get up (move from a sitting or', 'lying position to a standing position; to rise', "from one's bed) (intransitive, colloquial, of", 'windows, doors, etc.) to be open (to not be closed', 'or shut)'] }\\
\end{tabular}
}
%===aufwachen===
\card{\normalfont \Huge aufwachen}{
\begin{tabular}{lll}
\parbox[t][][t]{2.0 cm}{\normalfont \raggedleft ich\\du\\er/sie/es\\wir\\ihr\\sie} &    
\parbox[t][][t]{2cm}{\normalfont wache auf\\wachst auf\\wacht auf\\wachen auf\\wacht auf\\wachen auf} &
\parbox[t][][t]{2cm}{\normalfont wachte auf\\wachtest auf\\wachte auf\\wachten auf\\wachtet auf\\wachten auf}\\
\end{tabular}
\begin{tabular}{l}
\parbox[t][][t]{8cm}{}\\
\parbox[t][][t]{8cm}{\normalfont \small ['(intransitive) to wake up'] }\\
\end{tabular}
}
%===ausgeben===
\card{\normalfont \Huge ausgeben}{
\begin{tabular}{lll}
\parbox[t][][t]{2.0 cm}{\normalfont \raggedleft ich\\du\\er/sie/es\\wir\\ihr\\sie} &    
\parbox[t][][t]{2cm}{\normalfont gebe aus\\gibst aus\\gibt aus\\geben aus\\gebt aus\\geben aus} &
\parbox[t][][t]{2cm}{\normalfont gab aus\\gabst aus\\gab aus\\gaben aus\\gabt aus\\gaben aus}\\
\end{tabular}
\begin{tabular}{l}
\parbox[t][][t]{8cm}{}\\
\parbox[t][][t]{8cm}{\normalfont \small ['to spend, to expend; to pay out; to issue (money)', 'to output (data) (reflexive) to pretend to be'] }\\
\end{tabular}
}
%===ausgehen===
\card{\normalfont \Huge ausgehen}{
\begin{tabular}{lll}
\parbox[t][][t]{2.0 cm}{\normalfont \raggedleft ich\\du\\er/sie/es\\wir\\ihr\\sie} &    
\parbox[t][][t]{2cm}{\normalfont gehe aus\\gehst aus\\geht aus\\gehen aus\\geht aus\\gehen aus} &
\parbox[t][][t]{2cm}{\normalfont ging aus\\gingst aus\\ging aus\\gingen aus\\gingt aus\\gingen aus}\\
\end{tabular}
\begin{tabular}{l}
\parbox[t][][t]{8cm}{}\\
\parbox[t][][t]{8cm}{\normalfont \small ["(intransitive) to go out (to leave one's abode to", 'go to public places) Ich gehe nicht in die Disko,', 'weil ich tanzen will, sondern weil ich ausgehen', 'will.I am not going to the nightclub because I', 'want to dance, but because I want to go out.', '(intransitive, colloquial, of a light, etc.) to go', 'out (to be turned off or extinguished) 1913, Fanny', 'zu Reventlow, Herrn Dames Aufzeichnungen, Albert', 'Langen, page 30: Dann ging die Flamme aus, und die', 'Lampe wurde wieder angezndet. Then the flame went', 'out, and the lamp was lit again. (intransitive) to', 'run out (to be completely used up or consumed) Das', 'Geld fr den Hausbau ist ausgegangen.The money for', 'building the house has run out. (intransitive,', 'especially of hair, teeth, etc.) to fall out (to', 'come out without being made to do so) Meine Haare', 'sind mir schon ausgegangen.My hair is already', 'falling out. (intransitive) to start, begin (von', '("at")); to come, stem, lead off, radiate (von', '("from")) (to originate (at or from a certain', 'location)) (intransitive) to start (from), to take', "as one's starting point (intransitive) to end,", 'turn out (to have a given result) (intransitive)', 'to leave, get away, come away (to depart, implying', 'a certain consequence or result, or lack thereof)', 'leer ausgehen  to leave empty-handed (reflexive)', 'to work out (to have a satisfactory result)', '(reflexive) to be sufficient, be enough (to be', 'present in adequate quantity)'] }\\
\end{tabular}
}
%===ausmachen===
\card{\normalfont \Huge ausmachen}{
\begin{tabular}{lll}
\parbox[t][][t]{2.0 cm}{\normalfont \raggedleft ich\\du\\er/sie/es\\wir\\ihr\\sie} &    
\parbox[t][][t]{2cm}{\normalfont mache aus\\machst aus\\macht aus\\machen aus\\macht aus\\machen aus} &
\parbox[t][][t]{2cm}{\normalfont machte aus\\machtest aus\\machte aus\\machten aus\\machtet aus\\machten aus}\\
\end{tabular}
\begin{tabular}{l}
\parbox[t][][t]{8cm}{}\\
\parbox[t][][t]{8cm}{\normalfont \small ['to turn off, switch off Er macht den Fernseher', 'aus.  He is turning off the TV. Antonyms:', 'anstellen, anmachen, anschalten, einschalten,', 'anschmeien Synonyms: abschalten, abstellen,', 'ausstellen, ausschalten to put out, to extinguish', 'Ich mache die Kerze aus.  I am putting out the', 'candle. Machst du bitte das Gas aus?  Would you', 'put out the gas, please? Antonyms: anmachen,', 'anbrennen, entznden, anznden Synonyms: erlschen,', 'auslschen, lschen to agree to to make a difference', 'Es macht viel aus.  It makes a big difference. to', 'make out'] }\\
\end{tabular}
}
%===ausschalten===
\card{\normalfont \Huge ausschalten}{
\begin{tabular}{lll}
\parbox[t][][t]{2.0 cm}{\normalfont \raggedleft ich\\du\\er/sie/es\\wir\\ihr\\sie} &    
\parbox[t][][t]{2cm}{\normalfont schalte aus\\schaltest aus\\schaltet aus\\schalten aus\\schaltet aus\\schalten aus} &
\parbox[t][][t]{2cm}{\normalfont schaltete aus\\schaltetest aus\\schaltete aus\\schalteten aus\\schaltetet aus\\schalteten aus}\\
\end{tabular}
\begin{tabular}{l}
\parbox[t][][t]{8cm}{}\\
\parbox[t][][t]{8cm}{\normalfont \small ['to turn off, to switch off, to disable (to put a', 'device out of operation or deactivate its', 'functionality) Synonyms: ausmachen, ausstellen,', 'abstellen, abschalten, deaktivieren Antonyms:', 'einschalten, anmachen, anstellen, anschalten,', 'anschmeien, aktivieren'] }\\
\end{tabular}
}
%===ausschließen===
\card{\normalfont \Huge ausschließen}{
\begin{tabular}{lll}
\parbox[t][][t]{2.0 cm}{\normalfont \raggedleft ich\\du\\er/sie/es\\wir\\ihr\\sie} &    
\parbox[t][][t]{2cm}{\normalfont schließe aus\\schließt aus\\schließt aus\\schließen aus\\schließt aus\\schließen aus} &
\parbox[t][][t]{2cm}{\normalfont schloss aus\\schlossest aus\\schloss aus\\schlossen aus\\schlosst aus/ schlosset aus (obsolete)\\schlossen aus}\\
\end{tabular}
\begin{tabular}{l}
\parbox[t][][t]{8cm}{}\\
\parbox[t][][t]{8cm}{\normalfont \small ['to exclude, to debar to rule out 2015, Tim Krohn,', 'Nachts in Vals: denn auch das Einerzimmer hatte', 'Doppelbett, und natrlich schloss Marlette gleich', 'kategorisch aus, im Kinderbett zu schlafen []. For', 'even the single room had a double bed, and', 'naturally Marlette categorically ruled out', "sleeping on the child's bed. (reflexive) to lock", 'out oneself'] }\\
\end{tabular}
}
%===aussehen===
\card{\normalfont \Huge aussehen}{
\begin{tabular}{lll}
\parbox[t][][t]{2.0 cm}{\normalfont \raggedleft ich\\du\\er/sie/es\\wir\\ihr\\sie} &    
\parbox[t][][t]{2cm}{\normalfont sehe aus\\siehst aus\\sieht aus\\sehen aus\\seht aus\\sehen aus} &
\parbox[t][][t]{2cm}{\normalfont sah aus\\sahst aus\\sah aus\\sahen aus\\saht aus\\sahen aus}\\
\end{tabular}
\begin{tabular}{l}
\parbox[t][][t]{8cm}{}\\
\parbox[t][][t]{8cm}{\normalfont \small ['(intransitive) to look, seem (wie/nach like) Du', 'siehst mde aus.  You look tired. 2017, Simone', 'Meier, Fleisch, Kein  Aber, p. 7: Der Mann sah aus', 'wie eine geschlte Kellerassel, und sie fragte', 'sich: Wieso sind Schnheitschirurgen nie schn? The', 'man looked like a peeled woodlouse, and she', 'wondered: How come cosmetic surgeons are never', 'good looking? (intransitive, colloquial) to look', 'good or bad; depending on intonation Das sieht', 'aus!  Ich find schon, dass das aussieht.That looks', 'terrible!  I actually think it looks nice.', '(intransitive, with nach) to look for; to watch', 'out for Er sa am Ufer und sah nach den Booten', 'aus.He sat at the shore and watched out for the', 'boats. (transitive, usually with reflexive dative,', 'possibly dated) to look for and find; to pick; to', 'select Sieh dir den Besten aus!Find yourself the', 'best!'] }\\
\end{tabular}
}
%===aussprechen===
\card{\normalfont \Huge aussprechen}{
\begin{tabular}{lll}
\parbox[t][][t]{2.0 cm}{\normalfont \raggedleft ich\\du\\er/sie/es\\wir\\ihr\\sie} &    
\parbox[t][][t]{2cm}{\normalfont spreche aus\\sprichst aus\\spricht aus\\sprechen aus\\sprecht aus\\sprechen aus} &
\parbox[t][][t]{2cm}{\normalfont sprach aus\\sprachst aus\\sprach aus\\sprachen aus\\spracht aus\\sprachen aus}\\
\end{tabular}
\begin{tabular}{l}
\parbox[t][][t]{8cm}{}\\
\parbox[t][][t]{8cm}{\normalfont \small ['to pronounce (to sound out a word), utter', '(reflexive, with fr) to declare oneself in favor', 'of, to declare oneself for, to recommend'] }\\
\end{tabular}
}
%===ausstellen===
\card{\normalfont \Huge ausstellen}{
\begin{tabular}{lll}
\parbox[t][][t]{2.0 cm}{\normalfont \raggedleft ich\\du\\er/sie/es\\wir\\ihr\\sie} &    
\parbox[t][][t]{2cm}{\normalfont stelle aus\\stellst aus\\stellt aus\\stellen aus\\stellt aus\\stellen aus} &
\parbox[t][][t]{2cm}{\normalfont stellte aus\\stelltest aus\\stellte aus\\stellten aus\\stelltet aus\\stellten aus}\\
\end{tabular}
\begin{tabular}{l}
\parbox[t][][t]{8cm}{}\\
\parbox[t][][t]{8cm}{\normalfont \small ['to exhibit, display to issue, certificate to', 'switch off Antonyms: anstellen, anmachen,', 'anschalten, einschalten, anschmeien Synonyms:', 'ausmachen, abschalten, ausschalten, abstellen'] }\\
\end{tabular}
}
%===ausziehen===
\card{\normalfont \Huge ausziehen}{
\begin{tabular}{lll}
\parbox[t][][t]{2.0 cm}{\normalfont \raggedleft ich\\du\\er/sie/es\\wir\\ihr\\sie} &    
\parbox[t][][t]{2cm}{\normalfont ziehe aus\\ziehst aus\\zieht aus\\ziehen aus\\zieht aus\\ziehen aus} &
\parbox[t][][t]{2cm}{\normalfont zog aus\\zogst aus\\zog aus\\zogen aus\\zogt aus\\zogen aus}\\
\end{tabular}
\begin{tabular}{l}
\parbox[t][][t]{8cm}{}\\
\parbox[t][][t]{8cm}{\normalfont \small ['to pull out (auxiliary verb: haben) (transitive,', 'of clothing) to take off (auxiliary verb: haben)', '(transitive, a person) to undress (auxiliary verb:', 'haben) (reflexive: sich ausziehen) to undress, to', "take off one's clothes (auxiliary verb: haben)", '(housing) to move out (auxiliary verb: sein) to', 'unscrew (auxiliary verb: haben)'] }\\
\end{tabular}
}
%===backen===
\card{\normalfont \Huge backen}{
\begin{tabular}{lll}
\parbox[t][][t]{2.0 cm}{\normalfont \raggedleft ich\\du\\er/sie/es\\wir\\ihr\\sie} &    
\parbox[t][][t]{2cm}{\normalfont backe\\backst\\backt\\backen\\backt\\backen} &
\parbox[t][][t]{2cm}{\normalfont backte\\backtest\\backte\\backten\\backtet\\backten}\\
\end{tabular}
\begin{tabular}{l}
\parbox[t][][t]{8cm}{}\\
\parbox[t][][t]{8cm}{\normalfont \small ['(transitive or intransitive) to bake; to roast Der', 'Bcker backt jeden Morgen 30 Laib Brot.The baker', 'bakes 30 loaves of bread every morning. Ist der', 'Kuchen schon gebacken?Is the cake baked yet?', '(transitive or intransitive, colloquial, regional)', 'to fry (transitive or intransitive, chiefly', 'pottery) to fire Die Tonfigur muss mindestens zwei', 'Stunden im Ofen backen.The clay piece must be', 'fired in the oven for at least two hours.', '(intransitive) to stick together; to cake. Der', 'Schnee backte gestern besser.The snow caked better', 'yesterday. (transitive) to stick (something to', 'something else).'] }\\
\end{tabular}
}
%===baden===
\card{\normalfont \Huge baden}{
\begin{tabular}{lll}
\parbox[t][][t]{2.0 cm}{\normalfont \raggedleft ich\\du\\er/sie/es\\wir\\ihr\\sie} &    
\parbox[t][][t]{2cm}{\normalfont bade\\badest\\badet\\baden\\badet\\baden} &
\parbox[t][][t]{2cm}{\normalfont badete\\badetest\\badete\\badeten\\badetet\\badeten}\\
\end{tabular}
\begin{tabular}{l}
\parbox[t][][t]{8cm}{}\\
\parbox[t][][t]{8cm}{\normalfont \small ['to bathe'] }\\
\end{tabular}
}
%===bauen===
\card{\normalfont \Huge bauen}{
\begin{tabular}{lll}
\parbox[t][][t]{2.0 cm}{\normalfont \raggedleft ich\\du\\er/sie/es\\wir\\ihr\\sie} &    
\parbox[t][][t]{2cm}{\normalfont baue\\baust\\baut\\bauen\\baut\\bauen} &
\parbox[t][][t]{2cm}{\normalfont baute\\bautest\\baute\\bauten\\bautet\\bauten}\\
\end{tabular}
\begin{tabular}{l}
\parbox[t][][t]{8cm}{}\\
\parbox[t][][t]{8cm}{\normalfont \small ['to build; to construct (drug-related) to roll a', 'joint Baust du einen? Ich kann grad nicht mehr.Can', "you roll one? I just can't anymore. to rely [+ auf", '(object) = on]'] }\\
\end{tabular}
}
%===beachten===
\card{\normalfont \Huge beachten}{
\begin{tabular}{lll}
\parbox[t][][t]{2.0 cm}{\normalfont \raggedleft ich\\du\\er/sie/es\\wir\\ihr\\sie} &    
\parbox[t][][t]{2cm}{\normalfont beachte\\beachtest\\beachtet\\beachten\\beachtet\\beachten} &
\parbox[t][][t]{2cm}{\normalfont beachtete\\beachtetest\\beachtete\\beachteten\\beachtetet\\beachteten}\\
\end{tabular}
\begin{tabular}{l}
\parbox[t][][t]{8cm}{}\\
\parbox[t][][t]{8cm}{\normalfont \small ['To note, notice, observe To mind, heed'] }\\
\end{tabular}
}
%===beantragen===
\card{\normalfont \Huge beantragen}{
\begin{tabular}{lll}
\parbox[t][][t]{2.0 cm}{\normalfont \raggedleft ich\\du\\er/sie/es\\wir\\ihr\\sie} &    
\parbox[t][][t]{2cm}{\normalfont beantrage\\beantragst\\beantragt\\beantragen\\beantragt\\beantragen} &
\parbox[t][][t]{2cm}{\normalfont beantragte\\beantragtest\\beantragte\\beantragten\\beantragtet\\beantragten}\\
\end{tabular}
\begin{tabular}{l}
\parbox[t][][t]{8cm}{}\\
\parbox[t][][t]{8cm}{\normalfont \small ['to apply for to request, to propose'] }\\
\end{tabular}
}
%===bedeuten===
\card{\normalfont \Huge bedeuten}{
\begin{tabular}{lll}
\parbox[t][][t]{2.0 cm}{\normalfont \raggedleft ich\\du\\er/sie/es\\wir\\ihr\\sie} &    
\parbox[t][][t]{2cm}{\normalfont bedeute\\bedeutest\\bedeutet\\bedeuten\\bedeutet\\bedeuten} &
\parbox[t][][t]{2cm}{\normalfont bedeutete\\bedeutetest\\bedeutete\\bedeuteten\\bedeutetet\\bedeuteten}\\
\end{tabular}
\begin{tabular}{l}
\parbox[t][][t]{8cm}{}\\
\parbox[t][][t]{8cm}{\normalfont \small ['to mean, to signify'] }\\
\end{tabular}
}
%===bedienen===
\card{\normalfont \Huge bedienen}{
\begin{tabular}{lll}
\parbox[t][][t]{2.0 cm}{\normalfont \raggedleft ich\\du\\er/sie/es\\wir\\ihr\\sie} &    
\parbox[t][][t]{2cm}{\normalfont bediene\\bedienst\\bedient\\bedienen\\bedient\\bedienen} &
\parbox[t][][t]{2cm}{\normalfont bediente\\bedientest\\bediente\\bedienten\\bedientet\\bedienten}\\
\end{tabular}
\begin{tabular}{l}
\parbox[t][][t]{8cm}{}\\
\parbox[t][][t]{8cm}{\normalfont \small ['(transitive) to serve (transitive) to operate', '(transitive or intransitive, card games) to play a', 'card according to the grouping of the first card', 'of a trick; to follow suit Du musst Trumpf', 'bedienen.You must follow suit and play trump.', '(reflexive, + genitive) to help oneself (to); to', 'use'] }\\
\end{tabular}
}
%===bedingen===
\card{\normalfont \Huge bedingen}{
\begin{tabular}{lll}
\parbox[t][][t]{2.0 cm}{\normalfont \raggedleft ich\\du\\er/sie/es\\wir\\ihr\\sie} &    
\parbox[t][][t]{2cm}{\normalfont bedinge\\bedingst\\bedingt\\bedingen\\bedingt\\bedingen} &
\parbox[t][][t]{2cm}{\normalfont bedingte\\bedingtest\\bedingte\\bedingten\\bedingtet\\bedingten}\\
\end{tabular}
\begin{tabular}{l}
\parbox[t][][t]{8cm}{}\\
\parbox[t][][t]{8cm}{\normalfont \small ['to cause Der Stau auf der A1 bedingt eine lngere', 'Versptung.The traffic jam on the A1 is causing a', 'rather long delay. to require, have as a', 'prerequisite or condition Eine Jahressonderzahlung', 'bedingt berdurchschnittlichen Einsatz.A yearly', 'bonus requires/is conditional upon above-average', 'effort. Motivation und Erfolg bedingen sich', 'gegenseitig.Motivation and success are', 'prerequisites for each other.'] }\\
\end{tabular}
}
%===befehlen===
\card{\normalfont \Huge befehlen}{
\begin{tabular}{lll}
\parbox[t][][t]{2.0 cm}{\normalfont \raggedleft ich\\du\\er/sie/es\\wir\\ihr\\sie} &    
\parbox[t][][t]{2cm}{\normalfont befehle\\befiehlst\\befiehlt\\befehlen\\befehlt\\befehlen} &
\parbox[t][][t]{2cm}{\normalfont befahl\\befahlst\\befahl\\befahlen\\befahlt\\befahlen}\\
\end{tabular}
\begin{tabular}{l}
\parbox[t][][t]{8cm}{}\\
\parbox[t][][t]{8cm}{\normalfont \small ['(transitive) to command; to order (someone to do', 'something) (intransitive) to give orders; to be in', 'command (dated) to entrust'] }\\
\end{tabular}
}
%===befinden===
\card{\normalfont \Huge befinden}{
\begin{tabular}{lll}
\parbox[t][][t]{2.0 cm}{\normalfont \raggedleft ich\\du\\er/sie/es\\wir\\ihr\\sie} &    
\parbox[t][][t]{2cm}{\normalfont befinde\\befindest\\befindet\\befinden\\befindet\\befinden} &
\parbox[t][][t]{2cm}{\normalfont befand\\befandest\\befand\\befanden\\befandet\\befanden}\\
\end{tabular}
\begin{tabular}{l}
\parbox[t][][t]{8cm}{}\\
\parbox[t][][t]{8cm}{\normalfont \small ['(reflexive) to occupy a place; to be located; to', 'be situated Der Tresor befindet sich hinter dem', 'Bild im Wohnzimmer.  The safe is situated behind', 'the picture in the living room. (transitive,', 'formal) to find, to consider jemanden fr schuldig', 'eines Mordes befindento find someone guilty of', 'murder, to convict someone of murder'] }\\
\end{tabular}
}
%===befreien===
\card{\normalfont \Huge befreien}{
\begin{tabular}{lll}
\parbox[t][][t]{2.0 cm}{\normalfont \raggedleft ich\\du\\er/sie/es\\wir\\ihr\\sie} &    
\parbox[t][][t]{2cm}{\normalfont befreie\\befreist\\befreit\\befreien\\befreit\\befreien} &
\parbox[t][][t]{2cm}{\normalfont befreite\\befreitest\\befreite\\befreiten\\befreitet\\befreiten}\\
\end{tabular}
\begin{tabular}{l}
\parbox[t][][t]{8cm}{}\\
\parbox[t][][t]{8cm}{\normalfont \small ['to free (make free), to liberate'] }\\
\end{tabular}
}
%===begegnen===
\card{\normalfont \Huge begegnen}{
\begin{tabular}{lll}
\parbox[t][][t]{2.0 cm}{\normalfont \raggedleft ich\\du\\er/sie/es\\wir\\ihr\\sie} &    
\parbox[t][][t]{2cm}{\normalfont begegne\\begegnest\\begegnet\\begegnen\\begegnet\\begegnen} &
\parbox[t][][t]{2cm}{\normalfont begegnete\\begegnetest\\begegnete\\begegneten\\begegnetet\\begegneten}\\
\end{tabular}
\begin{tabular}{l}
\parbox[t][][t]{8cm}{}\\
\parbox[t][][t]{8cm}{\normalfont \small ['(intransitive) to meet, encounter'] }\\
\end{tabular}
}
%===beginnen===
\card{\normalfont \Huge beginnen}{
\begin{tabular}{lll}
\parbox[t][][t]{2.0 cm}{\normalfont \raggedleft ich\\du\\er/sie/es\\wir\\ihr\\sie} &    
\parbox[t][][t]{2cm}{\normalfont beginne\\beginnst\\beginnt\\beginnen\\beginnt\\beginnen} &
\parbox[t][][t]{2cm}{\normalfont begann\\begannst\\begann\\begannen\\begannt\\begannen}\\
\end{tabular}
\begin{tabular}{l}
\parbox[t][][t]{8cm}{}\\
\parbox[t][][t]{8cm}{\normalfont \small ['(intransitive) to begin; to commence; to be', 'started Der Vortrag hat begonnen.  "The lecture', 'has begun." (chiefly literary, transitive  or with', 'mit) to start something; to begin something Er hat', 'den Vortrag begonnen.  "He has started the', 'lecture." Er hat mit dem Vortrag begonnen.  "He', 'has started the lecture."'] }\\
\end{tabular}
}
%===begleiten===
\card{\normalfont \Huge begleiten}{
\begin{tabular}{lll}
\parbox[t][][t]{2.0 cm}{\normalfont \raggedleft ich\\du\\er/sie/es\\wir\\ihr\\sie} &    
\parbox[t][][t]{2cm}{\normalfont begleite\\begleitest\\begleitet\\begleiten\\begleitet\\begleiten} &
\parbox[t][][t]{2cm}{\normalfont begleitete\\begleitetest\\begleitete\\begleiteten\\begleitetet\\begleiteten}\\
\end{tabular}
\begin{tabular}{l}
\parbox[t][][t]{8cm}{}\\
\parbox[t][][t]{8cm}{\normalfont \small ['to accompany to conduct, escort'] }\\
\end{tabular}
}
%===begründen===
\card{\normalfont \Huge begründen}{
\begin{tabular}{lll}
\parbox[t][][t]{2.0 cm}{\normalfont \raggedleft ich\\du\\er/sie/es\\wir\\ihr\\sie} &    
\parbox[t][][t]{2cm}{\normalfont begründe\\begründest\\begründet\\begründen\\begründet\\begründen} &
\parbox[t][][t]{2cm}{\normalfont begründete\\begründetest\\begründete\\begründeten\\begründetet\\begründeten}\\
\end{tabular}
\begin{tabular}{l}
\parbox[t][][t]{8cm}{}\\
\parbox[t][][t]{8cm}{\normalfont \small ['to make an argument for something, to justify, to', 'give a reason to establish, to found'] }\\
\end{tabular}
}
%===begrüßen===
\card{\normalfont \Huge begrüßen}{
\begin{tabular}{lll}
\parbox[t][][t]{2.0 cm}{\normalfont \raggedleft ich\\du\\er/sie/es\\wir\\ihr\\sie} &    
\parbox[t][][t]{2cm}{\normalfont begrüße\\begrüßt\\begrüßt\\begrüßen\\begrüßt\\begrüßen} &
\parbox[t][][t]{2cm}{\normalfont begrüßte\\begrüßtest\\begrüßte\\begrüßten\\begrüßtet\\begrüßten}\\
\end{tabular}
\begin{tabular}{l}
\parbox[t][][t]{8cm}{}\\
\parbox[t][][t]{8cm}{\normalfont \small ['to welcome Wir begren euch zu unserer Webseite.We', 'welcome you to our website.'] }\\
\end{tabular}
}
%===behalten===
\card{\normalfont \Huge behalten}{
\begin{tabular}{lll}
\parbox[t][][t]{2.0 cm}{\normalfont \raggedleft ich\\du\\er/sie/es\\wir\\ihr\\sie} &    
\parbox[t][][t]{2cm}{\normalfont behalte\\behältst\\behält\\behalten\\behaltet\\behalten} &
\parbox[t][][t]{2cm}{\normalfont behielt\\behieltest\\behielt\\behielten\\behieltet\\behielten}\\
\end{tabular}
\begin{tabular}{l}
\parbox[t][][t]{8cm}{}\\
\parbox[t][][t]{8cm}{\normalfont \small ['to keep to remember'] }\\
\end{tabular}
}
%===behandeln===
\card{\normalfont \Huge behandeln}{
\begin{tabular}{lll}
\parbox[t][][t]{2.0 cm}{\normalfont \raggedleft ich\\du\\er/sie/es\\wir\\ihr\\sie} &    
\parbox[t][][t]{2cm}{\normalfont behandleich behandeleich behandel\\behandelst\\behandelt\\behandeln\\behandelt\\behandeln} &
\parbox[t][][t]{2cm}{\normalfont behandelte\\behandeltest\\behandelte\\behandelten\\behandeltet\\behandelten}\\
\end{tabular}
\begin{tabular}{l}
\parbox[t][][t]{8cm}{}\\
\parbox[t][][t]{8cm}{\normalfont \small ['to treat (medicinally) Der Arzt behandelte mich.', 'The physician treated me. to treat (physically,', 'psychologically) Ich habe dich immer gut', 'behandelt. I always treated you well. to cover der', 'Reporter behandelt das Thema. The reporter is', 'covering the subject.'] }\\
\end{tabular}
}
%===behaupten===
\card{\normalfont \Huge behaupten}{
\begin{tabular}{lll}
\parbox[t][][t]{2.0 cm}{\normalfont \raggedleft ich\\du\\er/sie/es\\wir\\ihr\\sie} &    
\parbox[t][][t]{2cm}{\normalfont behaupte\\behauptest\\behauptet\\behaupten\\behauptet\\behaupten} &
\parbox[t][][t]{2cm}{\normalfont behauptete\\behauptetest\\behauptete\\behaupteten\\behauptetet\\behaupteten}\\
\end{tabular}
\begin{tabular}{l}
\parbox[t][][t]{8cm}{}\\
\parbox[t][][t]{8cm}{\normalfont \small ['to claim, maintain, assert (reflexive) to stand', "one's ground"] }\\
\end{tabular}
}
%===behindern===
\card{\normalfont \Huge behindern}{
\begin{tabular}{lll}
\parbox[t][][t]{2.0 cm}{\normalfont \raggedleft ich\\du\\er/sie/es\\wir\\ihr\\sie} &    
\parbox[t][][t]{2cm}{\normalfont behindreich behindereich behinder\\behinderst\\behindert\\behindern\\behindert\\behindern} &
\parbox[t][][t]{2cm}{\normalfont behinderte\\behindertest\\behinderte\\behinderten\\behindertet\\behinderten}\\
\end{tabular}
\begin{tabular}{l}
\parbox[t][][t]{8cm}{}\\
\parbox[t][][t]{8cm}{\normalfont \small ['to hinder'] }\\
\end{tabular}
}
%===beißen===
\card{\normalfont \Huge beißen}{
\begin{tabular}{lll}
\parbox[t][][t]{2.0 cm}{\normalfont \raggedleft ich\\du\\er/sie/es\\wir\\ihr\\sie} &    
\parbox[t][][t]{2cm}{\normalfont beiße\\beißt\\beißt\\beißen\\beißt\\beißen} &
\parbox[t][][t]{2cm}{\normalfont biss\\bisst\\biss\\bissen\\bisst\\bissen}\\
\end{tabular}
\begin{tabular}{l}
\parbox[t][][t]{8cm}{}\\
\parbox[t][][t]{8cm}{\normalfont \small ['(transitive or intransitive) to bite (transitive', 'or intransitive) to sting; to burn; to be sharp;', 'to be spicy (reflexive, slightly informal, of', 'colors and clothes) to clash; to jar'] }\\
\end{tabular}
}
%===bekommen===
\card{\normalfont \Huge bekommen}{
\begin{tabular}{lll}
\parbox[t][][t]{2.0 cm}{\normalfont \raggedleft ich\\du\\er/sie/es\\wir\\ihr\\sie} &    
\parbox[t][][t]{2cm}{\normalfont bekomme\\bekommst\\bekommt\\bekommen\\bekommt\\bekommen} &
\parbox[t][][t]{2cm}{\normalfont bekam\\bekamst\\bekam\\bekamen\\bekamt\\bekamen}\\
\end{tabular}
\begin{tabular}{l}
\parbox[t][][t]{8cm}{}\\
\parbox[t][][t]{8cm}{\normalfont \small ['(transitive, auxiliary: "haben") to receive; to', 'get 2003, Franz Eugen Schlachter, Die Bibel', '("Schlachter 2000"), Genfer Bibelgesellschaft,', 'Numbers 34:14: Denn der Stamm der Kinder Rubens', 'mit ihren Vaterhusern, und der Stamm der Kinder', 'Gads mit ihren Vaterhusern haben ihren Teil', 'empfangen, auch der halbe Stamm Manasse hat seinen', "Teil bekommen. Then the tribe of Reuben's children", 'with the houses of their father, and the tribe of', "Gad's children with the houses of their father", 'have obtained their part, also half the tribe of', 'Manasseh got their part. Hunger bekommen  to get', 'hungry, become hungry (literally, "to get hunger")', 'ein Kind bekommen  to have a child (transitive,', 'auxiliary: "haben") to catch den Schnupfen', 'bekommen  to catch a cold (intransitive, with a', 'dative case object, of food or drink, auxiliary:', '"sein") to agree with (someone); to sit well with', '(someone) Das Essen bekommt ihm nicht.The food', 'does not agree with him.'] }\\
\end{tabular}
}
%===beleidigen===
\card{\normalfont \Huge beleidigen}{
\begin{tabular}{lll}
\parbox[t][][t]{2.0 cm}{\normalfont \raggedleft ich\\du\\er/sie/es\\wir\\ihr\\sie} &    
\parbox[t][][t]{2cm}{\normalfont beleidige\\beleidigst\\beleidigt\\beleidigen\\beleidigt\\beleidigen} &
\parbox[t][][t]{2cm}{\normalfont beleidigte\\beleidigtest\\beleidigte\\beleidigten\\beleidigtet\\beleidigten}\\
\end{tabular}
\begin{tabular}{l}
\parbox[t][][t]{8cm}{}\\
\parbox[t][][t]{8cm}{\normalfont \small ['To offend'] }\\
\end{tabular}
}
%===bellen===
\card{\normalfont \Huge bellen}{
\begin{tabular}{lll}
\parbox[t][][t]{2.0 cm}{\normalfont \raggedleft ich\\du\\er/sie/es\\wir\\ihr\\sie} &    
\parbox[t][][t]{2cm}{\normalfont belle\\bellst\\bellt\\bellen\\bellt\\bellen} &
\parbox[t][][t]{2cm}{\normalfont bellte\\belltest\\bellte\\bellten\\belltet\\bellten}\\
\end{tabular}
\begin{tabular}{l}
\parbox[t][][t]{8cm}{}\\
\parbox[t][][t]{8cm}{\normalfont \small ['(intransitive) to bark: (literally) like a canine', '1929, Kurt Tucholsky, Das Lcheln der Mona Lisa', '(Sammelband), Ernst Rowohlt Verlag, page 138: Ein', 'Hund bellt, wenn er mit den Sinnen etwas', 'wahrgenommen hat; daraufhin, weil ihn sein Bellen', 'erschreckt und aufregt, und des weiteren, weil', 'sich das wahrgenommene Objekt um ihn kmmert, nicht', 'um ihn kmmert oder davonluft. A dog barks when he', 'perceived something with the senses; thereupon,', 'because his barking scares and upsets him, and', 'furthermore, because the perceived object looks', 'after him, does not look after him, or runs away.', '(figuratively) in a rude, loud human voice'] }\\
\end{tabular}
}
%===benutzen===
\card{\normalfont \Huge benutzen}{
\begin{tabular}{lll}
\parbox[t][][t]{2.0 cm}{\normalfont \raggedleft ich\\du\\er/sie/es\\wir\\ihr\\sie} &    
\parbox[t][][t]{2cm}{\normalfont benutze\\benutzt\\benutzt\\benutzen\\benutzt\\benutzen} &
\parbox[t][][t]{2cm}{\normalfont benutzte\\benutztest\\benutzte\\benutzten\\benutztet\\benutzten}\\
\end{tabular}
\begin{tabular}{l}
\parbox[t][][t]{8cm}{}\\
\parbox[t][][t]{8cm}{\normalfont \small ['to use (employ, apply)'] }\\
\end{tabular}
}
%===beobachten===
\card{\normalfont \Huge beobachten}{
\begin{tabular}{lll}
\parbox[t][][t]{2.0 cm}{\normalfont \raggedleft ich\\du\\er/sie/es\\wir\\ihr\\sie} &    
\parbox[t][][t]{2cm}{\normalfont beobachte\\beobachtest\\beobachtet\\beobachten\\beobachtet\\beobachten} &
\parbox[t][][t]{2cm}{\normalfont beobachtete\\beobachtetest\\beobachtete\\beobachteten\\beobachtetet\\beobachteten}\\
\end{tabular}
\begin{tabular}{l}
\parbox[t][][t]{8cm}{}\\
\parbox[t][][t]{8cm}{\normalfont \small ['to watch, to observe 2010, Der Spiegel, issue', '5/2010, page 106: Eine neue Generation', 'hochauflsender Lichtmikroskope revolutioniert die', 'Biologie: Erstmals knnen die Forscher auch', 'molekulare Strukturen in lebenden Zellen', 'beobachten. A new generation of high-resolution', 'optical microscopes is revolutionizing biology:', 'for the first time researchers are able to also', 'watch molecular structures in living cells.'] }\\
\end{tabular}
}
%===beraten===
\card{\normalfont \Huge beraten}{
\begin{tabular}{lll}
\parbox[t][][t]{2.0 cm}{\normalfont \raggedleft ich\\du\\er/sie/es\\wir\\ihr\\sie} &    
\parbox[t][][t]{2cm}{\normalfont berate\\berätst\\berät\\beraten\\beratet\\beraten} &
\parbox[t][][t]{2cm}{\normalfont beriet\\berietest\\beriet\\berieten\\berietet\\berieten}\\
\end{tabular}
\begin{tabular}{l}
\parbox[t][][t]{8cm}{}\\
\parbox[t][][t]{8cm}{\normalfont \small ['(intransitive or reflexive) to discuss', '(transitive) to advise'] }\\
\end{tabular}
}
%===bereiten===
\card{\normalfont \Huge bereiten}{
\begin{tabular}{lll}
\parbox[t][][t]{2.0 cm}{\normalfont \raggedleft ich\\du\\er/sie/es\\wir\\ihr\\sie} &    
\parbox[t][][t]{2cm}{\normalfont bereite\\bereitest\\bereitet\\bereiten\\bereitet\\bereiten} &
\parbox[t][][t]{2cm}{\normalfont bereitete\\bereitetest\\bereitete\\bereiteten\\bereitetet\\bereiteten}\\
\end{tabular}
\begin{tabular}{l}
\parbox[t][][t]{8cm}{}\\
\parbox[t][][t]{8cm}{\normalfont \small ['(meals, coffee, etc.) to prepare, to make 1924,', 'Friedrich Bernhard Strzner, Das alte Kruterweib', 'von Reinhardtswalde, in: Reinhardtswalder', 'Sagenbchlein, Buchhandlung Otto Schmidt, page 18:', 'Auf einem Brettersims an der Wand standen', 'zahlreiche Flschchen und Bchsen, gefllt mit', 'Mixturen und Salben, die das Weib aus den', 'gesammelten Heilkrutern selbst bereitete. On a', 'shelf at the wall were numerous flasks and cans,', 'filled with mixtures and ointments, which the', 'woman prepared herself from the collected', 'medicinal herbs. (pleasure, trouble, etc.) to', 'give, to cause'] }\\
\end{tabular}
}
%===berichten===
\card{\normalfont \Huge berichten}{
\begin{tabular}{lll}
\parbox[t][][t]{2.0 cm}{\normalfont \raggedleft ich\\du\\er/sie/es\\wir\\ihr\\sie} &    
\parbox[t][][t]{2cm}{\normalfont berichte\\berichtest\\berichtet\\berichten\\berichtet\\berichten} &
\parbox[t][][t]{2cm}{\normalfont berichtete\\berichtetest\\berichtete\\berichteten\\berichtetet\\berichteten}\\
\end{tabular}
\begin{tabular}{l}
\parbox[t][][t]{8cm}{}\\
\parbox[t][][t]{8cm}{\normalfont \small ['to report'] }\\
\end{tabular}
}
%===bersten===
\card{\normalfont \Huge bersten}{
\begin{tabular}{lll}
\parbox[t][][t]{2.0 cm}{\normalfont \raggedleft ich\\du\\er/sie/es\\wir\\ihr\\sie} &    
\parbox[t][][t]{2cm}{\normalfont berste\\birst\\birst\\bersten\\berstet\\bersten} &
\parbox[t][][t]{2cm}{\normalfont barst\\barstest\\barst\\barsten\\barstet\\barsten}\\
\end{tabular}
\begin{tabular}{l}
\parbox[t][][t]{8cm}{}\\
\parbox[t][][t]{8cm}{\normalfont \small ['(intransitive, chiefly literary) to burst'] }\\
\end{tabular}
}
%===berufen===
\card{\normalfont \Huge berufen}{
\begin{tabular}{lll}
\parbox[t][][t]{2.0 cm}{\normalfont \raggedleft ich\\du\\er/sie/es\\wir\\ihr\\sie} &    
\parbox[t][][t]{2cm}{\normalfont berufe\\berufst\\beruft\\berufen\\beruft\\berufen} &
\parbox[t][][t]{2cm}{\normalfont berief\\beriefst\\berief\\beriefen\\berieft\\beriefen}\\
\end{tabular}
\begin{tabular}{l}
\parbox[t][][t]{8cm}{}\\
\parbox[t][][t]{8cm}{\normalfont \small ['to appoint (law) to invoke'] }\\
\end{tabular}
}
%===beruhigen===
\card{\normalfont \Huge beruhigen}{
\begin{tabular}{lll}
\parbox[t][][t]{2.0 cm}{\normalfont \raggedleft ich\\du\\er/sie/es\\wir\\ihr\\sie} &    
\parbox[t][][t]{2cm}{\normalfont beruhige\\beruhigst\\beruhigt\\beruhigen\\beruhigt\\beruhigen} &
\parbox[t][][t]{2cm}{\normalfont beruhigte\\beruhigtest\\beruhigte\\beruhigten\\beruhigtet\\beruhigten}\\
\end{tabular}
\begin{tabular}{l}
\parbox[t][][t]{8cm}{}\\
\parbox[t][][t]{8cm}{\normalfont \small ['to calm'] }\\
\end{tabular}
}
%===beschädigen===
\card{\normalfont \Huge beschädigen}{
\begin{tabular}{lll}
\parbox[t][][t]{2.0 cm}{\normalfont \raggedleft ich\\du\\er/sie/es\\wir\\ihr\\sie} &    
\parbox[t][][t]{2cm}{\normalfont beschädige\\beschädigst\\beschädigt\\beschädigen\\beschädigt\\beschädigen} &
\parbox[t][][t]{2cm}{\normalfont beschädigte\\beschädigtest\\beschädigte\\beschädigten\\beschädigtet\\beschädigten}\\
\end{tabular}
\begin{tabular}{l}
\parbox[t][][t]{8cm}{}\\
\parbox[t][][t]{8cm}{\normalfont \small ['to damage'] }\\
\end{tabular}
}
%===beschließen===
\card{\normalfont \Huge beschließen}{
\begin{tabular}{lll}
\parbox[t][][t]{2.0 cm}{\normalfont \raggedleft ich\\du\\er/sie/es\\wir\\ihr\\sie} &    
\parbox[t][][t]{2cm}{\normalfont beschließe\\beschließt\\beschließt\\beschließen\\beschließt\\beschließen} &
\parbox[t][][t]{2cm}{\normalfont beschloss\\beschlossest\\beschloss\\beschlossen\\beschlosst / beschlosset  (obsolete)\\beschlossen}\\
\end{tabular}
\begin{tabular}{l}
\parbox[t][][t]{8cm}{}\\
\parbox[t][][t]{8cm}{\normalfont \small ['to conclude, end; to terminate to resolve, decide,', 'determine'] }\\
\end{tabular}
}
%===beschreiben===
\card{\normalfont \Huge beschreiben}{
\begin{tabular}{lll}
\parbox[t][][t]{2.0 cm}{\normalfont \raggedleft ich\\du\\er/sie/es\\wir\\ihr\\sie} &    
\parbox[t][][t]{2cm}{\normalfont beschreibe\\beschreibst\\beschreibt\\beschreiben\\beschreibt\\beschreiben} &
\parbox[t][][t]{2cm}{\normalfont beschrieb\\beschriebst\\beschrieb\\beschrieben\\beschriebt\\beschrieben}\\
\end{tabular}
\begin{tabular}{l}
\parbox[t][][t]{8cm}{}\\
\parbox[t][][t]{8cm}{\normalfont \small ['(transitive, to represent in words) to describe', '(transitive, mathematics, to give rise to a', 'geometrical structure) to describe (transitive) to', 'write on (computing, transitive) to write (record', 'data) to'] }\\
\end{tabular}
}
%===beschweren===
\card{\normalfont \Huge beschweren}{
\begin{tabular}{lll}
\parbox[t][][t]{2.0 cm}{\normalfont \raggedleft ich\\du\\er/sie/es\\wir\\ihr\\sie} &    
\parbox[t][][t]{2cm}{\normalfont beschwere\\beschwerst\\beschwert\\beschweren\\beschwert\\beschweren} &
\parbox[t][][t]{2cm}{\normalfont beschwerte\\beschwertest\\beschwerte\\beschwerten\\beschwertet\\beschwerten}\\
\end{tabular}
\begin{tabular}{l}
\parbox[t][][t]{8cm}{}\\
\parbox[t][][t]{8cm}{\normalfont \small ['(reflexive) to complain (transitive) to weight, to', 'weight down'] }\\
\end{tabular}
}
%===besichtigen===
\card{\normalfont \Huge besichtigen}{
\begin{tabular}{lll}
\parbox[t][][t]{2.0 cm}{\normalfont \raggedleft ich\\du\\er/sie/es\\wir\\ihr\\sie} &    
\parbox[t][][t]{2cm}{\normalfont besichtige\\besichtigst\\besichtigt\\besichtigen\\besichtigt\\besichtigen} &
\parbox[t][][t]{2cm}{\normalfont besichtigte\\besichtigtest\\besichtigte\\besichtigten\\besichtigtet\\besichtigten}\\
\end{tabular}
\begin{tabular}{l}
\parbox[t][][t]{8cm}{}\\
\parbox[t][][t]{8cm}{\normalfont \small ['to visit (a city, tourist site, monument etc.)', '2018, 20 Minuten, 19 December: Der neu entdeckte', 'Kamelknochen kann whrend der Basler Museumsnacht', 'vom 18. Januar 2019 im Zelt der Archologischen', 'Bodenforschung besichtigt werden. The newly', 'discovered camel bones can be visited during Basel', 'Museum Night on the 18th January in the', 'Archaeological Soil Research tent. to have a look', 'around; to view (a property)'] }\\
\end{tabular}
}
%===besitzen===
\card{\normalfont \Huge besitzen}{
\begin{tabular}{lll}
\parbox[t][][t]{2.0 cm}{\normalfont \raggedleft ich\\du\\er/sie/es\\wir\\ihr\\sie} &    
\parbox[t][][t]{2cm}{\normalfont besitze\\besitzt\\besitzt\\besitzen\\besitzt\\besitzen} &
\parbox[t][][t]{2cm}{\normalfont besaß\\besaßt\\besaß\\besaßen\\besaßt\\besaßen}\\
\end{tabular}
\begin{tabular}{l}
\parbox[t][][t]{8cm}{}\\
\parbox[t][][t]{8cm}{\normalfont \small ['to own, to possess 1599, Philipp Nicolai, Wie schn', 'leuchtet der Morgenstern Du Sohn Davids aus Jakobs', 'Stamm, mein Knig und mein Brutigam, hast mir mein', 'Herz besessen. You, the Son of David from the root', 'of Jacob, my King and my bridegroom, have', 'possessed my heart. to have 1919, Walther Kabel,', 'Irrende Seelen, Werner Dietsch Verlag, p. 93: Der', 'Fremde war entschieden noch jung, mittelgro und', 'besa einen kleinen blonden Schnurrbart [] The', 'stranger was definitly still young, medium height', 'and had a small blonde moustache []'] }\\
\end{tabular}
}
%===besorgen===
\card{\normalfont \Huge besorgen}{
\begin{tabular}{lll}
\parbox[t][][t]{2.0 cm}{\normalfont \raggedleft ich\\du\\er/sie/es\\wir\\ihr\\sie} &    
\parbox[t][][t]{2cm}{\normalfont besorge\\besorgst\\besorgt\\besorgen\\besorgt\\besorgen} &
\parbox[t][][t]{2cm}{\normalfont besorgte\\besorgtest\\besorgte\\besorgten\\besorgtet\\besorgten}\\
\end{tabular}
\begin{tabular}{l}
\parbox[t][][t]{8cm}{}\\
\parbox[t][][t]{8cm}{\normalfont \small ['(transitive) to deal, to manage, to procure Seine', 'Geschfte wurden durch einen Stellvertreter', 'besorgt.His operations have been managed by a', 'proxy. (transitive) to get, to obtain Ich gehe', 'noch mal schnell Brtchen besorgen.  I go get bread', 'rolls only. (vulgar, with expletive and dative) to', "give it to someone, to satisfycarnally Er hat's", 'ihr richtig besorgt.  He has really given it to', 'her. (transitive, from 20th century  usually', 'officialese) to be afraid of, to worry about 1788,', 'Christoph Martin Wieland,  Lgengeschichten und', 'Dialoge von Lukian von Samosata, Chapter', '16:brigens hie er uns gutes Muthes seyn und keine', 'Gefahr besorgen; wir sollten mit allem, was wir', 'nthig htten, versehen werden.Else he hight us be', 'of good mood and fear no danger; we should be', 'furnished with all that we might need.'] }\\
\end{tabular}
}
%===bestätigen===
\card{\normalfont \Huge bestätigen}{
\begin{tabular}{lll}
\parbox[t][][t]{2.0 cm}{\normalfont \raggedleft ich\\du\\er/sie/es\\wir\\ihr\\sie} &    
\parbox[t][][t]{2cm}{\normalfont bestätige\\bestätigst\\bestätigt\\bestätigen\\bestätigt\\bestätigen} &
\parbox[t][][t]{2cm}{\normalfont bestätigte\\bestätigtest\\bestätigte\\bestätigten\\bestätigtet\\bestätigten}\\
\end{tabular}
\begin{tabular}{l}
\parbox[t][][t]{8cm}{}\\
\parbox[t][][t]{8cm}{\normalfont \small ['to confirm, corroborate'] }\\
\end{tabular}
}
%===bestehen===
\card{\normalfont \Huge bestehen}{
\begin{tabular}{lll}
\parbox[t][][t]{2.0 cm}{\normalfont \raggedleft ich\\du\\er/sie/es\\wir\\ihr\\sie} &    
\parbox[t][][t]{2cm}{\normalfont bestehe\\bestehst\\besteht\\bestehen\\besteht\\bestehen} &
\parbox[t][][t]{2cm}{\normalfont bestand\\bestandest\\bestand\\bestanden\\bestandet\\bestanden}\\
\end{tabular}
\begin{tabular}{l}
\parbox[t][][t]{8cm}{}\\
\parbox[t][][t]{8cm}{\normalfont \small ['(transitive) to succeed, to pass (an exam) Er hat', 'die Prfung bestanden.He has passed the exam.', '(intransitive) to consist (aus ("of")) 1931,', 'Gebhard Mehring, Schrift und Schrifttum,', 'Silberburg-Verlag, page 21: Das rmische', 'Zahlensystem [] besteht aus 7 Buchstaben, die zur', 'Bezeichnung von Zahlenwerten verwendet werden: M D', 'C L X V I The Roman numeral system [] consists of', '7 letters, which are used for the representation', 'of numerical values: M D C L X V I (intransitive)', 'to exist Durch geschicktes Management blieb die', 'Firma auch im starken Wettbewerb bestehen.By smart', 'management the company survived in the strong', 'competition. (intransitive) to insist (auf ("on"))', 'Er besteht darauf, dass dem Druck nicht', 'nachgegeben wird.He insists on not giving in to', 'the pressure.'] }\\
\end{tabular}
}
%===bestellen===
\card{\normalfont \Huge bestellen}{
\begin{tabular}{lll}
\parbox[t][][t]{2.0 cm}{\normalfont \raggedleft ich\\du\\er/sie/es\\wir\\ihr\\sie} &    
\parbox[t][][t]{2cm}{\normalfont bestelle\\bestellst\\bestellt\\bestellen\\bestellt\\bestellen} &
\parbox[t][][t]{2cm}{\normalfont bestellte\\bestelltest\\bestellte\\bestellten\\bestelltet\\bestellten}\\
\end{tabular}
\begin{tabular}{l}
\parbox[t][][t]{8cm}{}\\
\parbox[t][][t]{8cm}{\normalfont \small ['(transitive) To order to come, to summon', '(transitive or intransitive) To demand delivery,', 'to order (e.g. food) (transitive or intransitive)', 'To reserve (e.g. a table, hotel room) (transitive)', 'To convey (a message, greetings) (transitive) To', 'cultivate, till'] }\\
\end{tabular}
}
%===bestimmen===
\card{\normalfont \Huge bestimmen}{
\begin{tabular}{lll}
\parbox[t][][t]{2.0 cm}{\normalfont \raggedleft ich\\du\\er/sie/es\\wir\\ihr\\sie} &    
\parbox[t][][t]{2cm}{\normalfont bestimme\\bestimmst\\bestimmt\\bestimmen\\bestimmt\\bestimmen} &
\parbox[t][][t]{2cm}{\normalfont bestimmte\\bestimmtest\\bestimmte\\bestimmten\\bestimmtet\\bestimmten}\\
\end{tabular}
\begin{tabular}{l}
\parbox[t][][t]{8cm}{}\\
\parbox[t][][t]{8cm}{\normalfont \small ['to determine'] }\\
\end{tabular}
}
%===besuchen===
\card{\normalfont \Huge besuchen}{
\begin{tabular}{lll}
\parbox[t][][t]{2.0 cm}{\normalfont \raggedleft ich\\du\\er/sie/es\\wir\\ihr\\sie} &    
\parbox[t][][t]{2cm}{\normalfont besuche\\besuchst\\besucht\\besuchen\\besucht\\besuchen} &
\parbox[t][][t]{2cm}{\normalfont besuchte\\besuchtest\\besuchte\\besuchten\\besuchtet\\besuchten}\\
\end{tabular}
\begin{tabular}{l}
\parbox[t][][t]{8cm}{}\\
\parbox[t][][t]{8cm}{\normalfont \small ['to visit Wir haben heute unsere Mama in der', 'Seniorenwohnanlage besucht.We visited our mom in', 'the retirement home today. to attend something Er', 'besuchte das Dsseldorfer Gymnasium.He attended the', 'high school of Dsseldorf.'] }\\
\end{tabular}
}
%===beteiligen===
\card{\normalfont \Huge beteiligen}{
\begin{tabular}{lll}
\parbox[t][][t]{2.0 cm}{\normalfont \raggedleft ich\\du\\er/sie/es\\wir\\ihr\\sie} &    
\parbox[t][][t]{2cm}{\normalfont beteilige\\beteiligst\\beteiligt\\beteiligen\\beteiligt\\beteiligen} &
\parbox[t][][t]{2cm}{\normalfont beteiligte\\beteiligtest\\beteiligte\\beteiligten\\beteiligtet\\beteiligten}\\
\end{tabular}
\begin{tabular}{l}
\parbox[t][][t]{8cm}{}\\
\parbox[t][][t]{8cm}{\normalfont \small ['(transitive) to let someone take part in something', '(reflexive) participate'] }\\
\end{tabular}
}
%===beten===
\card{\normalfont \Huge beten}{
\begin{tabular}{lll}
\parbox[t][][t]{2.0 cm}{\normalfont \raggedleft ich\\du\\er/sie/es\\wir\\ihr\\sie} &    
\parbox[t][][t]{2cm}{\normalfont bete\\betest\\betet\\beten\\betet\\beten} &
\parbox[t][][t]{2cm}{\normalfont betete\\betetest\\betete\\beteten\\betetet\\beteten}\\
\end{tabular}
\begin{tabular}{l}
\parbox[t][][t]{8cm}{}\\
\parbox[t][][t]{8cm}{\normalfont \small ['to pray'] }\\
\end{tabular}
}
%===betonen===
\card{\normalfont \Huge betonen}{
\begin{tabular}{lll}
\parbox[t][][t]{2.0 cm}{\normalfont \raggedleft ich\\du\\er/sie/es\\wir\\ihr\\sie} &    
\parbox[t][][t]{2cm}{\normalfont betone\\betonst\\betont\\betonen\\betont\\betonen} &
\parbox[t][][t]{2cm}{\normalfont betonte\\betontest\\betonte\\betonten\\betontet\\betonten}\\
\end{tabular}
\begin{tabular}{l}
\parbox[t][][t]{8cm}{}\\
\parbox[t][][t]{8cm}{\normalfont \small ['to emphasize to stress'] }\\
\end{tabular}
}
%===betragen===
\card{\normalfont \Huge betragen}{
\begin{tabular}{lll}
\parbox[t][][t]{2.0 cm}{\normalfont \raggedleft ich\\du\\er/sie/es\\wir\\ihr\\sie} &    
\parbox[t][][t]{2cm}{\normalfont betrage\\beträgst\\beträgt\\betragen\\betragt\\betragen} &
\parbox[t][][t]{2cm}{\normalfont betrug\\betrugst\\betrug\\betrugen\\betrugt\\betrugen}\\
\end{tabular}
\begin{tabular}{l}
\parbox[t][][t]{8cm}{}\\
\parbox[t][][t]{8cm}{\normalfont \small ['(transitive) to amount to, to be 2010, Der', 'Spiegel, issue 2/2010, page 125: Kohlendioxid trgt', 'durch seine Menge mehr als jedes andere', 'Treibhausgas zum Klimawandel bei, doch seine', 'Konzentration in der Luft betrgt weniger als 0,04', 'Prozent.Carbon dioxide contributes by its amount', 'more than any other greenhouse gas to climate', 'change, but its concentration in the air is less', 'than 0.04 percent. Die Atommasse wird auf eine', 'Skala bezogen, auf der die Masse des Atoms'] }\\
\end{tabular}
}
%===betrügen===
\card{\normalfont \Huge betrügen}{
\begin{tabular}{lll}
\parbox[t][][t]{2.0 cm}{\normalfont \raggedleft ich\\du\\er/sie/es\\wir\\ihr\\sie} &    
\parbox[t][][t]{2cm}{\normalfont betrüge\\betrügst\\betrügt\\betrügen\\betrügt\\betrügen} &
\parbox[t][][t]{2cm}{\normalfont betrog\\betrogst\\betrog\\betrogen\\betrogt\\betrogen}\\
\end{tabular}
\begin{tabular}{l}
\parbox[t][][t]{8cm}{}\\
\parbox[t][][t]{8cm}{\normalfont \small ['(transitive) to cheat; to swindle Meine Frau', 'betrgt mich aus verschiedenen Grnden... My wife', 'cheates on me due to different reasons...', '(transitive, law) to defraud (transitive,', 'marriage) to deceive'] }\\
\end{tabular}
}
%===bewegen===
\card{\normalfont \Huge bewegen}{
\begin{tabular}{lll}
\parbox[t][][t]{2.0 cm}{\normalfont \raggedleft ich\\du\\er/sie/es\\wir\\ihr\\sie} &    
\parbox[t][][t]{2cm}{\normalfont bewege\\bewegst\\bewegt\\bewegen\\bewegt\\bewegen} &
\parbox[t][][t]{2cm}{\normalfont bewog\\bewogst\\bewog\\bewogen\\bewogt\\bewogen}\\
\end{tabular}
\begin{tabular}{l}
\parbox[t][][t]{8cm}{}\\
\parbox[t][][t]{8cm}{\normalfont \small ['(transitive) to persuade; to prompt (someone to do', 'something); to make (someone do something); to', 'induce; to get (someone to do something)'] }\\
\end{tabular}
}
%===beweisen===
\card{\normalfont \Huge beweisen}{
\begin{tabular}{lll}
\parbox[t][][t]{2.0 cm}{\normalfont \raggedleft ich\\du\\er/sie/es\\wir\\ihr\\sie} &    
\parbox[t][][t]{2cm}{\normalfont beweise\\beweist\\beweist\\beweisen\\beweist\\beweisen} &
\parbox[t][][t]{2cm}{\normalfont bewies\\bewiest\\bewies\\bewiesen\\bewiest\\bewiesen}\\
\end{tabular}
\begin{tabular}{l}
\parbox[t][][t]{8cm}{}\\
\parbox[t][][t]{8cm}{\normalfont \small ['to prove'] }\\
\end{tabular}
}
%===bezahlen===
\card{\normalfont \Huge bezahlen}{
\begin{tabular}{lll}
\parbox[t][][t]{2.0 cm}{\normalfont \raggedleft ich\\du\\er/sie/es\\wir\\ihr\\sie} &    
\parbox[t][][t]{2cm}{\normalfont bezahle\\bezahlst\\bezahlt\\bezahlen\\bezahlt\\bezahlen} &
\parbox[t][][t]{2cm}{\normalfont bezahlte\\bezahltest\\bezahlte\\bezahlten\\bezahltet\\bezahlten}\\
\end{tabular}
\begin{tabular}{l}
\parbox[t][][t]{8cm}{}\\
\parbox[t][][t]{8cm}{\normalfont \small ['to pay Dafr wirst du bezahlen!You will pay for it!', 'Ich mchte bitte bezahlen.I want to pay please.'] }\\
\end{tabular}
}
%===bezeichnen===
\card{\normalfont \Huge bezeichnen}{
\begin{tabular}{lll}
\parbox[t][][t]{2.0 cm}{\normalfont \raggedleft ich\\du\\er/sie/es\\wir\\ihr\\sie} &    
\parbox[t][][t]{2cm}{\normalfont bezeichne\\bezeichnest\\bezeichnet\\bezeichnen\\bezeichnet\\bezeichnen} &
\parbox[t][][t]{2cm}{\normalfont bezeichnete\\bezeichnetest\\bezeichnete\\bezeichneten\\bezeichnetet\\bezeichneten}\\
\end{tabular}
\begin{tabular}{l}
\parbox[t][][t]{8cm}{}\\
\parbox[t][][t]{8cm}{\normalfont \small ['to name (give a name to), call, designate to', 'identify, indicate (reflexive) to describe', 'oneself, to identify (als as) Bezeichnen Sie sich', 'als Hilfreiche?Do you consider yourself a helpful', 'woman?'] }\\
\end{tabular}
}
%===beziehen===
\card{\normalfont \Huge beziehen}{
\begin{tabular}{lll}
\parbox[t][][t]{2.0 cm}{\normalfont \raggedleft ich\\du\\er/sie/es\\wir\\ihr\\sie} &    
\parbox[t][][t]{2cm}{\normalfont beziehe\\beziehst\\bezieht\\beziehen\\bezieht\\beziehen} &
\parbox[t][][t]{2cm}{\normalfont bezog\\bezogst\\bezog\\bezogen\\bezogt\\bezogen}\\
\end{tabular}
\begin{tabular}{l}
\parbox[t][][t]{8cm}{}\\
\parbox[t][][t]{8cm}{\normalfont \small ['to move in to cover, to upholster (reflexive) to', 'refer'] }\\
\end{tabular}
}
%===biegen===
\card{\normalfont \Huge biegen}{
\begin{tabular}{lll}
\parbox[t][][t]{2.0 cm}{\normalfont \raggedleft ich\\du\\er/sie/es\\wir\\ihr\\sie} &    
\parbox[t][][t]{2cm}{\normalfont biege\\biegst\\biegt\\biegen\\biegt\\biegen} &
\parbox[t][][t]{2cm}{\normalfont bog\\bogst\\bog\\bogen\\bogt\\bogen}\\
\end{tabular}
\begin{tabular}{l}
\parbox[t][][t]{8cm}{}\\
\parbox[t][][t]{8cm}{\normalfont \small ['(transitive, auxiliary "haben") to bend something', '(to form something into a curve) Diese Stange kann', 'man leicht biegen.You can easily bend this pole.', '(reflexive, auxiliary "haben") to bend; to be bent', '(to form oneself or be formed into a curve) Die', 'Bume biegen sich im Wind.The trees are bending in', 'the wind. (intransitive, auxiliary "sein") to', 'turn; to round a corner; to drive into a street;', 'always requires some adverbial of location with', 'it; otherwise use abbiegen Er ist um die Ecke', "gebogen.He's turned around the corner. Er biegt", 'auf die Hauptstrae.He turns into the main street.'] }\\
\end{tabular}
}
%===bieten===
\card{\normalfont \Huge bieten}{
\begin{tabular}{lll}
\parbox[t][][t]{2.0 cm}{\normalfont \raggedleft ich\\du\\er/sie/es\\wir\\ihr\\sie} &    
\parbox[t][][t]{2cm}{\normalfont biete\\bietest\\bietet\\bieten\\bietet\\bieten} &
\parbox[t][][t]{2cm}{\normalfont bot\\botest\\bot\\boten\\botet\\boten}\\
\end{tabular}
\begin{tabular}{l}
\parbox[t][][t]{8cm}{}\\
\parbox[t][][t]{8cm}{\normalfont \small ['(transitive) to offer; to present (transitive or', 'intransitive) to bid; to offer to pay a certain', 'price (reflexive, of an opportunity) to arise; to', 'occur'] }\\
\end{tabular}
}
%===binden===
\card{\normalfont \Huge binden}{
\begin{tabular}{lll}
\parbox[t][][t]{2.0 cm}{\normalfont \raggedleft ich\\du\\er/sie/es\\wir\\ihr\\sie} &    
\parbox[t][][t]{2cm}{\normalfont binde\\bindest\\bindet\\binden\\bindet\\binden} &
\parbox[t][][t]{2cm}{\normalfont band\\bandest\\band\\banden\\bandet\\banden}\\
\end{tabular}
\begin{tabular}{l}
\parbox[t][][t]{8cm}{}\\
\parbox[t][][t]{8cm}{\normalfont \small ['(transitive) to tie up; to fasten; to bind', 'together ein Buch binden  "to bind a book"', '(transitive) to knot (intransitive) to congeal; to', 'thicken; to set; to bond (reflexive) to become', 'involved; to commit (oneself)'] }\\
\end{tabular}
}
%===bitten===
\card{\normalfont \Huge bitten}{
\begin{tabular}{lll}
\parbox[t][][t]{2.0 cm}{\normalfont \raggedleft ich\\du\\er/sie/es\\wir\\ihr\\sie} &    
\parbox[t][][t]{2cm}{\normalfont bitte\\bittest\\bittet\\bitten\\bittet\\bitten} &
\parbox[t][][t]{2cm}{\normalfont bat\\batest\\bat\\baten\\batet\\baten}\\
\end{tabular}
\begin{tabular}{l}
\parbox[t][][t]{8cm}{}\\
\parbox[t][][t]{8cm}{\normalfont \small ['(transitive or intransitive) to ask, to beg, to', 'plead, to request Wir lieben euchEntschuldigtWir', "bitten um VerzeihungVielen DankWe love you, we're", 'sorry, please forgive us, much thanks.'] }\\
\end{tabular}
}
%===blasen===
\card{\normalfont \Huge blasen}{
\begin{tabular}{lll}
\parbox[t][][t]{2.0 cm}{\normalfont \raggedleft ich\\du\\er/sie/es\\wir\\ihr\\sie} &    
\parbox[t][][t]{2cm}{\normalfont blase\\bläst\\bläst\\blasen\\blast\\blasen} &
\parbox[t][][t]{2cm}{\normalfont blies\\bliest\\blies\\bliesen\\bliest\\bliesen}\\
\end{tabular}
\begin{tabular}{l}
\parbox[t][][t]{8cm}{}\\
\parbox[t][][t]{8cm}{\normalfont \small ['(transitive or intransitive) to blow (transitive,', 'music) to play (a wind instrument) (vulgar) to', 'fellate, to perform oral sex'] }\\
\end{tabular}
}
%===bleiben===
\card{\normalfont \Huge bleiben}{
\begin{tabular}{lll}
\parbox[t][][t]{2.0 cm}{\normalfont \raggedleft ich\\du\\er/sie/es\\wir\\ihr\\sie} &    
\parbox[t][][t]{2cm}{\normalfont bleibe\\bleibst\\bleibt\\bleiben\\bleibt\\bleiben} &
\parbox[t][][t]{2cm}{\normalfont blieb\\bliebst\\blieb\\blieben\\bliebt\\blieben}\\
\end{tabular}
\begin{tabular}{l}
\parbox[t][][t]{8cm}{}\\
\parbox[t][][t]{8cm}{\normalfont \small ['(intransitive) to remain (to continue to be) 1929,', 'Kurt Tucholsky, Das Lcheln der Mona Lisa', '(Sammelband), Rowohlt Verlag, page 239: Um populr', 'zu werden, kann man seine eigene Meinung behalten.', 'Um populr zu bleiben, weniger. In order to become', "popular, one can keep one's own opinion. In order", 'to stay popular, less so. Er blieb sein ganzes', 'Leben ein glhender Anhnger der Monarchie. He', 'remained a devoted supporter of the monarchy for', 'all his life. (intransitive, + infinitive) to keep', '(on); to continue (see usage notes below) Ich', "bleibe noch ein bisschen liegen. I'll keep lying", 'here for a bit. (intransitive) to stay; to remain', 'in a place Du kannst ja schon fahren, aber ich', "bleibe noch. Feel free to leave, but I'm staying", 'some more. (intransitive) to be; to be stuck;', 'implying tardiness Wo bleibst du? Wir sind schon', 'seit ber einer Stunde da. Where are you? We', 'arrived more than an hour ago. (intransitive, +', 'dative) to be left for someone Was bleibt ihm', 'jetzt noch, wo seine Frau gestorben ist? What does', 'he have left now his wife has died? (literally:', 'What is left for him...) (intransitive, + bei) to', "stick with; to stay with Ich hab's ihm erklrt,", "aber er bleibt bei seiner Meinung. I've explained", 'it to him, but he sticks with his opinion.'] }\\
\end{tabular}
}
%===bleichen===
\card{\normalfont \Huge bleichen}{
\begin{tabular}{lll}
\parbox[t][][t]{2.0 cm}{\normalfont \raggedleft ich\\du\\er/sie/es\\wir\\ihr\\sie} &    
\parbox[t][][t]{2cm}{\normalfont bleiche\\bleichst\\bleicht\\bleichen\\bleicht\\bleichen} &
\parbox[t][][t]{2cm}{\normalfont bleichte\\bleichtest\\bleichte\\bleichten\\bleichtet\\bleichten}\\
\end{tabular}
\begin{tabular}{l}
\parbox[t][][t]{8cm}{}\\
\parbox[t][][t]{8cm}{\normalfont \small ['(transitive, auxiliary haben) to bleach', '(intransitive, rather  rare, auxiliary sein) to', 'fade, to lose colour'] }\\
\end{tabular}
}
%===blühen===
\card{\normalfont \Huge blühen}{
\begin{tabular}{lll}
\parbox[t][][t]{2.0 cm}{\normalfont \raggedleft ich\\du\\er/sie/es\\wir\\ihr\\sie} &    
\parbox[t][][t]{2cm}{\normalfont blühe\\blühst\\blüht\\blühen\\blüht\\blühen} &
\parbox[t][][t]{2cm}{\normalfont blühte\\blühtest\\blühte\\blühten\\blühtet\\blühten}\\
\end{tabular}
\begin{tabular}{l}
\parbox[t][][t]{8cm}{}\\
\parbox[t][][t]{8cm}{\normalfont \small ['to blossom or bloom to thrive or flourish'] }\\
\end{tabular}
}
%===braten===
\card{\normalfont \Huge braten}{
\begin{tabular}{lll}
\parbox[t][][t]{2.0 cm}{\normalfont \raggedleft ich\\du\\er/sie/es\\wir\\ihr\\sie} &    
\parbox[t][][t]{2cm}{\normalfont brate\\brätst\\brät\\braten\\bratet\\braten} &
\parbox[t][][t]{2cm}{\normalfont briet\\brietest\\briet\\brieten\\brietet\\brieten}\\
\end{tabular}
\begin{tabular}{l}
\parbox[t][][t]{8cm}{}\\
\parbox[t][][t]{8cm}{\normalfont \small ['(transitive or intransitive) to pan-fry', '(transitive or intransitive) to roast; to grill;', 'to broil'] }\\
\end{tabular}
}
%===brauchen===
\card{\normalfont \Huge brauchen}{
\begin{tabular}{lll}
\parbox[t][][t]{2.0 cm}{\normalfont \raggedleft ich\\du\\er/sie/es\\wir\\ihr\\sie} &    
\parbox[t][][t]{2cm}{\normalfont brauche\\brauchst\\braucht\\brauchen\\braucht\\brauchen} &
\parbox[t][][t]{2cm}{\normalfont brauchte\\brauchtest\\brauchte\\brauchten\\brauchtet\\brauchten}\\
\end{tabular}
\begin{tabular}{l}
\parbox[t][][t]{8cm}{}\\
\parbox[t][][t]{8cm}{\normalfont \small ['to need, to be in need of Ich brauche deine', 'Hilfe.I need your help. to need to, to have to (in', 'negation or with the adverb nur ("just, only")) Du', "brauchst nicht auf mich (zu) warten.You don't need", 'to wait for me. Sie braucht mich nur', 'an(zu)rufen.She just needs to call me.'] }\\
\end{tabular}
}
%===brechen===
\card{\normalfont \Huge brechen}{
\begin{tabular}{lll}
\parbox[t][][t]{2.0 cm}{\normalfont \raggedleft ich\\du\\er/sie/es\\wir\\ihr\\sie} &    
\parbox[t][][t]{2cm}{\normalfont breche\\brichst\\bricht\\brechen\\brecht\\brechen} &
\parbox[t][][t]{2cm}{\normalfont brach\\brachst\\brach\\brachen\\bracht\\brachen}\\
\end{tabular}
\begin{tabular}{l}
\parbox[t][][t]{8cm}{}\\
\parbox[t][][t]{8cm}{\normalfont \small ['(transitive, auxiliary: "haben") to break', '(transitive, physics, auxiliary: "haben") to', 'refract (transitive or intransitive, colloquial,', 'auxiliary: "haben") to vomit (transitive,', 'auxiliary: "haben") to fold (intransitive,', 'auxiliary: "sein") to become broken; to break; to', 'fracture'] }\\
\end{tabular}
}
%===brennen===
\card{\normalfont \Huge brennen}{
\begin{tabular}{lll}
\parbox[t][][t]{2.0 cm}{\normalfont \raggedleft ich\\du\\er/sie/es\\wir\\ihr\\sie} &    
\parbox[t][][t]{2cm}{\normalfont brenne\\brennst\\brennt\\brennen\\brennt\\brennen} &
\parbox[t][][t]{2cm}{\normalfont brannte\\branntest\\brannte\\brannten\\branntet\\brannten}\\
\end{tabular}
\begin{tabular}{l}
\parbox[t][][t]{8cm}{}\\
\parbox[t][][t]{8cm}{\normalfont \small ['(intransitive) to burn; to light on fire Ich', 'beobachtete wie das Haus brannte.  I watched the', 'house burn. Es brennt!  There is a fire!', '(intransitive) to burn; to be on fire Mein Haus', 'brennt!  My house is on fire! Trockenes Holz', 'brennt am besten.  Dry wood burns best.', '(intransitive) to have a strong affection for; to', 'be affectionate Ich brenne darauf sie zu besuchen!', 'I would really like to visit her. (intransitive)', 'to be lit, to be on (of a light or lamp) Das Licht', 'in der Kche brannte noch immer.  The light in the', 'kitchen was still on. (intransitive) to irritate;', 'to induce pain or another pianful sensation; to', 'bite; to sting Die Zwiebeln brennen in meinen', 'Augen!  The onions sting in my eyes! Dieser Senf', 'brennt wie Teufel auf meiner Zunge!  This mustard', 'bites my tongue like hell! (intransitive) to', 'smart; to sting (a sore or wound) (transitive) to', 'fire; to bake; to kiln (tiles and pottery)', 'Dachziegel werden im Brennofen gebrannt.  Tiles', 'are baked in a kiln. Nach drei Tagen kann die Vase', 'gebrannt werden.  The vase can be fired after', 'three days. Keramikwaren mssen gebrannt werden', 'bevor sie genutzt werden knnen.  Pottery needs to', 'be fired before one can use it. (transitive) to', 'distil (alcoholic beverages such as schnapps)', '(transitive, computing) to burn; to archive data', 'on a storage medium. (such as CDs, DVDs, etc.)', "Kannst du mir 'Das weie Album' von den Beatles", "brennen?  Can you burn 'The White Album' from The", 'Beatles for me? (intransitive, figuratively) to', 'emit heat Die Sonne brannte auf sie herab.  The', 'sun was shining upon them with great heat. to', 'roast to bream (clean a ship etc. by fire and', 'scraping)'] }\\
\end{tabular}
}
%===bringen===
\card{\normalfont \Huge bringen}{
\begin{tabular}{lll}
\parbox[t][][t]{2.0 cm}{\normalfont \raggedleft ich\\du\\er/sie/es\\wir\\ihr\\sie} &    
\parbox[t][][t]{2cm}{\normalfont bringe\\bringst\\bringt\\bringen\\bringt\\bringen} &
\parbox[t][][t]{2cm}{\normalfont brachte\\brachtest\\brachte\\brachten\\brachtet\\brachten}\\
\end{tabular}
\begin{tabular}{l}
\parbox[t][][t]{8cm}{}\\
\parbox[t][][t]{8cm}{\normalfont \small ['(transitive) to bring; to fetch (transitive) to', 'take; to convey (transitive) to lead; to cause; to', 'bear (transitive, with "an sich") to acquire; to', 'take possession of'] }\\
\end{tabular}
}
%===buchen===
\card{\normalfont \Huge buchen}{
\begin{tabular}{lll}
\parbox[t][][t]{2.0 cm}{\normalfont \raggedleft ich\\du\\er/sie/es\\wir\\ihr\\sie} &    
\parbox[t][][t]{2cm}{\normalfont buche\\buchst\\bucht\\buchen\\bucht\\buchen} &
\parbox[t][][t]{2cm}{\normalfont buchte\\buchtest\\buchte\\buchten\\buchtet\\buchten}\\
\end{tabular}
\begin{tabular}{l}
\parbox[t][][t]{8cm}{}\\
\parbox[t][][t]{8cm}{\normalfont \small ['to book'] }\\
\end{tabular}
}
%===danken===
\card{\normalfont \Huge danken}{
\begin{tabular}{lll}
\parbox[t][][t]{2.0 cm}{\normalfont \raggedleft ich\\du\\er/sie/es\\wir\\ihr\\sie} &    
\parbox[t][][t]{2cm}{\normalfont danke\\dankst\\dankt\\danken\\dankt\\danken} &
\parbox[t][][t]{2cm}{\normalfont dankte\\danktest\\dankte\\dankten\\danktet\\dankten}\\
\end{tabular}
\begin{tabular}{l}
\parbox[t][][t]{8cm}{}\\
\parbox[t][][t]{8cm}{\normalfont \small ['(intransitive, with dative) to thank Peter hat', 'Michaela mit einem groen Blumenstrau gedankt.Peter', 'thanked Michaela with a big bouquet.'] }\\
\end{tabular}
}
%===dauern===
\card{\normalfont \Huge dauern}{
\begin{tabular}{lll}
\parbox[t][][t]{2.0 cm}{\normalfont \raggedleft ich\\du\\er/sie/es\\wir\\ihr\\sie} &    
\parbox[t][][t]{2cm}{\normalfont daureich dauereich dauer\\dauerst\\dauert\\dauern\\dauert\\dauern} &
\parbox[t][][t]{2cm}{\normalfont dauerte\\dauertest\\dauerte\\dauerten\\dauertet\\dauerten}\\
\end{tabular}
\begin{tabular}{l}
\parbox[t][][t]{8cm}{}\\
\parbox[t][][t]{8cm}{\normalfont \small ['to last, to continue over time to persist to take', '(time)'] }\\
\end{tabular}
}
%===denken===
\card{\normalfont \Huge denken}{
\begin{tabular}{lll}
\parbox[t][][t]{2.0 cm}{\normalfont \raggedleft ich\\du\\er/sie/es\\wir\\ihr\\sie} &    
\parbox[t][][t]{2cm}{\normalfont denke\\denkst\\denkt\\denken\\denkt\\denken} &
\parbox[t][][t]{2cm}{\normalfont dachte\\dachtest\\dachte\\dachten\\dachtet\\dachten}\\
\end{tabular}
\begin{tabular}{l}
\parbox[t][][t]{8cm}{}\\
\parbox[t][][t]{8cm}{\normalfont \small ['(intransitive, rarely transitive) to think Ich', 'denke, also bin ich.I think, therefore I am. Er', 'denkt gewichtige Dinge.He is thinking weighty', 'thoughts. (intransitive, with an + accusative) to', 'think about Ich denke an frher.I am thinking about', 'the past. (intransitive, with an + accusative) not', 'to forget; to remember (to bring along, etc.) Denk', "an den Schlssel!Don't forget the key. (transitive,", 'with reflexive dative) to imagine Das kann ich mir', 'denken.I can imagine that. Ich denke ihn mir als', 'brtigen Einsiedler.I imagine him as a bearded', 'hermit.'] }\\
\end{tabular}
}
%===dienen===
\card{\normalfont \Huge dienen}{
\begin{tabular}{lll}
\parbox[t][][t]{2.0 cm}{\normalfont \raggedleft ich\\du\\er/sie/es\\wir\\ihr\\sie} &    
\parbox[t][][t]{2cm}{\normalfont diene\\dienst\\dient\\dienen\\dient\\dienen} &
\parbox[t][][t]{2cm}{\normalfont diente\\dientest\\diente\\dienten\\dientet\\dienten}\\
\end{tabular}
\begin{tabular}{l}
\parbox[t][][t]{8cm}{}\\
\parbox[t][][t]{8cm}{\normalfont \small ['to serve [+dative]'] }\\
\end{tabular}
}
%===diskutieren===
\card{\normalfont \Huge diskutieren}{
\begin{tabular}{lll}
\parbox[t][][t]{2.0 cm}{\normalfont \raggedleft ich\\du\\er/sie/es\\wir\\ihr\\sie} &    
\parbox[t][][t]{2cm}{\normalfont diskutiere\\diskutierst\\diskutiert\\diskutieren\\diskutiert\\diskutieren} &
\parbox[t][][t]{2cm}{\normalfont diskutierte\\diskutiertest\\diskutierte\\diskutierten\\diskutiertet\\diskutierten}\\
\end{tabular}
\begin{tabular}{l}
\parbox[t][][t]{8cm}{}\\
\parbox[t][][t]{8cm}{\normalfont \small ['to discuss ber dieses Thema mssen wir diskutieren.', 'We need to discuss this issue.'] }\\
\end{tabular}
}
%===dreschen===
\card{\normalfont \Huge dreschen}{
\begin{tabular}{lll}
\parbox[t][][t]{2.0 cm}{\normalfont \raggedleft ich\\du\\er/sie/es\\wir\\ihr\\sie} &    
\parbox[t][][t]{2cm}{\normalfont dresche\\drischst\\drischt\\dreschen\\drescht\\dreschen} &
\parbox[t][][t]{2cm}{\normalfont drosch\\droschst\\drosch\\droschen\\droscht\\droschen}\\
\end{tabular}
\begin{tabular}{l}
\parbox[t][][t]{8cm}{}\\
\parbox[t][][t]{8cm}{\normalfont \small ['(transitive or intransitive) to thresh; to thrash'] }\\
\end{tabular}
}
%===dringen===
\card{\normalfont \Huge dringen}{
\begin{tabular}{lll}
\parbox[t][][t]{2.0 cm}{\normalfont \raggedleft ich\\du\\er/sie/es\\wir\\ihr\\sie} &    
\parbox[t][][t]{2cm}{\normalfont dringe\\dringst\\dringt\\dringen\\dringt\\dringen} &
\parbox[t][][t]{2cm}{\normalfont drang\\drangst\\drang\\drangen\\drangt\\drangen}\\
\end{tabular}
\begin{tabular}{l}
\parbox[t][][t]{8cm}{}\\
\parbox[t][][t]{8cm}{\normalfont \small ['(intransitive, auxiliary: "haben") to insist; to', 'press auf etwas dringen  "to press for something"', 'or "to insist on something" (intransitive,', 'auxiliary: "sein") to ooze; to seep in etwas', 'dringen  "to seep into something" (intransitive,', 'auxiliary: "sein") to force one\'s way durch etwas', 'dringen  "to penetrate something" (literally, "to', 'force one\'s way into something")'] }\\
\end{tabular}
}
%===drucken===
\card{\normalfont \Huge drucken}{
\begin{tabular}{lll}
\parbox[t][][t]{2.0 cm}{\normalfont \raggedleft ich\\du\\er/sie/es\\wir\\ihr\\sie} &    
\parbox[t][][t]{2cm}{\normalfont drucke\\druckst\\druckt\\drucken\\druckt\\drucken} &
\parbox[t][][t]{2cm}{\normalfont druckte\\drucktest\\druckte\\druckten\\drucktet\\druckten}\\
\end{tabular}
\begin{tabular}{l}
\parbox[t][][t]{8cm}{}\\
\parbox[t][][t]{8cm}{\normalfont \small ['to print'] }\\
\end{tabular}
}
%===drücken===
\card{\normalfont \Huge drücken}{
\begin{tabular}{lll}
\parbox[t][][t]{2.0 cm}{\normalfont \raggedleft ich\\du\\er/sie/es\\wir\\ihr\\sie} &    
\parbox[t][][t]{2cm}{\normalfont drücke\\drückst\\drückt\\drücken\\drückt\\drücken} &
\parbox[t][][t]{2cm}{\normalfont drückte\\drücktest\\drückte\\drückten\\drücktet\\drückten}\\
\end{tabular}
\begin{tabular}{l}
\parbox[t][][t]{8cm}{}\\
\parbox[t][][t]{8cm}{\normalfont \small ['(transitive) to press; to push (e.g., a door', 'handle) (transitive) to hug (somebody)', '(intransitive, reflexive) to shirk'] }\\
\end{tabular}
}
%===dürfen===
\card{\normalfont \Huge dürfen}{
\begin{tabular}{lll}
\parbox[t][][t]{2.0 cm}{\normalfont \raggedleft ich\\du\\er/sie/es\\wir\\ihr\\sie} &    
\parbox[t][][t]{2cm}{\normalfont darf\\darfst\\darf\\dürfen\\dürft\\dürfen} &
\parbox[t][][t]{2cm}{\normalfont durfte\\durftest\\durfte\\durften\\durftet\\durften}\\
\end{tabular}
\begin{tabular}{l}
\parbox[t][][t]{8cm}{}\\
\parbox[t][][t]{8cm}{\normalfont \small ['(auxiliary, infinitive replaces past participle)', 'To be allowed (to do something); to be permitted', '(to do something); may. 1930,  Bertolt Brecht', '(lyrics), Kurt Weil (music),  "act 2, scene 2", in', 'Aufstieg und Fall der Stadt Mahagonny:Erstens,', 'vergesst nicht, kommt das Fressen, / zweitens', 'kommt der Liebesakt, / drittens das Boxen nicht', 'vergessen, / viertens Saufen, laut Kontrakt. / Vor', 'allem aber achtet scharf, dass man hier alles', 'drfen darf.(please add an English translation of', 'this quote) Darf ich gehen?  May I go? Ich habe', 'gehen drfen.  I was allowed to go. (intransitive', 'or transitive, past participle as above) To be', 'allowed or permitted to do something implied or', 'previously stated; may. Ja, du darfst.  Yes, you', 'may. Ich habe es gedurft.  I was allowed [to do]', 'it. (subjunctive present, modal auxiliary verb)', 'Expresses that something is estimated or probable.', '1934, Walther Kabel (as Max Schraut), Der Bluffer,', 'Verlag moderner Lektre, page 54: Die Wohnungen', 'drften ein wenig hellhrig sein. The walls of the', 'apartments are probably a bit thin. (colloquial)', 'to must, to have to Und ich darf dann wieder', 'hinter euch aufrumen.And I can clean up after you', 'once again then.'] }\\
\end{tabular}
}
%===ehren===
\card{\normalfont \Huge ehren}{
\begin{tabular}{lll}
\parbox[t][][t]{2.0 cm}{\normalfont \raggedleft ich\\du\\er/sie/es\\wir\\ihr\\sie} &    
\parbox[t][][t]{2cm}{\normalfont ehre\\ehrst\\ehrt\\ehren\\ehrt\\ehren} &
\parbox[t][][t]{2cm}{\normalfont ehrte\\ehrtest\\ehrte\\ehrten\\ehrtet\\ehrten}\\
\end{tabular}
\begin{tabular}{l}
\parbox[t][][t]{8cm}{}\\
\parbox[t][][t]{8cm}{\normalfont \small ['to honor'] }\\
\end{tabular}
}
%===einfallen===
\card{\normalfont \Huge einfallen}{
\begin{tabular}{lll}
\parbox[t][][t]{2.0 cm}{\normalfont \raggedleft ich\\du\\er/sie/es\\wir\\ihr\\sie} &    
\parbox[t][][t]{2cm}{\normalfont falle ein\\fällst ein\\fällt ein\\fallen ein\\fallt ein\\fallen ein} &
\parbox[t][][t]{2cm}{\normalfont fiel ein\\fielst ein\\fiel ein\\fielen ein\\fielt ein\\fielen ein}\\
\end{tabular}
\begin{tabular}{l}
\parbox[t][][t]{8cm}{}\\
\parbox[t][][t]{8cm}{\normalfont \small ['(with dative) to occur (to somebody), to come (to', 'somebody); to come to mind was fllt dir ein? - how', 'dare you? (with dative) to remember Es ist mir', 'wieder eingefallen - I remember again ("It has', 'again fallen into my mind") to invade die Soldaten', 'fielen in das Land ein - the soldiers invaded the', 'land'] }\\
\end{tabular}
}
%===einkaufen===
\card{\normalfont \Huge einkaufen}{
\begin{tabular}{lll}
\parbox[t][][t]{2.0 cm}{\normalfont \raggedleft ich\\du\\er/sie/es\\wir\\ihr\\sie} &    
\parbox[t][][t]{2cm}{\normalfont kaufe ein\\kaufst ein\\kauft ein\\kaufen ein\\kauft ein\\kaufen ein} &
\parbox[t][][t]{2cm}{\normalfont kaufte ein\\kauftest ein\\kaufte ein\\kauften ein\\kauftet ein\\kauften ein}\\
\end{tabular}
\begin{tabular}{l}
\parbox[t][][t]{8cm}{}\\
\parbox[t][][t]{8cm}{\normalfont \small ['to shop'] }\\
\end{tabular}
}
%===einladen===
\card{\normalfont \Huge einladen}{
\begin{tabular}{lll}
\parbox[t][][t]{2.0 cm}{\normalfont \raggedleft ich\\du\\er/sie/es\\wir\\ihr\\sie} &    
\parbox[t][][t]{2cm}{\normalfont lade ein\\lädst ein\\lädt ein\\laden ein\\ladet ein\\laden ein} &
\parbox[t][][t]{2cm}{\normalfont lud ein\\ludest ein/ ludst ein\\lud ein\\luden ein\\ludet ein\\luden ein}\\
\end{tabular}
\begin{tabular}{l}
\parbox[t][][t]{8cm}{}\\
\parbox[t][][t]{8cm}{\normalfont \small ['to invite 1801,  Wilhelm von Humboldt,  Werke in', 'fnf Bnden. Herausgegeben von Andreas Flitner und', 'Klaus Giel, volume II, Darmstadt: WBG, published', '2010, S.547: Der Anblick dieser krftigen und', 'frohgesinnten Menschen lud mich ein, sie in ihren', 'Wohnungen aufzusuchen, und fast jeden Nachmittag', 'machte ich einen Spaziergang nach einem der nah', 'gelegenen Ackerhfe.(please add an English', 'translation of this quote) Antonym: ausladen to', 'treat (entertain with food or drink, especially at', "one's own expense)"] }\\
\end{tabular}
}
%===einrichten===
\card{\normalfont \Huge einrichten}{
\begin{tabular}{lll}
\parbox[t][][t]{2.0 cm}{\normalfont \raggedleft ich\\du\\er/sie/es\\wir\\ihr\\sie} &    
\parbox[t][][t]{2cm}{\normalfont richte ein\\richtest ein\\richtet ein\\richten ein\\richtet ein\\richten ein} &
\parbox[t][][t]{2cm}{\normalfont richtete ein\\richtetest ein\\richtete ein\\richteten ein\\richtetet ein\\richteten ein}\\
\end{tabular}
\begin{tabular}{l}
\parbox[t][][t]{8cm}{}\\
\parbox[t][][t]{8cm}{\normalfont \small ['to furnish 1918, Elisabeth von Heyking, Die', 'Orgelpfeifen, in: Zwei Erzhlungen, Phillipp Reclam', 'jun. Verlag, page 19: Die eigenen Zimmer hatten', 'sich die Enkel nach persnlichem Geschmack', 'eingerichtet. The grandchildren had furnished', 'their own rooms according to their personal taste.', 'to arrange'] }\\
\end{tabular}
}
%===einschalten===
\card{\normalfont \Huge einschalten}{
\begin{tabular}{lll}
\parbox[t][][t]{2.0 cm}{\normalfont \raggedleft ich\\du\\er/sie/es\\wir\\ihr\\sie} &    
\parbox[t][][t]{2cm}{\normalfont schalte ein\\schaltest ein\\schaltet ein\\schalten ein\\schaltet ein\\schalten ein} &
\parbox[t][][t]{2cm}{\normalfont schaltete ein\\schaltetest ein\\schaltete ein\\schalteten ein\\schaltetet ein\\schalteten ein}\\
\end{tabular}
\begin{tabular}{l}
\parbox[t][][t]{8cm}{}\\
\parbox[t][][t]{8cm}{\normalfont \small ['(transitive) to switch on, to power up, to enable', '(to put a mechanism, device or system into', 'operation) Synonyms: anstellen, anmachen,', 'anschalten, anschmeien, aktivieren Antonyms:', 'ausschalten, abstellen, ausstellen, ausmachen,', 'abschalten, deaktivieren (transitive) to call in', '(reflexive) to intervene 2010, Der Spiegel, issue', '49/2010, page 89: Deutsche und niederlndische', 'Staatsbahnen wollten einen Milliardenmarkt unter', 'sich aufteilen. Jetzt hat sich das', 'Bundeskartellamt eingeschaltet. German and Dutch', 'state railroads wanted to carve up a market worth', 'billions between themselves. Now the federal', 'antitrust agency has intervened.'] }\\
\end{tabular}
}
%===einsetzen===
\card{\normalfont \Huge einsetzen}{
\begin{tabular}{lll}
\parbox[t][][t]{2.0 cm}{\normalfont \raggedleft ich\\du\\er/sie/es\\wir\\ihr\\sie} &    
\parbox[t][][t]{2cm}{\normalfont setze ein\\setzt ein\\setzt ein\\setzen ein\\setzt ein\\setzen ein} &
\parbox[t][][t]{2cm}{\normalfont setzte ein\\setztest ein\\setzte ein\\setzten ein\\setztet ein\\setzten ein}\\
\end{tabular}
\begin{tabular}{l}
\parbox[t][][t]{8cm}{}\\
\parbox[t][][t]{8cm}{\normalfont \small ['to put in; set in; insert to use; employ to', 'appoint to risk to stake (reflexive, with fr +', 'accusative) to speak up for 2017, Xenia Tchoumi,', '20 Minuten, 6 March: Ich wurde vom International', 'Trade Center eingeladen, einer Unterorganisation', 'der UNO, die sich auch fr Frauen in der', 'Arbeitswelt einsetzt. I was invited by the', 'International Trade Center, a UN subsidiary that', 'really speaks up for women in the workplace.'] }\\
\end{tabular}
}
%===einsteigen===
\card{\normalfont \Huge einsteigen}{
\begin{tabular}{lll}
\parbox[t][][t]{2.0 cm}{\normalfont \raggedleft ich\\du\\er/sie/es\\wir\\ihr\\sie} &    
\parbox[t][][t]{2cm}{\normalfont steige ein\\steigst ein\\steigt ein\\steigen ein\\steigt ein\\steigen ein} &
\parbox[t][][t]{2cm}{\normalfont stieg ein\\stiegst ein\\stieg ein\\stiegen ein\\stiegt ein\\stiegen ein}\\
\end{tabular}
\begin{tabular}{l}
\parbox[t][][t]{8cm}{}\\
\parbox[t][][t]{8cm}{\normalfont \small ['to get in to get on'] }\\
\end{tabular}
}
%===einstellen===
\card{\normalfont \Huge einstellen}{
\begin{tabular}{lll}
\parbox[t][][t]{2.0 cm}{\normalfont \raggedleft ich\\du\\er/sie/es\\wir\\ihr\\sie} &    
\parbox[t][][t]{2cm}{\normalfont stelle ein\\stellst ein\\stellt ein\\stellen ein\\stellt ein\\stellen ein} &
\parbox[t][][t]{2cm}{\normalfont stellte ein\\stelltest ein\\stellte ein\\stellten ein\\stelltet ein\\stellten ein}\\
\end{tabular}
\begin{tabular}{l}
\parbox[t][][t]{8cm}{}\\
\parbox[t][][t]{8cm}{\normalfont \small ['to employ, to hire to adjust (reflexive) to', 'prepare (reflexive) to appear to cease, to stop'] }\\
\end{tabular}
}
%===einziehen===
\card{\normalfont \Huge einziehen}{
\begin{tabular}{lll}
\parbox[t][][t]{2.0 cm}{\normalfont \raggedleft ich\\du\\er/sie/es\\wir\\ihr\\sie} &    
\parbox[t][][t]{2cm}{\normalfont ziehe ein\\ziehst ein\\zieht ein\\ziehen ein\\zieht ein\\ziehen ein} &
\parbox[t][][t]{2cm}{\normalfont zog ein\\zogst ein\\zog ein\\zogen ein\\zogt ein\\zogen ein}\\
\end{tabular}
\begin{tabular}{l}
\parbox[t][][t]{8cm}{}\\
\parbox[t][][t]{8cm}{\normalfont \small ['(transitive) to thread (elastic, thread etc.)', '(transitive) to put in (a joist, wall etc.)', '(transitive) to retract, pull in; to lower (a', 'periscope); to take in (a rudder) (transitive,', 'military) to conscript, draft (transitive) to', 'collect (taxes etc.) (transitive, typography) to', 'indent'] }\\
\end{tabular}
}
%===empfangen===
\card{\normalfont \Huge empfangen}{
\begin{tabular}{lll}
\parbox[t][][t]{2.0 cm}{\normalfont \raggedleft ich\\du\\er/sie/es\\wir\\ihr\\sie} &    
\parbox[t][][t]{2cm}{\normalfont empfange\\empfängst\\empfängt\\empfangen\\empfangt\\empfangen} &
\parbox[t][][t]{2cm}{\normalfont empfing\\empfingst\\empfing\\empfingen\\empfingt\\empfingen}\\
\end{tabular}
\begin{tabular}{l}
\parbox[t][][t]{8cm}{}\\
\parbox[t][][t]{8cm}{\normalfont \small ['(transitive) to receive (e.g. a present, a guest)', '(transitive) to welcome; to greet; to receive', 'cordially (transitive or intransitive, biology) to', 'conceive; to become pregnant (with)'] }\\
\end{tabular}
}
%===empfehlen===
\card{\normalfont \Huge empfehlen}{
\begin{tabular}{lll}
\parbox[t][][t]{2.0 cm}{\normalfont \raggedleft ich\\du\\er/sie/es\\wir\\ihr\\sie} &    
\parbox[t][][t]{2cm}{\normalfont empfehle\\empfiehlst\\empfiehlt\\empfehlen\\empfehlt\\empfehlen} &
\parbox[t][][t]{2cm}{\normalfont empfahl\\empfahlst\\empfahl\\empfahlen\\empfahlt\\empfahlen}\\
\end{tabular}
\begin{tabular}{l}
\parbox[t][][t]{8cm}{}\\
\parbox[t][][t]{8cm}{\normalfont \small ['(transitive) to recommend; to advocate (reflexive)', "to offer one's services"] }\\
\end{tabular}
}
%===enthalten===
\card{\normalfont \Huge enthalten}{
\begin{tabular}{lll}
\parbox[t][][t]{2.0 cm}{\normalfont \raggedleft ich\\du\\er/sie/es\\wir\\ihr\\sie} &    
\parbox[t][][t]{2cm}{\normalfont enthalte\\enthältst\\enthält\\enthalten\\enthaltet\\enthalten} &
\parbox[t][][t]{2cm}{\normalfont enthielt\\enthieltest\\enthielt\\enthielten\\enthieltet\\enthielten}\\
\end{tabular}
\begin{tabular}{l}
\parbox[t][][t]{8cm}{}\\
\parbox[t][][t]{8cm}{\normalfont \small ['to contain, comprise or include 1918, Elisabeth', 'von Heyking, Die Orgelpfeifen, in: Zwei', 'Erzhlungen, Phillipp Reclam jun. Verlag, page 14:', 'Es war aber auch eine Gegend, ber die sich viel', 'lernen lie, und weltbekannte Orte enthielt sie,', 'bei deren Nennung man stolz oder traurig werden', 'konnte, je nachdem. But it was really a region', 'that one could learn a lot about, and it contained', 'world-famous places, at the mention of which one', 'could become proud or sad, as the case may be.', '(reflexive, with genitive) to abstain from'] }\\
\end{tabular}
}
%===entlassen===
\card{\normalfont \Huge entlassen}{
\begin{tabular}{lll}
\parbox[t][][t]{2.0 cm}{\normalfont \raggedleft ich\\du\\er/sie/es\\wir\\ihr\\sie} &    
\parbox[t][][t]{2cm}{\normalfont entlasse\\entlässt\\entlässt\\entlassen\\entlasst\\entlassen} &
\parbox[t][][t]{2cm}{\normalfont entließ\\entließt\\entließ\\entließen\\entließt\\entließen}\\
\end{tabular}
\begin{tabular}{l}
\parbox[t][][t]{8cm}{}\\
\parbox[t][][t]{8cm}{\normalfont \small ['(a prisoner or patient) to release (from a job) to', 'dismiss, to fire'] }\\
\end{tabular}
}
%===entscheiden===
\card{\normalfont \Huge entscheiden}{
\begin{tabular}{lll}
\parbox[t][][t]{2.0 cm}{\normalfont \raggedleft ich\\du\\er/sie/es\\wir\\ihr\\sie} &    
\parbox[t][][t]{2cm}{\normalfont entscheide\\entscheidest\\entscheidet\\entscheiden\\entscheidet\\entscheiden} &
\parbox[t][][t]{2cm}{\normalfont entschied\\entschiedest\\entschied\\entschieden\\entschiedet\\entschieden}\\
\end{tabular}
\begin{tabular}{l}
\parbox[t][][t]{8cm}{}\\
\parbox[t][][t]{8cm}{\normalfont \small ['(transitive or intransitive or reflexive) to', 'decide'] }\\
\end{tabular}
}
%===entschuldigen===
\card{\normalfont \Huge entschuldigen}{
\begin{tabular}{lll}
\parbox[t][][t]{2.0 cm}{\normalfont \raggedleft ich\\du\\er/sie/es\\wir\\ihr\\sie} &    
\parbox[t][][t]{2cm}{\normalfont entschuldige\\entschuldigst\\entschuldigt\\entschuldigen\\entschuldigt\\entschuldigen} &
\parbox[t][][t]{2cm}{\normalfont entschuldigte\\entschuldigtest\\entschuldigte\\entschuldigten\\entschuldigtet\\entschuldigten}\\
\end{tabular}
\begin{tabular}{l}
\parbox[t][][t]{8cm}{}\\
\parbox[t][][t]{8cm}{\normalfont \small ['(transitive) to excuse (intransitive, chiefly in', 'imperative) to excuse/forgive something 1960,', "Marie Luise Kaschnitz, 'Der Strohhalm':", 'Entschuldige, aber ich liebe dich nicht mehr. "I\'m', 'sorry, but I don\'t love you anymore." (reflexive)', 'to apologize, to apologise'] }\\
\end{tabular}
}
%===entsprechen===
\card{\normalfont \Huge entsprechen}{
\begin{tabular}{lll}
\parbox[t][][t]{2.0 cm}{\normalfont \raggedleft ich\\du\\er/sie/es\\wir\\ihr\\sie} &    
\parbox[t][][t]{2cm}{\normalfont entspreche\\entsprichst\\entspricht\\entsprechen\\entsprecht\\entsprechen} &
\parbox[t][][t]{2cm}{\normalfont entsprach\\entsprachst\\entsprach\\entsprachen\\entspracht\\entsprachen}\\
\end{tabular}
\begin{tabular}{l}
\parbox[t][][t]{8cm}{}\\
\parbox[t][][t]{8cm}{\normalfont \small ['to correspond (be equivalent or similar in', 'character, etc.) to meet (satisfy, comply with)'] }\\
\end{tabular}
}
%===entstehen===
\card{\normalfont \Huge entstehen}{
\begin{tabular}{lll}
\parbox[t][][t]{2.0 cm}{\normalfont \raggedleft ich\\du\\er/sie/es\\wir\\ihr\\sie} &    
\parbox[t][][t]{2cm}{\normalfont entstehe\\entstehst\\entsteht\\entstehen\\entsteht\\entstehen} &
\parbox[t][][t]{2cm}{\normalfont entstand\\entstandest\\entstand\\entstanden\\entstandet\\entstanden}\\
\end{tabular}
\begin{tabular}{l}
\parbox[t][][t]{8cm}{}\\
\parbox[t][][t]{8cm}{\normalfont \small ['to come into being to arise to develop'] }\\
\end{tabular}
}
%===enttäuschen===
\card{\normalfont \Huge enttäuschen}{
\begin{tabular}{lll}
\parbox[t][][t]{2.0 cm}{\normalfont \raggedleft ich\\du\\er/sie/es\\wir\\ihr\\sie} &    
\parbox[t][][t]{2cm}{\normalfont enttäusche\\enttäuschst\\enttäuscht\\enttäuschen\\enttäuscht\\enttäuschen} &
\parbox[t][][t]{2cm}{\normalfont enttäuschte\\enttäuschtest\\enttäuschte\\enttäuschten\\enttäuschtet\\enttäuschten}\\
\end{tabular}
\begin{tabular}{l}
\parbox[t][][t]{8cm}{}\\
\parbox[t][][t]{8cm}{\normalfont \small ['to disappoint'] }\\
\end{tabular}
}
%===entwickeln===
\card{\normalfont \Huge entwickeln}{
\begin{tabular}{lll}
\parbox[t][][t]{2.0 cm}{\normalfont \raggedleft ich\\du\\er/sie/es\\wir\\ihr\\sie} &    
\parbox[t][][t]{2cm}{\normalfont entwickleich entwickeleich entwickel\\entwickelst\\entwickelt\\entwickeln\\entwickelt\\entwickeln} &
\parbox[t][][t]{2cm}{\normalfont entwickelte\\entwickeltest\\entwickelte\\entwickelten\\entwickeltet\\entwickelten}\\
\end{tabular}
\begin{tabular}{l}
\parbox[t][][t]{8cm}{}\\
\parbox[t][][t]{8cm}{\normalfont \small ['(transitive) to develop 1990, Zwei-plus-Vier-', 'Vertrag, in: Bundesgesetzblatt 1990, part 2, page', '1318: entschlossen, in bereinstimmung mit ihren', 'Verpflichtungen aus der Charta der Vereinten', 'Nationen freundschaftliche, auf der Achtung vor', 'dem Grundsatz der Gleichberechtigung und', 'Selbstbestimmung der Vlker beruhende Beziehungen', 'zwischen den Nationen zu entwickeln und andere', 'geeignete Manahmen zur Festigung des Weltfriedens', 'zu treffen, [...] Resolved, in accordance with', 'their obligations under the Charter of the United', 'Nations to develop friendly relations among', 'nations based on respect for the principle of', 'equal rights and self-determination of peoples,', 'and to take other appropriate measures to', 'strengthen universal peace; [...] (reflexive) to', 'shape up'] }\\
\end{tabular}
}
%===erfahren===
\card{\normalfont \Huge erfahren}{
\begin{tabular}{lll}
\parbox[t][][t]{2.0 cm}{\normalfont \raggedleft ich\\du\\er/sie/es\\wir\\ihr\\sie} &    
\parbox[t][][t]{2cm}{\normalfont erfahre\\erfährst\\erfährt\\erfahren\\erfahrt\\erfahren} &
\parbox[t][][t]{2cm}{\normalfont erfuhr\\erfuhrst\\erfuhr\\erfuhren\\erfuhrt\\erfuhren}\\
\end{tabular}
\begin{tabular}{l}
\parbox[t][][t]{8cm}{}\\
\parbox[t][][t]{8cm}{\normalfont \small ['to find out, learn, to come to know Ich habe aus', 'der Zeitung von Helgas Tod erfahren.I have learned', "from the newspaper about Helga's death. Sie hat", 'gerade erfahren, dass sie schwanger ist.She has', 'just found out that she is pregnant. to experience', 'Er hat im Leben viel Gutes und Bses erfahren.He', 'has experienced a lot of good and bad things in', 'life.'] }\\
\end{tabular}
}
%===erfinden===
\card{\normalfont \Huge erfinden}{
\begin{tabular}{lll}
\parbox[t][][t]{2.0 cm}{\normalfont \raggedleft ich\\du\\er/sie/es\\wir\\ihr\\sie} &    
\parbox[t][][t]{2cm}{\normalfont erfinde\\erfindest\\erfindet\\erfinden\\erfindet\\erfinden} &
\parbox[t][][t]{2cm}{\normalfont erfand\\erfandest\\erfand\\erfanden\\erfandet\\erfanden}\\
\end{tabular}
\begin{tabular}{l}
\parbox[t][][t]{8cm}{}\\
\parbox[t][][t]{8cm}{\normalfont \small ['(transitive) to invent (transitive) to fabricate', '(something untrue); to make up (a story)', '(transitive) to find out, to discover'] }\\
\end{tabular}
}
%===erfüllen===
\card{\normalfont \Huge erfüllen}{
\begin{tabular}{lll}
\parbox[t][][t]{2.0 cm}{\normalfont \raggedleft ich\\du\\er/sie/es\\wir\\ihr\\sie} &    
\parbox[t][][t]{2cm}{\normalfont erfülle\\erfüllst\\erfüllt\\erfüllen\\erfüllt\\erfüllen} &
\parbox[t][][t]{2cm}{\normalfont erfüllte\\erfülltest\\erfüllte\\erfüllten\\erfülltet\\erfüllten}\\
\end{tabular}
\begin{tabular}{l}
\parbox[t][][t]{8cm}{}\\
\parbox[t][][t]{8cm}{\normalfont \small ['(transitive) to fill (with something physical or', 'abstract) (transitive) to fulfill / fulfil', '(reflexive) to come true'] }\\
\end{tabular}
}
%===erhalten===
\card{\normalfont \Huge erhalten}{
\begin{tabular}{lll}
\parbox[t][][t]{2.0 cm}{\normalfont \raggedleft ich\\du\\er/sie/es\\wir\\ihr\\sie} &    
\parbox[t][][t]{2cm}{\normalfont erhalte\\erhältst\\erhält\\erhalten\\erhaltet\\erhalten} &
\parbox[t][][t]{2cm}{\normalfont erhielt\\erhieltest\\erhielt\\erhielten\\erhieltet\\erhielten}\\
\end{tabular}
\begin{tabular}{l}
\parbox[t][][t]{8cm}{}\\
\parbox[t][][t]{8cm}{\normalfont \small ['to maintain, uphold to receive, obtain to save', 'Gott erhalte den Kaiser!God save the Emperor!'] }\\
\end{tabular}
}
%===erhöhen===
\card{\normalfont \Huge erhöhen}{
\begin{tabular}{lll}
\parbox[t][][t]{2.0 cm}{\normalfont \raggedleft ich\\du\\er/sie/es\\wir\\ihr\\sie} &    
\parbox[t][][t]{2cm}{\normalfont erhöhe\\erhöhst\\erhöht\\erhöhen\\erhöht\\erhöhen} &
\parbox[t][][t]{2cm}{\normalfont erhöhte\\erhöhtest\\erhöhte\\erhöhten\\erhöhtet\\erhöhten}\\
\end{tabular}
\begin{tabular}{l}
\parbox[t][][t]{8cm}{}\\
\parbox[t][][t]{8cm}{\normalfont \small ['to heighten, raise unsere Moral erhhento heighten', 'our morality to increase to exalt Wer sich selbst', 'erhhet, der soll erniedriget werden.[1] The one', 'who exalts oneself, that one shall be humbled.'] }\\
\end{tabular}
}
%===erinnern===
\card{\normalfont \Huge erinnern}{
\begin{tabular}{lll}
\parbox[t][][t]{2.0 cm}{\normalfont \raggedleft ich\\du\\er/sie/es\\wir\\ihr\\sie} &    
\parbox[t][][t]{2cm}{\normalfont erinnreich erinnereich erinner\\erinnerst\\erinnert\\erinnern\\erinnert\\erinnern} &
\parbox[t][][t]{2cm}{\normalfont erinnerte\\erinnertest\\erinnerte\\erinnerten\\erinnertet\\erinnerten}\\
\end{tabular}
\begin{tabular}{l}
\parbox[t][][t]{8cm}{}\\
\parbox[t][][t]{8cm}{\normalfont \small ['(ditransitive) to remind [+accusative = whom]', '[+genitive = which matter] or [+ an (object) =', 'which matter] (reflexive, transitive,', 'intransitive) to remember [+genitive = which', 'matter] or [+ an (object) = which matter]', '(transitive, colloquial, rare) to remember', '[+accusative]'] }\\
\end{tabular}
}
%===erkälten===
\card{\normalfont \Huge erkälten}{
\begin{tabular}{lll}
\parbox[t][][t]{2.0 cm}{\normalfont \raggedleft ich\\du\\er/sie/es\\wir\\ihr\\sie} &    
\parbox[t][][t]{2cm}{\normalfont erkälte\\erkältest\\erkältet\\erkälten\\erkältet\\erkälten} &
\parbox[t][][t]{2cm}{\normalfont erkältete\\erkältetest\\erkältete\\erkälteten\\erkältetet\\erkälteten}\\
\end{tabular}
\begin{tabular}{l}
\parbox[t][][t]{8cm}{}\\
\parbox[t][][t]{8cm}{\normalfont \small ['(reflexive) to catch a cold'] }\\
\end{tabular}
}
%===erkennen===
\card{\normalfont \Huge erkennen}{
\begin{tabular}{lll}
\parbox[t][][t]{2.0 cm}{\normalfont \raggedleft ich\\du\\er/sie/es\\wir\\ihr\\sie} &    
\parbox[t][][t]{2cm}{\normalfont erkenne\\erkennst\\erkennt\\erkennen\\erkennt\\erkennen} &
\parbox[t][][t]{2cm}{\normalfont erkannte\\erkanntest\\erkannte\\erkannten\\erkanntet\\erkannten}\\
\end{tabular}
\begin{tabular}{l}
\parbox[t][][t]{8cm}{}\\
\parbox[t][][t]{8cm}{\normalfont \small ['to recognize, perceive, acknowledge to realize,', 'detect, see, know,  identify, discover, understand'] }\\
\end{tabular}
}
%===erklären===
\card{\normalfont \Huge erklären}{
\begin{tabular}{lll}
\parbox[t][][t]{2.0 cm}{\normalfont \raggedleft ich\\du\\er/sie/es\\wir\\ihr\\sie} &    
\parbox[t][][t]{2cm}{\normalfont erkläre\\erklärst\\erklärt\\erklären\\erklärt\\erklären} &
\parbox[t][][t]{2cm}{\normalfont erklärte\\erklärtest\\erklärte\\erklärten\\erklärtet\\erklärten}\\
\end{tabular}
\begin{tabular}{l}
\parbox[t][][t]{8cm}{}\\
\parbox[t][][t]{8cm}{\normalfont \small ['to explain to declare to assert 1919, Walther', 'Kabel, Irrende Seelen, Werner Dietsch Verlag, page', '11: Und letztens, vor einem Monat, hat er mir', 'erklrt, er wrde mir auch nicht mehr mit einem', 'Pfennig aushelfen. And recently, one month ago, he', 'asserted to me that he would not even help me out', 'with one penny anymore.'] }\\
\end{tabular}
}
%===erlauben===
\card{\normalfont \Huge erlauben}{
\begin{tabular}{lll}
\parbox[t][][t]{2.0 cm}{\normalfont \raggedleft ich\\du\\er/sie/es\\wir\\ihr\\sie} &    
\parbox[t][][t]{2cm}{\normalfont erlaube\\erlaubst\\erlaubt\\erlauben\\erlaubt\\erlauben} &
\parbox[t][][t]{2cm}{\normalfont erlaubte\\erlaubtest\\erlaubte\\erlaubten\\erlaubtet\\erlaubten}\\
\end{tabular}
\begin{tabular}{l}
\parbox[t][][t]{8cm}{}\\
\parbox[t][][t]{8cm}{\normalfont \small ['(with dative) to allow; to permit'] }\\
\end{tabular}
}
%===erleben===
\card{\normalfont \Huge erleben}{
\begin{tabular}{lll}
\parbox[t][][t]{2.0 cm}{\normalfont \raggedleft ich\\du\\er/sie/es\\wir\\ihr\\sie} &    
\parbox[t][][t]{2cm}{\normalfont erlebe\\erlebst\\erlebt\\erleben\\erlebt\\erleben} &
\parbox[t][][t]{2cm}{\normalfont erlebte\\erlebtest\\erlebte\\erlebten\\erlebtet\\erlebten}\\
\end{tabular}
\begin{tabular}{l}
\parbox[t][][t]{8cm}{}\\
\parbox[t][][t]{8cm}{\normalfont \small ['to experience 1930, Paul Joachimsen, Der', 'Humanismus und die Entwicklung des deutschen', 'Geistes, in: Deutsche Vierteljahrsschrift fr', 'Literaturwissenschaft und Geistesgeschichte, 8,', 'page 467: Und nun kommt die Reformation selbst.', 'Die grte geistige Umwlzung, die je ein Volk des', 'Abendlandes erlebt hat. And now comes the', 'Reformation itself. The largest spiritual upheaval', 'that was ever experienced by a nation of the', 'Occident. to see, live to see, witness to undergo'] }\\
\end{tabular}
}
%===erledigen===
\card{\normalfont \Huge erledigen}{
\begin{tabular}{lll}
\parbox[t][][t]{2.0 cm}{\normalfont \raggedleft ich\\du\\er/sie/es\\wir\\ihr\\sie} &    
\parbox[t][][t]{2cm}{\normalfont erledige\\erledigst\\erledigt\\erledigen\\erledigt\\erledigen} &
\parbox[t][][t]{2cm}{\normalfont erledigte\\erledigtest\\erledigte\\erledigten\\erledigtet\\erledigten}\\
\end{tabular}
\begin{tabular}{l}
\parbox[t][][t]{8cm}{}\\
\parbox[t][][t]{8cm}{\normalfont \small ['to finish, to complete to take care of, to see to,', 'to handle (slang) to kill'] }\\
\end{tabular}
}
%===erlöschen===
\card{\normalfont \Huge erlöschen}{
\begin{tabular}{lll}
\parbox[t][][t]{2.0 cm}{\normalfont \raggedleft ich\\du\\er/sie/es\\wir\\ihr\\sie} &    
\parbox[t][][t]{2cm}{\normalfont lösche\\löschst\\löscht\\löschen\\löscht\\löschen} &
\parbox[t][][t]{2cm}{\normalfont löschte\\löschtest\\löschte\\löschten\\löschtet\\löschten}\\
\end{tabular}
\begin{tabular}{l}
\parbox[t][][t]{8cm}{}\\
\parbox[t][][t]{8cm}{\normalfont \small ['(transitive) to extinguish; to quench, to douse', '1823 July 4,  "Die Stimme der Natur", in  Dresdner', 'Abend-Zeitung[1], number 159, page 633:Nichts', 'erlscht des Geistes Flammen.(please add an English', 'translation of this quote) Synonyms: lschen,', 'ersticken'] }\\
\end{tabular}
}
%===eröffnen===
\card{\normalfont \Huge eröffnen}{
\begin{tabular}{lll}
\parbox[t][][t]{2.0 cm}{\normalfont \raggedleft ich\\du\\er/sie/es\\wir\\ihr\\sie} &    
\parbox[t][][t]{2cm}{\normalfont eröffne\\eröffnest\\eröffnet\\eröffnen\\eröffnet\\eröffnen} &
\parbox[t][][t]{2cm}{\normalfont eröffnete\\eröffnetest\\eröffnete\\eröffneten\\eröffnetet\\eröffneten}\\
\end{tabular}
\begin{tabular}{l}
\parbox[t][][t]{8cm}{}\\
\parbox[t][][t]{8cm}{\normalfont \small ['(a ceremony, game, shop, etc.) to open', '(figuratively, a possibility, perspective, etc.)', 'to open up 1931, Arthur Schnitzler, Flucht in die', 'Finsternis, S. Fischer Verlag, page 119120: Die', 'Mglichkeit einer neuen, und zwar einer', 'journalistischen Laufbahn, schien sich ihm zu', 'erffnen. The possibility of a new, namely a', 'journalistic career, seemed to open up to him.'] }\\
\end{tabular}
}
%===erreichen===
\card{\normalfont \Huge erreichen}{
\begin{tabular}{lll}
\parbox[t][][t]{2.0 cm}{\normalfont \raggedleft ich\\du\\er/sie/es\\wir\\ihr\\sie} &    
\parbox[t][][t]{2cm}{\normalfont erreiche\\erreichst\\erreicht\\erreichen\\erreicht\\erreichen} &
\parbox[t][][t]{2cm}{\normalfont erreichte\\erreichtest\\erreichte\\erreichten\\erreichtet\\erreichten}\\
\end{tabular}
\begin{tabular}{l}
\parbox[t][][t]{8cm}{}\\
\parbox[t][][t]{8cm}{\normalfont \small ['to reach, to catch, to accomplish'] }\\
\end{tabular}
}
%===erscheinen===
\card{\normalfont \Huge erscheinen}{
\begin{tabular}{lll}
\parbox[t][][t]{2.0 cm}{\normalfont \raggedleft ich\\du\\er/sie/es\\wir\\ihr\\sie} &    
\parbox[t][][t]{2cm}{\normalfont erscheine\\erscheinst\\erscheint\\erscheinen\\erscheint\\erscheinen} &
\parbox[t][][t]{2cm}{\normalfont erschien\\erschienst\\erschien\\erschienen\\erschient\\erschienen}\\
\end{tabular}
\begin{tabular}{l}
\parbox[t][][t]{8cm}{}\\
\parbox[t][][t]{8cm}{\normalfont \small ['to appear 2017, Neue Zrcher Zeitung, 30 March: Im', 'Korruptionsskandal um eine ihrer Vertrauten ist', 'Sdkoreas frhere Prsidentin Park Geun Hye vor einem', 'Ermittlungsrichter erschienen. In the corruption', 'scandal surrounding one of her confidantes, South', "Korea's former President Park Geun-hye has", 'appeared before an investigator. to come out, be', 'released to appear, to seem'] }\\
\end{tabular}
}
%===erschrecken===
\card{\normalfont \Huge erschrecken}{
\begin{tabular}{lll}
\parbox[t][][t]{2.0 cm}{\normalfont \raggedleft ich\\du\\er/sie/es\\wir\\ihr\\sie} &    
\parbox[t][][t]{2cm}{\normalfont erschrecke\\erschrickst\\erschrickt\\erschrecken\\erschreckt\\erschrecken} &
\parbox[t][][t]{2cm}{\normalfont erschrak\\erschrakst\\erschrak\\erschraken\\erschrakt\\erschraken}\\
\end{tabular}
\begin{tabular}{l}
\parbox[t][][t]{8cm}{}\\
\parbox[t][][t]{8cm}{\normalfont \small ['(standard, intransitive, auxiliary sein) to be', 'frightened; to be startled Ich bin erschrocken.  I', 'got scared. (chiefly colloquial, reflexive,', 'auxiliary haben) to be frightened; to be startled', 'Ich habe mich erschrocken.  I got scared. (dated,', 'transitive, auxiliary haben) to frighten; to scare', '(someone) Heinrich von Kleist, Mdchenrtsel Bebt', 'er, ihr Schwestern, was, / Redet, erschrickt', 'ihn?When he quakes, you sisters, what, / Speak,', 'scares him?'] }\\
\end{tabular}
}
%===erwähnen===
\card{\normalfont \Huge erwähnen}{
\begin{tabular}{lll}
\parbox[t][][t]{2.0 cm}{\normalfont \raggedleft ich\\du\\er/sie/es\\wir\\ihr\\sie} &    
\parbox[t][][t]{2cm}{\normalfont erwähne\\erwähnst\\erwähnt\\erwähnen\\erwähnt\\erwähnen} &
\parbox[t][][t]{2cm}{\normalfont erwähnte\\erwähntest\\erwähnte\\erwähnten\\erwähntet\\erwähnten}\\
\end{tabular}
\begin{tabular}{l}
\parbox[t][][t]{8cm}{}\\
\parbox[t][][t]{8cm}{\normalfont \small ['to mention (to speak of something)'] }\\
\end{tabular}
}
%===erwarten===
\card{\normalfont \Huge erwarten}{
\begin{tabular}{lll}
\parbox[t][][t]{2.0 cm}{\normalfont \raggedleft ich\\du\\er/sie/es\\wir\\ihr\\sie} &    
\parbox[t][][t]{2cm}{\normalfont erwarte\\erwartest\\erwartet\\erwarten\\erwartet\\erwarten} &
\parbox[t][][t]{2cm}{\normalfont erwartete\\erwartetest\\erwartete\\erwarteten\\erwartetet\\erwarteten}\\
\end{tabular}
\begin{tabular}{l}
\parbox[t][][t]{8cm}{}\\
\parbox[t][][t]{8cm}{\normalfont \small ['to expect to await'] }\\
\end{tabular}
}
%===erzählen===
\card{\normalfont \Huge erzählen}{
\begin{tabular}{lll}
\parbox[t][][t]{2.0 cm}{\normalfont \raggedleft ich\\du\\er/sie/es\\wir\\ihr\\sie} &    
\parbox[t][][t]{2cm}{\normalfont erzähle\\erzählst\\erzählt\\erzählen\\erzählt\\erzählen} &
\parbox[t][][t]{2cm}{\normalfont erzählte\\erzähltest\\erzählte\\erzählten\\erzähltet\\erzählten}\\
\end{tabular}
\begin{tabular}{l}
\parbox[t][][t]{8cm}{}\\
\parbox[t][][t]{8cm}{\normalfont \small ['to tell; to narrate; to tell a story Erzhl mir ein', 'Mrchen.Tell me a fairytale. Erzhl mir von deiner', 'Kindheit.Tell me about your childhood. (somewhat', 'informal, perhaps  regional) to talk a lot; to', 'tell nonsense; to gabble Wir haben den ganzen', 'Nachmittag erzhlt.We talked the whole afternoon.', 'Der erzhlt hier wieder einen! (strong stress on', '"erzhlen")This guy\'s talking some [bullshit] again', 'today!'] }\\
\end{tabular}
}
%===essen===
\card{\normalfont \Huge essen}{
\begin{tabular}{lll}
\parbox[t][][t]{2.0 cm}{\normalfont \raggedleft ich\\du\\er/sie/es\\wir\\ihr\\sie} &    
\parbox[t][][t]{2cm}{\normalfont esse\\isst\\isst\\essen\\esst\\essen} &
\parbox[t][][t]{2cm}{\normalfont aß\\aßt\\aß\\aßen\\aßt\\aßen}\\
\end{tabular}
\begin{tabular}{l}
\parbox[t][][t]{8cm}{}\\
\parbox[t][][t]{8cm}{\normalfont \small ['(transitive) to eat Er isst gern Schokolade.He', 'likes eating chocolate. Ich esse einen Apfel.I am', 'eating an apple. (intransitive) to eat; to dine', 'Wir haben noch nicht gegessen.We have not eaten', 'yet. Sie haben immer vor 9 Uhr abends', "gegessen.They always ate before 9 o'clock in the", 'evening.'] }\\
\end{tabular}
}
%===existieren===
\card{\normalfont \Huge existieren}{
\begin{tabular}{lll}
\parbox[t][][t]{2.0 cm}{\normalfont \raggedleft ich\\du\\er/sie/es\\wir\\ihr\\sie} &    
\parbox[t][][t]{2cm}{\normalfont existiere\\existierst\\existiert\\existieren\\existiert\\existieren} &
\parbox[t][][t]{2cm}{\normalfont existierte\\existiertest\\existierte\\existierten\\existiertet\\existierten}\\
\end{tabular}
\begin{tabular}{l}
\parbox[t][][t]{8cm}{}\\
\parbox[t][][t]{8cm}{\normalfont \small ['to exist Wir existieren, weil Gott uns liebt.We', 'exist because God loves us.'] }\\
\end{tabular}
}
%===fahren===
\card{\normalfont \Huge fahren}{
\begin{tabular}{lll}
\parbox[t][][t]{2.0 cm}{\normalfont \raggedleft ich\\du\\er/sie/es\\wir\\ihr\\sie} &    
\parbox[t][][t]{2cm}{\normalfont fahre\\fährst\\fährt\\fahren\\fahrt\\fahren} &
\parbox[t][][t]{2cm}{\normalfont fuhr\\fuhrst\\fuhr\\fuhren\\fuhrt\\fuhren}\\
\end{tabular}
\begin{tabular}{l}
\parbox[t][][t]{8cm}{}\\
\parbox[t][][t]{8cm}{\normalfont \small ['(intransitive, of a person) to go (by vehicle); to', 'sail; to travel Wir fahren diesen Sommer nach', "Holland.We're going to Holland this summer.", '[Implying a trip by car, bike, train, or ship.]', '(intransitive, of a person) to leave (by vehicle)', "Wir fahren jetzt.  We're leaving now.", '(intransitive, of a vehicle) to go; to run; to', 'drive; to sail Autos knnen schneller fahren als', 'Fahrrder.Cars can go faster than bikes.', '(intransitive, of a vehicle) to leave; to depart', 'Beeil dich! Der Zug fhrt jetzt gleich.Hurry up!', 'The train is departing in a moment. (transitive or', 'intransitive) to drive; to ride; to sail (a', 'vehicle) Sie fhrt einen roten Wagen.  She drives a', 'red car. Er fhrt wie ein Bekloppter.  He drives', 'like a maniac. (transitive) to take (someone', 'somewhere by vehicle); to drive; to transport Ich', "fahre dich zum Bahnhof.I'll take you to the train", 'station.'] }\\
\end{tabular}
}
%===fallen===
\card{\normalfont \Huge fallen}{
\begin{tabular}{lll}
\parbox[t][][t]{2.0 cm}{\normalfont \raggedleft ich\\du\\er/sie/es\\wir\\ihr\\sie} &    
\parbox[t][][t]{2cm}{\normalfont falle\\fällst\\fällt\\fallen\\fallt\\fallen} &
\parbox[t][][t]{2cm}{\normalfont fiel\\fielst\\fiel\\fielen\\fielt\\fielen}\\
\end{tabular}
\begin{tabular}{l}
\parbox[t][][t]{8cm}{}\\
\parbox[t][][t]{8cm}{\normalfont \small ['(intransitive) to fall; to drop 1960, Marie Luise', "Kaschnitz, 'Gespenster': Das Programm fiel ihr aus", 'der Hand. The programme fell from her hand. Der', 'Regen fiel wie aus Eimern.It rained cats and dogs.', "(literally: 'The rain fell as if out of buckets.')", 'Sie fiel zu Boden.She fell to the floor.', '(intransitive, military) to die; to fall in', 'battle; to die in battle; to be killed in action', '1918, Elisabeth von Heyking, Die Orgelpfeifen, in:', 'Zwei Erzhlungen, Phillipp Reclam jun. Verlag, page', '31: Bei einem Patrouillenritt, zu dem er sich', 'freiwillig gemeldet, war der lteste der Enkel', 'gefallen. Ruhte nun fern in Feindesland. On a', 'patrolling ride, for which he had volunteered, the', 'oldest of the grandchildren had died. Rested now', 'far away in enemy country. (intransitive) to fall,', 'to collapse, to be overthrown. Das Rmische Reich', 'fiel auf Grund der Vlkerwanderung.The Roman Empire', 'was overthrown by the consequences of the', 'Migration period. (intransitive) to become lower,', 'to decrease, to decline Zur Zeit der Finanzkrise', 'fielen viele Aktienkurse um zahlreiche', 'Prozentpunkte.During the banking scandal many', 'stock prices decreased by a large percentage.', '(reflexive) (with zurck) To be allotted to; to be', "able to change one's fate. Und somit fiel die", 'Entscheidung was zu tun ist auf mich zurck.And', 'thus the decision of what is to be done fell once', 'more to me.'] }\\
\end{tabular}
}
%===fangen===
\card{\normalfont \Huge fangen}{
\begin{tabular}{lll}
\parbox[t][][t]{2.0 cm}{\normalfont \raggedleft ich\\du\\er/sie/es\\wir\\ihr\\sie} &    
\parbox[t][][t]{2cm}{\normalfont fange\\fängst\\fängt\\fangen\\fangt\\fangen} &
\parbox[t][][t]{2cm}{\normalfont fing\\fingst\\fing\\fingen\\fingt\\fingen}\\
\end{tabular}
\begin{tabular}{l}
\parbox[t][][t]{8cm}{}\\
\parbox[t][][t]{8cm}{\normalfont \small ['(transitive) to catch (grab something flying in', 'the air) Der Torwart fngt den Ball.  The', 'goalkeeper catches the ball. (transitive) to', 'catch; to capture (to take hold of a person or an', 'animal) Die Frster fingen das entlaufene', 'Wildschwein.  The rangers caught the run-away', 'boar. (reflexive) to improve in health; do well', 'again; to do better Es sah schlecht um sie aus,', 'aber jetzt hat sie sich wieder gefangen. It looked', 'quite bad for her, but now she has improved again.', '(reflexive) to calm down; to compose oneself Wenn', 'du dich gefangen hast, kannst du jetzt essen', "kommen. If you've composed yourself, you may come", 'to dinner now. (colloquial, transitive, with', 'reflexive dative) to catch (a disease; something', "unpleasant) Ich hab mir auf der Kirmes nur 'ne", 'Grippe gefangen. All I caught at the fun fair was', 'a flu. (colloquial, with eine and reflexive', 'dative) to be slapped Sag das noch mal und du', "fngst dir eine. Say that again and you'll catch", 'it.'] }\\
\end{tabular}
}
%===fassen===
\card{\normalfont \Huge fassen}{
\begin{tabular}{lll}
\parbox[t][][t]{2.0 cm}{\normalfont \raggedleft ich\\du\\er/sie/es\\wir\\ihr\\sie} &    
\parbox[t][][t]{2cm}{\normalfont fasse\\fasst\\fasst\\fassen\\fasst\\fassen} &
\parbox[t][][t]{2cm}{\normalfont fasste\\fasstest\\fasste\\fassten\\fasstet\\fassten}\\
\end{tabular}
\begin{tabular}{l}
\parbox[t][][t]{8cm}{}\\
\parbox[t][][t]{8cm}{\normalfont \small ['to gird, to surround, to confine into a form 1981,', 'Rezzori, Gregor von,  Der arbeitslose Knig.', 'Maghrebinische Mrchen, Gtersloh: C.Bertelsmann,', 'page 98:Einer von ihnen hat diese Quelle fassen', 'und in den Brunnen leiten lassen, damit das brave', 'ackerbebauende Volk daran seinen Durst stillen und', 'sein Vieh trnken knne.(please add an English', 'translation of this quote) to grasp, to catch', 'Synonyms: greifen, packen to seize, to capture', 'Synonym: ergreifen (reflexive) to compose oneself', '(to calm, to free from agitation) Synonym:', 'beruhigen'] }\\
\end{tabular}
}
%===fehlen===
\card{\normalfont \Huge fehlen}{
\begin{tabular}{lll}
\parbox[t][][t]{2.0 cm}{\normalfont \raggedleft ich\\du\\er/sie/es\\wir\\ihr\\sie} &    
\parbox[t][][t]{2cm}{\normalfont fehle\\fehlst\\fehlt\\fehlen\\fehlt\\fehlen} &
\parbox[t][][t]{2cm}{\normalfont fehlte\\fehltest\\fehlte\\fehlten\\fehltet\\fehlten}\\
\end{tabular}
\begin{tabular}{l}
\parbox[t][][t]{8cm}{}\\
\parbox[t][][t]{8cm}{\normalfont \small ['to lack to be absent (dated) to fail'] }\\
\end{tabular}
}
%===feiern===
\card{\normalfont \Huge feiern}{
\begin{tabular}{lll}
\parbox[t][][t]{2.0 cm}{\normalfont \raggedleft ich\\du\\er/sie/es\\wir\\ihr\\sie} &    
\parbox[t][][t]{2cm}{\normalfont feireich feiereich feier\\feierst\\feiert\\feiern\\feiert\\feiern} &
\parbox[t][][t]{2cm}{\normalfont feierte\\feiertest\\feierte\\feierten\\feiertet\\feierten}\\
\end{tabular}
\begin{tabular}{l}
\parbox[t][][t]{8cm}{}\\
\parbox[t][][t]{8cm}{\normalfont \small ['(transitive or intransitive) to celebrate; to', 'party; applicable from the most solemn to the most', 'revelrous forms In dieser Kirche wird jeden Abend', 'die Heilige Messe gefeiert. In this church, Holy', 'Mass is celebrated every evening. Die', 'Hochzeitsgste feierten bis zum frhen Morgen. The', 'wedding guests partied until the early morning.', '(transitive, colloquial, youth  slang) to love; to', 'adore Alter, ich feier das Lied voll! I just love', 'this song, mate!'] }\\
\end{tabular}
}
%===feixen===
\card{\normalfont \Huge feixen}{
\begin{tabular}{lll}
\parbox[t][][t]{2.0 cm}{\normalfont \raggedleft ich\\du\\er/sie/es\\wir\\ihr\\sie} &    
\parbox[t][][t]{2cm}{\normalfont feixe\\feixt\\feixt\\feixen\\feixt\\feixen} &
\parbox[t][][t]{2cm}{\normalfont feixte\\feixtest\\feixte\\feixten\\feixtet\\feixten}\\
\end{tabular}
\begin{tabular}{l}
\parbox[t][][t]{8cm}{}\\
\parbox[t][][t]{8cm}{\normalfont \small ['(colloquial) to smirk'] }\\
\end{tabular}
}
%===festhalten===
\card{\normalfont \Huge festhalten}{
\begin{tabular}{lll}
\parbox[t][][t]{2.0 cm}{\normalfont \raggedleft ich\\du\\er/sie/es\\wir\\ihr\\sie} &    
\parbox[t][][t]{2cm}{\normalfont halte fest\\hältst fest\\hält fest\\halten fest\\haltet fest\\halten fest} &
\parbox[t][][t]{2cm}{\normalfont hielt fest\\hieltest fest\\hielt fest\\hielten fest\\hieltet fest\\hielten fest}\\
\end{tabular}
\begin{tabular}{l}
\parbox[t][][t]{8cm}{}\\
\parbox[t][][t]{8cm}{\normalfont \small ['to hold; to adhere to retain to record; to capture', 'to detain'] }\\
\end{tabular}
}
%===finden===
\card{\normalfont \Huge finden}{
\begin{tabular}{lll}
\parbox[t][][t]{2.0 cm}{\normalfont \raggedleft ich\\du\\er/sie/es\\wir\\ihr\\sie} &    
\parbox[t][][t]{2cm}{\normalfont finde\\findest\\findet\\finden\\findet\\finden} &
\parbox[t][][t]{2cm}{\normalfont fand\\fandest\\fand\\fanden\\fandet\\fanden}\\
\end{tabular}
\begin{tabular}{l}
\parbox[t][][t]{8cm}{}\\
\parbox[t][][t]{8cm}{\normalfont \small ['(transitive) to find; to discover (transitive,', 'with a noun phrase and a predicate adjective) to', 'think that (something) is (a certain way); to', 'consider (something) to be (a certain way); to', "find (intransitive) to find one's way"] }\\
\end{tabular}
}
%===flechten===
\card{\normalfont \Huge flechten}{
\begin{tabular}{lll}
\parbox[t][][t]{2.0 cm}{\normalfont \raggedleft ich\\du\\er/sie/es\\wir\\ihr\\sie} &    
\parbox[t][][t]{2cm}{\normalfont flechte\\flichtst\\flicht\\flechten\\flechtet\\flechten} &
\parbox[t][][t]{2cm}{\normalfont flocht\\flochtest\\flocht\\flochten\\flochtet\\flochten}\\
\end{tabular}
\begin{tabular}{l}
\parbox[t][][t]{8cm}{}\\
\parbox[t][][t]{8cm}{\normalfont \small ['(transitive) to plait; to braid; to weave'] }\\
\end{tabular}
}
%===fliegen===
\card{\normalfont \Huge fliegen}{
\begin{tabular}{lll}
\parbox[t][][t]{2.0 cm}{\normalfont \raggedleft ich\\du\\er/sie/es\\wir\\ihr\\sie} &    
\parbox[t][][t]{2cm}{\normalfont fliege\\fliegst\\fliegt\\fliegen\\fliegt\\fliegen} &
\parbox[t][][t]{2cm}{\normalfont flog\\flogst\\flog\\flogen\\flogt\\flogen}\\
\end{tabular}
\begin{tabular}{l}
\parbox[t][][t]{8cm}{}\\
\parbox[t][][t]{8cm}{\normalfont \small ['(intransitive, auxiliary sein) to fly; to travel', 'by air 2010, Der Spiegel, issue 52/2010, page 16:', 'Passagiere, die aus den USA nach Europa fliegen', 'und dort umsteigen, sollen ab dem 1. April im', 'Transitbereich nicht mehr kontrolliert werden. It', 'is planned that passengers who fly from the United', 'States to Europe and change planes there are not', 'checked in the transit area anymore after April 1.', '(intransitive, figuratively, auxiliary sein) to', 'rush; to fly; to go quickly (transitive, auxiliary', 'haben) to fly; to pilot (transitive, auxiliary', 'haben) to transport by air (intransitive,', 'colloquial, auxiliary sein) to get the axe, to get', 'kicked out (intransitive, colloquial, auxiliary', 'sein) to fall; to fall down'] }\\
\end{tabular}
}
%===fliehen===
\card{\normalfont \Huge fliehen}{
\begin{tabular}{lll}
\parbox[t][][t]{2.0 cm}{\normalfont \raggedleft ich\\du\\er/sie/es\\wir\\ihr\\sie} &    
\parbox[t][][t]{2cm}{\normalfont fliehe\\fliehst\\flieht\\fliehen\\flieht\\fliehen} &
\parbox[t][][t]{2cm}{\normalfont floh\\flohst\\floh\\flohen\\floht\\flohen}\\
\end{tabular}
\begin{tabular}{l}
\parbox[t][][t]{8cm}{}\\
\parbox[t][][t]{8cm}{\normalfont \small ['(intransitive, auxiliary: "sein") to flee; to', 'escape (intransitive, auxiliary: "sein") to', 'diverge Die Linien fliehen.  "The lines diverge."', '(transitive, auxiliary: "haben") to flee from', '(someone); to avoid'] }\\
\end{tabular}
}
%===fließen===
\card{\normalfont \Huge fließen}{
\begin{tabular}{lll}
\parbox[t][][t]{2.0 cm}{\normalfont \raggedleft ich\\du\\er/sie/es\\wir\\ihr\\sie} &    
\parbox[t][][t]{2cm}{\normalfont fließe\\fließt\\fließt\\fließen\\fließt\\fließen} &
\parbox[t][][t]{2cm}{\normalfont floss\\flosst\\floss\\flossen\\flosst\\flossen}\\
\end{tabular}
\begin{tabular}{l}
\parbox[t][][t]{8cm}{}\\
\parbox[t][][t]{8cm}{\normalfont \small ['(intransitive) to flow'] }\\
\end{tabular}
}
%===fluchen===
\card{\normalfont \Huge fluchen}{
\begin{tabular}{lll}
\parbox[t][][t]{2.0 cm}{\normalfont \raggedleft ich\\du\\er/sie/es\\wir\\ihr\\sie} &    
\parbox[t][][t]{2cm}{\normalfont fluche\\fluchst\\flucht\\fluchen\\flucht\\fluchen} &
\parbox[t][][t]{2cm}{\normalfont fluchte\\fluchtest\\fluchte\\fluchten\\fluchtet\\fluchten}\\
\end{tabular}
\begin{tabular}{l}
\parbox[t][][t]{8cm}{}\\
\parbox[t][][t]{8cm}{\normalfont \small ['(intransitive) to swear, to curse'] }\\
\end{tabular}
}
%===folgen===
\card{\normalfont \Huge folgen}{
\begin{tabular}{lll}
\parbox[t][][t]{2.0 cm}{\normalfont \raggedleft ich\\du\\er/sie/es\\wir\\ihr\\sie} &    
\parbox[t][][t]{2cm}{\normalfont folge\\folgst\\folgt\\folgen\\folgt\\folgen} &
\parbox[t][][t]{2cm}{\normalfont folgte\\folgtest\\folgte\\folgten\\folgtet\\folgten}\\
\end{tabular}
\begin{tabular}{l}
\parbox[t][][t]{8cm}{}\\
\parbox[t][][t]{8cm}{\normalfont \small ['(with dative) to follow'] }\\
\end{tabular}
}
%===fordern===
\card{\normalfont \Huge fordern}{
\begin{tabular}{lll}
\parbox[t][][t]{2.0 cm}{\normalfont \raggedleft ich\\du\\er/sie/es\\wir\\ihr\\sie} &    
\parbox[t][][t]{2cm}{\normalfont fordreich fordereich forder\\forderst\\fordert\\fordern\\fordert\\fordern} &
\parbox[t][][t]{2cm}{\normalfont forderte\\fordertest\\forderte\\forderten\\fordertet\\forderten}\\
\end{tabular}
\begin{tabular}{l}
\parbox[t][][t]{8cm}{}\\
\parbox[t][][t]{8cm}{\normalfont \small ['(transitive) to demand, ask (transitive) to claim', '(transitive) to require (transitive, figuratively)', 'to challenge'] }\\
\end{tabular}
}
%===fressen===
\card{\normalfont \Huge fressen}{
\begin{tabular}{lll}
\parbox[t][][t]{2.0 cm}{\normalfont \raggedleft ich\\du\\er/sie/es\\wir\\ihr\\sie} &    
\parbox[t][][t]{2cm}{\normalfont fresse\\frisst\\frisst\\fressen\\fresst\\fressen} &
\parbox[t][][t]{2cm}{\normalfont fraß\\fraßt\\fraß\\fraßen\\fraßt\\fraßen}\\
\end{tabular}
\begin{tabular}{l}
\parbox[t][][t]{8cm}{}\\
\parbox[t][][t]{8cm}{\normalfont \small ['(transitive or intransitive, of an animal) to eat;', 'to feed on; to devour (transitive or intransitive,', 'of a person, derogatory) to stuff oneself; to', 'gorge oneself; to eat like a pig Erst kommt das', 'Fressen, dann kommt die Moral.First comes the', 'stomach, then comes ethics. (figuratively, chiefly', 'reflexive  with durch) to eat away (e.g. metal)', '(figuratively, transitive) to consume, to guzzle,', 'to burn (e.g. fuel, money) (transitive,', 'colloquial, perfect only) to despise, to have a', 'pet peeve against Den hab ich ja gefressen!  I', 'can\'t stand that guy! (literally, "I have eaten', 'that one.")'] }\\
\end{tabular}
}
%===freuen===
\card{\normalfont \Huge freuen}{
\begin{tabular}{lll}
\parbox[t][][t]{2.0 cm}{\normalfont \raggedleft ich\\du\\er/sie/es\\wir\\ihr\\sie} &    
\parbox[t][][t]{2cm}{\normalfont freue\\freust\\freut\\freuen\\freut\\freuen} &
\parbox[t][][t]{2cm}{\normalfont freute\\freutest\\freute\\freuten\\freutet\\freuten}\\
\end{tabular}
\begin{tabular}{l}
\parbox[t][][t]{8cm}{}\\
\parbox[t][][t]{8cm}{\normalfont \small ['(transitive) to gladden, make glad (reflexive) to', 'be glad (about something: ber + accusative)', '(reflexive) to look forward (to something: auf +', 'accusative)'] }\\
\end{tabular}
}
%===frieren===
\card{\normalfont \Huge frieren}{
\begin{tabular}{lll}
\parbox[t][][t]{2.0 cm}{\normalfont \raggedleft ich\\du\\er/sie/es\\wir\\ihr\\sie} &    
\parbox[t][][t]{2cm}{\normalfont friere\\frierst\\friert\\frieren\\friert\\frieren} &
\parbox[t][][t]{2cm}{\normalfont fror\\frorst\\fror\\froren\\frort\\froren}\\
\end{tabular}
\begin{tabular}{l}
\parbox[t][][t]{8cm}{}\\
\parbox[t][][t]{8cm}{\normalfont \small ['(intransitive, of a liquid, auxiliary: "sein") to', 'freeze (intransitive, of a person, auxiliary:', '"haben") to feel cold (intransitive, impersonal,', 'auxiliary: "haben") to be freezing Es friert.', "It's freezing."] }\\
\end{tabular}
}
%===frühstücken===
\card{\normalfont \Huge frühstücken}{
\begin{tabular}{lll}
\parbox[t][][t]{2.0 cm}{\normalfont \raggedleft ich\\du\\er/sie/es\\wir\\ihr\\sie} &    
\parbox[t][][t]{2cm}{\normalfont frühstücke\\frühstückst\\frühstückt\\frühstücken\\frühstückt\\frühstücken} &
\parbox[t][][t]{2cm}{\normalfont frühstückte\\frühstücktest\\frühstückte\\frühstückten\\frühstücktet\\frühstückten}\\
\end{tabular}
\begin{tabular}{l}
\parbox[t][][t]{8cm}{}\\
\parbox[t][][t]{8cm}{\normalfont \small ['(intransitive) to have breakfast; breakfast (to', 'eat the morning meal) (transitive) to breakfast', "on; have for breakfast (to eat as one's morning", 'meal)'] }\\
\end{tabular}
}
%===fühlen===
\card{\normalfont \Huge fühlen}{
\begin{tabular}{lll}
\parbox[t][][t]{2.0 cm}{\normalfont \raggedleft ich\\du\\er/sie/es\\wir\\ihr\\sie} &    
\parbox[t][][t]{2cm}{\normalfont fühle\\fühlst\\fühlt\\fühlen\\fühlt\\fühlen} &
\parbox[t][][t]{2cm}{\normalfont fühlte\\fühltest\\fühlte\\fühlten\\fühltet\\fühlten}\\
\end{tabular}
\begin{tabular}{l}
\parbox[t][][t]{8cm}{}\\
\parbox[t][][t]{8cm}{\normalfont \small ['(transitive) to feel (reflexive) to feel 1931,', 'Arthur Schnitzler, Flucht in die Finsternis, S.', 'Fischer Verlag, page 141: "Vielleicht bin ich', 'sogar verrckt. Ich will es nicht in Abrede', 'stellen. Aber wenn ich es bin, so fhle ich mich', 'sehr wohl dabei. Und das ist doch die Hauptsache,', 'nicht?" "Maybe I am even crazy. I don\'t want to', 'deny it. But if I am, I am feeling very good with', "it. And that is the most important thing, isn't", 'it?"'] }\\
\end{tabular}
}
%===führen===
\card{\normalfont \Huge führen}{
\begin{tabular}{lll}
\parbox[t][][t]{2.0 cm}{\normalfont \raggedleft ich\\du\\er/sie/es\\wir\\ihr\\sie} &    
\parbox[t][][t]{2cm}{\normalfont führe\\führst\\führt\\führen\\führt\\führen} &
\parbox[t][][t]{2cm}{\normalfont führte\\führtest\\führte\\führten\\führtet\\führten}\\
\end{tabular}
\begin{tabular}{l}
\parbox[t][][t]{8cm}{}\\
\parbox[t][][t]{8cm}{\normalfont \small ['(transitive) to lead (formal) to carry, to sell', "Der Laden fhrt keine Zigaretten.The shop doesn't", 'carry cigarettes. (of a discussion, negotiation,', 'etc.) to hold, to have 1919, Walther Kabel,', 'Irrende Seelen, Werner Dietsch Verlag, page 108:', 'Unsere Unterredung wurde jetzt im leichten', 'Plauderton gefhrt wie ein harmloses Gesprch unter', 'guten Bekannten. Our discussion was now held in a', 'light conversational tone like a harmless chat', 'between friendly acquaintances.'] }\\
\end{tabular}
}
%===füllen===
\card{\normalfont \Huge füllen}{
\begin{tabular}{lll}
\parbox[t][][t]{2.0 cm}{\normalfont \raggedleft ich\\du\\er/sie/es\\wir\\ihr\\sie} &    
\parbox[t][][t]{2cm}{\normalfont fülle\\füllst\\füllt\\füllen\\füllt\\füllen} &
\parbox[t][][t]{2cm}{\normalfont füllte\\fülltest\\füllte\\füllten\\fülltet\\füllten}\\
\end{tabular}
\begin{tabular}{l}
\parbox[t][][t]{8cm}{}\\
\parbox[t][][t]{8cm}{\normalfont \small ['to fill (cooking) to stuff'] }\\
\end{tabular}
}
%===funktionieren===
\card{\normalfont \Huge funktionieren}{
\begin{tabular}{lll}
\parbox[t][][t]{2.0 cm}{\normalfont \raggedleft ich\\du\\er/sie/es\\wir\\ihr\\sie} &    
\parbox[t][][t]{2cm}{\normalfont funktioniere\\funktionierst\\funktioniert\\funktionieren\\funktioniert\\funktionieren} &
\parbox[t][][t]{2cm}{\normalfont funktionierte\\funktioniertest\\funktionierte\\funktionierten\\funktioniertet\\funktionierten}\\
\end{tabular}
\begin{tabular}{l}
\parbox[t][][t]{8cm}{}\\
\parbox[t][][t]{8cm}{\normalfont \small ['to work, function'] }\\
\end{tabular}
}
%===fürchten===
\card{\normalfont \Huge fürchten}{
\begin{tabular}{lll}
\parbox[t][][t]{2.0 cm}{\normalfont \raggedleft ich\\du\\er/sie/es\\wir\\ihr\\sie} &    
\parbox[t][][t]{2cm}{\normalfont fürchte\\fürchtest\\fürchtet\\fürchten\\fürchtet\\fürchten} &
\parbox[t][][t]{2cm}{\normalfont fürchtete\\fürchtetest\\fürchtete\\fürchteten\\fürchtetet\\fürchteten}\\
\end{tabular}
\begin{tabular}{l}
\parbox[t][][t]{8cm}{}\\
\parbox[t][][t]{8cm}{\normalfont \small ['(transitive) to fear (reflexive) to be afraid'] }\\
\end{tabular}
}
%===gären===
\card{\normalfont \Huge gären}{
\begin{tabular}{lll}
\parbox[t][][t]{2.0 cm}{\normalfont \raggedleft ich\\du\\er/sie/es\\wir\\ihr\\sie} &    
\parbox[t][][t]{2cm}{\normalfont gäre\\gärst\\gärt\\gären\\gärt\\gären} &
\parbox[t][][t]{2cm}{\normalfont gor\\gorst\\gor\\goren\\gort\\goren}\\
\end{tabular}
\begin{tabular}{l}
\parbox[t][][t]{8cm}{}\\
\parbox[t][][t]{8cm}{\normalfont \small ['(intransitive) to ferment'] }\\
\end{tabular}
}
%===geben===
\card{\normalfont \Huge geben}{
\begin{tabular}{lll}
\parbox[t][][t]{2.0 cm}{\normalfont \raggedleft ich\\du\\er/sie/es\\wir\\ihr\\sie} &    
\parbox[t][][t]{2cm}{\normalfont gebe\\gibst\\gibt\\geben\\gebt\\geben} &
\parbox[t][][t]{2cm}{\normalfont gab\\gabst\\gab\\gaben\\gabt\\gaben}\\
\end{tabular}
\begin{tabular}{l}
\parbox[t][][t]{8cm}{}\\
\parbox[t][][t]{8cm}{\normalfont \small ['(transitive) To give; to hand Gib mir das!  Give', 'me that! (transitive) To present; to put', '(impersonal, transitive) there to be (there is;', 'there are); indicates that something exists. 2000,', 'Eurobarometer: Public Opinion in the European', 'Union, ISBN, Page 8: Es gibt eine europische', 'kulturelle Identitt, die von allen Europern', 'geteilt wird. "There is a European cultural', 'identity, which is shared by all Europeans."', '(transitive) To result in'] }\\
\end{tabular}
}
%===gebrauchen===
\card{\normalfont \Huge gebrauchen}{
\begin{tabular}{lll}
\parbox[t][][t]{2.0 cm}{\normalfont \raggedleft ich\\du\\er/sie/es\\wir\\ihr\\sie} &    
\parbox[t][][t]{2cm}{\normalfont gebrauche\\gebrauchst\\gebraucht\\gebrauchen\\gebraucht\\gebrauchen} &
\parbox[t][][t]{2cm}{\normalfont gebrauchte\\gebrauchtest\\gebrauchte\\gebrauchten\\gebrauchtet\\gebrauchten}\\
\end{tabular}
\begin{tabular}{l}
\parbox[t][][t]{8cm}{}\\
\parbox[t][][t]{8cm}{\normalfont \small ['to use'] }\\
\end{tabular}
}
%===gedeihen===
\card{\normalfont \Huge gedeihen}{
\begin{tabular}{lll}
\parbox[t][][t]{2.0 cm}{\normalfont \raggedleft ich\\du\\er/sie/es\\wir\\ihr\\sie} &    
\parbox[t][][t]{2cm}{\normalfont gedeihe\\gedeihst\\gedeiht\\gedeihen\\gedeiht\\gedeihen} &
\parbox[t][][t]{2cm}{\normalfont gedieh\\gediehst\\gedieh\\gediehen\\gedieht\\gediehen}\\
\end{tabular}
\begin{tabular}{l}
\parbox[t][][t]{8cm}{}\\
\parbox[t][][t]{8cm}{\normalfont \small ['(intransitive) to thrive; to flourish; to prosper'] }\\
\end{tabular}
}
%===gehen===
\card{\normalfont \Huge gehen}{
\begin{tabular}{lll}
\parbox[t][][t]{2.0 cm}{\normalfont \raggedleft ich\\du\\er/sie/es\\wir\\ihr\\sie} &    
\parbox[t][][t]{2cm}{\normalfont gehe\\gehst\\geht\\gehen\\geht\\gehen} &
\parbox[t][][t]{2cm}{\normalfont ging\\gingst\\ging\\gingen\\gingt\\gingen}\\
\end{tabular}
\begin{tabular}{l}
\parbox[t][][t]{8cm}{}\\
\parbox[t][][t]{8cm}{\normalfont \small ['(intransitive) to go, to walk (transitive) to walk', '(some distance); to go (some distance) by foot', "(intransitive) to leave Ich gehe jetzt.  I'm", 'leaving now. (intransitive) to leave, to take off', '(airplane, train) Wann geht dein Zug?When is your', 'train leaving? (impersonal, intransitive) to be', 'going; to be alright; indicates how the dative', 'object fares Wie geht es dir?  How are you doing?', 'Es geht mir gut.  I\'m doing well. (Literally, "It', 'goes well for me.") Es geht.  It\'s alright.', '(slightly informal, intransitive) to be possible', 'Das wrde vielleicht gehen.  That might be', 'possible. (colloquial, intransitive) to work, to', 'function (of a machine, method or the like)', 'Synonym: funktionieren Der Kaffeeautomat geht', "nicht.  The coffee dispenser doesn't work. 2014,", 'Der Spiegel, issue 21/2014, page 62: Aber erst in', 'Beirut lernte sie, wie professionelles Kochen', 'geht, die Logistik, das Timing, die Organisation,', 'um mehrere Hundert Mahlzeiten zuzubereiten. But', 'not until Beirut she learned how professional', 'cooking works, the logistics, the timing, the', 'organization for preparing several hundred meals.', '(colloquial, intransitive) to be in progress; to', 'last Die Sitzung geht bis ein Uhr.  The session is', "scheduled until one o'clock. (regional or dated,", 'impersonal, intransitive, with auf followed by a', 'time) to approach; to be going (on some one) Es', "geht auf 8 Uhr.  It's going on 8 o'clock."] }\\
\end{tabular}
}
%===gehören===
\card{\normalfont \Huge gehören}{
\begin{tabular}{lll}
\parbox[t][][t]{2.0 cm}{\normalfont \raggedleft ich\\du\\er/sie/es\\wir\\ihr\\sie} &    
\parbox[t][][t]{2cm}{\normalfont gehöre\\gehörst\\gehört\\gehören\\gehört\\gehören} &
\parbox[t][][t]{2cm}{\normalfont gehörte\\gehörtest\\gehörte\\gehörten\\gehörtet\\gehörten}\\
\end{tabular}
\begin{tabular}{l}
\parbox[t][][t]{8cm}{}\\
\parbox[t][][t]{8cm}{\normalfont \small ['to belong (zu to) Das Buch gehrt ihm.  The book', 'belongs to him. Ich gehre zur Organisation dazu.I', 'belong to the organization. Es gehrt sehr viel', 'Selbstvertrauen dazu um so etwas zu tun.To do', 'something like this needs a lot of self-', 'confidence. (reflexive) to be proper Wie es sich', 'gehrt.  As is right and proper. Das gehrt sich', "nicht.  That's just not done."] }\\
\end{tabular}
}
%===gelten===
\card{\normalfont \Huge gelten}{
\begin{tabular}{lll}
\parbox[t][][t]{2.0 cm}{\normalfont \raggedleft ich\\du\\er/sie/es\\wir\\ihr\\sie} &    
\parbox[t][][t]{2cm}{\normalfont gelte\\giltst\\gilt\\gelten\\geltet\\gelten} &
\parbox[t][][t]{2cm}{\normalfont galt\\galtest\\galt\\galten\\galtet\\galten}\\
\end{tabular}
\begin{tabular}{l}
\parbox[t][][t]{8cm}{}\\
\parbox[t][][t]{8cm}{\normalfont \small ['(intransitive) to be valid; to count; to hold true', '(intransitive) to be effective; to be in force', '2010, Der Spiegel, issue 21/2010, page 37: Doch', 'das Gesetz der Demokratie gilt nur zwischen den', 'Brgern und ihrem Staat. In der Auenpolitik gilt', 'traditionell das Primat der Regierung. But the law', 'of democracy is only in force between the citizens', 'and their state. In foreign policy the primacy of', 'the government is traditionally in force.', '(transitive) to be worth Kurt Honolka (tr.), "Das', 'Pfand der Liebe", in Antonn Dvok, Sechs Klnge aus', 'Mahren. Was gilt uns das Wort meiner Mutter? /', 'Meine Mutter, die beherrscht uns nicht!What does', "my mother's word matter to us? / My mother doesn't", 'own us. (intransitive, formal) to be for/of (+', "dative) 1902, Stefan Zweig, 'Die Wanderung': Sein", 'erster Gedanke galt der Zeit, und er sprang rasch', 'vom Lager, um nach der Sonne zu sehen. His first', 'thought was of the time, and he sprang quickly out', 'of bed to see the sun. (intransitive, with "als"', 'or "fr") to be regarded (as something); to pass', '(for something) (impersonal) to be necessary 2010,', 'Der Spiegel, issue 25/2010, page 77: Es gilt', 'deshalb, die richtigen Lehren aus der Krise zu', 'ziehen, aus den Fehlern der Vergangenheit zu', 'lernen, um die Zukunft zu sichern. Therefore it is', 'necessary to draw the right lessons from the', 'crisis, to learn from the mistakes of the past for', 'securing the future. (intransitive, with "lassen")', 'to pass; to go; (Often translated with lassen as a', 'unit into allow, endure, permit, accept, etc.) Das', 'will ich ausnahmsweise mal gelten lassen.I will', 'let that pass [allow, concede, endure, permit,', 'accept that] as an exception.'] }\\
\end{tabular}
}
%===genesen===
\card{\normalfont \Huge genesen}{
\begin{tabular}{lll}
\parbox[t][][t]{2.0 cm}{\normalfont \raggedleft ich\\du\\er/sie/es\\wir\\ihr\\sie} &    
\parbox[t][][t]{2cm}{\normalfont genese\\genest\\genest\\genesen\\genest\\genesen} &
\parbox[t][][t]{2cm}{\normalfont genas\\genast\\genas\\genasen\\genast\\genasen}\\
\end{tabular}
\begin{tabular}{l}
\parbox[t][][t]{8cm}{}\\
\parbox[t][][t]{8cm}{\normalfont \small ['(intransitive) to recover; to recuperate nach', 'langer Krankheit genesen - recovered after a long', 'illness'] }\\
\end{tabular}
}
%===genießen===
\card{\normalfont \Huge genießen}{
\begin{tabular}{lll}
\parbox[t][][t]{2.0 cm}{\normalfont \raggedleft ich\\du\\er/sie/es\\wir\\ihr\\sie} &    
\parbox[t][][t]{2cm}{\normalfont genieße\\genießt\\genießt\\genießen\\genießt\\genießen} &
\parbox[t][][t]{2cm}{\normalfont genoss\\genosst\\genoss\\genossen\\genosst\\genossen}\\
\end{tabular}
\begin{tabular}{l}
\parbox[t][][t]{8cm}{}\\
\parbox[t][][t]{8cm}{\normalfont \small ['(transitive) to enjoy (an experience); to relish;', 'to savor das Leben genieen  "to enjoy life"', '(transitive, formal) to have (food or beverage),', 'to eat, to drink (figuratively) to receive; to', 'have eine Erziehung genieen  "to receive an', 'education"'] }\\
\end{tabular}
}
%===geraten===
\card{\normalfont \Huge geraten}{
\begin{tabular}{lll}
\parbox[t][][t]{2.0 cm}{\normalfont \raggedleft ich\\du\\er/sie/es\\wir\\ihr\\sie} &    
\parbox[t][][t]{2cm}{\normalfont gerate\\gerätst\\gerät\\geraten\\geratet\\geraten} &
\parbox[t][][t]{2cm}{\normalfont geriet\\gerietest\\geriet\\gerieten\\gerietet\\gerieten}\\
\end{tabular}
\begin{tabular}{l}
\parbox[t][][t]{8cm}{}\\
\parbox[t][][t]{8cm}{\normalfont \small ['(intransitive) to turn out, succeed (intransitive)', 'to thrive (intransitive, with "an ..." or "auf', '..." and an accusative noun) to come (across, by);', 'to fall (upon) (intransitive, with "in ..." or', '"unter ..." and a state or condition in the', 'accusative) to get (into); to fall (into); to come', '(to); to fly (into) 2010, Der Spiegel, issue', '45/2010, page 90: Drei Jahre nach Ausbruch der', 'Finanzkrise mehren sich die Spannungen zwischen', 'den Industrienationen, der freie Welthandel gert', 'unter Druck. Three years after the outbreak of the', 'financial crisis, the tensions between the', 'industrial nations are increasing, global free', 'trade is coming under pressure.'] }\\
\end{tabular}
}
%===geschehen===
\card{\normalfont \Huge geschehen}{
\begin{tabular}{lll}
\parbox[t][][t]{2.0 cm}{\normalfont \raggedleft ich\\du\\er/sie/es\\wir\\ihr\\sie} &    
\parbox[t][][t]{2cm}{\normalfont geschehe\\geschiehst\\geschieht\\geschehen\\gescheht\\geschehen} &
\parbox[t][][t]{2cm}{\normalfont geschah\\geschahst\\geschah\\geschahen\\geschaht\\geschahen}\\
\end{tabular}
\begin{tabular}{l}
\parbox[t][][t]{8cm}{}\\
\parbox[t][][t]{8cm}{\normalfont \small ['(intransitive) to occur; to happen Wie konnte das', 'nur geschehen?How could that have happened? Dein', 'Wille geschehe, wie im Himmel so auf Erden.Thy', "will be done on earth as it is in Heaven. (Lord's", 'Prayer) (with dative or with "mit ...",', 'impersonal) to happen (to someone); to serve', '(someone) Es geschieht ihm recht.  It serves him', 'right.'] }\\
\end{tabular}
}
%===gewinnen===
\card{\normalfont \Huge gewinnen}{
\begin{tabular}{lll}
\parbox[t][][t]{2.0 cm}{\normalfont \raggedleft ich\\du\\er/sie/es\\wir\\ihr\\sie} &    
\parbox[t][][t]{2cm}{\normalfont gewinne\\gewinnst\\gewinnt\\gewinnen\\gewinnt\\gewinnen} &
\parbox[t][][t]{2cm}{\normalfont gewann\\gewannst\\gewann\\gewannen\\gewannt\\gewannen}\\
\end{tabular}
\begin{tabular}{l}
\parbox[t][][t]{8cm}{}\\
\parbox[t][][t]{8cm}{\normalfont \small ['(intransitive) to win; to be victorious', '(transitive) to win something; to gain', '(transitive) to win over; to persuade (transitive)', 'to win or extract a resource'] }\\
\end{tabular}
}
%===gewöhnen===
\card{\normalfont \Huge gewöhnen}{
\begin{tabular}{lll}
\parbox[t][][t]{2.0 cm}{\normalfont \raggedleft ich\\du\\er/sie/es\\wir\\ihr\\sie} &    
\parbox[t][][t]{2cm}{\normalfont gewöhne\\gewöhnst\\gewöhnt\\gewöhnen\\gewöhnt\\gewöhnen} &
\parbox[t][][t]{2cm}{\normalfont gewöhnte\\gewöhntest\\gewöhnte\\gewöhnten\\gewöhntet\\gewöhnten}\\
\end{tabular}
\begin{tabular}{l}
\parbox[t][][t]{8cm}{}\\
\parbox[t][][t]{8cm}{\normalfont \small ['(transitive with an) to get used to; to accustom', 'oneself to; to adapt to Ich habe mich daran', 'gewhnt.  I got used to it. Ich kann mich nicht', 'daran gewhnen.  I cannot get used to it.'] }\\
\end{tabular}
}
%===gießen===
\card{\normalfont \Huge gießen}{
\begin{tabular}{lll}
\parbox[t][][t]{2.0 cm}{\normalfont \raggedleft ich\\du\\er/sie/es\\wir\\ihr\\sie} &    
\parbox[t][][t]{2cm}{\normalfont gieße\\gießt\\gießt\\gießen\\gießt\\gießen} &
\parbox[t][][t]{2cm}{\normalfont goss\\gosst\\goss\\gossen\\gosst\\gossen}\\
\end{tabular}
\begin{tabular}{l}
\parbox[t][][t]{8cm}{}\\
\parbox[t][][t]{8cm}{\normalfont \small ['(transitive) to pour; usually only of liquids,', 'especially of large quantities Synonym: schtten', '(preferred with small quantities, also used of', 'solids) Das l wird in die Tanks gegossen.  The oil', 'is poured into the tanks. Bleigieen  lead-pouring', '(transitive) to pour; to cast; to found (shape', 'molten metal or glass by pouring) Die Statue wurde', 'vom Knstler selbst gegossen.  The statue was cast', 'by the artist himself. Stahlgieen  steel pouring', '(transitive, horticulture) to water die Blumen', 'gieen  to water the flowers Giekanne  watering can', '(impersonal, intransitive, of  rain) to pour down;', "to rain strongly Synonym: schtten Es giet.  It's", "pouring. Es giet wie aus Kbeln.  It's raining cats", 'and dogs. (literally, "It\'s pouring as though out', 'of buckets.")'] }\\
\end{tabular}
}
%===glauben===
\card{\normalfont \Huge glauben}{
\begin{tabular}{lll}
\parbox[t][][t]{2.0 cm}{\normalfont \raggedleft ich\\du\\er/sie/es\\wir\\ihr\\sie} &    
\parbox[t][][t]{2cm}{\normalfont glaube\\glaubst\\glaubt\\glauben\\glaubt\\glauben} &
\parbox[t][][t]{2cm}{\normalfont glaubte\\glaubtest\\glaubte\\glaubten\\glaubtet\\glaubten}\\
\end{tabular}
\begin{tabular}{l}
\parbox[t][][t]{8cm}{}\\
\parbox[t][][t]{8cm}{\normalfont \small ['to believe (to think someone/something exists = an', '+ acc.; to think something someone says is correct', '= dat.) Glaubst du an Engel? Do you believe in', 'angels? Niemand kann ihm glauben. No one can', 'believe him. Woran glaubst du? What do you believe', 'in? to think Glaubst du das ist so richtig? Do you', 'think this is right like this?'] }\\
\end{tabular}
}
%===gleichen===
\card{\normalfont \Huge gleichen}{
\begin{tabular}{lll}
\parbox[t][][t]{2.0 cm}{\normalfont \raggedleft ich\\du\\er/sie/es\\wir\\ihr\\sie} &    
\parbox[t][][t]{2cm}{\normalfont gleiche\\gleichst\\gleicht\\gleichen\\gleicht\\gleichen} &
\parbox[t][][t]{2cm}{\normalfont glich\\glichst\\glich\\glichen\\glicht\\glichen}\\
\end{tabular}
\begin{tabular}{l}
\parbox[t][][t]{8cm}{}\\
\parbox[t][][t]{8cm}{\normalfont \small ['(with a dative case object) to be like', '(something); to equal (something); to resemble', '(reflexive) to be alike'] }\\
\end{tabular}
}
%===gleiten===
\card{\normalfont \Huge gleiten}{
\begin{tabular}{lll}
\parbox[t][][t]{2.0 cm}{\normalfont \raggedleft ich\\du\\er/sie/es\\wir\\ihr\\sie} &    
\parbox[t][][t]{2cm}{\normalfont gleite\\gleitest\\gleitet\\gleiten\\gleitet\\gleiten} &
\parbox[t][][t]{2cm}{\normalfont glitt\\glittest\\glitt\\glitten\\glittet\\glitten}\\
\end{tabular}
\begin{tabular}{l}
\parbox[t][][t]{8cm}{}\\
\parbox[t][][t]{8cm}{\normalfont \small ['(intransitive) to glide; to float; to move', 'effortlessly (intransitive) to slide; to slip; to', 'transition smoothly'] }\\
\end{tabular}
}
%===glimmen===
\card{\normalfont \Huge glimmen}{
\begin{tabular}{lll}
\parbox[t][][t]{2.0 cm}{\normalfont \raggedleft ich\\du\\er/sie/es\\wir\\ihr\\sie} &    
\parbox[t][][t]{2cm}{\normalfont glimme\\glimmst\\glimmt\\glimmen\\glimmt\\glimmen} &
\parbox[t][][t]{2cm}{\normalfont glomm\\glommst\\glomm\\glommen\\glommt\\glommen}\\
\end{tabular}
\begin{tabular}{l}
\parbox[t][][t]{8cm}{}\\
\parbox[t][][t]{8cm}{\normalfont \small ['to shine to glow 1931, Arthur Schnitzler, Flucht', 'in die Finsternis, S. Fischer Verlag, page 158159:', 'Als in dem grnlichen Kachelofen nach einiger Zeit', 'die Holzscheite zu glimmen und zu knistern', 'begannen, setzte er sich, noch immer im Pelz, auf', 'den schwarzen, ans Bett gerckten breitlehnigen', 'Lederstuhl. When after some time the logs in the', 'greenish tiled stove started glowing and', 'crackling, he sat down, still in the fur, on the', 'black leather chair, which was moved next to the', 'bed.'] }\\
\end{tabular}
}
%===graben===
\card{\normalfont \Huge graben}{
\begin{tabular}{lll}
\parbox[t][][t]{2.0 cm}{\normalfont \raggedleft ich\\du\\er/sie/es\\wir\\ihr\\sie} &    
\parbox[t][][t]{2cm}{\normalfont grabe\\gräbst\\gräbt\\graben\\grabt\\graben} &
\parbox[t][][t]{2cm}{\normalfont grub\\grubst\\grub\\gruben\\grubt\\gruben}\\
\end{tabular}
\begin{tabular}{l}
\parbox[t][][t]{8cm}{}\\
\parbox[t][][t]{8cm}{\normalfont \small ['(transitive or intransitive) to dig (transitive,', 'intransitive or reflexive, of an animal) to burrow'] }\\
\end{tabular}
}
%===greifen===
\card{\normalfont \Huge greifen}{
\begin{tabular}{lll}
\parbox[t][][t]{2.0 cm}{\normalfont \raggedleft ich\\du\\er/sie/es\\wir\\ihr\\sie} &    
\parbox[t][][t]{2cm}{\normalfont greife\\greifst\\greift\\greifen\\greift\\greifen} &
\parbox[t][][t]{2cm}{\normalfont griff\\griffst\\griff\\griffen\\grifft\\griffen}\\
\end{tabular}
\begin{tabular}{l}
\parbox[t][][t]{8cm}{}\\
\parbox[t][][t]{8cm}{\normalfont \small ['(transitive) to grab; to grasp; to grip', '(something) (transitive) to grab; to seize; to', 'snatch (in an aggressive way) (intransitive) to', 'reach; to grab an etwas greifen  "to touch', 'something" (literally, "to grab onto something")', 'nach etwas greifen  "to reach for something" zu', 'etwas greifen  "to reach for something" in etwas', 'greifen  "to reach into something" (transitive) to', 'capture (someone) (transitive, music, chords) to', 'strike 1931, Arthur Schnitzler, Flucht in die', 'Finsternis, S. Fischer Verlag, page 14: Er begab', 'sich ins Klavierzimmer, griff ein paar Akkorde auf', 'dem verstimmten Flgel, verlie aber bald wieder den', 'Raum, [...] He went to the piano room, struck a', 'few chords on the out-of-tune grand piano, but', 'soon left the room again, [...] (intransitive) to', 'take hold; to bite Trotz des schlechten Wetters', 'griffen die Reifen hervorragend  The tires did', 'bite perfectly despite the bad weather.'] }\\
\end{tabular}
}
%===gründen===
\card{\normalfont \Huge gründen}{
\begin{tabular}{lll}
\parbox[t][][t]{2.0 cm}{\normalfont \raggedleft ich\\du\\er/sie/es\\wir\\ihr\\sie} &    
\parbox[t][][t]{2cm}{\normalfont gründe\\gründest\\gründet\\gründen\\gründet\\gründen} &
\parbox[t][][t]{2cm}{\normalfont gründete\\gründetest\\gründete\\gründeten\\gründetet\\gründeten}\\
\end{tabular}
\begin{tabular}{l}
\parbox[t][][t]{8cm}{}\\
\parbox[t][][t]{8cm}{\normalfont \small ['to build to establish to found to institute to', 'plant'] }\\
\end{tabular}
}
%===grüßen===
\card{\normalfont \Huge grüßen}{
\begin{tabular}{lll}
\parbox[t][][t]{2.0 cm}{\normalfont \raggedleft ich\\du\\er/sie/es\\wir\\ihr\\sie} &    
\parbox[t][][t]{2cm}{\normalfont grüße\\grüßt\\grüßt\\grüßen\\grüßt\\grüßen} &
\parbox[t][][t]{2cm}{\normalfont grüßte\\grüßtest\\grüßte\\grüßten\\grüßtet\\grüßten}\\
\end{tabular}
\begin{tabular}{l}
\parbox[t][][t]{8cm}{}\\
\parbox[t][][t]{8cm}{\normalfont \small ['to greet'] }\\
\end{tabular}
}
%===gucken===
\card{\normalfont \Huge gucken}{
\begin{tabular}{lll}
\parbox[t][][t]{2.0 cm}{\normalfont \raggedleft ich\\du\\er/sie/es\\wir\\ihr\\sie} &    
\parbox[t][][t]{2cm}{\normalfont gucke\\guckst\\guckt\\gucken\\guckt\\gucken} &
\parbox[t][][t]{2cm}{\normalfont guckte\\gucktest\\guckte\\guckten\\gucktet\\guckten}\\
\end{tabular}
\begin{tabular}{l}
\parbox[t][][t]{8cm}{}\\
\parbox[t][][t]{8cm}{\normalfont \small ["(colloquial) to look, to direct one's gaze at", 'something Synonyms: schauen, sehen, hinsehen,', 'betrachten, ansehen 1960, Marie Luise Kaschnitz,', "'Der Strohhalm': Ich kann in die Stadt gehen und", 'mich irgendwo hinsetzen und in die Luft gucken. I', 'can go into town and sit myself down somewhere and', 'stare into space. (colloquial) to watch, to direct', "one's gaze at something for some time (colloquial)", 'to look, to have a certain facial expression', 'Synonym: aussehen Warum guckst du so traurig?  Why', 'are you looking so sad?'] }\\
\end{tabular}
}
%===haben===
\card{\normalfont \Huge haben}{
\begin{tabular}{lll}
\parbox[t][][t]{2.0 cm}{\normalfont \raggedleft ich\\du\\er/sie/es\\wir\\ihr\\sie} &    
\parbox[t][][t]{2cm}{\normalfont habe\\hast\\hat\\haben\\habt\\haben} &
\parbox[t][][t]{2cm}{\normalfont hatte\\hattest\\hatte\\hatten\\hattet\\hatten}\\
\end{tabular}
\begin{tabular}{l}
\parbox[t][][t]{8cm}{}\\
\parbox[t][][t]{8cm}{\normalfont \small ['(auxiliary, with a past participle) to have (forms', 'the perfect and past perfect tense) (transitive)', 'to have; to own (to possess, have ownership of; to', 'possess a certain characteristic) (transitive) to', 'have; to hold (to contain within itself/oneself)', "Glaub und hab keine Angst.Believe and don't be", 'afraid or Believe and have no fear. (transitive)', 'to have, get (to obtain, acquire) (transitive) to', 'get (to receive) (transitive) to have (to be', 'scheduled to attend) (transitive) to have (to be', 'afflicted with, suffer from) (transitive, of units', 'of measure) to contain, be composed of, equal Ein', 'Meter hat 100 Zentimeter.There are 100 centimetres', 'in one metre.(literally, "One metre has 100', 'centimetres.") (impersonal, dialectal, with es)', 'there be, there is, there are Es hat zwei', 'Bcher.There are two books. (reflexive, colloquial)', "to make a fuss Hab dich nicht so!Don't make such a", 'fuss! (colloquial, with es and mit) to be occupied', "with, to like Ich hab's nich so mit Hunden.I'm not", 'a great fan of dogs.(literally, "I don\'t have it', 'that much with dogs.")'] }\\
\end{tabular}
}
%===halten===
\card{\normalfont \Huge halten}{
\begin{tabular}{lll}
\parbox[t][][t]{2.0 cm}{\normalfont \raggedleft ich\\du\\er/sie/es\\wir\\ihr\\sie} &    
\parbox[t][][t]{2cm}{\normalfont halte\\hältst\\hält\\halten\\haltet\\halten} &
\parbox[t][][t]{2cm}{\normalfont hielt\\hieltest\\hielt\\hielten\\hieltet\\hielten}\\
\end{tabular}
\begin{tabular}{l}
\parbox[t][][t]{8cm}{}\\
\parbox[t][][t]{8cm}{\normalfont \small ['(transitive) to hold (transitive) to stop; to', 'halt; to hold back Haltet den Dieb!  Stop the', 'thief! (transitive) to support; to hold up', '(transitive) to keep; to maintain; to hold Der', 'Wein hlt mich jung.Wine keeps me young.', '(transitive, of animals) to keep 2012 June 21, Die', 'Welt [1], page 22: In der Sahara haben', 'prhistorische Bewohner schon vor 7000 Jahren', 'Milchvieh gehalten.In the Sahara, prehistoric', 'inhabitants kept dairy cattle already 7000 years', 'ago. (intransitive) to hold; to keep; to stay', '(intransitive) to stop Der Zug hlt nicht in', "Wrzburg.The train doesn't stop in Wrzburg.", '(transitive, with preposition fr) to consider, to', 'regard 2010, Der Spiegel, issue 32/2010, page 45:', 'Wieder einmal hatte es erhitzte Debatten gegeben', 'zwischen Tierschtzern und solchen, die Stierkampf', 'fr ein schtzenswertes Kulturerbe Spaniens', 'halten.Once again there had been heated debates', 'between animal rights activists and ones who', 'consider bullfighting a cultural heritage of Spain', 'worth protecting. (reflexive, with preposition:', 'an) to adhere (to) 2010, Der Spiegel, issue', '46/2010, page 103: Die boomende Branche privater', 'Sicherheitsfirmen will sich knftig an strengere', 'Regeln in Krisenregionen haltenThe booming sector', 'of private security companies wants to adhere to', 'stricter rules in crisis regions in the future.', '(with preposition: von) expresses a positive or', 'negative opinion depending on quantity Ich halte', 'viel von Dieter.I like Dieter a lot. Ich halte', 'wenig von Tomaten.I do not like tomatoes very', 'much.'] }\\
\end{tabular}
}
%===handeln===
\card{\normalfont \Huge handeln}{
\begin{tabular}{lll}
\parbox[t][][t]{2.0 cm}{\normalfont \raggedleft ich\\du\\er/sie/es\\wir\\ihr\\sie} &    
\parbox[t][][t]{2cm}{\normalfont handleich handeleich handel\\handelst\\handelt\\handeln\\handelt\\handeln} &
\parbox[t][][t]{2cm}{\normalfont handelte\\handeltest\\handelte\\handelten\\handeltet\\handelten}\\
\end{tabular}
\begin{tabular}{l}
\parbox[t][][t]{8cm}{}\\
\parbox[t][][t]{8cm}{\normalfont \small ['(intransitive) to act; to take action Jeder ist', 'dafr verantwortlich, wie er handelt.Everybody is', 'responsible for the way they act. Wir mssen', 'handeln, bevor es zu spt ist.We must take action', 'before it is too late. (intransitive) to', 'negotiate; to bargain; to haggle Da lass ich nicht', "mit mir handeln.I won't bargain about this.", '(intransitive  with mit) to trade in; to deal; to', 'sell Er handelt mit Teppichen.  He trades in', 'carpets. (transitive, stock exchange  and the', 'like) to trade Eine Aktiengesellschaft ist eine', 'Firma, die an der Brse gehandelt wird.A stock', 'company is an enterprise that is traded in the', 'stock market. 2010, Der Spiegel, issue 46/2010,', 'page 95: In Toronto werden mehr als 1400', 'Bergbaubetriebe gehandelt, jede Woche kommen im', 'Schnitt drei Unternehmen hinzu. More than 1400', 'mining companies are traded in Toronto, three', 'companies are added each week on average.', '(transitive, figuratively) to tip as; to take into', 'consideration as being Er wird als der nchste', "Superstar gehandelt.He's being tipped as the next", 'superstar. (intransitive, with von, of a person,', 'formal, literary) to discuss; to deal with; to', 'write or speak about Der Autor handelt auf 32', 'Seiten von dieser Nebenschlichkeit.The author', 'discusses this marginal issue in 32 pages.', '(intransitive, with von, of a text, film, etc.) to', 'be about; to deal with Das Buch handelt von Hasen.', 'The book is about hares. (impersonal, reflexive,', 'with um) to be Es handelt sich um ein', 'Missverstndnis.It is a misunderstanding. Bei', 'diesen Tieren handelt es sich um Hasen.These', 'animals are hares.(literally, "In these animals it', 'deals itself about hares.")'] }\\
\end{tabular}
}
%===hängen===
\card{\normalfont \Huge hängen}{
\begin{tabular}{lll}
\parbox[t][][t]{2.0 cm}{\normalfont \raggedleft ich\\du\\er/sie/es\\wir\\ihr\\sie} &    
\parbox[t][][t]{2cm}{\normalfont hänge\\hängst\\hängt\\hängen\\hängt\\hängen} &
\parbox[t][][t]{2cm}{\normalfont hing\\hingst\\hing\\hingen\\hingt\\hingen}\\
\end{tabular}
\begin{tabular}{l}
\parbox[t][][t]{8cm}{}\\
\parbox[t][][t]{8cm}{\normalfont \small ['(intransitive) to hang; to be suspended Der Apfel', 'hngt am Baum. The apple hangs on the tree.', '(intransitive) to be attached to; to be fond of;', 'to be devoted to; to cling to Er hngt sehr an', 'seiner Schwester. He is very attached to his', 'sister. (intransitive, somewhat informal) to', 'depend on Es hngt alles daran, wie du dich', 'entscheidest. It all depends on what decision you', 'take. (transitive, colloquial, otherwise', 'proscribed) to hang something; to suspend Ich hab', "meine Jacke an die Garderobe gehangen. I've hung", 'my jacket on the hallstand.'] }\\
\end{tabular}
}
%===hassen===
\card{\normalfont \Huge hassen}{
\begin{tabular}{lll}
\parbox[t][][t]{2.0 cm}{\normalfont \raggedleft ich\\du\\er/sie/es\\wir\\ihr\\sie} &    
\parbox[t][][t]{2cm}{\normalfont hasse\\hasst\\hasst\\hassen\\hasst\\hassen} &
\parbox[t][][t]{2cm}{\normalfont hasste\\hasstest\\hasste\\hassten\\hasstet\\hassten}\\
\end{tabular}
\begin{tabular}{l}
\parbox[t][][t]{8cm}{}\\
\parbox[t][][t]{8cm}{\normalfont \small ['to hate'] }\\
\end{tabular}
}
%===heben===
\card{\normalfont \Huge heben}{
\begin{tabular}{lll}
\parbox[t][][t]{2.0 cm}{\normalfont \raggedleft ich\\du\\er/sie/es\\wir\\ihr\\sie} &    
\parbox[t][][t]{2cm}{\normalfont hebe\\hebst\\hebt\\heben\\hebt\\heben} &
\parbox[t][][t]{2cm}{\normalfont hob\\hobst\\hob\\hoben\\hobt\\hoben}\\
\end{tabular}
\begin{tabular}{l}
\parbox[t][][t]{8cm}{}\\
\parbox[t][][t]{8cm}{\normalfont \small ['(transitive) to lift; to raise (transitive) to', 'heave; to hoist (reflexive) to rise; to lift'] }\\
\end{tabular}
}
%===heiraten===
\card{\normalfont \Huge heiraten}{
\begin{tabular}{lll}
\parbox[t][][t]{2.0 cm}{\normalfont \raggedleft ich\\du\\er/sie/es\\wir\\ihr\\sie} &    
\parbox[t][][t]{2cm}{\normalfont heirate\\heiratest\\heiratet\\heiraten\\heiratet\\heiraten} &
\parbox[t][][t]{2cm}{\normalfont heiratete\\heiratetest\\heiratete\\heirateten\\heiratetet\\heirateten}\\
\end{tabular}
\begin{tabular}{l}
\parbox[t][][t]{8cm}{}\\
\parbox[t][][t]{8cm}{\normalfont \small ['to marry'] }\\
\end{tabular}
}
%===heißen===
\card{\normalfont \Huge heißen}{
\begin{tabular}{lll}
\parbox[t][][t]{2.0 cm}{\normalfont \raggedleft ich\\du\\er/sie/es\\wir\\ihr\\sie} &    
\parbox[t][][t]{2cm}{\normalfont heiße\\heißt\\heißt\\heißen\\heißt\\heißen} &
\parbox[t][][t]{2cm}{\normalfont hieß\\hießt\\hieß\\hießen\\hießt\\hießen}\\
\end{tabular}
\begin{tabular}{l}
\parbox[t][][t]{8cm}{}\\
\parbox[t][][t]{8cm}{\normalfont \small ['(intransitive) to have a name; to be named; to be', 'called; but implying that one "owns" this name,', 'not necessarily that one goes by it Wie heit', "du?What is your name? Ich heie ...  I'm called ...", 'Ich werde Lutz genannt, aber ich heie Ludger.They', 'call me Lutz, but my name is Ludger. (intransitive', 'or transitive) to mean; to have a meaning Das', 'heit, dass wir nur noch wenig Zeit haben.This', 'means that we have but little time left. Was [or:', 'Wie] heit "Auto" auf Englisch?What is \'Auto\' in', 'English? (impersonal) to say, to be said; to go,', 'run (like) Es heit, dass ...  It is said that ...;', 'They say that ... (transitive, archaic  except in', 'fixed expressions) to call (someone something) Er', 'hat mich einen Idioten geheien.He called me an', 'idiot. Ich heie Sie herzlich willkommen!I welcome', 'you cordially!(literally, "I call you cordially', 'welcome!") (transitive, archaic) to order, to', 'direct, to call to do something Sie hie ihn, nach', 'der Schule anzurufen.She told him to call after', 'school'] }\\
\end{tabular}
}
%===heizen===
\card{\normalfont \Huge heizen}{
\begin{tabular}{lll}
\parbox[t][][t]{2.0 cm}{\normalfont \raggedleft ich\\du\\er/sie/es\\wir\\ihr\\sie} &    
\parbox[t][][t]{2cm}{\normalfont heize\\heizt\\heizt\\heizen\\heizt\\heizen} &
\parbox[t][][t]{2cm}{\normalfont heizte\\heiztest\\heizte\\heizten\\heiztet\\heizten}\\
\end{tabular}
\begin{tabular}{l}
\parbox[t][][t]{8cm}{}\\
\parbox[t][][t]{8cm}{\normalfont \small ['to heat'] }\\
\end{tabular}
}
%===helfen===
\card{\normalfont \Huge helfen}{
\begin{tabular}{lll}
\parbox[t][][t]{2.0 cm}{\normalfont \raggedleft ich\\du\\er/sie/es\\wir\\ihr\\sie} &    
\parbox[t][][t]{2cm}{\normalfont helfe\\hilfst\\hilft\\helfen\\helft\\helfen} &
\parbox[t][][t]{2cm}{\normalfont half\\halfst\\half\\halfen\\halft\\halfen}\\
\end{tabular}
\begin{tabular}{l}
\parbox[t][][t]{8cm}{}\\
\parbox[t][][t]{8cm}{\normalfont \small ['to help (someone); to assist; to aid [+dative] Ich', 'habe ihnen bei der Reparatur des Wagens geholfen.I', 'helped them with the repair of the car. 1832,', 'Auswahl vorzglicher Predigten auf alle Sonn- und', 'Feiertage des Jahres, wie auch bei verschiedenen', 'Gelegenheiten und der heiligen Fastenzeit, von', 'einer Gesellschaft katholischer Geistlicher.', 'Erster Band, page 83: Er, der allen half, hlft', 'auch uns  [] He, who helped all, also helps us  []', '1845, Wm. Wittich, German tales for beginners,', 'arranged in a progressive order, London, page 130:', 'Was hlft das Aufschieben?Of what use is the', 'procrastination?'] }\\
\end{tabular}
}
%===herkommen===
\card{\normalfont \Huge herkommen}{
\begin{tabular}{lll}
\parbox[t][][t]{2.0 cm}{\normalfont \raggedleft ich\\du\\er/sie/es\\wir\\ihr\\sie} &    
\parbox[t][][t]{2cm}{\normalfont komme her\\kommst her\\kommt her\\kommen her\\kommt her\\kommen her} &
\parbox[t][][t]{2cm}{\normalfont kam her\\kamst her\\kam her\\kamen her\\kamt her\\kamen her}\\
\end{tabular}
\begin{tabular}{l}
\parbox[t][][t]{8cm}{}\\
\parbox[t][][t]{8cm}{\normalfont \small ['to come over, come here (towards the speaker) to', 'come from'] }\\
\end{tabular}
}
%===herstellen===
\card{\normalfont \Huge herstellen}{
\begin{tabular}{lll}
\parbox[t][][t]{2.0 cm}{\normalfont \raggedleft ich\\du\\er/sie/es\\wir\\ihr\\sie} &    
\parbox[t][][t]{2cm}{\normalfont stelle her\\stellst her\\stellt her\\stellen her\\stellt her\\stellen her} &
\parbox[t][][t]{2cm}{\normalfont stellte her\\stelltest her\\stellte her\\stellten her\\stelltet her\\stellten her}\\
\end{tabular}
\begin{tabular}{l}
\parbox[t][][t]{8cm}{}\\
\parbox[t][][t]{8cm}{\normalfont \small ['to produce, manufacture to establish, make,', 'install, connect'] }\\
\end{tabular}
}
%===hinterlassen===
\card{\normalfont \Huge hinterlassen}{
\begin{tabular}{lll}
\parbox[t][][t]{2.0 cm}{\normalfont \raggedleft ich\\du\\er/sie/es\\wir\\ihr\\sie} &    
\parbox[t][][t]{2cm}{\normalfont hinterlasse\\hinterlässt\\hinterlässt\\hinterlassen\\hinterlasst\\hinterlassen} &
\parbox[t][][t]{2cm}{\normalfont hinterließ\\hinterließt\\hinterließ\\hinterließen\\hinterließt\\hinterließen}\\
\end{tabular}
\begin{tabular}{l}
\parbox[t][][t]{8cm}{}\\
\parbox[t][][t]{8cm}{\normalfont \small ['to leave to leave behind 2010, Der Spiegel, issue', '5/2010, page 101: Ein 20-kpfiges Team von Top-', 'Bergsteigern aus Nepal will sich Ende April auf', 'den Weg in die Gipfelregion des Mount Everest', 'machen, um den Zivilisationsmll zu beseitigen, den', 'frhere Expeditionen dort hinterlassen haben. A', 'team of 20 top mountain climbers from Nepal wants', 'to head to the summit region of Mount Everest at', 'the end of April to clean up the waste of', 'civilization that earlier expeditions have left', 'behind there. to bequeath'] }\\
\end{tabular}
}
%===hoffen===
\card{\normalfont \Huge hoffen}{
\begin{tabular}{lll}
\parbox[t][][t]{2.0 cm}{\normalfont \raggedleft ich\\du\\er/sie/es\\wir\\ihr\\sie} &    
\parbox[t][][t]{2cm}{\normalfont hoffe\\hoffst\\hofft\\hoffen\\hofft\\hoffen} &
\parbox[t][][t]{2cm}{\normalfont hoffte\\hofftest\\hoffte\\hofften\\hofftet\\hofften}\\
\end{tabular}
\begin{tabular}{l}
\parbox[t][][t]{8cm}{}\\
\parbox[t][][t]{8cm}{\normalfont \small ['to hope'] }\\
\end{tabular}
}
%===holen===
\card{\normalfont \Huge holen}{
\begin{tabular}{lll}
\parbox[t][][t]{2.0 cm}{\normalfont \raggedleft ich\\du\\er/sie/es\\wir\\ihr\\sie} &    
\parbox[t][][t]{2cm}{\normalfont hole\\holst\\holt\\holen\\holt\\holen} &
\parbox[t][][t]{2cm}{\normalfont holte\\holtest\\holte\\holten\\holtet\\holten}\\
\end{tabular}
\begin{tabular}{l}
\parbox[t][][t]{8cm}{}\\
\parbox[t][][t]{8cm}{\normalfont \small ['(transitive) to (go) get, to fetch (to go', 'somewhere and take something) Hol noch einen', 'Stuhl!Go get another chair! (transitive, always', 'with reflexive dative, colloquial) to get (to', "acquire, buy) Ich hol mir morgen 'n neuen", "Fernseher.I'm getting a new TV tomorrow."] }\\
\end{tabular}
}
%===hören===
\card{\normalfont \Huge hören}{
\begin{tabular}{lll}
\parbox[t][][t]{2.0 cm}{\normalfont \raggedleft ich\\du\\er/sie/es\\wir\\ihr\\sie} &    
\parbox[t][][t]{2cm}{\normalfont höre\\hörst\\hört\\hören\\hört\\hören} &
\parbox[t][][t]{2cm}{\normalfont hörte\\hörtest\\hörte\\hörten\\hörtet\\hörten}\\
\end{tabular}
\begin{tabular}{l}
\parbox[t][][t]{8cm}{}\\
\parbox[t][][t]{8cm}{\normalfont \small ['(transitive or intransitive) to hear (to perceive', 'sounds (or a sound) through the ear) (transitive)', 'to listen to, pay attention to (to give (someone)', "one's attention) (transitive, of a lecture) to", 'attend, to go to, to sit in on (transitive, of a', 'radio signal) to get, to receive (intransitive) to', 'listen (to pay attention to a sound or speech; to', 'accept advice or obey instruction) (intransitive)', 'to hear (to receive information; to come to learn)', '(intransitive) to hear (to be contacted (by))'] }\\
\end{tabular}
}
%===informieren===
\card{\normalfont \Huge informieren}{
\begin{tabular}{lll}
\parbox[t][][t]{2.0 cm}{\normalfont \raggedleft ich\\du\\er/sie/es\\wir\\ihr\\sie} &    
\parbox[t][][t]{2cm}{\normalfont informiere\\informierst\\informiert\\informieren\\informiert\\informieren} &
\parbox[t][][t]{2cm}{\normalfont informierte\\informiertest\\informierte\\informierten\\informiertet\\informierten}\\
\end{tabular}
\begin{tabular}{l}
\parbox[t][][t]{8cm}{}\\
\parbox[t][][t]{8cm}{\normalfont \small ['(transitive) to inform, to update Er informierte', 'uns ber das bevorstehende Unwetter. (reflexive) to', 'be informed, to be updated Ich informiere mich in', 'Zeitungen ber die neuesten Trends.'] }\\
\end{tabular}
}
%===interessieren===
\card{\normalfont \Huge interessieren}{
\begin{tabular}{lll}
\parbox[t][][t]{2.0 cm}{\normalfont \raggedleft ich\\du\\er/sie/es\\wir\\ihr\\sie} &    
\parbox[t][][t]{2cm}{\normalfont interessiere\\interessierst\\interessiert\\interessieren\\interessiert\\interessieren} &
\parbox[t][][t]{2cm}{\normalfont interessierte\\interessiertest\\interessierte\\interessierten\\interessiertet\\interessierten}\\
\end{tabular}
\begin{tabular}{l}
\parbox[t][][t]{8cm}{}\\
\parbox[t][][t]{8cm}{\normalfont \small ['(transitive) to interest Theater interessiert mich', 'nicht. - Theatre does not interest me. (reflexive,', 'with fr) to be interested in Claudia interessiert', 'sich fr hollndische Architektur. - Claudia is', 'interested in Dutch architecture.'] }\\
\end{tabular}
}
%===kämpfen===
\card{\normalfont \Huge kämpfen}{
\begin{tabular}{lll}
\parbox[t][][t]{2.0 cm}{\normalfont \raggedleft ich\\du\\er/sie/es\\wir\\ihr\\sie} &    
\parbox[t][][t]{2cm}{\normalfont kämpfe\\kämpfst\\kämpft\\kämpfen\\kämpft\\kämpfen} &
\parbox[t][][t]{2cm}{\normalfont kämpfte\\kämpftest\\kämpfte\\kämpften\\kämpftet\\kämpften}\\
\end{tabular}
\begin{tabular}{l}
\parbox[t][][t]{8cm}{}\\
\parbox[t][][t]{8cm}{\normalfont \small ['to fight, to struggle Mit der Dummheit kmpfen', 'Gtter selbst vergebens.Against stupidity, the gods', 'themselves contend in vain.'] }\\
\end{tabular}
}
%===kaufen===
\card{\normalfont \Huge kaufen}{
\begin{tabular}{lll}
\parbox[t][][t]{2.0 cm}{\normalfont \raggedleft ich\\du\\er/sie/es\\wir\\ihr\\sie} &    
\parbox[t][][t]{2cm}{\normalfont kaufe\\kaufst\\kauft\\kaufen\\kauft\\kaufen} &
\parbox[t][][t]{2cm}{\normalfont kaufte\\kauftest\\kaufte\\kauften\\kauftet\\kauften}\\
\end{tabular}
\begin{tabular}{l}
\parbox[t][][t]{8cm}{}\\
\parbox[t][][t]{8cm}{\normalfont \small ['to buy Sie kauft ein Auto.She is buying a car. Ich', 'glaube, wir haben zu viel gekauft.I think we', 'bought too much.'] }\\
\end{tabular}
}
%===kehren===
\card{\normalfont \Huge kehren}{
\begin{tabular}{lll}
\parbox[t][][t]{2.0 cm}{\normalfont \raggedleft ich\\du\\er/sie/es\\wir\\ihr\\sie} &    
\parbox[t][][t]{2cm}{\normalfont kehre\\kehrst\\kehrt\\kehren\\kehrt\\kehren} &
\parbox[t][][t]{2cm}{\normalfont kehrte\\kehrtest\\kehrte\\kehrten\\kehrtet\\kehrten}\\
\end{tabular}
\begin{tabular}{l}
\parbox[t][][t]{8cm}{}\\
\parbox[t][][t]{8cm}{\normalfont \small ['to sweep'] }\\
\end{tabular}
}
%===keimen===
\card{\normalfont \Huge keimen}{
\begin{tabular}{lll}
\parbox[t][][t]{2.0 cm}{\normalfont \raggedleft ich\\du\\er/sie/es\\wir\\ihr\\sie} &    
\parbox[t][][t]{2cm}{\normalfont keime\\keimst\\keimt\\keimen\\keimt\\keimen} &
\parbox[t][][t]{2cm}{\normalfont keimte\\keimtest\\keimte\\keimten\\keimtet\\keimten}\\
\end{tabular}
\begin{tabular}{l}
\parbox[t][][t]{8cm}{}\\
\parbox[t][][t]{8cm}{\normalfont \small ['to sprout, germinate to stir (e.g. hope) to form', '(e.g. thought, decision) to awaken (e.g. love,', 'yearning)'] }\\
\end{tabular}
}
%===kennen===
\card{\normalfont \Huge kennen}{
\begin{tabular}{lll}
\parbox[t][][t]{2.0 cm}{\normalfont \raggedleft ich\\du\\er/sie/es\\wir\\ihr\\sie} &    
\parbox[t][][t]{2cm}{\normalfont kenne\\kennst\\kennt\\kennen\\kennt\\kennen} &
\parbox[t][][t]{2cm}{\normalfont kannte\\kanntest\\kannte\\kannten\\kanntet\\kannten}\\
\end{tabular}
\begin{tabular}{l}
\parbox[t][][t]{8cm}{}\\
\parbox[t][][t]{8cm}{\normalfont \small ['(transitive) to know; to be acquainted with; to be', 'familiar with'] }\\
\end{tabular}
}
%===klagen===
\card{\normalfont \Huge klagen}{
\begin{tabular}{lll}
\parbox[t][][t]{2.0 cm}{\normalfont \raggedleft ich\\du\\er/sie/es\\wir\\ihr\\sie} &    
\parbox[t][][t]{2cm}{\normalfont klage\\klagst\\klagt\\klagen\\klagt\\klagen} &
\parbox[t][][t]{2cm}{\normalfont klagte\\klagtest\\klagte\\klagten\\klagtet\\klagten}\\
\end{tabular}
\begin{tabular}{l}
\parbox[t][][t]{8cm}{}\\
\parbox[t][][t]{8cm}{\normalfont \small ['to complain (to express feelings of pain,', 'dissatisfaction, or resentment) to wail, lament', '(law) to sue'] }\\
\end{tabular}
}
%===klappen===
\card{\normalfont \Huge klappen}{
\begin{tabular}{lll}
\parbox[t][][t]{2.0 cm}{\normalfont \raggedleft ich\\du\\er/sie/es\\wir\\ihr\\sie} &    
\parbox[t][][t]{2cm}{\normalfont klappe\\klappst\\klappt\\klappen\\klappt\\klappen} &
\parbox[t][][t]{2cm}{\normalfont klappte\\klapptest\\klappte\\klappten\\klapptet\\klappten}\\
\end{tabular}
\begin{tabular}{l}
\parbox[t][][t]{8cm}{}\\
\parbox[t][][t]{8cm}{\normalfont \small ['(intransitive) to clap (make a soft clapping', 'sound, particularly of something being closed)', '(transitive) to fold; to flip; to bend (to close', 'or open a hinge) Synonyms: umbiegen, umschlagen', 'Zum Transport muss diese Liege geklappt werden.In', 'order to be transportable, this deckchair needs to', 'be folded. (intransitive, chiefly colloquial) to', 'work (out) Synonyms: funktionieren; fluppen', '(colloquial, regional) Wenn alles klappt, bin ich', "um sieben zu Hause.If everything works out, I'll", 'be home by seven.'] }\\
\end{tabular}
}
%===kleben===
\card{\normalfont \Huge kleben}{
\begin{tabular}{lll}
\parbox[t][][t]{2.0 cm}{\normalfont \raggedleft ich\\du\\er/sie/es\\wir\\ihr\\sie} &    
\parbox[t][][t]{2cm}{\normalfont klebe\\klebst\\klebt\\kleben\\klebt\\kleben} &
\parbox[t][][t]{2cm}{\normalfont klebte\\klebtest\\klebte\\klebten\\klebtet\\klebten}\\
\end{tabular}
\begin{tabular}{l}
\parbox[t][][t]{8cm}{}\\
\parbox[t][][t]{8cm}{\normalfont \small ['(transitive, with an + accusative) to glue (onto)', '(intransitive, with an + dative) to stick (to)', '(intransitive) to be sticky'] }\\
\end{tabular}
}
%===klettern===
\card{\normalfont \Huge klettern}{
\begin{tabular}{lll}
\parbox[t][][t]{2.0 cm}{\normalfont \raggedleft ich\\du\\er/sie/es\\wir\\ihr\\sie} &    
\parbox[t][][t]{2cm}{\normalfont klettreich klettereich kletter\\kletterst\\klettert\\klettern\\klettert\\klettern} &
\parbox[t][][t]{2cm}{\normalfont kletterte\\klettertest\\kletterte\\kletterten\\klettertet\\kletterten}\\
\end{tabular}
\begin{tabular}{l}
\parbox[t][][t]{8cm}{}\\
\parbox[t][][t]{8cm}{\normalfont \small ['to climb'] }\\
\end{tabular}
}
%===klingen===
\card{\normalfont \Huge klingen}{
\begin{tabular}{lll}
\parbox[t][][t]{2.0 cm}{\normalfont \raggedleft ich\\du\\er/sie/es\\wir\\ihr\\sie} &    
\parbox[t][][t]{2cm}{\normalfont klinge\\klingst\\klingt\\klingen\\klingt\\klingen} &
\parbox[t][][t]{2cm}{\normalfont klang\\klangst\\klang\\klangen\\klangt\\klangen}\\
\end{tabular}
\begin{tabular}{l}
\parbox[t][][t]{8cm}{}\\
\parbox[t][][t]{8cm}{\normalfont \small ['(intransitive) to sound; to clink Es klingt gut.It', 'sounds good (e.g. music)'] }\\
\end{tabular}
}
%===klopfen===
\card{\normalfont \Huge klopfen}{
\begin{tabular}{lll}
\parbox[t][][t]{2.0 cm}{\normalfont \raggedleft ich\\du\\er/sie/es\\wir\\ihr\\sie} &    
\parbox[t][][t]{2cm}{\normalfont klopfe\\klopfst\\klopft\\klopfen\\klopft\\klopfen} &
\parbox[t][][t]{2cm}{\normalfont klopfte\\klopftest\\klopfte\\klopften\\klopftet\\klopften}\\
\end{tabular}
\begin{tabular}{l}
\parbox[t][][t]{8cm}{}\\
\parbox[t][][t]{8cm}{\normalfont \small ['to knock, to rap (strike rather gently with', 'something hard) (of the heart) to throb; to beat', 'quickly or audibly'] }\\
\end{tabular}
}
%===knien===
\card{\normalfont \Huge knien}{
\begin{tabular}{lll}
\parbox[t][][t]{2.0 cm}{\normalfont \raggedleft ich\\du\\er/sie/es\\wir\\ihr\\sie} &    
\parbox[t][][t]{2cm}{\normalfont knie\\kniest\\kniet\\knien\\kniet\\knien} &
\parbox[t][][t]{2cm}{\normalfont kniete\\knietest\\kniete\\knieten\\knietet\\knieten}\\
\end{tabular}
\begin{tabular}{l}
\parbox[t][][t]{8cm}{}\\
\parbox[t][][t]{8cm}{\normalfont \small ['to kneel'] }\\
\end{tabular}
}
%===kochen===
\card{\normalfont \Huge kochen}{
\begin{tabular}{lll}
\parbox[t][][t]{2.0 cm}{\normalfont \raggedleft ich\\du\\er/sie/es\\wir\\ihr\\sie} &    
\parbox[t][][t]{2cm}{\normalfont koche\\kochst\\kocht\\kochen\\kocht\\kochen} &
\parbox[t][][t]{2cm}{\normalfont kochte\\kochtest\\kochte\\kochten\\kochtet\\kochten}\\
\end{tabular}
\begin{tabular}{l}
\parbox[t][][t]{8cm}{}\\
\parbox[t][][t]{8cm}{\normalfont \small ['(intransitive) to boil (transitive) to boil (to', 'heat a liquid) (transitive) to boil (cook in', 'boiling water) to cook, to prepare food'] }\\
\end{tabular}
}
%===kommen===
\card{\normalfont \Huge kommen}{
\begin{tabular}{lll}
\parbox[t][][t]{2.0 cm}{\normalfont \raggedleft ich\\du\\er/sie/es\\wir\\ihr\\sie} &    
\parbox[t][][t]{2cm}{\normalfont komme\\kommst\\kommt\\kommen\\kommt\\kommen} &
\parbox[t][][t]{2cm}{\normalfont kam\\kamst\\kam\\kamen\\kamt\\kamen}\\
\end{tabular}
\begin{tabular}{l}
\parbox[t][][t]{8cm}{}\\
\parbox[t][][t]{8cm}{\normalfont \small ['(intransitive) to come; to arrive Er kam letzte', 'Nacht sehr spt nach Hause.  He came home very late', 'last night. Als ich nach Wuppertal kam, hatte es', 'gerade geschneit  When I arrived in Wuppertal, it', 'had just snowed. (intransitive) to come to; to', 'come over (go somewhere so as to join someone', 'else) Bleib sitzen! Ich komme zu dir.  Keep your', "seat! I'm coming over to you. Und viele kamen zu", 'ihm und sprachen...  And many resorted unto him', 'and said... (John 10:41) (intransitive) to get; to', 'make it (go somewhere in a way that implies an', 'obstacle or difficulty to be overcome) Ich komme', "nicht ber die Mauer.  I can't get over this wall.", 'Wenn er den Zug verpasst, kommt er heute nicht', "nach Nrnberg.  If he misses the train, he won't", 'make it to Nuremberg today. (intransitive) to go', 'to; to be put in (go somewhere in a way that is', 'predetermined or prearranged) Hartnckige Snder', 'kommen in die Hlle.  Persistant sinners will go to', 'hell. Die Gruppensieger kommen ins Halbfinale.', 'The group winners will go to the semifinals.', '(intransitive) to come on Ach komm, das wird so', "schlimm nicht werden.  Aw, come on, it won't be so", 'bad. Kommt, deckt schon mal den Tisch!  Come on,', 'just set the table already. (intransitive,', 'impersonal) to occur; to happen; to come to be', 'Dann kam, was alle befrchtet hatten.  Then', 'happened that which everybody had feared. Wie', 'kommt es, dass...?  Why is it that ...? How come', 'that...? (intransitive) to be played (of a song or', 'film) Eben kam mein Lieblingslied.  They just', 'played my favourite song. (intransitive, with von', 'or durch) to be due to; to be the result of Das', 'kommt alles von deiner Faulheit.  All of that is', 'due to your laziness. (intransitive, with aus +', 'dative) to come from (to have a social or', 'geographic background) Sie kommt aus der Schweiz.', 'She comes from Switzerland. Sie kommt aus einer', 'Diplomatenfamilie.  She comes a family of', 'diplomats. (intransitive, personal or impersonal', "+ dative) to orgasm; to cum Ich komme gleich!  I'm", "about to cum! Mir kommt's gleich!  I'm about to", 'cum! (intransitive, with auf + accusative) to be', 'statistically equivalent to; to be there for Auf', 'jeden Verkehrstoten kommen zwanzig Verletzte.For', 'each traffic fatality there are twenty injured', 'people. (intransitive, with auf + accusative) to', 'obtain (a solution or result) Die Werte wurden', 'frisiert, um auf das gewnschte Ergebnis zu', 'kommen.The values were manipulated in order to', 'obtain the desired result. (intransitive, with auf', '+ accusative) to get an idea; to think of; to', 'remember; to imagine Ich komme im Moment nicht', "drauf, aber ich sag's dir spter.I can't think of", "it right now, but I'll tell you later. Ich wei", 'wirklich nicht, wie du immer auf diese Einflle', "kommst.I really don't know how you always get all", 'those ideas. (intransitive, with um) to lose; to', 'forfeit; not to get Er hat Angst, dass er um', "seinen Anteil kommt.He fears that he won't get his", 'share.'] }\\
\end{tabular}
}
%===können===
\card{\normalfont \Huge können}{
\begin{tabular}{lll}
\parbox[t][][t]{2.0 cm}{\normalfont \raggedleft ich\\du\\er/sie/es\\wir\\ihr\\sie} &    
\parbox[t][][t]{2cm}{\normalfont kann\\kannst\\kann\\können\\könnt\\können} &
\parbox[t][][t]{2cm}{\normalfont konnte\\konntest\\konnte\\konnten\\konntet\\konnten}\\
\end{tabular}
\begin{tabular}{l}
\parbox[t][][t]{8cm}{}\\
\parbox[t][][t]{8cm}{\normalfont \small ['(auxiliary, with an infinitive, past participle:', '"knnen") To be able (to do something); can. Kannst', 'du ihm helfen?  "Are you able to help him?" Ich', 'htte das machen knnen.  "I could have done that."', '(auxiliary, with an infinitive, past participle:', '"knnen") To be allowed (to do something); to be', 'permitted (to do something); may. Kann ich', 'mitkommen?  "May I come along?" Er hat nicht ins', 'Kino gehen knnen.  "He was not allowed to go to', 'the cinema." (transitive, past participle:', '"gekonnt") To know how to do (something); to know;', 'to understand; to be able to do (something); to be', 'capable of; can do (something). Ich kann Deutsch', 'und Englisch.  "I know German and English." Kannst', 'du es?  "Can you do it?" Das htte ich nicht', 'gekonnt.  "I couldn\'t have done that." or "I', 'wouldn\'t have been capable of that."', '(intransitive, past participle: "gekonnt") To be', 'able to do something implied; can. Nein, ich kann', 'nicht.  "No, I can\'t." Er hat gekonnt.  "He was', 'able to [do it]." (intransitive, colloquial,', 'usually in negation) to be possible, to make sense', "Nchstes Jahr is'n Schaltjahr.  Das kann nich'.", 'Letztes Jahr war doch Schaltjahr! Next year is a', "leap year.  That's not possible. Last year was a", 'leap year!'] }\\
\end{tabular}
}
%===konzentrieren===
\card{\normalfont \Huge konzentrieren}{
\begin{tabular}{lll}
\parbox[t][][t]{2.0 cm}{\normalfont \raggedleft ich\\du\\er/sie/es\\wir\\ihr\\sie} &    
\parbox[t][][t]{2cm}{\normalfont konzentriere\\konzentrierst\\konzentriert\\konzentrieren\\konzentriert\\konzentrieren} &
\parbox[t][][t]{2cm}{\normalfont konzentrierte\\konzentriertest\\konzentrierte\\konzentrierten\\konzentriertet\\konzentrierten}\\
\end{tabular}
\begin{tabular}{l}
\parbox[t][][t]{8cm}{}\\
\parbox[t][][t]{8cm}{\normalfont \small ["(reflexive) to concentrate (one's mind or", 'attention) In den nchsten Monaten konzentriere ich', 'mich auf mein neues Buch. In the next few months I', 'will focus myself on my new book. (transitive) to', 'concentrate (something)'] }\\
\end{tabular}
}
%===korrigieren===
\card{\normalfont \Huge korrigieren}{
\begin{tabular}{lll}
\parbox[t][][t]{2.0 cm}{\normalfont \raggedleft ich\\du\\er/sie/es\\wir\\ihr\\sie} &    
\parbox[t][][t]{2cm}{\normalfont korrigiere\\korrigierst\\korrigiert\\korrigieren\\korrigiert\\korrigieren} &
\parbox[t][][t]{2cm}{\normalfont korrigierte\\korrigiertest\\korrigierte\\korrigierten\\korrigiertet\\korrigierten}\\
\end{tabular}
\begin{tabular}{l}
\parbox[t][][t]{8cm}{}\\
\parbox[t][][t]{8cm}{\normalfont \small ['to correct (homework or assignment) to mark'] }\\
\end{tabular}
}
%===kosten===
\card{\normalfont \Huge kosten}{
\begin{tabular}{lll}
\parbox[t][][t]{2.0 cm}{\normalfont \raggedleft ich\\du\\er/sie/es\\wir\\ihr\\sie} &    
\parbox[t][][t]{2cm}{\normalfont koste\\kostest\\kostet\\kosten\\kostet\\kosten} &
\parbox[t][][t]{2cm}{\normalfont kostete\\kostetest\\kostete\\kosteten\\kostetet\\kosteten}\\
\end{tabular}
\begin{tabular}{l}
\parbox[t][][t]{8cm}{}\\
\parbox[t][][t]{8cm}{\normalfont \small ['to cost'] }\\
\end{tabular}
}
%===kreischen===
\card{\normalfont \Huge kreischen}{
\begin{tabular}{lll}
\parbox[t][][t]{2.0 cm}{\normalfont \raggedleft ich\\du\\er/sie/es\\wir\\ihr\\sie} &    
\parbox[t][][t]{2cm}{\normalfont kreische\\kreischst\\kreischt\\kreischen\\kreischt\\kreischen} &
\parbox[t][][t]{2cm}{\normalfont kreischte\\kreischtest\\kreischte\\kreischten\\kreischtet\\kreischten}\\
\end{tabular}
\begin{tabular}{l}
\parbox[t][][t]{8cm}{}\\
\parbox[t][][t]{8cm}{\normalfont \small ['to squeal, to shriek, to scream to screech', '(colloquial, regional) to cry, to weep', '(birdwatching) to squawk'] }\\
\end{tabular}
}
%===kriechen===
\card{\normalfont \Huge kriechen}{
\begin{tabular}{lll}
\parbox[t][][t]{2.0 cm}{\normalfont \raggedleft ich\\du\\er/sie/es\\wir\\ihr\\sie} &    
\parbox[t][][t]{2cm}{\normalfont krieche\\kriechst\\kriecht\\kriechen\\kriecht\\kriechen} &
\parbox[t][][t]{2cm}{\normalfont kroch\\krochst\\kroch\\krochen\\krocht\\krochen}\\
\end{tabular}
\begin{tabular}{l}
\parbox[t][][t]{8cm}{}\\
\parbox[t][][t]{8cm}{\normalfont \small ['(intransitive) to creep; to crawl'] }\\
\end{tabular}
}
%===kriegen===
\card{\normalfont \Huge kriegen}{
\begin{tabular}{lll}
\parbox[t][][t]{2.0 cm}{\normalfont \raggedleft ich\\du\\er/sie/es\\wir\\ihr\\sie} &    
\parbox[t][][t]{2cm}{\normalfont kriege\\kriegst\\kriegt\\kriegen\\kriegt\\kriegen} &
\parbox[t][][t]{2cm}{\normalfont kriegte\\kriegtest\\kriegte\\kriegten\\kriegtet\\kriegten}\\
\end{tabular}
\begin{tabular}{l}
\parbox[t][][t]{8cm}{}\\
\parbox[t][][t]{8cm}{\normalfont \small ['(chiefly colloquial) to get (chiefly colloquial)', 'to catch, to come down with (dated, rare) to war'] }\\
\end{tabular}
}
%===kümmern===
\card{\normalfont \Huge kümmern}{
\begin{tabular}{lll}
\parbox[t][][t]{2.0 cm}{\normalfont \raggedleft ich\\du\\er/sie/es\\wir\\ihr\\sie} &    
\parbox[t][][t]{2cm}{\normalfont kümmreich kümmereich kümmer\\kümmerst\\kümmert\\kümmern\\kümmert\\kümmern} &
\parbox[t][][t]{2cm}{\normalfont kümmerte\\kümmertest\\kümmerte\\kümmerten\\kümmertet\\kümmerten}\\
\end{tabular}
\begin{tabular}{l}
\parbox[t][][t]{8cm}{}\\
\parbox[t][][t]{8cm}{\normalfont \small ['(reflexive) to take care, to look after. Wer', 'kmmert sich um die Ausbildung der Lehrlinge?Who', 'takes care of training the interns? (transitive)', 'to grieve, to afflict, to trouble, to concern.'] }\\
\end{tabular}
}
%===kündigen===
\card{\normalfont \Huge kündigen}{
\begin{tabular}{lll}
\parbox[t][][t]{2.0 cm}{\normalfont \raggedleft ich\\du\\er/sie/es\\wir\\ihr\\sie} &    
\parbox[t][][t]{2cm}{\normalfont kündige\\kündigst\\kündigt\\kündigen\\kündigt\\kündigen} &
\parbox[t][][t]{2cm}{\normalfont kündigte\\kündigtest\\kündigte\\kündigten\\kündigtet\\kündigten}\\
\end{tabular}
\begin{tabular}{l}
\parbox[t][][t]{8cm}{}\\
\parbox[t][][t]{8cm}{\normalfont \small ['can, sack (to fire or terminate an employee) quit', "(to quit one's job or position in a firm) cancel", '(a contract)'] }\\
\end{tabular}
}
%===kürzen===
\card{\normalfont \Huge kürzen}{
\begin{tabular}{lll}
\parbox[t][][t]{2.0 cm}{\normalfont \raggedleft ich\\du\\er/sie/es\\wir\\ihr\\sie} &    
\parbox[t][][t]{2cm}{\normalfont kürze\\kürzt\\kürzt\\kürzen\\kürzt\\kürzen} &
\parbox[t][][t]{2cm}{\normalfont kürzte\\kürztest\\kürzte\\kürzten\\kürztet\\kürzten}\\
\end{tabular}
\begin{tabular}{l}
\parbox[t][][t]{8cm}{}\\
\parbox[t][][t]{8cm}{\normalfont \small ['to shorten to abbreviate (literature) to abridge', '(mathematics) to cancel (a fraction, equation or', 'term)'] }\\
\end{tabular}
}
%===lächeln===
\card{\normalfont \Huge lächeln}{
\begin{tabular}{lll}
\parbox[t][][t]{2.0 cm}{\normalfont \raggedleft ich\\du\\er/sie/es\\wir\\ihr\\sie} &    
\parbox[t][][t]{2cm}{\normalfont lächleich lächeleich lächel\\lächelst\\lächelt\\lächeln\\lächelt\\lächeln} &
\parbox[t][][t]{2cm}{\normalfont lächelte\\lächeltest\\lächelte\\lächelten\\lächeltet\\lächelten}\\
\end{tabular}
\begin{tabular}{l}
\parbox[t][][t]{8cm}{}\\
\parbox[t][][t]{8cm}{\normalfont \small ['to smile'] }\\
\end{tabular}
}
%===lachen===
\card{\normalfont \Huge lachen}{
\begin{tabular}{lll}
\parbox[t][][t]{2.0 cm}{\normalfont \raggedleft ich\\du\\er/sie/es\\wir\\ihr\\sie} &    
\parbox[t][][t]{2cm}{\normalfont lache\\lachst\\lacht\\lachen\\lacht\\lachen} &
\parbox[t][][t]{2cm}{\normalfont lachte\\lachtest\\lachte\\lachten\\lachtet\\lachten}\\
\end{tabular}
\begin{tabular}{l}
\parbox[t][][t]{8cm}{}\\
\parbox[t][][t]{8cm}{\normalfont \small ['to laugh'] }\\
\end{tabular}
}
%===laden===
\card{\normalfont \Huge laden}{
\begin{tabular}{lll}
\parbox[t][][t]{2.0 cm}{\normalfont \raggedleft ich\\du\\er/sie/es\\wir\\ihr\\sie} &    
\parbox[t][][t]{2cm}{\normalfont lade\\lädst\\lädt\\laden\\ladet\\laden} &
\parbox[t][][t]{2cm}{\normalfont lud\\ludest / ludst\\lud\\luden\\ludet\\luden}\\
\end{tabular}
\begin{tabular}{l}
\parbox[t][][t]{8cm}{}\\
\parbox[t][][t]{8cm}{\normalfont \small ['(transitive, intransitive) to load (something)', 'e.g. into a container or onto a vehicle, to load', 'up (transitive, intransitive, weaponry) to load', '(some weapon) (transitive, computing) to load', '(some data) from a store (transitive, computing)', 'to download from a network (transitive,', 'engineering) to charge (a battery or capacitor)', 'with electricity'] }\\
\end{tabular}
}
%===landen===
\card{\normalfont \Huge landen}{
\begin{tabular}{lll}
\parbox[t][][t]{2.0 cm}{\normalfont \raggedleft ich\\du\\er/sie/es\\wir\\ihr\\sie} &    
\parbox[t][][t]{2cm}{\normalfont lande\\landest\\landet\\landen\\landet\\landen} &
\parbox[t][][t]{2cm}{\normalfont landete\\landetest\\landete\\landeten\\landetet\\landeten}\\
\end{tabular}
\begin{tabular}{l}
\parbox[t][][t]{8cm}{}\\
\parbox[t][][t]{8cm}{\normalfont \small ['(intransitive, auxiliary verb: sein, aviation) to', 'land das Flugzeug landet auf der Landebahn - the', 'plane lands on the runway (transitive, auxiliary', 'verb: haben, aviation) to land'] }\\
\end{tabular}
}
%===lassen===
\card{\normalfont \Huge lassen}{
\begin{tabular}{lll}
\parbox[t][][t]{2.0 cm}{\normalfont \raggedleft ich\\du\\er/sie/es\\wir\\ihr\\sie} &    
\parbox[t][][t]{2cm}{\normalfont lasse\\lässt\\lässt\\lassen\\lasst\\lassen} &
\parbox[t][][t]{2cm}{\normalfont ließ\\ließt\\ließ\\ließen\\ließt\\ließen}\\
\end{tabular}
\begin{tabular}{l}
\parbox[t][][t]{8cm}{}\\
\parbox[t][][t]{8cm}{\normalfont \small ['(transitive, with an infinitive) to allow; to', 'permit; to let (transitive, with an infinitive) to', 'have someone (do something); to have (something', 'done); to make (something happen); to cause', '(something to be done) etwas machen lassen  "to', 'have something done" jemanden etwas tun lassen', '"to have someone do something" (transitive) to', 'let; to leave (transitive) to stop (something); to', 'quit; to refrain from; to help doing (something)', '(intransitive) to cease; to desist'] }\\
\end{tabular}
}
%===laufen===
\card{\normalfont \Huge laufen}{
\begin{tabular}{lll}
\parbox[t][][t]{2.0 cm}{\normalfont \raggedleft ich\\du\\er/sie/es\\wir\\ihr\\sie} &    
\parbox[t][][t]{2cm}{\normalfont laufe\\läufst\\läuft\\laufen\\lauft\\laufen} &
\parbox[t][][t]{2cm}{\normalfont lief\\liefst\\lief\\liefen\\lieft\\liefen}\\
\end{tabular}
\begin{tabular}{l}
\parbox[t][][t]{8cm}{}\\
\parbox[t][][t]{8cm}{\normalfont \small ['(transitive or intransitive) to walk; to jog; to', 'run (to move on foot; either at a normal or an', 'increased speed) Wir knnen mit dem Bus fahren oder', 'laufen.We could take the bus or walk. Lasst uns', "etwas schneller laufen.Let's move a little faster.", 'Joggen bedeutet entspannter aber auch bewusster zu', 'laufen.Jogging means to run in a less exhausting', 'but more conscious way. (intransitive, of a fluid)', 'to flow; to leak; to run (intransitive, of an', 'event) to be in progress; to run Das Projekt luft', 'erfolgreich.The project is progressing', 'successfully. (intransitive, computing) to run, to', 'execute (a program) Das Programm luft einwandfrei.', 'The program runs flawlessly. (intransitive, of an', 'event) to be in order; to work; to function Alles', 'luft wie es soll.  Everything works just fine.', '(intransitive, of time) to pass; to flow'] }\\
\end{tabular}
}
%===leben===
\card{\normalfont \Huge leben}{
\begin{tabular}{lll}
\parbox[t][][t]{2.0 cm}{\normalfont \raggedleft ich\\du\\er/sie/es\\wir\\ihr\\sie} &    
\parbox[t][][t]{2cm}{\normalfont lebe\\lebst\\lebt\\leben\\lebt\\leben} &
\parbox[t][][t]{2cm}{\normalfont lebte\\lebtest\\lebte\\lebten\\lebtet\\lebten}\\
\end{tabular}
\begin{tabular}{l}
\parbox[t][][t]{8cm}{}\\
\parbox[t][][t]{8cm}{\normalfont \small ['(intransitive) to live, to be alive (intransitive)', 'to dwell, to reside 2010, Der Spiegel, issue', '35/2010, page 102: Es leben etwa 300 000 Brger des', 'ehemaligen Jugoslawien in der Schweiz, kaum ein', 'Staat hat damals im Verhltnis zu seiner', 'Einwohnerzahl so viele Flchtlinge aufgenommen.', 'There are (=reside) about 300,000 citizens of the', 'former Yugoslavia living in Switzerland, hardly', 'any state took in so many refugees in relation to', 'its population at that time. Ich lebe in der', 'Schillerstrae in der Nhe des Stadtzentrums. - I', "live in the Schiller-street near the city's", 'center. (intransitive) to live, to exist, to', 'occupy a place Die Dinosaurier lebten fr', 'Jahrmillionen auf der Erde bevor der Mensch', 'erschien. - The dinosaurs existed on Earth for', 'millions of years prior to the rise of man.', '(intransitive, hyperbolic) To cope with, to live', 'with, to deal with. Du wirst wohl damit leben', "mssen! - You'll have to cope with it! Jeder muss", 'mit seinen eigenen Problemen leben! - Everybody', 'has to deal with his own issues.'] }\\
\end{tabular}
}
%===legen===
\card{\normalfont \Huge legen}{
\begin{tabular}{lll}
\parbox[t][][t]{2.0 cm}{\normalfont \raggedleft ich\\du\\er/sie/es\\wir\\ihr\\sie} &    
\parbox[t][][t]{2cm}{\normalfont lege\\legst\\legt\\legen\\legt\\legen} &
\parbox[t][][t]{2cm}{\normalfont legte\\legtest\\legte\\legten\\legtet\\legten}\\
\end{tabular}
\begin{tabular}{l}
\parbox[t][][t]{8cm}{}\\
\parbox[t][][t]{8cm}{\normalfont \small ['(transitive) to lay, to put, to place, to', 'position, so that it afterwands lies as opposed to', 'being gesetzt, gestellt Leg deine Sachen auf den', 'Stuhl!Leave your stuff on the chair! Ich lege mich', 'auf das Bett.I lie down on the bed. Eier legento', 'lay eggs (animal husbandry) to castrate Ich habe', 'meinen Hengst legen lassen.I let my stallion be', 'castrated. Synonyms: verschneiden, geilen,', 'leichten, kastrieren, entmannen'] }\\
\end{tabular}
}
%===lehnen===
\card{\normalfont \Huge lehnen}{
\begin{tabular}{lll}
\parbox[t][][t]{2.0 cm}{\normalfont \raggedleft ich\\du\\er/sie/es\\wir\\ihr\\sie} &    
\parbox[t][][t]{2cm}{\normalfont lehne\\lehnst\\lehnt\\lehnen\\lehnt\\lehnen} &
\parbox[t][][t]{2cm}{\normalfont lehnte\\lehntest\\lehnte\\lehnten\\lehntet\\lehnten}\\
\end{tabular}
\begin{tabular}{l}
\parbox[t][][t]{8cm}{}\\
\parbox[t][][t]{8cm}{\normalfont \small ['(intransitive or reflexive) to lean (transitive)', 'to lean something'] }\\
\end{tabular}
}
%===lehren===
\card{\normalfont \Huge lehren}{
\begin{tabular}{lll}
\parbox[t][][t]{2.0 cm}{\normalfont \raggedleft ich\\du\\er/sie/es\\wir\\ihr\\sie} &    
\parbox[t][][t]{2cm}{\normalfont lehre\\lehrst\\lehrt\\lehren\\lehrt\\lehren} &
\parbox[t][][t]{2cm}{\normalfont lehrte\\lehrtest\\lehrte\\lehrten\\lehrtet\\lehrten}\\
\end{tabular}
\begin{tabular}{l}
\parbox[t][][t]{8cm}{}\\
\parbox[t][][t]{8cm}{\normalfont \small ['(transitive or intransitive) to teach a class, a', 'subject; to be a teacher Ich lehre Geschichte.  I', "teach history. Sie lehrt an der Uni.  She's a", 'teacher at the university. (formal, transitive) to', 'teach somebody something Meine Gromutter lehrte', 'mich das Stricken. My grandmother taught me', 'knitting.'] }\\
\end{tabular}
}
%===leiden===
\card{\normalfont \Huge leiden}{
\begin{tabular}{lll}
\parbox[t][][t]{2.0 cm}{\normalfont \raggedleft ich\\du\\er/sie/es\\wir\\ihr\\sie} &    
\parbox[t][][t]{2cm}{\normalfont leide\\leidest\\leidet\\leiden\\leidet\\leiden} &
\parbox[t][][t]{2cm}{\normalfont litt\\littest\\litt\\litten\\littet\\litten}\\
\end{tabular}
\begin{tabular}{l}
\parbox[t][][t]{8cm}{}\\
\parbox[t][][t]{8cm}{\normalfont \small ['(transitive) to bear; to endure; to undergo (some', 'hardship) Lerne leiden ohne zu klagen.Learn to', 'suffer without complaining. (intransitive) to', 'suffer; to feel pain (intransitive) to suffer', '(from; a disease) 2012 April 20, Die Welt [1],', 'page 22: Durch Passivrauchen steigt bei Kindern', 'das Risiko, dass sie als Erwachsene an einer', 'chronisch-obstruktiven Lungenerkrankung leiden. By', 'passive smoking, the risk increases in children', 'that they suffer from chronic obstructive lung', 'disease as adults.'] }\\
\end{tabular}
}
%===leihen===
\card{\normalfont \Huge leihen}{
\begin{tabular}{lll}
\parbox[t][][t]{2.0 cm}{\normalfont \raggedleft ich\\du\\er/sie/es\\wir\\ihr\\sie} &    
\parbox[t][][t]{2cm}{\normalfont leihe\\leihst\\leiht\\leihen\\leiht\\leihen} &
\parbox[t][][t]{2cm}{\normalfont lieh\\liehst\\lieh\\liehen\\lieht\\liehen}\\
\end{tabular}
\begin{tabular}{l}
\parbox[t][][t]{8cm}{}\\
\parbox[t][][t]{8cm}{\normalfont \small ['(transitive) to lend; (US) to loan (transitive) to', 'borrow 2010, Der Spiegel, issue 25/2010, page 80:', 'Ein Verbot sollte es nach Ansicht vieler konomen', 'auch fr die sogenannten Leerverkufe geben. Banken', 'verkaufen dabei Aktien oder Whrungen, die sie noch', 'gar nicht besitzen oder allenfalls geliehen haben.', 'In the opinion of many economists there should', 'also exist a prohibition for the so-called short', 'sales. In these banks sell shares or currencies', 'that they do not own at all yet or have borrowed', 'at best.'] }\\
\end{tabular}
}
%===leisten===
\card{\normalfont \Huge leisten}{
\begin{tabular}{lll}
\parbox[t][][t]{2.0 cm}{\normalfont \raggedleft ich\\du\\er/sie/es\\wir\\ihr\\sie} &    
\parbox[t][][t]{2cm}{\normalfont leiste\\leistest\\leistet\\leisten\\leistet\\leisten} &
\parbox[t][][t]{2cm}{\normalfont leistete\\leistetest\\leistete\\leisteten\\leistetet\\leisteten}\\
\end{tabular}
\begin{tabular}{l}
\parbox[t][][t]{8cm}{}\\
\parbox[t][][t]{8cm}{\normalfont \small ['to perform (reflexive) to afford'] }\\
\end{tabular}
}
%===leiten===
\card{\normalfont \Huge leiten}{
\begin{tabular}{lll}
\parbox[t][][t]{2.0 cm}{\normalfont \raggedleft ich\\du\\er/sie/es\\wir\\ihr\\sie} &    
\parbox[t][][t]{2cm}{\normalfont leite\\leitest\\leitet\\leiten\\leitet\\leiten} &
\parbox[t][][t]{2cm}{\normalfont leitete\\leitetest\\leitete\\leiteten\\leitetet\\leiteten}\\
\end{tabular}
\begin{tabular}{l}
\parbox[t][][t]{8cm}{}\\
\parbox[t][][t]{8cm}{\normalfont \small ['to lead to manage (an organization) to conduct (a', 'liquid, electricity etc)'] }\\
\end{tabular}
}
%===lesen===
\card{\normalfont \Huge lesen}{
\begin{tabular}{lll}
\parbox[t][][t]{2.0 cm}{\normalfont \raggedleft ich\\du\\er/sie/es\\wir\\ihr\\sie} &    
\parbox[t][][t]{2cm}{\normalfont lese\\liest\\liest\\lesen\\lest\\lesen} &
\parbox[t][][t]{2cm}{\normalfont las\\last\\las\\lasen\\last\\lasen}\\
\end{tabular}
\begin{tabular}{l}
\parbox[t][][t]{8cm}{}\\
\parbox[t][][t]{8cm}{\normalfont \small ['(transitive or intransitive) to read (look at and', 'understand symbols, words, or data) to select and', 'gather or harvest (things like grapes)'] }\\
\end{tabular}
}
%===lieben===
\card{\normalfont \Huge lieben}{
\begin{tabular}{lll}
\parbox[t][][t]{2.0 cm}{\normalfont \raggedleft ich\\du\\er/sie/es\\wir\\ihr\\sie} &    
\parbox[t][][t]{2cm}{\normalfont liebe\\liebst\\liebt\\lieben\\liebt\\lieben} &
\parbox[t][][t]{2cm}{\normalfont liebte\\liebtest\\liebte\\liebten\\liebtet\\liebten}\\
\end{tabular}
\begin{tabular}{l}
\parbox[t][][t]{8cm}{}\\
\parbox[t][][t]{8cm}{\normalfont \small ['(usually transitive, sometimes  intransitive) to', 'love, to have a strong affection for (someone or', 'something) Ich liebe dich.  I love you. Ich liebe', 'die franzsische Sprache.  I love the French', 'language. (reflexive) to love one another', '(reflexive, poetic) to make love, to have sex'] }\\
\end{tabular}
}
%===liefern===
\card{\normalfont \Huge liefern}{
\begin{tabular}{lll}
\parbox[t][][t]{2.0 cm}{\normalfont \raggedleft ich\\du\\er/sie/es\\wir\\ihr\\sie} &    
\parbox[t][][t]{2cm}{\normalfont liefreich liefereich liefer\\lieferst\\liefert\\liefern\\liefert\\liefern} &
\parbox[t][][t]{2cm}{\normalfont lieferte\\liefertest\\lieferte\\lieferten\\liefertet\\lieferten}\\
\end{tabular}
\begin{tabular}{l}
\parbox[t][][t]{8cm}{}\\
\parbox[t][][t]{8cm}{\normalfont \small ['to supply, provide (goods etc.) (an to) to deliver', '(goods, an order etc.) 2017, Neue Zrcher Zeitung,', '7 March: Wichtiger ist, dass die Ware innerhalb', 'einer definierten Zeit und an einen genau', 'bestimmten Ort geliefert wird. More importantly,', 'the goods are delivered in a specified time and', 'place. (crops) to yield (produce outcome)', '(information) to research, to read up on something', '(battle, duel, war) to fight'] }\\
\end{tabular}
}
%===liegen===
\card{\normalfont \Huge liegen}{
\begin{tabular}{lll}
\parbox[t][][t]{2.0 cm}{\normalfont \raggedleft ich\\du\\er/sie/es\\wir\\ihr\\sie} &    
\parbox[t][][t]{2cm}{\normalfont liege\\liegst\\liegt\\liegen\\liegt\\liegen} &
\parbox[t][][t]{2cm}{\normalfont lag\\lagst\\lag\\lagen\\lagt\\lagen}\\
\end{tabular}
\begin{tabular}{l}
\parbox[t][][t]{8cm}{}\\
\parbox[t][][t]{8cm}{\normalfont \small ['(intransitive) to lie (to be in a horizontal', 'position) (Switzerland) to lie down (intransitive)', 'to be, to lie somewhere (of flat objects;', 'otherwise use stehen) (intransitive) to be', 'located, to lie somewhere (of countries, towns,', 'houses, etc.) (intransitive) to be, to stand (of', 'indices, measurements) 2012 June 19, Die Welt [1],', 'page 10: Der deutsche Energieverbrauch lag in den', 'ersten drei Monaten des Jahres rund zwei Prozent', 'unter dem Niveau des Vorjahreszeitraumes. In the', 'first three months of the year, the German energy', 'consumption was about two percent below the level', 'of the same period last year.'] }\\
\end{tabular}
}
%===loben===
\card{\normalfont \Huge loben}{
\begin{tabular}{lll}
\parbox[t][][t]{2.0 cm}{\normalfont \raggedleft ich\\du\\er/sie/es\\wir\\ihr\\sie} &    
\parbox[t][][t]{2cm}{\normalfont lobe\\lobst\\lobt\\loben\\lobt\\loben} &
\parbox[t][][t]{2cm}{\normalfont lobte\\lobtest\\lobte\\lobten\\lobtet\\lobten}\\
\end{tabular}
\begin{tabular}{l}
\parbox[t][][t]{8cm}{}\\
\parbox[t][][t]{8cm}{\normalfont \small ['to praise 2012, Hans-Ulrich Ldemann, ICH - dann', 'eine Weile nichts, page 128: Ich lob mich selbst.', '2013, Euripides, Karl-Maria Guth (editor),', 'Iphigenie in Aulis, Sammlung Hofenberg im Verlag', 'der Contumax GmbH, page 17: Ich lob es, Bruder, da', 'du mir ein wackres WortUnd deiner wrdiges wider', 'mein Erwarten beutst! 1839, Shakspeare, Aug. Wilh.', 'v. Schlegel and Ludwig Tieck (translators),', "Shakspeare's dramatische Werke bersetzt von Aug.", 'Wilh. v. Schlegel und Ludwig Tieck. Zehnter Band.', 'Antonius und Cleopatra. Maa fr Maa. Timon von', 'Athen., Berlin, page 131 (part of: Antonius und', "Cleopatra, fnfter Aufzug, zweite Scene): Ich lob'", 'euchFr eure Klugheit.'] }\\
\end{tabular}
}
%===lohnen===
\card{\normalfont \Huge lohnen}{
\begin{tabular}{lll}
\parbox[t][][t]{2.0 cm}{\normalfont \raggedleft ich\\du\\er/sie/es\\wir\\ihr\\sie} &    
\parbox[t][][t]{2cm}{\normalfont lohne\\lohnst\\lohnt\\lohnen\\lohnt\\lohnen} &
\parbox[t][][t]{2cm}{\normalfont lohnte\\lohntest\\lohnte\\lohnten\\lohntet\\lohnten}\\
\end{tabular}
\begin{tabular}{l}
\parbox[t][][t]{8cm}{}\\
\parbox[t][][t]{8cm}{\normalfont \small ['(reflexive) to be worthwhile (reflexive) to pay', 'off'] }\\
\end{tabular}
}
%===löschen===
\card{\normalfont \Huge löschen}{
\begin{tabular}{lll}
\parbox[t][][t]{2.0 cm}{\normalfont \raggedleft ich\\du\\er/sie/es\\wir\\ihr\\sie} &    
\parbox[t][][t]{2cm}{\normalfont lösche\\löschst\\löscht\\löschen\\löscht\\löschen} &
\parbox[t][][t]{2cm}{\normalfont löschte\\löschtest\\löschte\\löschten\\löschtet\\löschten}\\
\end{tabular}
\begin{tabular}{l}
\parbox[t][][t]{8cm}{}\\
\parbox[t][][t]{8cm}{\normalfont \small ['(transitive) to extinguish (fire) (transitive) to', 'satisfy (thirst) (transitive) to delete, to erase', '(remembrance, memory, data)'] }\\
\end{tabular}
}
%===lösen===
\card{\normalfont \Huge lösen}{
\begin{tabular}{lll}
\parbox[t][][t]{2.0 cm}{\normalfont \raggedleft ich\\du\\er/sie/es\\wir\\ihr\\sie} &    
\parbox[t][][t]{2cm}{\normalfont löse\\löst\\löst\\lösen\\löst\\lösen} &
\parbox[t][][t]{2cm}{\normalfont löste\\löstest\\löste\\lösten\\löstet\\lösten}\\
\end{tabular}
\begin{tabular}{l}
\parbox[t][][t]{8cm}{}\\
\parbox[t][][t]{8cm}{\normalfont \small ['(transitive) to loose; to loosen; to detach; to', 'remove (etwas) von (etwas anderes) lsen - "to', 'remove (something) from (something else)"', '(transitive) to separate 2010, Der Spiegel, issue', '35/2010, page 96: Viele Gewsser sind mit', 'Quecksilber belastet, das die Goldsucher', 'einsetzen, um das Metall vom Gestein zu lsen. Many', 'waters are polluted with mercury, which is used by', 'gold diggers to separate the metal from the rock.', '(transitive) to cast off; to remove (transitive,', 'of a problem, puzzle, question, conflict,', 'figuratively) to solve; to resolve; to answer', '(reflexive) to come loose (reflexive) to dissolve', '(of a ticket) to buy eine Fahrkarte lsen - "to buy', 'a ticket" (transitive) to release; to undo; to', 'untie; to ease (transitive, of a relationship,', 'figuratively) to dissolve; to disband; to break', 'up; to end'] }\\
\end{tabular}
}
%===lügen===
\card{\normalfont \Huge lügen}{
\begin{tabular}{lll}
\parbox[t][][t]{2.0 cm}{\normalfont \raggedleft ich\\du\\er/sie/es\\wir\\ihr\\sie} &    
\parbox[t][][t]{2cm}{\normalfont lüge\\lügst\\lügt\\lügen\\lügt\\lügen} &
\parbox[t][][t]{2cm}{\normalfont log\\logst\\log\\logen\\logt\\logen}\\
\end{tabular}
\begin{tabular}{l}
\parbox[t][][t]{8cm}{}\\
\parbox[t][][t]{8cm}{\normalfont \small ['(intransitive) to tell a lie; to lie (to', 'intentionally give false information)', '(intransitive, less often) to give false', 'information (unintentionally) Wie alt sind Sie?', 'Ehm... lassen Sie mich nicht lgen...', "Zweiunddreiig.How old are you?  Er... don't let me", 'tell you something wrong... Thirty-two.'] }\\
\end{tabular}
}
%===machen===
\card{\normalfont \Huge machen}{
\begin{tabular}{lll}
\parbox[t][][t]{2.0 cm}{\normalfont \raggedleft ich\\du\\er/sie/es\\wir\\ihr\\sie} &    
\parbox[t][][t]{2cm}{\normalfont mache\\machst\\macht\\machen\\macht\\machen} &
\parbox[t][][t]{2cm}{\normalfont machte\\machtest\\machte\\machten\\machtet\\machten}\\
\end{tabular}
\begin{tabular}{l}
\parbox[t][][t]{8cm}{}\\
\parbox[t][][t]{8cm}{\normalfont \small ['(transitive) to make, produce, create (an object,', 'arrangement, situation, etc.) Ich hab dir einen', 'Kuchen gemacht!  I have made you a pie! Du hast', 'einen Fehler gemacht.  You made a mistake.', '(transitive) to take (a photo) (transitive, of', 'food, drinks, etc.) to make, prepare Machst du', 'heute das Essen?  Are you making dinner today?', 'sich eine Pizza machen  to prepare a pizza for', 'oneself (transitive, informal) to do, perform,', 'carry out (to execute; to put into operation (an', 'action)) Mach es!  Do it! Das hat er ganz allein', 'gemacht!  He has done that all by himself! ein', 'Experiment machen  to perform an experiment', '(transitive, with a noun) to do; indicates an', 'activity associated with a noun Sport machen  do', 'sports eine Party machen  have a party', '(transitive) to go (to make the (specified) sound)', '(transitive) to make (to cause or compel (to do', 'something)) (transitive, of difficulties, pain,', 'etc.) to cause (to set off an event or action or', 'produce as a result) (transitive, with an', 'adjective) to make (to cause to be) (transitive,', 'with a noun) to make (aus ("into")) (to cause to', 'become) (transitive, usually not translated', 'literally) to make (to have as a feature)', '(transitive, informal, colloquial) to come to,', 'total, cost (to require the payment of) Wie viel', 'macht das?  How much does that come to?', '(transitive, arithmetic) to make, be (to have a', 'sum of) (transitive, informal, colloquial) to make', '(to earn, gain wages, profit, etc.) Der Herr Mller', 'ist echt reich; der macht mehr als 5000 im', 'Monat.Mr Mller is quite rich; he makes more than', '5000 bucks per month. (transitive) to be, play (to', 'act as the indicated role, especially in a', 'performance) (transitive, of a bed) to make (to', 'cover neatly with bedclothes) (transitive,', 'impersonal, colloquial) to matter (to be', "important) Das macht nichts!  That doesn't matter!", '(intransitive, with auf) to make, make oneself out', 'to be, act, play (to behave so as to give an', 'appearance of being; to act as if one were', '(something, or a certain way)) (intransitive,', "informal, euphemistic) to do one's business, do", 'number two or number one, go (to defecate or', 'urinate) (childish) gro machen  to go poop', '(childish) klein machen  to go pee (reflexive) to', 'do (to fare or perform (well or poorly))', '(reflexive) to look (to have an appearance of', 'being) Der Mantel macht sich sehr schn.  The coat', 'looks very nice. (reflexive  dative, colloquial)', 'to get cracking (an ("on," "with")), get a move on', '(it), to get down (an ("to")) (something); (in', "imperative:) come on, let's go"] }\\
\end{tabular}
}
%===mahlen===
\card{\normalfont \Huge mahlen}{
\begin{tabular}{lll}
\parbox[t][][t]{2.0 cm}{\normalfont \raggedleft ich\\du\\er/sie/es\\wir\\ihr\\sie} &    
\parbox[t][][t]{2cm}{\normalfont mahle\\mahlst\\mahlt\\mahlen\\mahlt\\mahlen} &
\parbox[t][][t]{2cm}{\normalfont mahlte\\mahltest\\mahlte\\mahlten\\mahltet\\mahlten}\\
\end{tabular}
\begin{tabular}{l}
\parbox[t][][t]{8cm}{}\\
\parbox[t][][t]{8cm}{\normalfont \small ['(transitive or intransitive) to grind'] }\\
\end{tabular}
}
%===malen===
\card{\normalfont \Huge malen}{
\begin{tabular}{lll}
\parbox[t][][t]{2.0 cm}{\normalfont \raggedleft ich\\du\\er/sie/es\\wir\\ihr\\sie} &    
\parbox[t][][t]{2cm}{\normalfont male\\malst\\malt\\malen\\malt\\malen} &
\parbox[t][][t]{2cm}{\normalfont malte\\maltest\\malte\\malten\\maltet\\malten}\\
\end{tabular}
\begin{tabular}{l}
\parbox[t][][t]{8cm}{}\\
\parbox[t][][t]{8cm}{\normalfont \small ['to paint (to create a painting) Synonym: streichen'] }\\
\end{tabular}
}
%===meiden===
\card{\normalfont \Huge meiden}{
\begin{tabular}{lll}
\parbox[t][][t]{2.0 cm}{\normalfont \raggedleft ich\\du\\er/sie/es\\wir\\ihr\\sie} &    
\parbox[t][][t]{2cm}{\normalfont meide\\meidest\\meidet\\meiden\\meidet\\meiden} &
\parbox[t][][t]{2cm}{\normalfont mied\\miedest\\mied\\mieden\\miedet\\mieden}\\
\end{tabular}
\begin{tabular}{l}
\parbox[t][][t]{8cm}{}\\
\parbox[t][][t]{8cm}{\normalfont \small ['(transitive) to avoid 2010, Der Spiegel, issue', '22/2010, page 126: Pasta, Kuchen, Msli, Brot  wer', 'an Zliakie leidet, muss viele gngige Lebensmittel', 'meiden: Das Eiwei Gluten, das bei den Betroffenen', 'zu chronischer Darmentzndung fhrt, kommt in den', 'meisten Getreidearten vor. Pasta, cakes, muesli,', 'bread  someone who suffers from celiac disease has', 'to avoid many common foods: the protein gluten,', 'which leads to chronic intestinal inflammation for', 'the sufferers, occurs in most types of grain.', '(transitive) to shun'] }\\
\end{tabular}
}
%===meinen===
\card{\normalfont \Huge meinen}{
\begin{tabular}{lll}
\parbox[t][][t]{2.0 cm}{\normalfont \raggedleft ich\\du\\er/sie/es\\wir\\ihr\\sie} &    
\parbox[t][][t]{2cm}{\normalfont meine\\meinst\\meint\\meinen\\meint\\meinen} &
\parbox[t][][t]{2cm}{\normalfont meinte\\meintest\\meinte\\meinten\\meintet\\meinten}\\
\end{tabular}
\begin{tabular}{l}
\parbox[t][][t]{8cm}{}\\
\parbox[t][][t]{8cm}{\normalfont \small ['to think; to believe; to suppose (have an opinion', 'or impression) Ich meine, das war letztes oder', 'vorletztes Jahr.I think it was last year or the', 'year before. to say; to utter; not used with', 'nouns; not used in the imperative and rarely in', 'the infinitive Entschuldige, was meintest du', 'gerade?Sorry, what did you just say? to mean; to', 'be convinced or sincere about something Das sagt', "er nicht nur, das meint er auch.He doesn't just", 'say it, he means it. to mean; to have in mind; to', 'convey Was meintest du damit?What did you mean by', 'that? Meinst du das rote oder das gelbe Haus?Do', 'you mean the red or the yellow house? (now  rare)', 'to mean; to signify Was meint dieses Wort?What', 'does this word mean?'] }\\
\end{tabular}
}
%===melden===
\card{\normalfont \Huge melden}{
\begin{tabular}{lll}
\parbox[t][][t]{2.0 cm}{\normalfont \raggedleft ich\\du\\er/sie/es\\wir\\ihr\\sie} &    
\parbox[t][][t]{2cm}{\normalfont melde\\meldest\\meldet\\melden\\meldet\\melden} &
\parbox[t][][t]{2cm}{\normalfont meldete\\meldetest\\meldete\\meldeten\\meldetet\\meldeten}\\
\end{tabular}
\begin{tabular}{l}
\parbox[t][][t]{8cm}{}\\
\parbox[t][][t]{8cm}{\normalfont \small ['(transitive) to report Der Direktor meldete den', 'Unfall.The principal reported the accident.', '(transitive) to tell on (someone) (reflexive) to', "put one's hand up (in school) (reflexive) to get", 'in touch with someone, touch base (make contact', 'with someone) "Eine Reise hrt sich gut an... ich', 'melde mich bei dir!"A trip sounds good... I\'m', 'getting in touch with you! to stay in touch (keep', 'in contact with someone) Synonym: von sich hren', 'lassen to follow up, get back to (return contact', 'with someone)'] }\\
\end{tabular}
}
%===merken===
\card{\normalfont \Huge merken}{
\begin{tabular}{lll}
\parbox[t][][t]{2.0 cm}{\normalfont \raggedleft ich\\du\\er/sie/es\\wir\\ihr\\sie} &    
\parbox[t][][t]{2cm}{\normalfont merke\\merkst\\merkt\\merken\\merkt\\merken} &
\parbox[t][][t]{2cm}{\normalfont merkte\\merktest\\merkte\\merkten\\merktet\\merkten}\\
\end{tabular}
\begin{tabular}{l}
\parbox[t][][t]{8cm}{}\\
\parbox[t][][t]{8cm}{\normalfont \small ['(transitive) to notice (reflexive) to memorize,', 'remember to realize 1912, Franz Kafka, Die', 'Verwandlung, in: Die Weien Bltter. Eine', 'Monatsschrift. year 2, issue 10, Verlag der Weien', 'Bcher (1915), page 1180: Zunchst wollte er ruhig', 'und ungestrt aufstehen, sich anziehen und vor', 'allem frhstcken, und dann erst das Weitere', 'berlegen, denn, das merkte er wohl, im Bett wrde', 'er mit dem Nachdenken zu keinem vernnftigen Ende', 'kommen. To begin with, he wanted to get up calmly', 'and undisturbed, get dressed and, above all, have', 'breakfast, and only then think about everything', 'else, because, as he realized well, in bed he', 'would not come to a sensible conclusion with the', 'thinking.'] }\\
\end{tabular}
}
%===messen===
\card{\normalfont \Huge messen}{
\begin{tabular}{lll}
\parbox[t][][t]{2.0 cm}{\normalfont \raggedleft ich\\du\\er/sie/es\\wir\\ihr\\sie} &    
\parbox[t][][t]{2cm}{\normalfont messe\\misst\\misst\\messen\\messt\\messen} &
\parbox[t][][t]{2cm}{\normalfont maß\\maßt\\maß\\maßen\\maßt\\maßen}\\
\end{tabular}
\begin{tabular}{l}
\parbox[t][][t]{8cm}{}\\
\parbox[t][][t]{8cm}{\normalfont \small ['(transitive) to measure (something) (reflexive) to', 'compete sich mit jemandem messen  "to compete with', 'someone" (intransitive) to measure; to be a given', 'size, height, width, length, etc. Der Tisch misst', '160 Zentimeter.The table measures 160 centimeters.'] }\\
\end{tabular}
}
%===mieten===
\card{\normalfont \Huge mieten}{
\begin{tabular}{lll}
\parbox[t][][t]{2.0 cm}{\normalfont \raggedleft ich\\du\\er/sie/es\\wir\\ihr\\sie} &    
\parbox[t][][t]{2cm}{\normalfont miete\\mietest\\mietet\\mieten\\mietet\\mieten} &
\parbox[t][][t]{2cm}{\normalfont mietete\\mietetest\\mietete\\mieteten\\mietetet\\mieteten}\\
\end{tabular}
\begin{tabular}{l}
\parbox[t][][t]{8cm}{}\\
\parbox[t][][t]{8cm}{\normalfont \small ['(transitive) to hire, to rent'] }\\
\end{tabular}
}
%===misslingen===
\card{\normalfont \Huge misslingen}{
\begin{tabular}{lll}
\parbox[t][][t]{2.0 cm}{\normalfont \raggedleft ich\\du\\er/sie/es\\wir\\ihr\\sie} &    
\parbox[t][][t]{2cm}{\normalfont misslinge\\misslingst\\misslingt\\misslingen\\misslingt\\misslingen} &
\parbox[t][][t]{2cm}{\normalfont misslang\\misslangst\\misslang\\misslangen\\misslangt\\misslangen}\\
\end{tabular}
\begin{tabular}{l}
\parbox[t][][t]{8cm}{}\\
\parbox[t][][t]{8cm}{\normalfont \small ['(intransitive) to fail; to be unsuccessful Das ist', 'dir aber grndlich misslungen!  "That was', 'thoroughly unsuccessful for you!"'] }\\
\end{tabular}
}
%===mitteilen===
\card{\normalfont \Huge mitteilen}{
\begin{tabular}{lll}
\parbox[t][][t]{2.0 cm}{\normalfont \raggedleft ich\\du\\er/sie/es\\wir\\ihr\\sie} &    
\parbox[t][][t]{2cm}{\normalfont teile mit\\teilst mit\\teilt mit\\teilen mit\\teilt mit\\teilen mit} &
\parbox[t][][t]{2cm}{\normalfont teilte mit\\teiltest mit\\teilte mit\\teilten mit\\teiltet mit\\teilten mit}\\
\end{tabular}
\begin{tabular}{l}
\parbox[t][][t]{8cm}{}\\
\parbox[t][][t]{8cm}{\normalfont \small ['to inform, to tell, to disclose to share with'] }\\
\end{tabular}
}
%===mögen===
\card{\normalfont \Huge mögen}{
\begin{tabular}{lll}
\parbox[t][][t]{2.0 cm}{\normalfont \raggedleft ich\\du\\er/sie/es\\wir\\ihr\\sie} &    
\parbox[t][][t]{2cm}{\normalfont mag\\magst\\mag\\mögen\\mögt\\mögen} &
\parbox[t][][t]{2cm}{\normalfont mochte\\mochtest\\mochte\\mochten\\mochtet\\mochten}\\
\end{tabular}
\begin{tabular}{l}
\parbox[t][][t]{8cm}{}\\
\parbox[t][][t]{8cm}{\normalfont \small ['(transitive) to like (something or someone) Ich', "mag keinen Kse.  I don't like cheese.", '(intransitive) to want to go (auxiliary, with', 'infinitive) may (expresses a possibility, never a', 'permission) Das mag ja alles stimmen.  That may', 'all be true. (auxiliary, in negation, with', 'infinitive) to be hesitant to (do something) Ich', 'mag sie nicht fragen.  I am hesitant to ask her.', '(auxiliary, in the present subjunctive, with', 'infinitive) may (paraphrases the optative). Mge', 'die Macht mit dir sein.  May the Force be with', 'you. (transitive, in the past subjunctive) to', 'want; would like (to have or to do something) Ich', 'mchte Kse.  I would like cheese. Ich mchte keinen', "Kse.  I don't want cheese. (auxiliary, in the past", 'subjunctive, with infinitive) to want to; would', 'like to; to wish to (do something) Ich mchte sie', "nicht fragen.  I don't want to ask her."] }\\
\end{tabular}
}
%===nehmen===
\card{\normalfont \Huge nehmen}{
\begin{tabular}{lll}
\parbox[t][][t]{2.0 cm}{\normalfont \raggedleft ich\\du\\er/sie/es\\wir\\ihr\\sie} &    
\parbox[t][][t]{2cm}{\normalfont nehme\\nimmst\\nimmt\\nehmen\\nehmt\\nehmen} &
\parbox[t][][t]{2cm}{\normalfont nahm\\nahmst\\nahm\\nahmen\\nahmt\\nahmen}\\
\end{tabular}
\begin{tabular}{l}
\parbox[t][][t]{8cm}{}\\
\parbox[t][][t]{8cm}{\normalfont \small ['(transitive) to take jemandem etwas nehmen  "to', 'take something from someone" einen Anfang nehmen', '"to begin" (Literally, "to take a beginning") ein', 'Haus in Pacht nehmen  "to lease a house"', '(Literally, "to take a house in lease") das Wort', 'nehmen  "to begin to speak" (Literally, "to take a', 'word") 1798, Wold und Ostar, zwo altteutsche', 'Gottheiten, von Karl, Freyherrn v. Mnchhausen, in:', 'Bragur. Ein Literarisches Magazin der Teutschen', 'und Nordischen Vorzeit. Herausgegeben von F. D.', 'Grter. Sechster Band. Erste Abtheilung.  Braga und', 'Hermode oder Neues Magazin fr die vaterlndischen', 'Alterthmer der Sprache, Kunst und Sitten.', 'Herausgegeben von F. D. Grter. Dritter Band. Erste', 'Abtheilung, Leipzig, 1798, p. 23: Hierauf nehmen', 'sie des Getrnks [...] (reflexive) to cause oneself', 'to be (in some state); to become; to take oneself', '(to some state) Nimm dich in Acht!Take care!', '(transitive) to seize; to capture (transitive) to', 'receive; to accept'] }\\
\end{tabular}
}
%===nennen===
\card{\normalfont \Huge nennen}{
\begin{tabular}{lll}
\parbox[t][][t]{2.0 cm}{\normalfont \raggedleft ich\\du\\er/sie/es\\wir\\ihr\\sie} &    
\parbox[t][][t]{2cm}{\normalfont nenne\\nennst\\nennt\\nennen\\nennt\\nennen} &
\parbox[t][][t]{2cm}{\normalfont nannte\\nanntest\\nannte\\nannten\\nanntet\\nannten}\\
\end{tabular}
\begin{tabular}{l}
\parbox[t][][t]{8cm}{}\\
\parbox[t][][t]{8cm}{\normalfont \small ['(transitive) to name; to give a name to (someone);', 'to call (someone something) (transitive) to call', '(someone or something by some name or title)', '(transitive) to mention 1918, Elisabeth von', 'Heyking, Die Orgelpfeifen, in Zwei Erzhlungen,', 'Phillipp Reclam jun., page 6465: Namen wurden in', 'diesen Berichten nicht genannt, und in dieser', 'Anonymitt der Leistungen lag eine besondere', 'entsagungsvolle Gre. Names were not mentioned in', 'these reports and in this anonymity of the', 'achievements was a particular sacrificing', 'greatness. (transitive) to call out; to give (e.g.', 'some request) (reflexive) to be called; to be', 'named; to go by some name'] }\\
\end{tabular}
}
%===nutzen===
\card{\normalfont \Huge nutzen}{
\begin{tabular}{lll}
\parbox[t][][t]{2.0 cm}{\normalfont \raggedleft ich\\du\\er/sie/es\\wir\\ihr\\sie} &    
\parbox[t][][t]{2cm}{\normalfont nutze\\nutzt\\nutzt\\nutzen\\nutzt\\nutzen} &
\parbox[t][][t]{2cm}{\normalfont nutzte\\nutztest\\nutzte\\nutzten\\nutztet\\nutzten}\\
\end{tabular}
\begin{tabular}{l}
\parbox[t][][t]{8cm}{}\\
\parbox[t][][t]{8cm}{\normalfont \small ['(transitive) to make use of; to deploy; to', 'exploit; to harness; to take (the opportunity of)', 'Du solltest deine Talente nutzen. You should make', 'use of your talents. Wenn sich ihnen so eine', 'Gelegenheit bietet, dann nutzen sie die. When such', "an opportunity offers itself to them, they'll take", 'it. (transitive or intransitive, most often', 'negated or in questions) to be useful, to be of', 'use, to do good Er hrt mit der Dit auf, weil sie', "nichts nutzt. He's stopping the diet because it is", 'no use. Was nutzt es schon, sich anzustrengen?', 'What is the use of making an effort anyway?', '(transitive or intransitive, + dative) to benefit', 'someone, to help, to do good to Er hat sich berall', 'angebiedert, aber das hat ihm auch nicht genutzt.', "He curried favour with everybody, but that didn't", 'help him either.'] }\\
\end{tabular}
}
%===öffnen===
\card{\normalfont \Huge öffnen}{
\begin{tabular}{lll}
\parbox[t][][t]{2.0 cm}{\normalfont \raggedleft ich\\du\\er/sie/es\\wir\\ihr\\sie} &    
\parbox[t][][t]{2cm}{\normalfont öffne\\öffnest\\öffnet\\öffnen\\öffnet\\öffnen} &
\parbox[t][][t]{2cm}{\normalfont öffnete\\öffnetest\\öffnete\\öffneten\\öffnetet\\öffneten}\\
\end{tabular}
\begin{tabular}{l}
\parbox[t][][t]{8cm}{}\\
\parbox[t][][t]{8cm}{\normalfont \small ['(transitive) to open (to make something accessible', 'or allow for passage by moving from a shut', 'position) Das Kind ffnete die Fenster.The child', 'opened the windows. (transitive) to open (to make', 'accessible to customers or clients) (transitive,', 'computing) to open (to load into memory for', 'viewing or editing) (intransitive) to open, get,', 'or answer the door (reflexive) to open (to become', 'open) (reflexive) to open up (to), confide (in)', '(to reveal oneself; share personal information', 'about oneself)'] }\\
\end{tabular}
}
%===ordnen===
\card{\normalfont \Huge ordnen}{
\begin{tabular}{lll}
\parbox[t][][t]{2.0 cm}{\normalfont \raggedleft ich\\du\\er/sie/es\\wir\\ihr\\sie} &    
\parbox[t][][t]{2cm}{\normalfont ordne\\ordnest\\ordnet\\ordnen\\ordnet\\ordnen} &
\parbox[t][][t]{2cm}{\normalfont ordnete\\ordnetest\\ordnete\\ordneten\\ordnetet\\ordneten}\\
\end{tabular}
\begin{tabular}{l}
\parbox[t][][t]{8cm}{}\\
\parbox[t][][t]{8cm}{\normalfont \small ['to put in order, to order'] }\\
\end{tabular}
}
%===packen===
\card{\normalfont \Huge packen}{
\begin{tabular}{lll}
\parbox[t][][t]{2.0 cm}{\normalfont \raggedleft ich\\du\\er/sie/es\\wir\\ihr\\sie} &    
\parbox[t][][t]{2cm}{\normalfont packe\\packst\\packt\\packen\\packt\\packen} &
\parbox[t][][t]{2cm}{\normalfont packte\\packtest\\packte\\packten\\packtet\\packten}\\
\end{tabular}
\begin{tabular}{l}
\parbox[t][][t]{8cm}{}\\
\parbox[t][][t]{8cm}{\normalfont \small ['(transitive or intransitive) to pack (luggage, a', 'bundle, etc.); to get packed Du musst deine Sachen', 'packen.  "You need to pack your stuff." Du musst', 'packen.  "You need to get packed." (transitive) to', 'grab, to grip, to take Er packte mich am Arm.  "He', 'grabbed me by the arm." (informal, transitive) to', 'manage, to stand, to cope Ich pack das alles', 'nich\'...  "I can\'t manage all of this..." (slang,', 'transitive) to take (a bus, train, etc.) Lass den', 'Bus packen!  "Let\'s take the bus!" (regional or', 'dated, reflexive) to beat it Pack dich!  "Beat', 'it!"'] }\\
\end{tabular}
}
%===passieren===
\card{\normalfont \Huge passieren}{
\begin{tabular}{lll}
\parbox[t][][t]{2.0 cm}{\normalfont \raggedleft ich\\du\\er/sie/es\\wir\\ihr\\sie} &    
\parbox[t][][t]{2cm}{\normalfont passiere\\passierst\\passiert\\passieren\\passiert\\passieren} &
\parbox[t][][t]{2cm}{\normalfont passierte\\passiertest\\passierte\\passierten\\passiertet\\passierten}\\
\end{tabular}
\begin{tabular}{l}
\parbox[t][][t]{8cm}{}\\
\parbox[t][][t]{8cm}{\normalfont \small ['(auxiliary sein) to happen Wie konnte das nur', 'passieren?How could that have happened? Synonyms:', 'geschehen (formal), sich ereignen (formal), los', 'sein (colloquial), abgehen (slang) (formal,', 'auxiliary haben) to move beyond; pass Synonyms:', 'vorbei sein (colloquial), vorbeikommen,', 'vorbeigehen (walking), vorbeifahren (driving),', 'berqueren (cross), berschreiten (cross) (formal)', 'Die Gruppe hat soeben die Grenze passiert.The', 'group has just passed the border. (cooking,', 'auxiliary haben) to pass through a sieve, to', 'strain Als nchstes muss die Brhe durch das Sieb', 'passiert werden.Next, the broth needs to be passed', 'through the sieve. Synonym: seihen'] }\\
\end{tabular}
}
%===pfeifen===
\card{\normalfont \Huge pfeifen}{
\begin{tabular}{lll}
\parbox[t][][t]{2.0 cm}{\normalfont \raggedleft ich\\du\\er/sie/es\\wir\\ihr\\sie} &    
\parbox[t][][t]{2cm}{\normalfont pfeife\\pfeifst\\pfeift\\pfeifen\\pfeift\\pfeifen} &
\parbox[t][][t]{2cm}{\normalfont pfiff\\pfiffst\\pfiff\\pfiffen\\pfifft\\pfiffen}\\
\end{tabular}
\begin{tabular}{l}
\parbox[t][][t]{8cm}{}\\
\parbox[t][][t]{8cm}{\normalfont \small ['(intransitive or transitive) to whistle (with', "one's mouth or a whistle) 1844, Heinrich Heine,", '"Tragdie III", in Neue Gedichte. Auf ihrem Grab da', 'steht eine Linde, / drin pfeifen die Vgel und', 'Abendwinde, / und drunter sitzt, auf dem grnen', 'Platz, / der Mllersknecht mit seinem', 'Schatz.(please add an English translation of this', 'quote) Synonym: flten (regional) Hr auf zu', 'pfeifen!  Stop whistling! Er pfeift ein Liedchen.', "He's whistling a song. (intransitive or", 'transitive, sports) to act as referee Synonym:', 'flten (regional) Der Schiedsrichter pfeift sehr', 'gut.  The referee is doing a great job. Wer pfeift', "das Spiel?  Who's refereeing the game?", '(colloquial, transitive with auf) to be', 'uninterested (in something or someone); to ignore;', 'not to give a damn Synonym: foutieren', "(Switzerland) Darauf pfeif ich!  I don't care one", 'bit about that!'] }\\
\end{tabular}
}
%===pflegen===
\card{\normalfont \Huge pflegen}{
\begin{tabular}{lll}
\parbox[t][][t]{2.0 cm}{\normalfont \raggedleft ich\\du\\er/sie/es\\wir\\ihr\\sie} &    
\parbox[t][][t]{2cm}{\normalfont pflege\\pflegst\\pflegt\\pflegen\\pflegt\\pflegen} &
\parbox[t][][t]{2cm}{\normalfont pflegte\\pflegtest\\pflegte\\pflegten\\pflegtet\\pflegten}\\
\end{tabular}
\begin{tabular}{l}
\parbox[t][][t]{8cm}{}\\
\parbox[t][][t]{8cm}{\normalfont \small ['(transitive) to look after; to care for (medicine,', 'transitive) to nurse (intransitive, with "zu"', 'followed by an infinitive verb) to perform', 'habitually; to be accustomed [to doing something];', 'to be in the habit [of doing something] Ich pflege', 'zu laufen."I usually walk." Er pflegte zu', 'reisen."He used to travel."'] }\\
\end{tabular}
}
%===planen===
\card{\normalfont \Huge planen}{
\begin{tabular}{lll}
\parbox[t][][t]{2.0 cm}{\normalfont \raggedleft ich\\du\\er/sie/es\\wir\\ihr\\sie} &    
\parbox[t][][t]{2cm}{\normalfont plane\\planst\\plant\\planen\\plant\\planen} &
\parbox[t][][t]{2cm}{\normalfont plante\\plantest\\plante\\planten\\plantet\\planten}\\
\end{tabular}
\begin{tabular}{l}
\parbox[t][][t]{8cm}{}\\
\parbox[t][][t]{8cm}{\normalfont \small ['to plan'] }\\
\end{tabular}
}
%===preisen===
\card{\normalfont \Huge preisen}{
\begin{tabular}{lll}
\parbox[t][][t]{2.0 cm}{\normalfont \raggedleft ich\\du\\er/sie/es\\wir\\ihr\\sie} &    
\parbox[t][][t]{2cm}{\normalfont preise\\preist\\preist\\preisen\\preist\\preisen} &
\parbox[t][][t]{2cm}{\normalfont pries\\priest\\pries\\priesen\\priest\\priesen}\\
\end{tabular}
\begin{tabular}{l}
\parbox[t][][t]{8cm}{}\\
\parbox[t][][t]{8cm}{\normalfont \small ['to hail, to laud, to praise'] }\\
\end{tabular}
}
%===probieren===
\card{\normalfont \Huge probieren}{
\begin{tabular}{lll}
\parbox[t][][t]{2.0 cm}{\normalfont \raggedleft ich\\du\\er/sie/es\\wir\\ihr\\sie} &    
\parbox[t][][t]{2cm}{\normalfont probiere\\probierst\\probiert\\probieren\\probiert\\probieren} &
\parbox[t][][t]{2cm}{\normalfont probierte\\probiertest\\probierte\\probierten\\probiertet\\probierten}\\
\end{tabular}
\begin{tabular}{l}
\parbox[t][][t]{8cm}{}\\
\parbox[t][][t]{8cm}{\normalfont \small ['to try to taste to sample'] }\\
\end{tabular}
}
%===protestieren===
\card{\normalfont \Huge protestieren}{
\begin{tabular}{lll}
\parbox[t][][t]{2.0 cm}{\normalfont \raggedleft ich\\du\\er/sie/es\\wir\\ihr\\sie} &    
\parbox[t][][t]{2cm}{\normalfont protestiere\\protestierst\\protestiert\\protestieren\\protestiert\\protestieren} &
\parbox[t][][t]{2cm}{\normalfont protestierte\\protestiertest\\protestierte\\protestierten\\protestiertet\\protestierten}\\
\end{tabular}
\begin{tabular}{l}
\parbox[t][][t]{8cm}{}\\
\parbox[t][][t]{8cm}{\normalfont \small ['to protest (to make a strong objection)'] }\\
\end{tabular}
}
%===prüfen===
\card{\normalfont \Huge prüfen}{
\begin{tabular}{lll}
\parbox[t][][t]{2.0 cm}{\normalfont \raggedleft ich\\du\\er/sie/es\\wir\\ihr\\sie} &    
\parbox[t][][t]{2cm}{\normalfont prüfe\\prüfst\\prüft\\prüfen\\prüft\\prüfen} &
\parbox[t][][t]{2cm}{\normalfont prüfte\\prüftest\\prüfte\\prüften\\prüftet\\prüften}\\
\end{tabular}
\begin{tabular}{l}
\parbox[t][][t]{8cm}{}\\
\parbox[t][][t]{8cm}{\normalfont \small ['to examine'] }\\
\end{tabular}
}
%===rächen===
\card{\normalfont \Huge rächen}{
\begin{tabular}{lll}
\parbox[t][][t]{2.0 cm}{\normalfont \raggedleft ich\\du\\er/sie/es\\wir\\ihr\\sie} &    
\parbox[t][][t]{2cm}{\normalfont räche\\rächst\\rächt\\rächen\\rächt\\rächen} &
\parbox[t][][t]{2cm}{\normalfont rächte\\rächtest\\rächte\\rächten\\rächtet\\rächten}\\
\end{tabular}
\begin{tabular}{l}
\parbox[t][][t]{8cm}{}\\
\parbox[t][][t]{8cm}{\normalfont \small ['(transitive) to revenge; to avenge (reflexive,', 'sich rchen) to take revenge; to avenge oneself'] }\\
\end{tabular}
}
%===raten===
\card{\normalfont \Huge raten}{
\begin{tabular}{lll}
\parbox[t][][t]{2.0 cm}{\normalfont \raggedleft ich\\du\\er/sie/es\\wir\\ihr\\sie} &    
\parbox[t][][t]{2cm}{\normalfont rate\\rätst\\rät\\raten\\ratet\\raten} &
\parbox[t][][t]{2cm}{\normalfont riet\\rietest\\riet\\rieten\\rietet\\rieten}\\
\end{tabular}
\begin{tabular}{l}
\parbox[t][][t]{8cm}{}\\
\parbox[t][][t]{8cm}{\normalfont \small ['(intransitive) to advise; to recommend jemandem', 'etwas zu tun ratento advise someone to do', 'something Ich rate dir dazu, die Chance zu', 'ergreifen.I advise you to take the chance.', '(transitive) to guess Das rtst du nie.You will', 'never guess that. Lass mich raten!Let me guess!'] }\\
\end{tabular}
}
%===reagieren===
\card{\normalfont \Huge reagieren}{
\begin{tabular}{lll}
\parbox[t][][t]{2.0 cm}{\normalfont \raggedleft ich\\du\\er/sie/es\\wir\\ihr\\sie} &    
\parbox[t][][t]{2cm}{\normalfont reagiere\\reagierst\\reagiert\\reagieren\\reagiert\\reagieren} &
\parbox[t][][t]{2cm}{\normalfont reagierte\\reagiertest\\reagierte\\reagierten\\reagiertet\\reagierten}\\
\end{tabular}
\begin{tabular}{l}
\parbox[t][][t]{8cm}{}\\
\parbox[t][][t]{8cm}{\normalfont \small ['to react; to respond'] }\\
\end{tabular}
}
%===rechnen===
\card{\normalfont \Huge rechnen}{
\begin{tabular}{lll}
\parbox[t][][t]{2.0 cm}{\normalfont \raggedleft ich\\du\\er/sie/es\\wir\\ihr\\sie} &    
\parbox[t][][t]{2cm}{\normalfont rechne\\rechnest\\rechnet\\rechnen\\rechnet\\rechnen} &
\parbox[t][][t]{2cm}{\normalfont rechnete\\rechnetest\\rechnete\\rechneten\\rechnetet\\rechneten}\\
\end{tabular}
\begin{tabular}{l}
\parbox[t][][t]{8cm}{}\\
\parbox[t][][t]{8cm}{\normalfont \small ["to count, reckon, calculate, compute (with 'mit')", 'to expect 2010, Der Spiegel, issue 22/2010, page', '13: In ihren konomischen Eckwerten im April', 'rechnete die Regierung frs kommende Jahr noch mit', '3,43 Millionen Arbeitslosen. In their economic', 'guidance values of April the government still', 'expected 3.43 million unemployed for the coming', 'year.'] }\\
\end{tabular}
}
%===reden===
\card{\normalfont \Huge reden}{
\begin{tabular}{lll}
\parbox[t][][t]{2.0 cm}{\normalfont \raggedleft ich\\du\\er/sie/es\\wir\\ihr\\sie} &    
\parbox[t][][t]{2cm}{\normalfont rede\\redest\\redet\\reden\\redet\\reden} &
\parbox[t][][t]{2cm}{\normalfont redete\\redetest\\redete\\redeten\\redetet\\redeten}\\
\end{tabular}
\begin{tabular}{l}
\parbox[t][][t]{8cm}{}\\
\parbox[t][][t]{8cm}{\normalfont \small ['to speak; to talk'] }\\
\end{tabular}
}
%===regeln===
\card{\normalfont \Huge regeln}{
\begin{tabular}{lll}
\parbox[t][][t]{2.0 cm}{\normalfont \raggedleft ich\\du\\er/sie/es\\wir\\ihr\\sie} &    
\parbox[t][][t]{2cm}{\normalfont regleich regeleich regel\\regelst\\regelt\\regeln\\regelt\\regeln} &
\parbox[t][][t]{2cm}{\normalfont regelte\\regeltest\\regelte\\regelten\\regeltet\\regelten}\\
\end{tabular}
\begin{tabular}{l}
\parbox[t][][t]{8cm}{}\\
\parbox[t][][t]{8cm}{\normalfont \small ['to regulate to control (to adjust an apparatus)'] }\\
\end{tabular}
}
%===regieren===
\card{\normalfont \Huge regieren}{
\begin{tabular}{lll}
\parbox[t][][t]{2.0 cm}{\normalfont \raggedleft ich\\du\\er/sie/es\\wir\\ihr\\sie} &    
\parbox[t][][t]{2cm}{\normalfont regiere\\regierst\\regiert\\regieren\\regiert\\regieren} &
\parbox[t][][t]{2cm}{\normalfont regierte\\regiertest\\regierte\\regierten\\regiertet\\regierten}\\
\end{tabular}
\begin{tabular}{l}
\parbox[t][][t]{8cm}{}\\
\parbox[t][][t]{8cm}{\normalfont \small ['to govern'] }\\
\end{tabular}
}
%===regnen===
\card{\normalfont \Huge regnen}{
\begin{tabular}{lll}
\parbox[t][][t]{2.0 cm}{\normalfont \raggedleft ich\\du\\er/sie/es\\wir\\ihr\\sie} &    
\parbox[t][][t]{2cm}{\normalfont regne\\regnest\\regnet\\regnen\\regnet\\regnen} &
\parbox[t][][t]{2cm}{\normalfont regnete\\regnetest\\regnete\\regneten\\regnetet\\regneten}\\
\end{tabular}
\begin{tabular}{l}
\parbox[t][][t]{8cm}{}\\
\parbox[t][][t]{8cm}{\normalfont \small ['(chiefly impersonal) to rain'] }\\
\end{tabular}
}
%===reiben===
\card{\normalfont \Huge reiben}{
\begin{tabular}{lll}
\parbox[t][][t]{2.0 cm}{\normalfont \raggedleft ich\\du\\er/sie/es\\wir\\ihr\\sie} &    
\parbox[t][][t]{2cm}{\normalfont reibe\\reibst\\reibt\\reiben\\reibt\\reiben} &
\parbox[t][][t]{2cm}{\normalfont rieb\\riebst\\rieb\\rieben\\riebt\\rieben}\\
\end{tabular}
\begin{tabular}{l}
\parbox[t][][t]{8cm}{}\\
\parbox[t][][t]{8cm}{\normalfont \small ['(transitive or intransitive) to rub; to chafe sich', 'die Augen reiben  "to rub one\'s eyes" 2003,', 'Obsidian Voice, Schatten der Vergangenheit Ich', 'reib mir die Augen mit Zucker und Lgen und taumle', 'bewutlos whrend ich lache. I rub my eyes with', 'sugar and lies and stagger senseless as I laugh.', '(transitive) to grate pfel reibento grate apples'] }\\
\end{tabular}
}
%===reichen===
\card{\normalfont \Huge reichen}{
\begin{tabular}{lll}
\parbox[t][][t]{2.0 cm}{\normalfont \raggedleft ich\\du\\er/sie/es\\wir\\ihr\\sie} &    
\parbox[t][][t]{2cm}{\normalfont reiche\\reichst\\reicht\\reichen\\reicht\\reichen} &
\parbox[t][][t]{2cm}{\normalfont reichte\\reichtest\\reichte\\reichten\\reichtet\\reichten}\\
\end{tabular}
\begin{tabular}{l}
\parbox[t][][t]{8cm}{}\\
\parbox[t][][t]{8cm}{\normalfont \small ['to reach to pass, to hand, to give to be', 'sufficient 2010, Der Spiegel, issue 25/2010, page', '129: Zudem schrumpfen in Deutschland die Jahrgnge.', 'Das Angebot an Arbeitnehmern, auch im Top-Bereich,', 'wird bald nicht mehr reichen, um den Bedarf zu', 'decken. In addition the age groups are shrinking', 'in Germany. The supply of workers, also in the top', 'region, will soon be no longer sufficient to cover', 'the demand.'] }\\
\end{tabular}
}
%===reisen===
\card{\normalfont \Huge reisen}{
\begin{tabular}{lll}
\parbox[t][][t]{2.0 cm}{\normalfont \raggedleft ich\\du\\er/sie/es\\wir\\ihr\\sie} &    
\parbox[t][][t]{2cm}{\normalfont reise\\reist\\reist\\reisen\\reist\\reisen} &
\parbox[t][][t]{2cm}{\normalfont reiste\\reistest\\reiste\\reisten\\reistet\\reisten}\\
\end{tabular}
\begin{tabular}{l}
\parbox[t][][t]{8cm}{}\\
\parbox[t][][t]{8cm}{\normalfont \small ['(intransitive) to travel'] }\\
\end{tabular}
}
%===reißen===
\card{\normalfont \Huge reißen}{
\begin{tabular}{lll}
\parbox[t][][t]{2.0 cm}{\normalfont \raggedleft ich\\du\\er/sie/es\\wir\\ihr\\sie} &    
\parbox[t][][t]{2cm}{\normalfont reiße\\reißt\\reißt\\reißen\\reißt\\reißen} &
\parbox[t][][t]{2cm}{\normalfont riss\\risst\\riss\\rissen\\risst\\rissen}\\
\end{tabular}
\begin{tabular}{l}
\parbox[t][][t]{8cm}{}\\
\parbox[t][][t]{8cm}{\normalfont \small ['(transitive, auxiliary: "haben") to tear', '(something); to pull (something) apart 1918,', 'Elisabeth von Heyking, Aus dem Lande der', 'Ostseeritter, in Zwei Erzhlungen, Phillipp Reclam', 'jun., page 100: Es war als rngen bestndig zwei', 'Mchte um sie, als wrde sie wehrlos von ihnen hin', 'und her gerissen. It was as if two powers', 'struggled over her continuously, as if she was', 'torn to and fro by them defenselessly.', '(intransitive, auxiliary: "sein") to break; to', 'become torn apart (transitive, auxiliary: "haben")', 'to snatch; to wrench; to yank; to drag; to tug; to', 'pull on (something)'] }\\
\end{tabular}
}
%===reiten===
\card{\normalfont \Huge reiten}{
\begin{tabular}{lll}
\parbox[t][][t]{2.0 cm}{\normalfont \raggedleft ich\\du\\er/sie/es\\wir\\ihr\\sie} &    
\parbox[t][][t]{2cm}{\normalfont reite\\reitest\\reitet\\reiten\\reitet\\reiten} &
\parbox[t][][t]{2cm}{\normalfont ritt\\rittest\\ritt\\ritten\\rittet\\ritten}\\
\end{tabular}
\begin{tabular}{l}
\parbox[t][][t]{8cm}{}\\
\parbox[t][][t]{8cm}{\normalfont \small ['(intransitive) to travel on a mount; to ride; to', 'do horseriding (transitive) to ride (someone or', 'something), to ride on the back of'] }\\
\end{tabular}
}
%===rennen===
\card{\normalfont \Huge rennen}{
\begin{tabular}{lll}
\parbox[t][][t]{2.0 cm}{\normalfont \raggedleft ich\\du\\er/sie/es\\wir\\ihr\\sie} &    
\parbox[t][][t]{2cm}{\normalfont renne\\rennst\\rennt\\rennen\\rennt\\rennen} &
\parbox[t][][t]{2cm}{\normalfont rannte\\ranntest\\rannte\\rannten\\ranntet\\rannten}\\
\end{tabular}
\begin{tabular}{l}
\parbox[t][][t]{8cm}{}\\
\parbox[t][][t]{8cm}{\normalfont \small ['(intransitive, auxiliary: "sein") to run; to race;', 'to sprint (said of competing sportsmen, animals', 'etc.) So schnell wie Mike rennt niemand in der', 'Klasse.In this class, nobody runs as fast as Mike.', 'Der Gepard ist das Sugetier, welches am', 'schnellsten rennen kann.The cheetah is the mammal', 'which can run the fastest. (transitive, auxiliary:', '"haben") to run over (someone) jemanden zu Boden', 'rennen  to run someone to the ground'] }\\
\end{tabular}
}
%===reservieren===
\card{\normalfont \Huge reservieren}{
\begin{tabular}{lll}
\parbox[t][][t]{2.0 cm}{\normalfont \raggedleft ich\\du\\er/sie/es\\wir\\ihr\\sie} &    
\parbox[t][][t]{2cm}{\normalfont reserviere\\reservierst\\reserviert\\reservieren\\reserviert\\reservieren} &
\parbox[t][][t]{2cm}{\normalfont reservierte\\reserviertest\\reservierte\\reservierten\\reserviertet\\reservierten}\\
\end{tabular}
\begin{tabular}{l}
\parbox[t][][t]{8cm}{}\\
\parbox[t][][t]{8cm}{\normalfont \small ['to reserve'] }\\
\end{tabular}
}
%===retten===
\card{\normalfont \Huge retten}{
\begin{tabular}{lll}
\parbox[t][][t]{2.0 cm}{\normalfont \raggedleft ich\\du\\er/sie/es\\wir\\ihr\\sie} &    
\parbox[t][][t]{2cm}{\normalfont rette\\rettest\\rettet\\retten\\rettet\\retten} &
\parbox[t][][t]{2cm}{\normalfont rettete\\rettetest\\rettete\\retteten\\rettetet\\retteten}\\
\end{tabular}
\begin{tabular}{l}
\parbox[t][][t]{8cm}{}\\
\parbox[t][][t]{8cm}{\normalfont \small ['to save to rescue'] }\\
\end{tabular}
}
%===riechen===
\card{\normalfont \Huge riechen}{
\begin{tabular}{lll}
\parbox[t][][t]{2.0 cm}{\normalfont \raggedleft ich\\du\\er/sie/es\\wir\\ihr\\sie} &    
\parbox[t][][t]{2cm}{\normalfont rieche\\riechst\\riecht\\riechen\\riecht\\riechen} &
\parbox[t][][t]{2cm}{\normalfont roch\\rochst\\roch\\rochen\\rocht\\rochen}\\
\end{tabular}
\begin{tabular}{l}
\parbox[t][][t]{8cm}{}\\
\parbox[t][][t]{8cm}{\normalfont \small ['(transitive) to smell (something); to sniff', '(something) (intransitive) to use the sense of', 'smell; to detect a smell (intransitive) to smell', 'something Ich rieche dein Parfm.I smell your', 'perfume. (intransitive) to reek; to smell bad', '(intransitive) to smell (nach like) Im Haus riecht', 'es nach (+ Dat) gebratenem Fisch.In the house it', 'smells like fried fish. (transitive, slang) to', 'tolerate (someone); to stand (someone) Ich kann', 'ihn nicht riechen.I cannot stand him.'] }\\
\end{tabular}
}
%===ringen===
\card{\normalfont \Huge ringen}{
\begin{tabular}{lll}
\parbox[t][][t]{2.0 cm}{\normalfont \raggedleft ich\\du\\er/sie/es\\wir\\ihr\\sie} &    
\parbox[t][][t]{2cm}{\normalfont ringe\\ringst\\ringt\\ringen\\ringt\\ringen} &
\parbox[t][][t]{2cm}{\normalfont rang\\rangst\\rang\\rangen\\rangt\\rangen}\\
\end{tabular}
\begin{tabular}{l}
\parbox[t][][t]{8cm}{}\\
\parbox[t][][t]{8cm}{\normalfont \small ['(intransitive) to wrestle (intransitive) to', 'struggle 1918, Elisabeth von Heyking, Aus dem', 'Lande der Ostseeritter, in Zwei Erzhlungen,', 'Phillipp Reclam jun., page 100: Es war als rngen', 'bestndig zwei Mchte um sie, als wrde sie wehrlos', 'von ihnen hin und her gerissen. It was as if two', 'powers struggled over her continuously, as if she', 'was torn to and fro by them defenselessly.', '(transitive) to wring (for example "hands"; but', 'not "clothes", for that see wringen)'] }\\
\end{tabular}
}
%===rufen===
\card{\normalfont \Huge rufen}{
\begin{tabular}{lll}
\parbox[t][][t]{2.0 cm}{\normalfont \raggedleft ich\\du\\er/sie/es\\wir\\ihr\\sie} &    
\parbox[t][][t]{2cm}{\normalfont rufe\\rufst\\ruft\\rufen\\ruft\\rufen} &
\parbox[t][][t]{2cm}{\normalfont rief\\riefst\\rief\\riefen\\rieft\\riefen}\\
\end{tabular}
\begin{tabular}{l}
\parbox[t][][t]{8cm}{}\\
\parbox[t][][t]{8cm}{\normalfont \small ['(intransitive) to call out; to shout; to cry; to', 'shriek um Hilfe rufen  "to cry for help"', '(intransitive, with "nach ...") to call (for', 'someone); to request the presence (of someone)', '(transitive) to call (something) out (with dative', 'object) to ask (someone) to do something; to call', 'for (someone) to do something jemandem zu', 'antworten rufen  "to ask someone to answer"', '(transitive) to call (someone), e.g. by telephone'] }\\
\end{tabular}
}
%===ruhen===
\card{\normalfont \Huge ruhen}{
\begin{tabular}{lll}
\parbox[t][][t]{2.0 cm}{\normalfont \raggedleft ich\\du\\er/sie/es\\wir\\ihr\\sie} &    
\parbox[t][][t]{2cm}{\normalfont ruhe\\ruhst\\ruht\\ruhen\\ruht\\ruhen} &
\parbox[t][][t]{2cm}{\normalfont ruhte\\ruhtest\\ruhte\\ruhten\\ruhtet\\ruhten}\\
\end{tabular}
\begin{tabular}{l}
\parbox[t][][t]{8cm}{}\\
\parbox[t][][t]{8cm}{\normalfont \small ['(intransitive, of something animate) to rest; to', 'sleep (intransitive, euphemistic) to be buried, to', 'lie 1918, Elisabeth von Heyking, Die Orgelpfeifen,', 'in: Zwei Erzhlungen, Phillipp Reclam jun. Verlag,', 'page 31: Bei einem Patrouillenritt, zu dem er sich', 'freiwillig gemeldet, war der lteste der Enkel', 'gefallen. Ruhte nun fern in Feindesland. On a', 'patrolling ride, for which he had volunteered, the', 'oldest of the grandchildren had died. He now lay', 'far away in enemy country. (intransitive, of', 'something inanimate) to be positioned; to rest', '(intransitive, of a process or event) to stall; to', 'be suspended'] }\\
\end{tabular}
}
%===rühren===
\card{\normalfont \Huge rühren}{
\begin{tabular}{lll}
\parbox[t][][t]{2.0 cm}{\normalfont \raggedleft ich\\du\\er/sie/es\\wir\\ihr\\sie} &    
\parbox[t][][t]{2cm}{\normalfont rühre\\rührst\\rührt\\rühren\\rührt\\rühren} &
\parbox[t][][t]{2cm}{\normalfont rührte\\rührtest\\rührte\\rührten\\rührtet\\rührten}\\
\end{tabular}
\begin{tabular}{l}
\parbox[t][][t]{8cm}{}\\
\parbox[t][][t]{8cm}{\normalfont \small ['(transitive or intransitive) to stir (usually', 'transitive) to stir; to move; to cause an emotion,', 'especially sentimentality or compassion', '(reflexive) to stir (oneself); to move slightly'] }\\
\end{tabular}
}
%===sagen===
\card{\normalfont \Huge sagen}{
\begin{tabular}{lll}
\parbox[t][][t]{2.0 cm}{\normalfont \raggedleft ich\\du\\er/sie/es\\wir\\ihr\\sie} &    
\parbox[t][][t]{2cm}{\normalfont sage\\sagst\\sagt\\sagen\\sagt\\sagen} &
\parbox[t][][t]{2cm}{\normalfont sagte\\sagtest\\sagte\\sagten\\sagtet\\sagten}\\
\end{tabular}
\begin{tabular}{l}
\parbox[t][][t]{8cm}{}\\
\parbox[t][][t]{8cm}{\normalfont \small ['(transitive) to say (to pronounce; communicate', 'verbally) 1931, Arthur Schnitzler, Flucht in die', 'Finsternis, S. Fischer Verlag, page 105: Sie', 'schwiegen lange. Als er endlich etwas sagen', 'wollte, wehrte sie leise ab. "Heute nichts mehr,', 'ich bitte dich darum".They were silent for a long', 'time. When he finally wanted to say something, she', 'softly refused. "Nothing more today, I beg you for', 'that." (transitive) to tell (to inform (someone)', 'verbally) (transitive) to mean (to convey or', 'signify) (with etwas) to speak up'] }\\
\end{tabular}
}
%===sammeln===
\card{\normalfont \Huge sammeln}{
\begin{tabular}{lll}
\parbox[t][][t]{2.0 cm}{\normalfont \raggedleft ich\\du\\er/sie/es\\wir\\ihr\\sie} &    
\parbox[t][][t]{2cm}{\normalfont sammleich sammeleich sammel\\sammelst\\sammelt\\sammeln\\sammelt\\sammeln} &
\parbox[t][][t]{2cm}{\normalfont sammelte\\sammeltest\\sammelte\\sammelten\\sammeltet\\sammelten}\\
\end{tabular}
\begin{tabular}{l}
\parbox[t][][t]{8cm}{}\\
\parbox[t][][t]{8cm}{\normalfont \small ['(transitive) to gather; to assemble; to collect', '(reflexive) to congregate; to assemble; to meet', '(intransitive) to collect money'] }\\
\end{tabular}
}
%===saufen===
\card{\normalfont \Huge saufen}{
\begin{tabular}{lll}
\parbox[t][][t]{2.0 cm}{\normalfont \raggedleft ich\\du\\er/sie/es\\wir\\ihr\\sie} &    
\parbox[t][][t]{2cm}{\normalfont saufe\\säufst\\säuft\\saufen\\sauft\\saufen} &
\parbox[t][][t]{2cm}{\normalfont soff\\soffst\\soff\\soffen\\sofft\\soffen}\\
\end{tabular}
\begin{tabular}{l}
\parbox[t][][t]{8cm}{}\\
\parbox[t][][t]{8cm}{\normalfont \small ['(transitive or intransitive, of an animal) to', 'drink (transitive or intransitive, informal, of a', 'person) to drink, especially in large quantities;', 'to quaff; to swig (intransitive, informal, of a', 'person) to booze; to consume alcohol excessively', '(in terms of quantity or frequence)'] }\\
\end{tabular}
}
%===saugen===
\card{\normalfont \Huge saugen}{
\begin{tabular}{lll}
\parbox[t][][t]{2.0 cm}{\normalfont \raggedleft ich\\du\\er/sie/es\\wir\\ihr\\sie} &    
\parbox[t][][t]{2cm}{\normalfont sauge\\saugst\\saugt\\saugen\\saugt\\saugen} &
\parbox[t][][t]{2cm}{\normalfont sog\\sogst\\sog\\sogen\\sogt\\sogen}\\
\end{tabular}
\begin{tabular}{l}
\parbox[t][][t]{8cm}{}\\
\parbox[t][][t]{8cm}{\normalfont \small ['(intransitive) to suck; to create underpressure', "with a tube-like object such as one's mouth an", 'etwas saugen  to suck on something (transitive) to', 'suck (something) (colloquial, transitive) to', 'download something'] }\\
\end{tabular}
}
%===schaden===
\card{\normalfont \Huge schaden}{
\begin{tabular}{lll}
\parbox[t][][t]{2.0 cm}{\normalfont \raggedleft ich\\du\\er/sie/es\\wir\\ihr\\sie} &    
\parbox[t][][t]{2cm}{\normalfont schade\\schadest\\schadet\\schaden\\schadet\\schaden} &
\parbox[t][][t]{2cm}{\normalfont schadete\\schadetest\\schadete\\schadeten\\schadetet\\schadeten}\\
\end{tabular}
\begin{tabular}{l}
\parbox[t][][t]{8cm}{}\\
\parbox[t][][t]{8cm}{\normalfont \small ['(intransitive) to damage; to harm; to hurt', 'jemandem schaden  to hurt someone'] }\\
\end{tabular}
}
%===schaffen===
\card{\normalfont \Huge schaffen}{
\begin{tabular}{lll}
\parbox[t][][t]{2.0 cm}{\normalfont \raggedleft ich\\du\\er/sie/es\\wir\\ihr\\sie} &    
\parbox[t][][t]{2cm}{\normalfont schaffe\\schaffst\\schafft\\schaffen\\schafft\\schaffen} &
\parbox[t][][t]{2cm}{\normalfont schuf\\schufst\\schuf\\schufen\\schuft\\schufen}\\
\end{tabular}
\begin{tabular}{l}
\parbox[t][][t]{8cm}{}\\
\parbox[t][][t]{8cm}{\normalfont \small ['(transitive) to create; to produce; to make; to', 'cause; to establish 1919, Rudolf Steiner,', 'Allgemeine Menschenkunde als Grundlage der', 'Pdagogik, Dornach, 1992, p.143 Es kann aber nur', 'Abhilfe geschaffen werden, wenn der Lehrer, der', 'Unterrichtende, sich selbst fortwhrend herausheben', 'will aus dem Hausbackenen, Pedantischen,', 'Philistrsen. This can only be solved if the', 'teacher constantly releases himself from -- or', 'lifts himself out of -- what is unadventurous,', 'pedantic, narrow-minded.'] }\\
\end{tabular}
}
%===schalten===
\card{\normalfont \Huge schalten}{
\begin{tabular}{lll}
\parbox[t][][t]{2.0 cm}{\normalfont \raggedleft ich\\du\\er/sie/es\\wir\\ihr\\sie} &    
\parbox[t][][t]{2cm}{\normalfont schalte\\schaltest\\schaltet\\schalten\\schaltet\\schalten} &
\parbox[t][][t]{2cm}{\normalfont schaltete\\schaltetest\\schaltete\\schalteten\\schaltetet\\schalteten}\\
\end{tabular}
\begin{tabular}{l}
\parbox[t][][t]{8cm}{}\\
\parbox[t][][t]{8cm}{\normalfont \small ['to switch'] }\\
\end{tabular}
}
%===schätzen===
\card{\normalfont \Huge schätzen}{
\begin{tabular}{lll}
\parbox[t][][t]{2.0 cm}{\normalfont \raggedleft ich\\du\\er/sie/es\\wir\\ihr\\sie} &    
\parbox[t][][t]{2cm}{\normalfont schätze\\schätzt\\schätzt\\schätzen\\schätzt\\schätzen} &
\parbox[t][][t]{2cm}{\normalfont schätzte\\schätztest\\schätzte\\schätzten\\schätztet\\schätzten}\\
\end{tabular}
\begin{tabular}{l}
\parbox[t][][t]{8cm}{}\\
\parbox[t][][t]{8cm}{\normalfont \small ['(transitive) to estimate; to guess; to suppose; to', 'assume (transitive) to evaluate; to price; to', 'assess (transitive) to esteem; to appreciate; to', 'value'] }\\
\end{tabular}
}
%===schauen===
\card{\normalfont \Huge schauen}{
\begin{tabular}{lll}
\parbox[t][][t]{2.0 cm}{\normalfont \raggedleft ich\\du\\er/sie/es\\wir\\ihr\\sie} &    
\parbox[t][][t]{2cm}{\normalfont schaue\\schaust\\schaut\\schauen\\schaut\\schauen} &
\parbox[t][][t]{2cm}{\normalfont schaute\\schautest\\schaute\\schauten\\schautet\\schauten}\\
\end{tabular}
\begin{tabular}{l}
\parbox[t][][t]{8cm}{}\\
\parbox[t][][t]{8cm}{\normalfont \small ['to look (at something)'] }\\
\end{tabular}
}
%===scheiden===
\card{\normalfont \Huge scheiden}{
\begin{tabular}{lll}
\parbox[t][][t]{2.0 cm}{\normalfont \raggedleft ich\\du\\er/sie/es\\wir\\ihr\\sie} &    
\parbox[t][][t]{2cm}{\normalfont scheide\\scheidest\\scheidet\\scheiden\\scheidet\\scheiden} &
\parbox[t][][t]{2cm}{\normalfont schied\\schiedest\\schied\\schieden\\schiedet\\schieden}\\
\end{tabular}
\begin{tabular}{l}
\parbox[t][][t]{8cm}{}\\
\parbox[t][][t]{8cm}{\normalfont \small ['(transitive, dated, literary) to separate', '(intransitive or reflexive, dated, literary) to', 'leave one another; to part; to be separated; to be', 'divided (transitive) to dissolve (a marriage); to', 'divorce (a couple) Der Richter weigerte sich, die', 'Ehe zu scheiden.  "The judge refused to dissolve', 'the marriage." (transitive, with lassen) to have', '(a marriage) dissolved Sie wollen ihre Ehe', 'scheiden lassen.  "They want to dissolve their', 'marriage." (reflexive, with lassen) to divorce', "(one's spouse); to get a divorce (from one's", 'spouse) Meine Frau will sich von mir scheiden', 'lassen.  "My wife wants to divorce me."'] }\\
\end{tabular}
}
%===scheinen===
\card{\normalfont \Huge scheinen}{
\begin{tabular}{lll}
\parbox[t][][t]{2.0 cm}{\normalfont \raggedleft ich\\du\\er/sie/es\\wir\\ihr\\sie} &    
\parbox[t][][t]{2cm}{\normalfont scheine\\scheinst\\scheint\\scheinen\\scheint\\scheinen} &
\parbox[t][][t]{2cm}{\normalfont schien\\schienst\\schien\\schienen\\schient\\schienen}\\
\end{tabular}
\begin{tabular}{l}
\parbox[t][][t]{8cm}{}\\
\parbox[t][][t]{8cm}{\normalfont \small ['to shine; to gleam 2002, Peter Bichsel,', 'Eisenbahnfahren, Insel Verlag 2015, p. 12: Wenn', 'man aus der Kirche kommt, schneit es, oder es', 'regnet, oder die Sonne scheint. When you come out', "of the church, it's snowing, or it's raining, or", "the sun's shining. to seem; to appear; to look Es", 'scheint mir, dass It seems to me that'] }\\
\end{tabular}
}
%===scheitern===
\card{\normalfont \Huge scheitern}{
\begin{tabular}{lll}
\parbox[t][][t]{2.0 cm}{\normalfont \raggedleft ich\\du\\er/sie/es\\wir\\ihr\\sie} &    
\parbox[t][][t]{2cm}{\normalfont scheitreich scheitereich scheiter\\scheiterst\\scheitert\\scheitern\\scheitert\\scheitern} &
\parbox[t][][t]{2cm}{\normalfont scheiterte\\scheitertest\\scheiterte\\scheiterten\\scheitertet\\scheiterten}\\
\end{tabular}
\begin{tabular}{l}
\parbox[t][][t]{8cm}{}\\
\parbox[t][][t]{8cm}{\normalfont \small ['to fail'] }\\
\end{tabular}
}
%===schelten===
\card{\normalfont \Huge schelten}{
\begin{tabular}{lll}
\parbox[t][][t]{2.0 cm}{\normalfont \raggedleft ich\\du\\er/sie/es\\wir\\ihr\\sie} &    
\parbox[t][][t]{2cm}{\normalfont schelte\\schiltst\\schilt\\schelten\\scheltet\\schelten} &
\parbox[t][][t]{2cm}{\normalfont schalt\\schaltest\\schalt\\schalten\\schaltet\\schalten}\\
\end{tabular}
\begin{tabular}{l}
\parbox[t][][t]{8cm}{}\\
\parbox[t][][t]{8cm}{\normalfont \small ['(transitive) to scold (someone); to rebuke; to', 'chide (intransitive) to scold'] }\\
\end{tabular}
}
%===schenken===
\card{\normalfont \Huge schenken}{
\begin{tabular}{lll}
\parbox[t][][t]{2.0 cm}{\normalfont \raggedleft ich\\du\\er/sie/es\\wir\\ihr\\sie} &    
\parbox[t][][t]{2cm}{\normalfont schenke\\schenkst\\schenkt\\schenken\\schenkt\\schenken} &
\parbox[t][][t]{2cm}{\normalfont schenkte\\schenktest\\schenkte\\schenkten\\schenktet\\schenkten}\\
\end{tabular}
\begin{tabular}{l}
\parbox[t][][t]{8cm}{}\\
\parbox[t][][t]{8cm}{\normalfont \small ['(archaic) to pour from a vessel, to serve', 'Synonyms: ausschenken, eingieen to give as a', 'present, to gift Synonyms: verehren,', 'schenkweisebereignen, zuwenden, bescheren'] }\\
\end{tabular}
}
%===schicken===
\card{\normalfont \Huge schicken}{
\begin{tabular}{lll}
\parbox[t][][t]{2.0 cm}{\normalfont \raggedleft ich\\du\\er/sie/es\\wir\\ihr\\sie} &    
\parbox[t][][t]{2cm}{\normalfont schicke\\schickst\\schickt\\schicken\\schickt\\schicken} &
\parbox[t][][t]{2cm}{\normalfont schickte\\schicktest\\schickte\\schickten\\schicktet\\schickten}\\
\end{tabular}
\begin{tabular}{l}
\parbox[t][][t]{8cm}{}\\
\parbox[t][][t]{8cm}{\normalfont \small ['(transitive) to send (reflexive) to hurry (rare)', '(reflexive) to be decent, to be appropriate'] }\\
\end{tabular}
}
%===schieben===
\card{\normalfont \Huge schieben}{
\begin{tabular}{lll}
\parbox[t][][t]{2.0 cm}{\normalfont \raggedleft ich\\du\\er/sie/es\\wir\\ihr\\sie} &    
\parbox[t][][t]{2cm}{\normalfont schiebe\\schiebst\\schiebt\\schieben\\schiebt\\schieben} &
\parbox[t][][t]{2cm}{\normalfont schob\\schobst\\schob\\schoben\\schobt\\schoben}\\
\end{tabular}
\begin{tabular}{l}
\parbox[t][][t]{8cm}{}\\
\parbox[t][][t]{8cm}{\normalfont \small ['(transitive) to push; to shove (transitive) to', 'slide; to slip; to put (intransitive) to push', '(transitive or intransitive, colloquial) to', 'traffic (something); to profiteer; to racketeer'] }\\
\end{tabular}
}
%===schießen===
\card{\normalfont \Huge schießen}{
\begin{tabular}{lll}
\parbox[t][][t]{2.0 cm}{\normalfont \raggedleft ich\\du\\er/sie/es\\wir\\ihr\\sie} &    
\parbox[t][][t]{2cm}{\normalfont schieße\\schießt\\schießt\\schießen\\schießt\\schießen} &
\parbox[t][][t]{2cm}{\normalfont schoss\\schosst\\schoss\\schossen\\schosst\\schossen}\\
\end{tabular}
\begin{tabular}{l}
\parbox[t][][t]{8cm}{}\\
\parbox[t][][t]{8cm}{\normalfont \small ['(transitive or intransitive, auxiliary: "haben")', 'to shoot; to fire auf etwas schieen  to shoot at', 'something Salut schieen  to fire a salute (sports,', 'auxiliary: "haben") to kick; to shoot ein Tor', 'schieen  to score (literally, "to shoot a goal")', 'den Ball ins Tor schieen  to shoot the ball into', 'the goal (photography, auxiliary: "haben") to', 'shoot (slang, drugs, auxiliary: "haben") to shoot', 'up (intransitive, auxiliary: "sein") to dart; to', 'shoot; to rush; to gush durch etwas schieen  to', 'rush through something aus etwas schieen  to gush', 'from something'] }\\
\end{tabular}
}
%===schimpfen===
\card{\normalfont \Huge schimpfen}{
\begin{tabular}{lll}
\parbox[t][][t]{2.0 cm}{\normalfont \raggedleft ich\\du\\er/sie/es\\wir\\ihr\\sie} &    
\parbox[t][][t]{2cm}{\normalfont schimpfe\\schimpfst\\schimpft\\schimpfen\\schimpft\\schimpfen} &
\parbox[t][][t]{2cm}{\normalfont schimpfte\\schimpftest\\schimpfte\\schimpften\\schimpftet\\schimpften}\\
\end{tabular}
\begin{tabular}{l}
\parbox[t][][t]{8cm}{}\\
\parbox[t][][t]{8cm}{\normalfont \small ['to tell off, scold'] }\\
\end{tabular}
}
%===schinden===
\card{\normalfont \Huge schinden}{
\begin{tabular}{lll}
\parbox[t][][t]{2.0 cm}{\normalfont \raggedleft ich\\du\\er/sie/es\\wir\\ihr\\sie} &    
\parbox[t][][t]{2cm}{\normalfont schinde\\schindest\\schindet\\schinden\\schindet\\schinden} &
\parbox[t][][t]{2cm}{\normalfont schund\\schundest\\schund\\schunden\\schundet\\schunden}\\
\end{tabular}
\begin{tabular}{l}
\parbox[t][][t]{8cm}{}\\
\parbox[t][][t]{8cm}{\normalfont \small ['to mistreat'] }\\
\end{tabular}
}
%===schlafen===
\card{\normalfont \Huge schlafen}{
\begin{tabular}{lll}
\parbox[t][][t]{2.0 cm}{\normalfont \raggedleft ich\\du\\er/sie/es\\wir\\ihr\\sie} &    
\parbox[t][][t]{2cm}{\normalfont schlafe\\schläfst\\schläft\\schlafen\\schlaft\\schlafen} &
\parbox[t][][t]{2cm}{\normalfont schlief\\schliefst\\schlief\\schliefen\\schlieft\\schliefen}\\
\end{tabular}
\begin{tabular}{l}
\parbox[t][][t]{8cm}{}\\
\parbox[t][][t]{8cm}{\normalfont \small ['to sleep Ich habe diese Nacht gut geschlafen.I', 'slept well last night. schlafen gehen  to go to', 'sleep sich schlafen legen  to lie down to sleep', '(with ber + accusative) to sleep on something; to', 'wait and think before making a decision Ich werde', "eine Nacht darber schlafen.I'll sleep on it", 'tonight. (with mit) to sleep with; to have sex'] }\\
\end{tabular}
}
%===schlagen===
\card{\normalfont \Huge schlagen}{
\begin{tabular}{lll}
\parbox[t][][t]{2.0 cm}{\normalfont \raggedleft ich\\du\\er/sie/es\\wir\\ihr\\sie} &    
\parbox[t][][t]{2cm}{\normalfont schlage\\schlägst\\schlägt\\schlagen\\schlagt\\schlagen} &
\parbox[t][][t]{2cm}{\normalfont schlug\\schlugst\\schlug\\schlugen\\schlugt\\schlugen}\\
\end{tabular}
\begin{tabular}{l}
\parbox[t][][t]{8cm}{}\\
\parbox[t][][t]{8cm}{\normalfont \small ['(transitive) to beat; to hit; to knock; to strike;', 'to punch; to hammer; to pound (transitive,', 'figuratively) to beat; to win against; to defeat', '(intransitive) to beat; to strike repeatedly; to', 'pound (transitive, cooking) to beat; to whip; to', 'mix food in a rapid aerating fashion (of a clock)', 'to chime 1919, Walther Kabel, Irrende Seelen,', 'Werner Dietsch Verlag, page 41: Die kleine', 'Stutzuhr auf dem Kamin, ein letzter Rest der Habe', 'meiner Eltern, schlug zehn. The small bracket', 'clock on the chimney, a last remnant of the', 'belongings of my parents, chimed ten. (reflexive)', 'to fight'] }\\
\end{tabular}
}
%===schleichen===
\card{\normalfont \Huge schleichen}{
\begin{tabular}{lll}
\parbox[t][][t]{2.0 cm}{\normalfont \raggedleft ich\\du\\er/sie/es\\wir\\ihr\\sie} &    
\parbox[t][][t]{2cm}{\normalfont schleiche\\schleichst\\schleicht\\schleichen\\schleicht\\schleichen} &
\parbox[t][][t]{2cm}{\normalfont schlich\\schlichst\\schlich\\schlichen\\schlicht\\schlichen}\\
\end{tabular}
\begin{tabular}{l}
\parbox[t][][t]{8cm}{}\\
\parbox[t][][t]{8cm}{\normalfont \small ['(intransitive) to move in a quiet and', 'inconspicuous manner, hence often slowly and/or', 'ducked: to crawl, to sneak, to steal, to prowl, to', 'creep, to slither (of a snake) (reflexive, with', 'some adverb of location) to go somewhere in the', 'above manner (informal, intransitive) to move', 'slowly (especially in a car) Was schleicht der da', "so?  Why's that guy driving so slowly? (informal,", 'reflexive) to slink away; to leave meekly or', 'sheepishly'] }\\
\end{tabular}
}
%===schleifen===
\card{\normalfont \Huge schleifen}{
\begin{tabular}{lll}
\parbox[t][][t]{2.0 cm}{\normalfont \raggedleft ich\\du\\er/sie/es\\wir\\ihr\\sie} &    
\parbox[t][][t]{2cm}{\normalfont schleife\\schleifst\\schleift\\schleifen\\schleift\\schleifen} &
\parbox[t][][t]{2cm}{\normalfont schliff\\schliffst\\schliff\\schliffen\\schlifft\\schliffen}\\
\end{tabular}
\begin{tabular}{l}
\parbox[t][][t]{8cm}{}\\
\parbox[t][][t]{8cm}{\normalfont \small ['to whet, to move across a surface abrasively'] }\\
\end{tabular}
}
%===schlingen===
\card{\normalfont \Huge schlingen}{
\begin{tabular}{lll}
\parbox[t][][t]{2.0 cm}{\normalfont \raggedleft ich\\du\\er/sie/es\\wir\\ihr\\sie} &    
\parbox[t][][t]{2cm}{\normalfont schlinge\\schlingst\\schlingt\\schlingen\\schlingt\\schlingen} &
\parbox[t][][t]{2cm}{\normalfont schlang\\schlangst\\schlang\\schlangen\\schlangt\\schlangen}\\
\end{tabular}
\begin{tabular}{l}
\parbox[t][][t]{8cm}{}\\
\parbox[t][][t]{8cm}{\normalfont \small ['(transitive) to wind (transitive) to tie, knot', '(reflexive) to wind, coil (reflexive, botany) to', 'creep, twine'] }\\
\end{tabular}
}
%===schmecken===
\card{\normalfont \Huge schmecken}{
\begin{tabular}{lll}
\parbox[t][][t]{2.0 cm}{\normalfont \raggedleft ich\\du\\er/sie/es\\wir\\ihr\\sie} &    
\parbox[t][][t]{2cm}{\normalfont schmecke\\schmeckst\\schmeckt\\schmecken\\schmeckt\\schmecken} &
\parbox[t][][t]{2cm}{\normalfont schmeckte\\schmecktest\\schmeckte\\schmeckten\\schmecktet\\schmeckten}\\
\end{tabular}
\begin{tabular}{l}
\parbox[t][][t]{8cm}{}\\
\parbox[t][][t]{8cm}{\normalfont \small ['to taste Das Bier schmeckt sehr gut. The beer', 'tastes very good. to enjoy (the taste of', 'something)'] }\\
\end{tabular}
}
%===schmeißen===
\card{\normalfont \Huge schmeißen}{
\begin{tabular}{lll}
\parbox[t][][t]{2.0 cm}{\normalfont \raggedleft ich\\du\\er/sie/es\\wir\\ihr\\sie} &    
\parbox[t][][t]{2cm}{\normalfont schmeiße\\schmeißt\\schmeißt\\schmeißen\\schmeißt\\schmeißen} &
\parbox[t][][t]{2cm}{\normalfont schmiss\\schmisst\\schmiss\\schmissen\\schmisst\\schmissen}\\
\end{tabular}
\begin{tabular}{l}
\parbox[t][][t]{8cm}{}\\
\parbox[t][][t]{8cm}{\normalfont \small ['(transitive or intransitive  with mit) to throw;', 'to fling; to hurl; to chuck; to slam Er schmiss', 'seine Tasche aufs Bett. He flung his bag on the', 'bed. Sie schmeit mit Kastanien auf die andern', "Kinder. She's throwing chestnuts at the other", 'children. (transitive, informal) to drop by', 'accident Pass auf, dass du die Flasche nicht auf', 'den Boden schmeit. Be careful not to drop the', 'bottle on the ground. (transitive, colloquial) to', 'manage; to organize;  to run Ich muss hier heut', 'den ganzen Laden alleine schmeien. I have to run', 'this whole thing on my own today.'] }\\
\end{tabular}
}
%===schmelzen===
\card{\normalfont \Huge schmelzen}{
\begin{tabular}{lll}
\parbox[t][][t]{2.0 cm}{\normalfont \raggedleft ich\\du\\er/sie/es\\wir\\ihr\\sie} &    
\parbox[t][][t]{2cm}{\normalfont schmelze\\schmilzt\\schmilzt\\schmelzen\\schmelzt\\schmelzen} &
\parbox[t][][t]{2cm}{\normalfont schmolz\\schmolzt\\schmolz\\schmolzen\\schmolzt\\schmolzen}\\
\end{tabular}
\begin{tabular}{l}
\parbox[t][][t]{8cm}{}\\
\parbox[t][][t]{8cm}{\normalfont \small ['(transitive, auxiliary: "haben") to melt; to', 'liquefy (transitive, metallurgy, auxiliary:', '"haben") to smelt; to fuse (intransitive,', 'auxiliary: "sein") to melt; to liquefy', '(intransitive, figuratively, auxiliary: "sein") to', 'dwindle; to melt away'] }\\
\end{tabular}
}
%===schnauben===
\card{\normalfont \Huge schnauben}{
\begin{tabular}{lll}
\parbox[t][][t]{2.0 cm}{\normalfont \raggedleft ich\\du\\er/sie/es\\wir\\ihr\\sie} &    
\parbox[t][][t]{2cm}{\normalfont schnaube\\schnaubst\\schnaubt\\schnauben\\schnaubt\\schnauben} &
\parbox[t][][t]{2cm}{\normalfont schnaubte\\schnaubtest\\schnaubte\\schnaubten\\schnaubtet\\schnaubten}\\
\end{tabular}
\begin{tabular}{l}
\parbox[t][][t]{8cm}{}\\
\parbox[t][][t]{8cm}{\normalfont \small ['(intransitive) to snort, to pant (to breathe', 'loudly)'] }\\
\end{tabular}
}
%===schneiden===
\card{\normalfont \Huge schneiden}{
\begin{tabular}{lll}
\parbox[t][][t]{2.0 cm}{\normalfont \raggedleft ich\\du\\er/sie/es\\wir\\ihr\\sie} &    
\parbox[t][][t]{2cm}{\normalfont schneide\\schneidest\\schneidet\\schneiden\\schneidet\\schneiden} &
\parbox[t][][t]{2cm}{\normalfont schnitt\\schnittest\\schnitt\\schnitten\\schnittet\\schnitten}\\
\end{tabular}
\begin{tabular}{l}
\parbox[t][][t]{8cm}{}\\
\parbox[t][][t]{8cm}{\normalfont \small ['(transitive) to cut; to carve; to slice', '(transitive) to pare; to clip; to mow; to prune;', 'to trim (transitive, driving, figuratively) to cut', '(someone) off; to cut in on (someone) (transitive,', 'film) to edit (transitive or reflexive) to', 'intersect Die beiden Geraden schneiden sich.', '"Both streets intersect." (reflexive) to cut', '(oneself) (reflexive, colloquial) to delude', '(oneself); to become mistaken to avoid somebody', '(to cut someone)'] }\\
\end{tabular}
}
%===schneien===
\card{\normalfont \Huge schneien}{
\begin{tabular}{lll}
\parbox[t][][t]{2.0 cm}{\normalfont \raggedleft ich\\du\\er/sie/es\\wir\\ihr\\sie} &    
\parbox[t][][t]{2cm}{\normalfont schneie\\schneist\\schneit\\schneien\\schneit\\schneien} &
\parbox[t][][t]{2cm}{\normalfont schneite\\schneitest\\schneite\\schneiten\\schneitet\\schneiten}\\
\end{tabular}
\begin{tabular}{l}
\parbox[t][][t]{8cm}{}\\
\parbox[t][][t]{8cm}{\normalfont \small ['to snow'] }\\
\end{tabular}
}
%===schrauben===
\card{\normalfont \Huge schrauben}{
\begin{tabular}{lll}
\parbox[t][][t]{2.0 cm}{\normalfont \raggedleft ich\\du\\er/sie/es\\wir\\ihr\\sie} &    
\parbox[t][][t]{2cm}{\normalfont schraube\\schraubst\\schraubt\\schrauben\\schraubt\\schrauben} &
\parbox[t][][t]{2cm}{\normalfont schraubte\\schraubtest\\schraubte\\schraubten\\schraubtet\\schraubten}\\
\end{tabular}
\begin{tabular}{l}
\parbox[t][][t]{8cm}{}\\
\parbox[t][][t]{8cm}{\normalfont \small ['to screw to unscrew'] }\\
\end{tabular}
}
%===schrecken===
\card{\normalfont \Huge schrecken}{
\begin{tabular}{lll}
\parbox[t][][t]{2.0 cm}{\normalfont \raggedleft ich\\du\\er/sie/es\\wir\\ihr\\sie} &    
\parbox[t][][t]{2cm}{\normalfont schrecke\\schreckst\\schreckt\\schrecken\\schreckt\\schrecken} &
\parbox[t][][t]{2cm}{\normalfont schreckte\\schrecktest\\schreckte\\schreckten\\schrecktet\\schreckten}\\
\end{tabular}
\begin{tabular}{l}
\parbox[t][][t]{8cm}{}\\
\parbox[t][][t]{8cm}{\normalfont \small ['(transitive) to frighten, to scare. (transitive)', 'to intimidate or discourage, to deter.'] }\\
\end{tabular}
}
%===schreiben===
\card{\normalfont \Huge schreiben}{
\begin{tabular}{lll}
\parbox[t][][t]{2.0 cm}{\normalfont \raggedleft ich\\du\\er/sie/es\\wir\\ihr\\sie} &    
\parbox[t][][t]{2cm}{\normalfont schreibe\\schreibst\\schreibt\\schreiben\\schreibt\\schreiben} &
\parbox[t][][t]{2cm}{\normalfont schrieb\\schriebst\\schrieb\\schrieben\\schriebt\\schrieben}\\
\end{tabular}
\begin{tabular}{l}
\parbox[t][][t]{8cm}{}\\
\parbox[t][][t]{8cm}{\normalfont \small ['(transitive) to write, write out (to form the', 'letters, words or symbols of) (transitive) to', 'spell, spell out (to write or say the letters that', 'form a word or part of a word) (intransitive) to', 'write (to form letters, words or symbols on a', 'surface in order to communicate) (intransitive, of', 'a book, etc.) to write, work (on) (to be an', 'author) (intransitive) to write (to) (to send', 'written information, communication) (impersonal,', 'reflexive) to write (to use for writing)', '(reflexive) to write (to exchange correspondance)', '(reflexive) to be spelled (reflexive) to spell', "one's name"] }\\
\end{tabular}
}
%===schreien===
\card{\normalfont \Huge schreien}{
\begin{tabular}{lll}
\parbox[t][][t]{2.0 cm}{\normalfont \raggedleft ich\\du\\er/sie/es\\wir\\ihr\\sie} &    
\parbox[t][][t]{2cm}{\normalfont schreie\\schreist\\schreit\\schreien\\schreit\\schreien} &
\parbox[t][][t]{2cm}{\normalfont schrie\\schriest\\schrie\\schrieen\\schriet\\schrieen}\\
\end{tabular}
\begin{tabular}{l}
\parbox[t][][t]{8cm}{}\\
\parbox[t][][t]{8cm}{\normalfont \small ['(transitive or intransitive) to shout; to yell; to', 'cry; to scream; to howl Warum schreist du?Why are', 'you shouting?'] }\\
\end{tabular}
}
%===schreiten===
\card{\normalfont \Huge schreiten}{
\begin{tabular}{lll}
\parbox[t][][t]{2.0 cm}{\normalfont \raggedleft ich\\du\\er/sie/es\\wir\\ihr\\sie} &    
\parbox[t][][t]{2cm}{\normalfont schreite\\schreitest\\schreitet\\schreiten\\schreitet\\schreiten} &
\parbox[t][][t]{2cm}{\normalfont schritt\\schrittest\\schritt\\schritten\\schrittet\\schritten}\\
\end{tabular}
\begin{tabular}{l}
\parbox[t][][t]{8cm}{}\\
\parbox[t][][t]{8cm}{\normalfont \small ['(intransitive, formal) to stride; to step, to', 'proceed'] }\\
\end{tabular}
}
%===schützen===
\card{\normalfont \Huge schützen}{
\begin{tabular}{lll}
\parbox[t][][t]{2.0 cm}{\normalfont \raggedleft ich\\du\\er/sie/es\\wir\\ihr\\sie} &    
\parbox[t][][t]{2cm}{\normalfont schütze\\schützt\\schützt\\schützen\\schützt\\schützen} &
\parbox[t][][t]{2cm}{\normalfont schützte\\schütztest\\schützte\\schützten\\schütztet\\schützten}\\
\end{tabular}
\begin{tabular}{l}
\parbox[t][][t]{8cm}{}\\
\parbox[t][][t]{8cm}{\normalfont \small ['(transitive) to defend; to protect; to shelter; to', 'guard Die Gtter schtzen die guten Leute.Gods', 'protect good people. 2010, Der Spiegel, issue', '27/2010, page 32: Von den Leibwchtern wird', 'erwartet, dass sie die Politiker mit ihrem Krper', 'schtzen. It is expected of the bodyguards that', 'they protect the politicians with their body.', '(transitive) to cover (reflexive) to protect', 'oneself'] }\\
\end{tabular}
}
%===schweben===
\card{\normalfont \Huge schweben}{
\begin{tabular}{lll}
\parbox[t][][t]{2.0 cm}{\normalfont \raggedleft ich\\du\\er/sie/es\\wir\\ihr\\sie} &    
\parbox[t][][t]{2cm}{\normalfont schwebe\\schwebst\\schwebt\\schweben\\schwebt\\schweben} &
\parbox[t][][t]{2cm}{\normalfont schwebte\\schwebtest\\schwebte\\schwebten\\schwebtet\\schwebten}\\
\end{tabular}
\begin{tabular}{l}
\parbox[t][][t]{8cm}{}\\
\parbox[t][][t]{8cm}{\normalfont \small ['to hover, to float to soar'] }\\
\end{tabular}
}
%===schweigen===
\card{\normalfont \Huge schweigen}{
\begin{tabular}{lll}
\parbox[t][][t]{2.0 cm}{\normalfont \raggedleft ich\\du\\er/sie/es\\wir\\ihr\\sie} &    
\parbox[t][][t]{2cm}{\normalfont schweige\\schweigst\\schweigt\\schweigen\\schweigt\\schweigen} &
\parbox[t][][t]{2cm}{\normalfont schwieg\\schwiegst\\schwieg\\schwiegen\\schwiegt\\schwiegen}\\
\end{tabular}
\begin{tabular}{l}
\parbox[t][][t]{8cm}{}\\
\parbox[t][][t]{8cm}{\normalfont \small ['(intransitive) to be silent; to keep quiet Wer', 'schweigt, stimmt zu!Who is silent agrees! 1931,', 'Arthur Schnitzler, Flucht in die Finsternis, S.', 'Fischer Verlag, page 105: Sie schwiegen lange. Als', 'er endlich etwas sagen wollte, wehrte sie leise', 'ab. "Heute nichts mehr, ich bitte dich darum" They', 'were silent for a long time. When he finally', 'wanted to say something, she softly refused.', '"Nothing more today, I beg you for that"', '(intransitive) to stop talking; to shut up'] }\\
\end{tabular}
}
%===schwellen===
\card{\normalfont \Huge schwellen}{
\begin{tabular}{lll}
\parbox[t][][t]{2.0 cm}{\normalfont \raggedleft ich\\du\\er/sie/es\\wir\\ihr\\sie} &    
\parbox[t][][t]{2cm}{\normalfont schwelle\\schwillst\\schwillt\\schwellen\\schwellt\\schwellen} &
\parbox[t][][t]{2cm}{\normalfont schwoll\\schwollst\\schwoll\\schwollen\\schwollt\\schwollen}\\
\end{tabular}
\begin{tabular}{l}
\parbox[t][][t]{8cm}{}\\
\parbox[t][][t]{8cm}{\normalfont \small ['(intransitive) to swell up; to bulge'] }\\
\end{tabular}
}
%===schwimmen===
\card{\normalfont \Huge schwimmen}{
\begin{tabular}{lll}
\parbox[t][][t]{2.0 cm}{\normalfont \raggedleft ich\\du\\er/sie/es\\wir\\ihr\\sie} &    
\parbox[t][][t]{2cm}{\normalfont schwimme\\schwimmst\\schwimmt\\schwimmen\\schwimmt\\schwimmen} &
\parbox[t][][t]{2cm}{\normalfont schwamm\\schwammst\\schwamm\\schwammen\\schwammt\\schwammen}\\
\end{tabular}
\begin{tabular}{l}
\parbox[t][][t]{8cm}{}\\
\parbox[t][][t]{8cm}{\normalfont \small ['(intransitive, auxiliary: "haben" or "sein") to', 'swim (intransitive, auxiliary: "sein") to float', '(intransitive, colloquial, auxiliary: "sein") to', 'be flooded'] }\\
\end{tabular}
}
%===schwinden===
\card{\normalfont \Huge schwinden}{
\begin{tabular}{lll}
\parbox[t][][t]{2.0 cm}{\normalfont \raggedleft ich\\du\\er/sie/es\\wir\\ihr\\sie} &    
\parbox[t][][t]{2cm}{\normalfont schwinde\\schwindest\\schwindet\\schwinden\\schwindet\\schwinden} &
\parbox[t][][t]{2cm}{\normalfont schwand\\schwandest\\schwand\\schwanden\\schwandet\\schwanden}\\
\end{tabular}
\begin{tabular}{l}
\parbox[t][][t]{8cm}{}\\
\parbox[t][][t]{8cm}{\normalfont \small ['(intransitive) to dwindle (to decrease, shrink,', 'vanish)'] }\\
\end{tabular}
}
%===schwingen===
\card{\normalfont \Huge schwingen}{
\begin{tabular}{lll}
\parbox[t][][t]{2.0 cm}{\normalfont \raggedleft ich\\du\\er/sie/es\\wir\\ihr\\sie} &    
\parbox[t][][t]{2cm}{\normalfont schwinge\\schwingst\\schwingt\\schwingen\\schwingt\\schwingen} &
\parbox[t][][t]{2cm}{\normalfont schwang\\schwangst\\schwang\\schwangen\\schwangt\\schwangen}\\
\end{tabular}
\begin{tabular}{l}
\parbox[t][][t]{8cm}{}\\
\parbox[t][][t]{8cm}{\normalfont \small ['(transitive or reflexive, auxiliary: "haben") to', 'swing (clarification of this definition is needed)', '(transitive, auxiliary: "haben") to wave; to', 'brandish (intransitive, physics, auxiliary:', '"haben") to vibrate (intransitive, physics,', 'auxiliary: "haben") to oscillate 2010, Der', 'Spiegel, issue 5/2010, page 106: Das grundlegende', 'Problem aller optischen Mikroskope ist das Licht:', 'Es schwingt, je nach Farbe, mit einer Wellenlnge', 'von einigen hundert Nanometern. The fundamental', 'problem of all optical microscopes is the light:', 'it oscillates, depending on the color, with a', 'wavelength of a few hundred nanometers.', '(intransitive, auxiliary: "sein") to swing', '(clarification of this definition is needed)'] }\\
\end{tabular}
}
%===schwitzen===
\card{\normalfont \Huge schwitzen}{
\begin{tabular}{lll}
\parbox[t][][t]{2.0 cm}{\normalfont \raggedleft ich\\du\\er/sie/es\\wir\\ihr\\sie} &    
\parbox[t][][t]{2cm}{\normalfont schwitze\\schwitzt\\schwitzt\\schwitzen\\schwitzt\\schwitzen} &
\parbox[t][][t]{2cm}{\normalfont schwitzte\\schwitztest\\schwitzte\\schwitzten\\schwitztet\\schwitzten}\\
\end{tabular}
\begin{tabular}{l}
\parbox[t][][t]{8cm}{}\\
\parbox[t][][t]{8cm}{\normalfont \small ['(transitive or intransitive) to sweat; to perspire', '(of living beings) (intransitive) to give off', 'water (of things) (intransitive, figuratively,', 'with ber + dative) to think hard; to work hard (on', 'a mental task) (intransitive, figuratively) to be', 'in fear; to worry'] }\\
\end{tabular}
}
%===schwören===
\card{\normalfont \Huge schwören}{
\begin{tabular}{lll}
\parbox[t][][t]{2.0 cm}{\normalfont \raggedleft ich\\du\\er/sie/es\\wir\\ihr\\sie} &    
\parbox[t][][t]{2cm}{\normalfont schwöre\\schwörst\\schwört\\schwören\\schwört\\schwören} &
\parbox[t][][t]{2cm}{\normalfont schwor\\schworst\\schwor\\schworen\\schwort\\schworen}\\
\end{tabular}
\begin{tabular}{l}
\parbox[t][][t]{8cm}{}\\
\parbox[t][][t]{8cm}{\normalfont \small ['(transitive or intransitive) to swear; to take an', 'oath bei etwas schwrento swear by something'] }\\
\end{tabular}
}
%===sehen===
\card{\normalfont \Huge sehen}{
\begin{tabular}{lll}
\parbox[t][][t]{2.0 cm}{\normalfont \raggedleft ich\\du\\er/sie/es\\wir\\ihr\\sie} &    
\parbox[t][][t]{2cm}{\normalfont sehe\\siehst\\sieht\\sehen\\seht\\sehen} &
\parbox[t][][t]{2cm}{\normalfont sah\\sahst\\sah\\sahen\\saht\\sahen}\\
\end{tabular}
\begin{tabular}{l}
\parbox[t][][t]{8cm}{}\\
\parbox[t][][t]{8cm}{\normalfont \small ['(intransitive) to see; to have sight Er sieht', "nicht gut.  He doesn't see well. (transitive) to", 'see (something); to perceive by vision 2016, Selma', 'Lagerlf, Mathilde Mann (translator), Karl-Maria', 'Guth (editor), Jerusalem. Erster und zweiter Teil,', 'Sammlung Hofenberg im Verlag der Contumax GmbH,', 'Berlin, page 225: Sahest du nicht den Patriarchen', 'der Armenier ebenso wie den der Griechen und der', 'Assyrer ihre Throne hier errichten? Und sahest du', 'nicht Kopten aus dem alten gypten und Abessinier', 'aus dem Herzen Afrikas kommen? Du sahest Jerusalem', 'wieder aufgebaut, eine Stadt von Kirchen und', 'Klstern, von Gasthusern und frommen', 'Stiftungen.(please add an English translation of', 'this quote) (transitive or intransitive) to', 'realize; to notice; to see; to find out', '(transitive) to meet (somebody); to meet up; to', 'see; but not in the sense of "pay a visit to", nor', 'as a euphemism for having a romantic or sexual', 'relation Siehst du den Markus noch?Do you still', 'see Markus? (Do you meet him regularly? Are you', 'still friends with him?) (intransitive, often with', 'auf, also nach) to look at; to watch; the', 'construction with nach often implies a turning of', 'the head; other prepositions can be used depending', 'on the context Synonyms: gucken, kucken, schauen', 'auf/nach etwas sehen  to look at something', '(intransitive, with nach) to check on; to look', 'after; to see to Synonyms: gucken, kucken, schauen', '(intransitive, informal) to decide spontaneously', 'and/or by personal preference; to wait and see', 'Synonyms: gucken, kucken, schauen Das werden wir', "dann sehen.We'll see then. / We'll play it by ear.", 'Soll ich Nudeln oder Pizza nehmen?  Das musst du', "selber sehen.Should I take pasta or pizza?  You'll", 'have to decide for yourself.'] }\\
\end{tabular}
}
%===sein===
\card{\normalfont \Huge sein}{
\begin{tabular}{lll}
\parbox[t][][t]{2.0 cm}{\normalfont \raggedleft ich\\du\\er/sie/es\\wir\\ihr\\sie} &    
\parbox[t][][t]{2cm}{\normalfont bin\\bist\\ist\\sind\\seid\\sind} &
\parbox[t][][t]{2cm}{\normalfont war\\warst\\war\\waren\\wart\\waren}\\
\end{tabular}
\begin{tabular}{l}
\parbox[t][][t]{8cm}{}\\
\parbox[t][][t]{8cm}{\normalfont \small ['(with a predicate adjective or predicate', 'nominative) to be Das ist schn.  That is', 'beautiful. Das ist ein Auto.  That is a car. (with', 'a predicate adjective and an indirect object) to', 'feel (to experience a certain condition) Mir ist', 'kalt.  I feel cold. (literally, "To me is cold.")', 'Mir ist bel.  I feel sick. Mir ist schwindelig.  I', 'feel dizzy. Mir ist wohl.  I feel well.', '(auxiliary) forms the present perfect and past', 'perfect tense of certain intransitive verbs Er ist', 'alt geworden.  He has become old. (intransitive)', 'to exist; there to be; to be alive (a common', 'proverb) Was nicht ist, kann noch werden. That', 'which does not exist now, may come into existence.', 'Wenn ich nicht mehr bin, erbst du das Haus.When I', "am no more, you'll inherit the house.", '(intransitive, colloquial) to have the next turn', "(in a game, in a queue, etc.) Du bist.  It's your", 'turn. Du bist nach mir.  Your turn is after mine.', '(intransitive, childish) to be "it"; to be the', "tagger in a game of tag Du bist!  You're it! Ich", "bin nicht mehr.  I'm not it anymore."] }\\
\end{tabular}
}
%===senden===
\card{\normalfont \Huge senden}{
\begin{tabular}{lll}
\parbox[t][][t]{2.0 cm}{\normalfont \raggedleft ich\\du\\er/sie/es\\wir\\ihr\\sie} &    
\parbox[t][][t]{2cm}{\normalfont sende\\sendest\\sendet\\senden\\sendet\\senden} &
\parbox[t][][t]{2cm}{\normalfont sendete\\sendetest\\sendete\\sendeten\\sendetet\\sendeten}\\
\end{tabular}
\begin{tabular}{l}
\parbox[t][][t]{8cm}{}\\
\parbox[t][][t]{8cm}{\normalfont \small ['(transitive or intransitive) to broadcast; to', 'transmit (transitive, chiefly literary) to send'] }\\
\end{tabular}
}
%===setzen===
\card{\normalfont \Huge setzen}{
\begin{tabular}{lll}
\parbox[t][][t]{2.0 cm}{\normalfont \raggedleft ich\\du\\er/sie/es\\wir\\ihr\\sie} &    
\parbox[t][][t]{2cm}{\normalfont setze\\setzt\\setzt\\setzen\\setzt\\setzen} &
\parbox[t][][t]{2cm}{\normalfont setzte\\setztest\\setzte\\setzten\\setztet\\setzten}\\
\end{tabular}
\begin{tabular}{l}
\parbox[t][][t]{8cm}{}\\
\parbox[t][][t]{8cm}{\normalfont \small ['(transitive) to set; to put (reflexive, of a', 'person) to sit down 1915, Franz Kafka, Der', 'Process, Verlag: Die Schmiede (1925), page 5: K.', 'wollte sich setzen, aber nun sah er, da im ganzen', 'Zimmer keine Sitzgelegenheit war, auer dem Sessel', 'beim Fenster. K. wanted to sit down, but now he', 'saw that there was no seat in the whole room apart', 'from the armchair at the window. (reflexive, of', 'particles or contents) to settle'] }\\
\end{tabular}
}
%===sichern===
\card{\normalfont \Huge sichern}{
\begin{tabular}{lll}
\parbox[t][][t]{2.0 cm}{\normalfont \raggedleft ich\\du\\er/sie/es\\wir\\ihr\\sie} &    
\parbox[t][][t]{2cm}{\normalfont sichreich sichereich sicher\\sicherst\\sichert\\sichern\\sichert\\sichern} &
\parbox[t][][t]{2cm}{\normalfont sicherte\\sichertest\\sicherte\\sicherten\\sichertet\\sicherten}\\
\end{tabular}
\begin{tabular}{l}
\parbox[t][][t]{8cm}{}\\
\parbox[t][][t]{8cm}{\normalfont \small ['to secure'] }\\
\end{tabular}
}
%===sieden===
\card{\normalfont \Huge sieden}{
\begin{tabular}{lll}
\parbox[t][][t]{2.0 cm}{\normalfont \raggedleft ich\\du\\er/sie/es\\wir\\ihr\\sie} &    
\parbox[t][][t]{2cm}{\normalfont siede\\siedest\\siedet\\sieden\\siedet\\sieden} &
\parbox[t][][t]{2cm}{\normalfont sott\\sottest\\sott\\sotten\\sottet\\sotten}\\
\end{tabular}
\begin{tabular}{l}
\parbox[t][][t]{8cm}{}\\
\parbox[t][][t]{8cm}{\normalfont \small ['to simmer, seethe to boil'] }\\
\end{tabular}
}
%===singen===
\card{\normalfont \Huge singen}{
\begin{tabular}{lll}
\parbox[t][][t]{2.0 cm}{\normalfont \raggedleft ich\\du\\er/sie/es\\wir\\ihr\\sie} &    
\parbox[t][][t]{2cm}{\normalfont singe\\singst\\singt\\singen\\singt\\singen} &
\parbox[t][][t]{2cm}{\normalfont sang\\sangst\\sang\\sangen\\sangt\\sangen}\\
\end{tabular}
\begin{tabular}{l}
\parbox[t][][t]{8cm}{}\\
\parbox[t][][t]{8cm}{\normalfont \small ['to sing 1931, Arthur Schnitzler, Flucht in die', 'Finsternis, S. Fischer Verlag, page 38: Er ging', 'rasch und sicher, trllerte vor sich hin, endlich', 'begann er sogar zu singen mit einer schnen dunklen', 'Stimme, die ihm selber fremd vorkam. He walked', 'fast and firmly, trilled to himself, finally he', 'even started to sing in a beautiful dark voice,', 'which seemed unfamiliar to himself.'] }\\
\end{tabular}
}
%===sinken===
\card{\normalfont \Huge sinken}{
\begin{tabular}{lll}
\parbox[t][][t]{2.0 cm}{\normalfont \raggedleft ich\\du\\er/sie/es\\wir\\ihr\\sie} &    
\parbox[t][][t]{2cm}{\normalfont sinke\\sinkst\\sinkt\\sinken\\sinkt\\sinken} &
\parbox[t][][t]{2cm}{\normalfont sank\\sankst\\sank\\sanken\\sankt\\sanken}\\
\end{tabular}
\begin{tabular}{l}
\parbox[t][][t]{8cm}{}\\
\parbox[t][][t]{8cm}{\normalfont \small ['(intransitive) to sink; to submerge; to set; to', 'fall from the sky Die Sonne ist gesunken.The sun', 'has set. (intransitive, figuratively, of prices,', 'temperature, quantities, rates, etc.) to fall; to', 'drop; to decline'] }\\
\end{tabular}
}
%===sinnen===
\card{\normalfont \Huge sinnen}{
\begin{tabular}{lll}
\parbox[t][][t]{2.0 cm}{\normalfont \raggedleft ich\\du\\er/sie/es\\wir\\ihr\\sie} &    
\parbox[t][][t]{2cm}{\normalfont sinne\\sinnst\\sinnt\\sinnen\\sinnt\\sinnen} &
\parbox[t][][t]{2cm}{\normalfont sann\\sannst\\sann\\sannen\\sannt\\sannen}\\
\end{tabular}
\begin{tabular}{l}
\parbox[t][][t]{8cm}{}\\
\parbox[t][][t]{8cm}{\normalfont \small ['(transitive, elevated) to think, to ponder, to', "cogitate (transitive, elevated) to direct one's", 'thoughts to something, to plan, to intend'] }\\
\end{tabular}
}
%===sitzen===
\card{\normalfont \Huge sitzen}{
\begin{tabular}{lll}
\parbox[t][][t]{2.0 cm}{\normalfont \raggedleft ich\\du\\er/sie/es\\wir\\ihr\\sie} &    
\parbox[t][][t]{2cm}{\normalfont sitze\\sitzt\\sitzt\\sitzen\\sitzt\\sitzen} &
\parbox[t][][t]{2cm}{\normalfont saß\\saßt\\saß\\saßen\\saßt\\saßen}\\
\end{tabular}
\begin{tabular}{l}
\parbox[t][][t]{8cm}{}\\
\parbox[t][][t]{8cm}{\normalfont \small ['(intransitive) to sit; to perch (intransitive) to', 'stay (in one place); to remain; to be (in a', 'particular place or state) Wir saen fest!  We were', 'stuck! (intransitive, of clothing) to fit', '(intransitive, in certain constructions, e.g. with', 'voller) to be Der Schrank sitzt voller Motten.', 'The cupboard is full of moths. (intransitive,', 'colloquial) to do time; to spend time in jail', '(intransitive, colloquial, of a strike, a comment,', 'etc.) to hit home; to have a significant effect', '(intransitive, Switzerland) to sit down'] }\\
\end{tabular}
}
%===sollen===
\card{\normalfont \Huge sollen}{
\begin{tabular}{lll}
\parbox[t][][t]{2.0 cm}{\normalfont \raggedleft ich\\du\\er/sie/es\\wir\\ihr\\sie} &    
\parbox[t][][t]{2cm}{\normalfont soll\\sollst\\soll\\sollen\\sollt\\sollen} &
\parbox[t][][t]{2cm}{\normalfont sollte\\solltest\\sollte\\sollten\\solltet\\sollten}\\
\end{tabular}
\begin{tabular}{l}
\parbox[t][][t]{8cm}{}\\
\parbox[t][][t]{8cm}{\normalfont \small ['(auxiliary) should; to be obligated (to do', 'something); ought; shall Ich soll das machen.  "I', 'should do that." Ich sollte das nicht tun.  "I', 'should not do it." (auxiliary) to be recommended', '(to do something); to be asked (to do something)', '(auxiliary) to be intended (to do something); to', 'be meant (to be something) (auxiliary) to be said', '(to do something); reportedly; they say that; I', 'hear that; so they say; rumor has it; supposedly.', 'Es soll da viele Leute geben.  "They say that', 'there are many people there." (auxiliary, in a', 'subordinate clause in the simple past tense)', 'would; indicates that the subordinate clause', 'indicates something that would happen in the past', 'but after the time frame of the main clause', '(auxiliary, in a subordinate clause in the', 'subjunctive) should; indicates that the', 'subordinate clause indicates a hypothetical and', 'unlikely condition for the main clause'] }\\
\end{tabular}
}
%===sorgen===
\card{\normalfont \Huge sorgen}{
\begin{tabular}{lll}
\parbox[t][][t]{2.0 cm}{\normalfont \raggedleft ich\\du\\er/sie/es\\wir\\ihr\\sie} &    
\parbox[t][][t]{2cm}{\normalfont sorge\\sorgst\\sorgt\\sorgen\\sorgt\\sorgen} &
\parbox[t][][t]{2cm}{\normalfont sorgte\\sorgtest\\sorgte\\sorgten\\sorgtet\\sorgten}\\
\end{tabular}
\begin{tabular}{l}
\parbox[t][][t]{8cm}{}\\
\parbox[t][][t]{8cm}{\normalfont \small ['to care (reflexive, sich sorgen) to worry'] }\\
\end{tabular}
}
%===spalten===
\card{\normalfont \Huge spalten}{
\begin{tabular}{lll}
\parbox[t][][t]{2.0 cm}{\normalfont \raggedleft ich\\du\\er/sie/es\\wir\\ihr\\sie} &    
\parbox[t][][t]{2cm}{\normalfont spalte\\spaltest\\spaltet\\spalten\\spaltet\\spalten} &
\parbox[t][][t]{2cm}{\normalfont spaltete\\spaltetest\\spaltete\\spalteten\\spaltetet\\spalteten}\\
\end{tabular}
\begin{tabular}{l}
\parbox[t][][t]{8cm}{}\\
\parbox[t][][t]{8cm}{\normalfont \small ['(transitive) to split (something); to cleave; to', 'chop (reflexive) to split up; to become divided'] }\\
\end{tabular}
}
%===sparen===
\card{\normalfont \Huge sparen}{
\begin{tabular}{lll}
\parbox[t][][t]{2.0 cm}{\normalfont \raggedleft ich\\du\\er/sie/es\\wir\\ihr\\sie} &    
\parbox[t][][t]{2cm}{\normalfont spare\\sparst\\spart\\sparen\\spart\\sparen} &
\parbox[t][][t]{2cm}{\normalfont sparte\\spartest\\sparte\\sparten\\spartet\\sparten}\\
\end{tabular}
\begin{tabular}{l}
\parbox[t][][t]{8cm}{}\\
\parbox[t][][t]{8cm}{\normalfont \small ['to save up to put aside to conserve (for example', 'energy)'] }\\
\end{tabular}
}
%===spazieren===
\card{\normalfont \Huge spazieren}{
\begin{tabular}{lll}
\parbox[t][][t]{2.0 cm}{\normalfont \raggedleft ich\\du\\er/sie/es\\wir\\ihr\\sie} &    
\parbox[t][][t]{2cm}{\normalfont spaziere\\spazierst\\spaziert\\spazieren\\spaziert\\spazieren} &
\parbox[t][][t]{2cm}{\normalfont spazierte\\spaziertest\\spazierte\\spazierten\\spaziertet\\spazierten}\\
\end{tabular}
\begin{tabular}{l}
\parbox[t][][t]{8cm}{}\\
\parbox[t][][t]{8cm}{\normalfont \small ['to take a stroll, to stroll, to take a walk, to', 'walk leisurely'] }\\
\end{tabular}
}
%===speien===
\card{\normalfont \Huge speien}{
\begin{tabular}{lll}
\parbox[t][][t]{2.0 cm}{\normalfont \raggedleft ich\\du\\er/sie/es\\wir\\ihr\\sie} &    
\parbox[t][][t]{2cm}{\normalfont speie\\speist\\speit\\speien\\speit\\speien} &
\parbox[t][][t]{2cm}{\normalfont spie\\spiest\\spie\\spieen\\spiet\\spieen}\\
\end{tabular}
\begin{tabular}{l}
\parbox[t][][t]{8cm}{}\\
\parbox[t][][t]{8cm}{\normalfont \small ['to spit (euphemistic) to vomit'] }\\
\end{tabular}
}
%===spielen===
\card{\normalfont \Huge spielen}{
\begin{tabular}{lll}
\parbox[t][][t]{2.0 cm}{\normalfont \raggedleft ich\\du\\er/sie/es\\wir\\ihr\\sie} &    
\parbox[t][][t]{2cm}{\normalfont spiele\\spielst\\spielt\\spielen\\spielt\\spielen} &
\parbox[t][][t]{2cm}{\normalfont spielte\\spieltest\\spielte\\spielten\\spieltet\\spielten}\\
\end{tabular}
\begin{tabular}{l}
\parbox[t][][t]{8cm}{}\\
\parbox[t][][t]{8cm}{\normalfont \small ['to play 1846, edited by G. Phillips and G. Grres,', 'Historisch-politische Bltter fr das katholische', 'Deutschland, 17th volume, Mnchen, page 809:', 'Entschieden zu verwerfen war dabei nur die', 'unwrdige Verkappung, welche mit der wahren', 'Herzensmeinung Versteckens spielen, den platten,', 'gedankenlosen Unglauben immer noch in den Mantel', 'einer gewissen Christlichkeit hllen, die oben', 'bezeichnete lustige Lebensphilosophie, die keine', 'ascetischen Gesichter sehen kann, immer noch fr', 'halbweg und gewissermaen katholisch ausgeben', 'wollte.'] }\\
\end{tabular}
}
%===spinnen===
\card{\normalfont \Huge spinnen}{
\begin{tabular}{lll}
\parbox[t][][t]{2.0 cm}{\normalfont \raggedleft ich\\du\\er/sie/es\\wir\\ihr\\sie} &    
\parbox[t][][t]{2cm}{\normalfont spinne\\spinnst\\spinnt\\spinnen\\spinnt\\spinnen} &
\parbox[t][][t]{2cm}{\normalfont spann\\spannst\\spann\\spannen\\spannt\\spannen}\\
\end{tabular}
\begin{tabular}{l}
\parbox[t][][t]{8cm}{}\\
\parbox[t][][t]{8cm}{\normalfont \small ['(transitive or intransitive) to spin (a thread, a', 'web, a cocoon, etc.) (transitive, figuratively) to', 'fabricate (an untrue story) (intransitive,', 'colloquial) to be crazy (intransitive, colloquial)', 'to act or talk foolishly (intransitive,', 'colloquial) to freak out (intransitive,', 'figuratively, of a cat) to purr'] }\\
\end{tabular}
}
%===sprechen===
\card{\normalfont \Huge sprechen}{
\begin{tabular}{lll}
\parbox[t][][t]{2.0 cm}{\normalfont \raggedleft ich\\du\\er/sie/es\\wir\\ihr\\sie} &    
\parbox[t][][t]{2cm}{\normalfont spreche\\sprichst\\spricht\\sprechen\\sprecht\\sprechen} &
\parbox[t][][t]{2cm}{\normalfont sprach\\sprachst\\sprach\\sprachen\\spracht\\sprachen}\\
\end{tabular}
\begin{tabular}{l}
\parbox[t][][t]{8cm}{}\\
\parbox[t][][t]{8cm}{\normalfont \small ['(transitive) to speak (some language, the truth,', 'etc.) (intransitive) to speak; to talk; to give a', 'speech (transitive, literary, poetic) to say', '(something) (Switzerland, transitive, officialese)', 'to grant, approve (a loan, funding, etc.)'] }\\
\end{tabular}
}
%===sprießen===
\card{\normalfont \Huge sprießen}{
\begin{tabular}{lll}
\parbox[t][][t]{2.0 cm}{\normalfont \raggedleft ich\\du\\er/sie/es\\wir\\ihr\\sie} &    
\parbox[t][][t]{2cm}{\normalfont sprieße\\sprießt\\sprießt\\sprießen\\sprießt\\sprießen} &
\parbox[t][][t]{2cm}{\normalfont spross\\sprosst\\spross\\sprossen\\sprosst\\sprossen}\\
\end{tabular}
\begin{tabular}{l}
\parbox[t][][t]{8cm}{}\\
\parbox[t][][t]{8cm}{\normalfont \small ['to sprout'] }\\
\end{tabular}
}
%===springen===
\card{\normalfont \Huge springen}{
\begin{tabular}{lll}
\parbox[t][][t]{2.0 cm}{\normalfont \raggedleft ich\\du\\er/sie/es\\wir\\ihr\\sie} &    
\parbox[t][][t]{2cm}{\normalfont springe\\springst\\springt\\springen\\springt\\springen} &
\parbox[t][][t]{2cm}{\normalfont sprang\\sprangst\\sprang\\sprangen\\sprangt\\sprangen}\\
\end{tabular}
\begin{tabular}{l}
\parbox[t][][t]{8cm}{}\\
\parbox[t][][t]{8cm}{\normalfont \small ['(intransitive) to spring; to leap; to bounce', '(intransitive, sports) to dive; to jump; to vault', '(intransitive) to break; to burst; to pop'] }\\
\end{tabular}
}
%===spülen===
\card{\normalfont \Huge spülen}{
\begin{tabular}{lll}
\parbox[t][][t]{2.0 cm}{\normalfont \raggedleft ich\\du\\er/sie/es\\wir\\ihr\\sie} &    
\parbox[t][][t]{2cm}{\normalfont spüle\\spülst\\spült\\spülen\\spült\\spülen} &
\parbox[t][][t]{2cm}{\normalfont spülte\\spültest\\spülte\\spülten\\spültet\\spülten}\\
\end{tabular}
\begin{tabular}{l}
\parbox[t][][t]{8cm}{}\\
\parbox[t][][t]{8cm}{\normalfont \small ['to rinse to do the dishes, wash up, wash', '(figuratively) to wash, to sweep 2010, Der', 'Spiegel, issue 34/2010, page 19: Die', 'Mehreinnahmen, die der rasante', 'Wirtschaftsaufschwung in die Staatskasse splen', 'wird, sind konjunkturell bedingt nur vorbergehend.', 'Due to the nature of business cycles, the', 'additional revenues which the rapid economic', "upturn will sweep into the government's coffers", 'will only be transient. to flush (the toilet)'] }\\
\end{tabular}
}
%===starten===
\card{\normalfont \Huge starten}{
\begin{tabular}{lll}
\parbox[t][][t]{2.0 cm}{\normalfont \raggedleft ich\\du\\er/sie/es\\wir\\ihr\\sie} &    
\parbox[t][][t]{2cm}{\normalfont starte\\startest\\startet\\starten\\startet\\starten} &
\parbox[t][][t]{2cm}{\normalfont startete\\startetest\\startete\\starteten\\startetet\\starteten}\\
\end{tabular}
\begin{tabular}{l}
\parbox[t][][t]{8cm}{}\\
\parbox[t][][t]{8cm}{\normalfont \small ['(intransitive, auxiliary sein) to start (race car,', 'airplane, etc.) Die Maschine ist pnktlich', 'gestartet. The plane started on time. (transitive,', 'auxiliary haben) to start something Ich habe den', "Computer gestartet. I've started the computer."] }\\
\end{tabular}
}
%===stattfinden===
\card{\normalfont \Huge stattfinden}{
\begin{tabular}{lll}
\parbox[t][][t]{2.0 cm}{\normalfont \raggedleft ich\\du\\er/sie/es\\wir\\ihr\\sie} &    
\parbox[t][][t]{2cm}{\normalfont finde statt\\findest statt\\findet statt\\finden statt\\findet statt\\finden statt} &
\parbox[t][][t]{2cm}{\normalfont fand statt\\fandest statt\\fand statt\\fanden statt\\fandet statt\\fanden statt}\\
\end{tabular}
\begin{tabular}{l}
\parbox[t][][t]{8cm}{}\\
\parbox[t][][t]{8cm}{\normalfont \small ['to take place, to happen'] }\\
\end{tabular}
}
%===stechen===
\card{\normalfont \Huge stechen}{
\begin{tabular}{lll}
\parbox[t][][t]{2.0 cm}{\normalfont \raggedleft ich\\du\\er/sie/es\\wir\\ihr\\sie} &    
\parbox[t][][t]{2cm}{\normalfont steche\\stichst\\sticht\\stechen\\stecht\\stechen} &
\parbox[t][][t]{2cm}{\normalfont stach\\stachst\\stach\\stachen\\stacht\\stachen}\\
\end{tabular}
\begin{tabular}{l}
\parbox[t][][t]{8cm}{}\\
\parbox[t][][t]{8cm}{\normalfont \small ['(transitive) to stick (someone or something); to', 'poke (transitive or intransitive) to sting; to', 'bite; to prick (transitive) to cut; to chop', '(transitive, of the sun) to burn (transitive, card', 'games) to take; to trump; to capture'] }\\
\end{tabular}
}
%===stecken===
\card{\normalfont \Huge stecken}{
\begin{tabular}{lll}
\parbox[t][][t]{2.0 cm}{\normalfont \raggedleft ich\\du\\er/sie/es\\wir\\ihr\\sie} &    
\parbox[t][][t]{2cm}{\normalfont stecke\\steckst\\steckt\\stecken\\steckt\\stecken} &
\parbox[t][][t]{2cm}{\normalfont steckte\\stecktest\\steckte\\steckten\\stecktet\\steckten}\\
\end{tabular}
\begin{tabular}{l}
\parbox[t][][t]{8cm}{}\\
\parbox[t][][t]{8cm}{\normalfont \small ['(transitive) to stick; to put; to insert', '(transitive) to pin (intransitive) to stick; to be', 'stuck (intransitive) to be (hiding) Nur seine', 'Eltern und zwei gute Freunde wissen, dass er', 'hinter der Seite steckt.  Only his parents and two', 'good friends know he is hiding behind the side.'] }\\
\end{tabular}
}
%===stehen===
\card{\normalfont \Huge stehen}{
\begin{tabular}{lll}
\parbox[t][][t]{2.0 cm}{\normalfont \raggedleft ich\\du\\er/sie/es\\wir\\ihr\\sie} &    
\parbox[t][][t]{2cm}{\normalfont stehe\\stehst\\steht\\stehen\\steht\\stehen} &
\parbox[t][][t]{2cm}{\normalfont stand\\standest\\stand\\standen\\standet\\standen}\\
\end{tabular}
\begin{tabular}{l}
\parbox[t][][t]{8cm}{}\\
\parbox[t][][t]{8cm}{\normalfont \small ['(intransitive) to stand (to be upright, support', 'oneself on the feet in an erect position)', '(intransitive) to be, to appear, to stand (to be', 'placed or located somewhere) Das steht nicht in', 'dem Wrterbuch.This does not appear in the', 'dictionary. 1931, Arthur Schnitzler, Flucht in die', 'Finsternis, S. Fischer Verlag, page 36: Ein frisch', 'geflltes Glas Champagner stand vor ihm. Er trank', 'es in einem Zug aus  mit Lust, fast mit Begier. A', 'freshly filled glass of champaign was in front of', 'him. He emptied it in one draught  with pleasure,', 'almost with greed. (intransitive) to stay; to be', 'still (Switzerland) to confront, surrender', 'Synonym: sich stellen'] }\\
\end{tabular}
}
%===stehlen===
\card{\normalfont \Huge stehlen}{
\begin{tabular}{lll}
\parbox[t][][t]{2.0 cm}{\normalfont \raggedleft ich\\du\\er/sie/es\\wir\\ihr\\sie} &    
\parbox[t][][t]{2cm}{\normalfont stehle\\stiehlst\\stiehlt\\stehlen\\stehlt\\stehlen} &
\parbox[t][][t]{2cm}{\normalfont stahl\\stahlst\\stahl\\stahlen\\stahlt\\stahlen}\\
\end{tabular}
\begin{tabular}{l}
\parbox[t][][t]{8cm}{}\\
\parbox[t][][t]{8cm}{\normalfont \small ['(transitive or intransitive) to steal (reflexive)', 'to skulk, to move secretly'] }\\
\end{tabular}
}
%===steigen===
\card{\normalfont \Huge steigen}{
\begin{tabular}{lll}
\parbox[t][][t]{2.0 cm}{\normalfont \raggedleft ich\\du\\er/sie/es\\wir\\ihr\\sie} &    
\parbox[t][][t]{2cm}{\normalfont steige\\steigst\\steigt\\steigen\\steigt\\steigen} &
\parbox[t][][t]{2cm}{\normalfont stieg\\stiegst\\stieg\\stiegen\\stiegt\\stiegen}\\
\end{tabular}
\begin{tabular}{l}
\parbox[t][][t]{8cm}{}\\
\parbox[t][][t]{8cm}{\normalfont \small ['(intransitive) to ascend; to climb (intransitive)', 'to rise 2010, Der Spiegel, issue 35/2010, page 93:', 'Seit der rumnische Wirtschaftsboom vor zwei Jahren', 'schlagartig mit der Finanzkrise zu Ende ging, ist', 'die Arbeitslosigkeit im Land auf mehr als sieben', 'Prozent gestiegen. Since the Romanian economic', 'boom abruptly came to an end with the financial', 'crisis two years ago, the unemployment in the', 'country rose to more than seven percent. 1931,', 'Arthur Schnitzler, Flucht in die Finsternis, S.', 'Fischer Verlag, page 36: Er lchelte oder wollte', 'vielmehr lcheln, denn pltzlich fhlte er seine', 'Lippen zucken, Trnen stiegen ihm in die Augen, und', 'er konnte sich eben noch mit Mhe zurckhalten, laut', 'aufzuschluchzen. He smiled or rather wanted to', 'smile, because suddenly he felt his lips', 'twitching, tears rose to his eyes, and only with', 'difficulty he could hold onto himself so as not to', 'give a loud sob. to step 2010, Der Spiegel, issue', '24/2010, page 128: Das Schiff legt an, und die', 'Besucher steigen in einen weien Bus, der sie ber', 'die Insel fhrt. The ship docks and the visitors', 'step into a white bus, which drives them across', 'the island. (intransitive, of a horse) to rear up'] }\\
\end{tabular}
}
%===stellen===
\card{\normalfont \Huge stellen}{
\begin{tabular}{lll}
\parbox[t][][t]{2.0 cm}{\normalfont \raggedleft ich\\du\\er/sie/es\\wir\\ihr\\sie} &    
\parbox[t][][t]{2cm}{\normalfont stelle\\stellst\\stellt\\stellen\\stellt\\stellen} &
\parbox[t][][t]{2cm}{\normalfont stellte\\stelltest\\stellte\\stellten\\stelltet\\stellten}\\
\end{tabular}
\begin{tabular}{l}
\parbox[t][][t]{8cm}{}\\
\parbox[t][][t]{8cm}{\normalfont \small ['(transitive) to put, to place, to position, so', 'that it afterwands stands as opposed to being', 'gesetzt, gelegt [+accusative] Stell die Flasche', 'auf den Boden!  Put the bottle on the floor!', '(figuratively and abstractly) to lodge, to', 'provide, to pose, etc. [+accusative] Die Beklagte', 'stellte den Antrag, die Klage abzuweisen.  The', 'defendant lodged the application to reject the', 'claim. Aus dem Sicherungsvertrage war die', 'A-Gesellschaft verpflichtet, eine Brgschaft zu', 'stellen.  From the surety agreement the A company', 'was obliged to provide a suretyship. Kann ich dir', 'eine Frage stellen?  Can I pose a question to you?', 'to encounterand stop [+accusative = a human] Die', 'Hunde haben den Hirsch gestellt.  The hounds', 'stopped the stag. Der Eigentmer stellte den Dieb.', 'The proprietor came upon the thief. Der Inhaber', 'stellte den Dieb zur Rede.  The detentor', 'confronted the thief and engaged him in speech.', 'Die Polizei stellte den Dieb.  The police caught', 'the thief. to set, adjust [+accusative = technical', 'applications that have a rhythm] Mssen wir am', 'Sonntag wieder die Uhren stellen?  Do we have to', 'adjust the clocks again on Sunday? Synonym:', 'einstellen (reflexive)to expose oneself, to', 'succumb, to come out to face, to confront', '[+dative] Du musst dich der Gefahr stellen.  You', 'have to face the danger. Der Dieb stellte sich der', 'Polizei.  The thief surrendered to the police.', '2006,  "Brief von der Front", in  Sturmabende,', 'performed by Arische Jugend, track 10,', '2:392:49:Wir stellen uns den Panzern und', 'Granaten,Dem Feinde, der uns gegenber ficht,Denn', 'was ist schon das Leben des Soldaten?Der Tod frs', 'Volk die Heldenpflicht.We will face the tanks and', 'grenadesThe enemy who fights in front of usFor', 'what is the life of the soldierDeath for the', 'people is the obligation of a hero (transitive) to', 'feign, to simulate, to pretend [+accusative] Es', 'war alles nur gestellt!  It was all fake! Sie', 'hatte ihre Krankheit nur gestellt.  She simulated', 'her ailment merely.'] }\\
\end{tabular}
}
%===sterben===
\card{\normalfont \Huge sterben}{
\begin{tabular}{lll}
\parbox[t][][t]{2.0 cm}{\normalfont \raggedleft ich\\du\\er/sie/es\\wir\\ihr\\sie} &    
\parbox[t][][t]{2cm}{\normalfont sterbe\\stirbst\\stirbt\\sterben\\sterbt\\sterben} &
\parbox[t][][t]{2cm}{\normalfont starb\\starbst\\starb\\starben\\starbt\\starben}\\
\end{tabular}
\begin{tabular}{l}
\parbox[t][][t]{8cm}{}\\
\parbox[t][][t]{8cm}{\normalfont \small ['(intransitive) to die Mein Hund ist gestorben.', '"My dog has died." (transitive, with accusative or', 'genitive) to die of a (particular kind of) death', 'Das Opfer ist einen schrecklichen Tod gestorben.', 'Das Opfer ist eines schrecklichen Todes gestorben.', '"The victim died a terrible death."'] }\\
\end{tabular}
}
%===stimmen===
\card{\normalfont \Huge stimmen}{
\begin{tabular}{lll}
\parbox[t][][t]{2.0 cm}{\normalfont \raggedleft ich\\du\\er/sie/es\\wir\\ihr\\sie} &    
\parbox[t][][t]{2cm}{\normalfont stimme\\stimmst\\stimmt\\stimmen\\stimmt\\stimmen} &
\parbox[t][][t]{2cm}{\normalfont stimmte\\stimmtest\\stimmte\\stimmten\\stimmtet\\stimmten}\\
\end{tabular}
\begin{tabular}{l}
\parbox[t][][t]{8cm}{}\\
\parbox[t][][t]{8cm}{\normalfont \small ['to vote (music, an instrument) to tune to be', 'right, be true 1969, Peter Bichsel,', 'Kindergeschichten: Tad und Nacht sa er ber seinen', 'Plnen und prfte sie nach, und sie stimmten. Day', 'and night he sat over his plans and checked them;', 'and they were right. (with an adjective of', 'emotion, transitive) to make (someone happy, sad,', 'etc.)'] }\\
\end{tabular}
}
%===stinken===
\card{\normalfont \Huge stinken}{
\begin{tabular}{lll}
\parbox[t][][t]{2.0 cm}{\normalfont \raggedleft ich\\du\\er/sie/es\\wir\\ihr\\sie} &    
\parbox[t][][t]{2cm}{\normalfont stinke\\stinkst\\stinkt\\stinken\\stinkt\\stinken} &
\parbox[t][][t]{2cm}{\normalfont stank\\stankst\\stank\\stanken\\stankt\\stanken}\\
\end{tabular}
\begin{tabular}{l}
\parbox[t][][t]{8cm}{}\\
\parbox[t][][t]{8cm}{\normalfont \small ['(intransitive) to stink'] }\\
\end{tabular}
}
%===stoppen===
\card{\normalfont \Huge stoppen}{
\begin{tabular}{lll}
\parbox[t][][t]{2.0 cm}{\normalfont \raggedleft ich\\du\\er/sie/es\\wir\\ihr\\sie} &    
\parbox[t][][t]{2cm}{\normalfont stoppe\\stoppst\\stoppt\\stoppen\\stoppt\\stoppen} &
\parbox[t][][t]{2cm}{\normalfont stoppte\\stopptest\\stoppte\\stoppten\\stopptet\\stoppten}\\
\end{tabular}
\begin{tabular}{l}
\parbox[t][][t]{8cm}{}\\
\parbox[t][][t]{8cm}{\normalfont \small ['(standard, transitive or intransitive) to stop Wir', 'mssen diese Entwicklung stoppen.  We must stop', 'this development. Das Auto hat vor der Kreuzung', 'gestoppt.  The car stopped at the crossroads.', '(colloquial, regional, northern and central', 'Germany) Alternative form of stopfen ("to stuff,', 'to plug") Musste dein Zeug da so unvorsichtig', "reinstoppen? D'you need to stuff your things in", 'there so carelessly?'] }\\
\end{tabular}
}
%===stören===
\card{\normalfont \Huge stören}{
\begin{tabular}{lll}
\parbox[t][][t]{2.0 cm}{\normalfont \raggedleft ich\\du\\er/sie/es\\wir\\ihr\\sie} &    
\parbox[t][][t]{2cm}{\normalfont störe\\störst\\stört\\stören\\stört\\stören} &
\parbox[t][][t]{2cm}{\normalfont störte\\störtest\\störte\\störten\\störtet\\störten}\\
\end{tabular}
\begin{tabular}{l}
\parbox[t][][t]{8cm}{}\\
\parbox[t][][t]{8cm}{\normalfont \small ['to disturb, to interfere, to bother Es strt mich,', 'dass It bothers me that'] }\\
\end{tabular}
}
%===stoßen===
\card{\normalfont \Huge stoßen}{
\begin{tabular}{lll}
\parbox[t][][t]{2.0 cm}{\normalfont \raggedleft ich\\du\\er/sie/es\\wir\\ihr\\sie} &    
\parbox[t][][t]{2cm}{\normalfont stoße\\stößt\\stößt\\stoßen\\stoßt\\stoßen} &
\parbox[t][][t]{2cm}{\normalfont stieß\\stießt\\stieß\\stießen\\stießt\\stießen}\\
\end{tabular}
\begin{tabular}{l}
\parbox[t][][t]{8cm}{}\\
\parbox[t][][t]{8cm}{\normalfont \small ['(transitive) to push; to shove; to thrust', '(transitive or reflexive) to bump; to knock; to', 'strike; to hurt (reflexive, figuratively, with an)', 'to take exception (to something) (intransitive) to', 'jolt; to kick; to thrust'] }\\
\end{tabular}
}
%===strahlen===
\card{\normalfont \Huge strahlen}{
\begin{tabular}{lll}
\parbox[t][][t]{2.0 cm}{\normalfont \raggedleft ich\\du\\er/sie/es\\wir\\ihr\\sie} &    
\parbox[t][][t]{2cm}{\normalfont strahle\\strahlst\\strahlt\\strahlen\\strahlt\\strahlen} &
\parbox[t][][t]{2cm}{\normalfont strahlte\\strahltest\\strahlte\\strahlten\\strahltet\\strahlten}\\
\end{tabular}
\begin{tabular}{l}
\parbox[t][][t]{8cm}{}\\
\parbox[t][][t]{8cm}{\normalfont \small ['to shine to radiate to smile in a big way'] }\\
\end{tabular}
}
%===streben===
\card{\normalfont \Huge streben}{
\begin{tabular}{lll}
\parbox[t][][t]{2.0 cm}{\normalfont \raggedleft ich\\du\\er/sie/es\\wir\\ihr\\sie} &    
\parbox[t][][t]{2cm}{\normalfont strebe\\strebst\\strebt\\streben\\strebt\\streben} &
\parbox[t][][t]{2cm}{\normalfont strebte\\strebtest\\strebte\\strebten\\strebtet\\strebten}\\
\end{tabular}
\begin{tabular}{l}
\parbox[t][][t]{8cm}{}\\
\parbox[t][][t]{8cm}{\normalfont \small ['To strive To aspire'] }\\
\end{tabular}
}
%===streichen===
\card{\normalfont \Huge streichen}{
\begin{tabular}{lll}
\parbox[t][][t]{2.0 cm}{\normalfont \raggedleft ich\\du\\er/sie/es\\wir\\ihr\\sie} &    
\parbox[t][][t]{2cm}{\normalfont streiche\\streichst\\streicht\\streichen\\streicht\\streichen} &
\parbox[t][][t]{2cm}{\normalfont strich\\strichst\\strich\\strichen\\stricht\\strichen}\\
\end{tabular}
\begin{tabular}{l}
\parbox[t][][t]{8cm}{}\\
\parbox[t][][t]{8cm}{\normalfont \small ['(transitive) to stroke (transitive) to cancel; to', 'discard; to delete; to strike out; to cross out', '(transitive) to spread; to rub; to apply', '(transitive) to paint'] }\\
\end{tabular}
}
%===streiten===
\card{\normalfont \Huge streiten}{
\begin{tabular}{lll}
\parbox[t][][t]{2.0 cm}{\normalfont \raggedleft ich\\du\\er/sie/es\\wir\\ihr\\sie} &    
\parbox[t][][t]{2cm}{\normalfont streite\\streitest\\streitet\\streiten\\streitet\\streiten} &
\parbox[t][][t]{2cm}{\normalfont stritt\\strittest\\stritt\\stritten\\strittet\\stritten}\\
\end{tabular}
\begin{tabular}{l}
\parbox[t][][t]{8cm}{}\\
\parbox[t][][t]{8cm}{\normalfont \small ['(intransitive or reflexive) to argue; to fight; to', 'quarrel'] }\\
\end{tabular}
}
%===studieren===
\card{\normalfont \Huge studieren}{
\begin{tabular}{lll}
\parbox[t][][t]{2.0 cm}{\normalfont \raggedleft ich\\du\\er/sie/es\\wir\\ihr\\sie} &    
\parbox[t][][t]{2cm}{\normalfont studiere\\studierst\\studiert\\studieren\\studiert\\studieren} &
\parbox[t][][t]{2cm}{\normalfont studierte\\studiertest\\studierte\\studierten\\studiertet\\studierten}\\
\end{tabular}
\begin{tabular}{l}
\parbox[t][][t]{8cm}{}\\
\parbox[t][][t]{8cm}{\normalfont \small ['(transitive or intransitive) to study at', 'university or college level; to be a student (of)', 'Sie studiert Chemie.She studies chemistry. Seit', 'wann studierst du?For how long have you been a', 'student? (transitive, perhaps slightly  dated) to', 'study  scientifically; to research; to perform a', 'study on Er studiert den Nestbau der Ameisen.He', 'studies the nest building of ants. (transitive,', 'slightly informal) to look at minutely; to study;', 'to peruse; to analyse Sie studierte seinen', 'Brief.She studied his letter. Er studierte ihren', 'Gesichtsausdruck.He studied her facial', 'expressions.'] }\\
\end{tabular}
}
%===stürzen===
\card{\normalfont \Huge stürzen}{
\begin{tabular}{lll}
\parbox[t][][t]{2.0 cm}{\normalfont \raggedleft ich\\du\\er/sie/es\\wir\\ihr\\sie} &    
\parbox[t][][t]{2cm}{\normalfont stürze\\stürzt\\stürzt\\stürzen\\stürzt\\stürzen} &
\parbox[t][][t]{2cm}{\normalfont stürzte\\stürztest\\stürzte\\stürzten\\stürztet\\stürzten}\\
\end{tabular}
\begin{tabular}{l}
\parbox[t][][t]{8cm}{}\\
\parbox[t][][t]{8cm}{\normalfont \small ['to fall down, to drop, to tumble to dash, to rush,', 'to sprint to something to throw, to hurl to upturn', 'to overthrow, to oust, to dethrone to drop off', 'steeply (reflexive) to plunge'] }\\
\end{tabular}
}
%===stützen===
\card{\normalfont \Huge stützen}{
\begin{tabular}{lll}
\parbox[t][][t]{2.0 cm}{\normalfont \raggedleft ich\\du\\er/sie/es\\wir\\ihr\\sie} &    
\parbox[t][][t]{2cm}{\normalfont stütze\\stützt\\stützt\\stützen\\stützt\\stützen} &
\parbox[t][][t]{2cm}{\normalfont stützte\\stütztest\\stützte\\stützten\\stütztet\\stützten}\\
\end{tabular}
\begin{tabular}{l}
\parbox[t][][t]{8cm}{}\\
\parbox[t][][t]{8cm}{\normalfont \small ['(transitive) to support (to keep from falling)', '(reflexive) (with preposition auf) to use', 'something or someone for support (transitive) to', 'corroborate, to abet (with preposition auf) to', 'rest, to lean 1931, Arthur Schnitzler, Flucht in', 'die Finsternis, S. Fischer Verlag, page 125: Er', 'sah sie, von Schneeflocken umweht, auf einem', 'kleinen Balkon stehen, die Hnde auf die Brstung', 'gesttzt und nach unten blickend. He saw her,', 'snowflakes blowing around her, standing on a small', 'balcony, the hands rested on the balustrade and', 'looking down.'] }\\
\end{tabular}
}
%===suchen===
\card{\normalfont \Huge suchen}{
\begin{tabular}{lll}
\parbox[t][][t]{2.0 cm}{\normalfont \raggedleft ich\\du\\er/sie/es\\wir\\ihr\\sie} &    
\parbox[t][][t]{2cm}{\normalfont suche\\suchst\\sucht\\suchen\\sucht\\suchen} &
\parbox[t][][t]{2cm}{\normalfont suchte\\suchtest\\suchte\\suchten\\suchtet\\suchten}\\
\end{tabular}
\begin{tabular}{l}
\parbox[t][][t]{8cm}{}\\
\parbox[t][][t]{8cm}{\normalfont \small ['to seek to search, to look for Ich suche  nach', "meinen Schlsseln.I'm looking for my keys."] }\\
\end{tabular}
}
%===tanzen===
\card{\normalfont \Huge tanzen}{
\begin{tabular}{lll}
\parbox[t][][t]{2.0 cm}{\normalfont \raggedleft ich\\du\\er/sie/es\\wir\\ihr\\sie} &    
\parbox[t][][t]{2cm}{\normalfont tanze\\tanzt\\tanzt\\tanzen\\tanzt\\tanzen} &
\parbox[t][][t]{2cm}{\normalfont tanzte\\tanztest\\tanzte\\tanzten\\tanztet\\tanzten}\\
\end{tabular}
\begin{tabular}{l}
\parbox[t][][t]{8cm}{}\\
\parbox[t][][t]{8cm}{\normalfont \small ['to dance'] }\\
\end{tabular}
}
%===teilen===
\card{\normalfont \Huge teilen}{
\begin{tabular}{lll}
\parbox[t][][t]{2.0 cm}{\normalfont \raggedleft ich\\du\\er/sie/es\\wir\\ihr\\sie} &    
\parbox[t][][t]{2cm}{\normalfont teile\\teilst\\teilt\\teilen\\teilt\\teilen} &
\parbox[t][][t]{2cm}{\normalfont teilte\\teiltest\\teilte\\teilten\\teiltet\\teilten}\\
\end{tabular}
\begin{tabular}{l}
\parbox[t][][t]{8cm}{}\\
\parbox[t][][t]{8cm}{\normalfont \small ['to split, to share to divide'] }\\
\end{tabular}
}
%===teilnehmen===
\card{\normalfont \Huge teilnehmen}{
\begin{tabular}{lll}
\parbox[t][][t]{2.0 cm}{\normalfont \raggedleft ich\\du\\er/sie/es\\wir\\ihr\\sie} &    
\parbox[t][][t]{2cm}{\normalfont nehme teil\\nimmst teil\\nimmt teil\\nehmen teil\\nehmt teil\\nehmen teil} &
\parbox[t][][t]{2cm}{\normalfont nahm teil\\nahmst teil\\nahm teil\\nahmen teil\\nahmt teil\\nahmen teil}\\
\end{tabular}
\begin{tabular}{l}
\parbox[t][][t]{8cm}{}\\
\parbox[t][][t]{8cm}{\normalfont \small ['to participate'] }\\
\end{tabular}
}
%===töten===
\card{\normalfont \Huge töten}{
\begin{tabular}{lll}
\parbox[t][][t]{2.0 cm}{\normalfont \raggedleft ich\\du\\er/sie/es\\wir\\ihr\\sie} &    
\parbox[t][][t]{2cm}{\normalfont töte\\tötest\\tötet\\töten\\tötet\\töten} &
\parbox[t][][t]{2cm}{\normalfont tötete\\tötetest\\tötete\\töteten\\tötetet\\töteten}\\
\end{tabular}
\begin{tabular}{l}
\parbox[t][][t]{8cm}{}\\
\parbox[t][][t]{8cm}{\normalfont \small ['to kill Zehn Menschen wurden bei dem Anschlag', 'gettet.Ten people were killed in the attack.'] }\\
\end{tabular}
}
%===tragen===
\card{\normalfont \Huge tragen}{
\begin{tabular}{lll}
\parbox[t][][t]{2.0 cm}{\normalfont \raggedleft ich\\du\\er/sie/es\\wir\\ihr\\sie} &    
\parbox[t][][t]{2cm}{\normalfont trage\\trägst\\trägt\\tragen\\tragt\\tragen} &
\parbox[t][][t]{2cm}{\normalfont trug\\trugst\\trug\\trugen\\trugt\\trugen}\\
\end{tabular}
\begin{tabular}{l}
\parbox[t][][t]{8cm}{}\\
\parbox[t][][t]{8cm}{\normalfont \small ['(transitive) to carry, to take Und was auch kommen', 'mag,sei es ein dunkler Tag,sei es Glck oder', 'Freude,wir tragen es beide! (Haussegen) And come', 'what may,be it a dark day,be it lucky or joyous,we', 'take them both! (House blessing) (transitive) to', 'hold (transitive or intransitive, fashion) to wear', '(clothing, jewelry) (transitive, agriculture) to', 'produce, to bear, to yield (transitive) to', 'support, to maintain (intransitive, agriculture)', 'to crop (reflexive, finance) to pay for itself'] }\\
\end{tabular}
}
%===trauen===
\card{\normalfont \Huge trauen}{
\begin{tabular}{lll}
\parbox[t][][t]{2.0 cm}{\normalfont \raggedleft ich\\du\\er/sie/es\\wir\\ihr\\sie} &    
\parbox[t][][t]{2cm}{\normalfont traue\\traust\\traut\\trauen\\traut\\trauen} &
\parbox[t][][t]{2cm}{\normalfont traute\\trautest\\traute\\trauten\\trautet\\trauten}\\
\end{tabular}
\begin{tabular}{l}
\parbox[t][][t]{8cm}{}\\
\parbox[t][][t]{8cm}{\normalfont \small ['to trust (transitive) to marry (reflexive) to dare'] }\\
\end{tabular}
}
%===träumen===
\card{\normalfont \Huge träumen}{
\begin{tabular}{lll}
\parbox[t][][t]{2.0 cm}{\normalfont \raggedleft ich\\du\\er/sie/es\\wir\\ihr\\sie} &    
\parbox[t][][t]{2cm}{\normalfont träume\\träumst\\träumt\\träumen\\träumt\\träumen} &
\parbox[t][][t]{2cm}{\normalfont träumte\\träumtest\\träumte\\träumten\\träumtet\\träumten}\\
\end{tabular}
\begin{tabular}{l}
\parbox[t][][t]{8cm}{}\\
\parbox[t][][t]{8cm}{\normalfont \small ['to dream 1931, Arthur Schnitzler, Flucht in die', 'Finsternis, S. Fischer Verlag, page 38: Vielleicht', 'trume ich? Vielleicht ist es mein letzter Traum,', 'den ich trume, der Traum auf dem Sterbebett? Maybe', 'I am dreaming? May it is my last dream, which I am', 'dreaming, the dream on the deathbed?'] }\\
\end{tabular}
}
%===treffen===
\card{\normalfont \Huge treffen}{
\begin{tabular}{lll}
\parbox[t][][t]{2.0 cm}{\normalfont \raggedleft ich\\du\\er/sie/es\\wir\\ihr\\sie} &    
\parbox[t][][t]{2cm}{\normalfont treffe\\triffst\\trifft\\treffen\\trefft\\treffen} &
\parbox[t][][t]{2cm}{\normalfont traf\\trafst\\traf\\trafen\\traft\\trafen}\\
\end{tabular}
\begin{tabular}{l}
\parbox[t][][t]{8cm}{}\\
\parbox[t][][t]{8cm}{\normalfont \small ['(transitive or reflexive) to meet; to encounter', '(transitive or intransitive) to hit; to strike', '(transitive) to affect; to concern (intransitive', 'or reflexive, colloquial, often with "gut" or', '"schlecht") to hit the mark; to suit; to be', 'convenient, fortunate Das trifft sich schlecht.', "That's unfortunate."] }\\
\end{tabular}
}
%===treiben===
\card{\normalfont \Huge treiben}{
\begin{tabular}{lll}
\parbox[t][][t]{2.0 cm}{\normalfont \raggedleft ich\\du\\er/sie/es\\wir\\ihr\\sie} &    
\parbox[t][][t]{2cm}{\normalfont treibe\\treibst\\treibt\\treiben\\treibt\\treiben} &
\parbox[t][][t]{2cm}{\normalfont trieb\\triebst\\trieb\\trieben\\triebt\\trieben}\\
\end{tabular}
\begin{tabular}{l}
\parbox[t][][t]{8cm}{}\\
\parbox[t][][t]{8cm}{\normalfont \small ['(transitive, auxiliary: "haben") to drive (e.g.', 'livestock); to propel; to force (transitive,', 'auxiliary: "haben") to put forth; to produce; to', 'sprout (transitive, figuratively, auxiliary:', '"haben") to urge (transitive, vulgar, slang,', 'auxiliary: "haben") to fuck (intransitive,', 'auxiliary: "sein") to drift; to float about', '(intransitive, auxiliary: "sein") to sprout', '(transitive) to do, to get up to Was treibst du', 'denn so den ganzen Tag?What do you get up to all', 'day?'] }\\
\end{tabular}
}
%===trennen===
\card{\normalfont \Huge trennen}{
\begin{tabular}{lll}
\parbox[t][][t]{2.0 cm}{\normalfont \raggedleft ich\\du\\er/sie/es\\wir\\ihr\\sie} &    
\parbox[t][][t]{2cm}{\normalfont trenne\\trennst\\trennt\\trennen\\trennt\\trennen} &
\parbox[t][][t]{2cm}{\normalfont trennte\\trenntest\\trennte\\trennten\\trenntet\\trennten}\\
\end{tabular}
\begin{tabular}{l}
\parbox[t][][t]{8cm}{}\\
\parbox[t][][t]{8cm}{\normalfont \small ['to separate, sever, part, disunite, uncouple,', 'segregate, unjoin, disjoin to sunder to delink,', 'unlink, disconnect, detach to isolate'] }\\
\end{tabular}
}
%===treten===
\card{\normalfont \Huge treten}{
\begin{tabular}{lll}
\parbox[t][][t]{2.0 cm}{\normalfont \raggedleft ich\\du\\er/sie/es\\wir\\ihr\\sie} &    
\parbox[t][][t]{2cm}{\normalfont trete\\trittst\\tritt\\treten\\tretet\\treten} &
\parbox[t][][t]{2cm}{\normalfont trat\\tratest\\trat\\traten\\tratet\\traten}\\
\end{tabular}
\begin{tabular}{l}
\parbox[t][][t]{8cm}{}\\
\parbox[t][][t]{8cm}{\normalfont \small ['(transitive, auxiliary: "haben") to step; to', 'tread; to trample Wasser treten  "to tread water"', '(transitive, auxiliary: "haben") to kick (a ball)', '(intransitive, auxiliary: "sein") to step; to', 'tread auf etwas treten  "to tread on something"', '(intransitive, auxiliary: "sein", figuratively) to', 'appear 1919, Walther Kabel, Irrende Seelen, Werner', 'Dietsch Verlag, page 107: Wieder begann mein Herz', 'jetzt in rasenden Schlgen zu klopfen, wieder fhlte', 'ich kalten Schwei auf meine Stirn treten. Again my', 'heart started now beating with racing beats, again', 'I felt cold sweat appear on my brow.', '(intransitive, auxiliary: "sein") to pedal', '(intransitive, auxiliary: "sein") to walk'] }\\
\end{tabular}
}
%===triefen===
\card{\normalfont \Huge triefen}{
\begin{tabular}{lll}
\parbox[t][][t]{2.0 cm}{\normalfont \raggedleft ich\\du\\er/sie/es\\wir\\ihr\\sie} &    
\parbox[t][][t]{2cm}{\normalfont triefe\\triefest\\triefet\\triefen\\triefet\\triefen} &
\parbox[t][][t]{2cm}{\normalfont troff\\troffest\\troff\\troffen\\troffet\\troffen}\\
\end{tabular}
\begin{tabular}{l}
\parbox[t][][t]{8cm}{}\\
\parbox[t][][t]{8cm}{\normalfont \small ['to drip heavily to be so wet that a large volume', 'of water drips off (of one)'] }\\
\end{tabular}
}
%===trinken===
\card{\normalfont \Huge trinken}{
\begin{tabular}{lll}
\parbox[t][][t]{2.0 cm}{\normalfont \raggedleft ich\\du\\er/sie/es\\wir\\ihr\\sie} &    
\parbox[t][][t]{2cm}{\normalfont trinke\\trinkst\\trinkt\\trinken\\trinkt\\trinken} &
\parbox[t][][t]{2cm}{\normalfont trank\\trankst\\trank\\tranken\\trankt\\tranken}\\
\end{tabular}
\begin{tabular}{l}
\parbox[t][][t]{8cm}{}\\
\parbox[t][][t]{8cm}{\normalfont \small ['(transitive) to drink (to consume (a liquid)', 'through the mouth or the liquid contained within', '(a bottle, glass, etc.)) (intransitive) to drink', '(to consume alcoholic beverages) (intransitive) to', 'drink; to toast (engage in a salutation (of', 'someone), accompanying the raising of glasses', "while drinking alcohol) (reflexive) to drink one's", 'fill; to drink to satiety'] }\\
\end{tabular}
}
%===trösten===
\card{\normalfont \Huge trösten}{
\begin{tabular}{lll}
\parbox[t][][t]{2.0 cm}{\normalfont \raggedleft ich\\du\\er/sie/es\\wir\\ihr\\sie} &    
\parbox[t][][t]{2cm}{\normalfont tröste\\tröstest\\tröstet\\trösten\\tröstet\\trösten} &
\parbox[t][][t]{2cm}{\normalfont tröstete\\tröstetest\\tröstete\\trösteten\\tröstetet\\trösteten}\\
\end{tabular}
\begin{tabular}{l}
\parbox[t][][t]{8cm}{}\\
\parbox[t][][t]{8cm}{\normalfont \small ['to console'] }\\
\end{tabular}
}
%===trügen===
\card{\normalfont \Huge trügen}{
\begin{tabular}{lll}
\parbox[t][][t]{2.0 cm}{\normalfont \raggedleft ich\\du\\er/sie/es\\wir\\ihr\\sie} &    
\parbox[t][][t]{2cm}{\normalfont trüge\\trügst\\trügt\\trügen\\trügt\\trügen} &
\parbox[t][][t]{2cm}{\normalfont trog\\trogst\\trog\\trogen\\trogt\\trogen}\\
\end{tabular}
\begin{tabular}{l}
\parbox[t][][t]{8cm}{}\\
\parbox[t][][t]{8cm}{\normalfont \small ['(transitive) to deceive (intransitive) to be', 'deceptive'] }\\
\end{tabular}
}
%===tun===
\card{\normalfont \Huge tun}{
\begin{tabular}{lll}
\parbox[t][][t]{2.0 cm}{\normalfont \raggedleft ich\\du\\er/sie/es\\wir\\ihr\\sie} &    
\parbox[t][][t]{2cm}{\normalfont tue\\tust\\tut\\tun\\tut\\tun} &
\parbox[t][][t]{2cm}{\normalfont tat\\tatest\\tat\\taten\\tatet\\taten}\\
\end{tabular}
\begin{tabular}{l}
\parbox[t][][t]{8cm}{}\\
\parbox[t][][t]{8cm}{\normalfont \small ['to do (to perform or execute an action) Synonym:', 'machen Tu es!  Do it! Man tut, was man kann.  One', 'does what one can. Er tat das, was man ihm gesagt', 'hat.  He did as he was told. Das einzige, was er', 'je tat, war arbeiten.  The only thing he ever did', 'was work. (with dative) to do something (positive', 'or negative) to someone Synonym: antun Der tut', "Ihnen nichts!  He won't hurt you! (said for", 'example about a dog) Mein Mann hat mir so viel', 'Gutes getan.  My husband has done me so much good.', '(reflexive, with an indefinite pronoun) to make a', 'difference; to be different Synonym: unterscheiden', 'Tut sich das viel?  Does that make much of a', 'difference? Die beiden Kameras tun sich nichts.', 'The two cameras are no different [i.e. neither', 'better than the other]. (somewhat  informal, with', '"so" or "als ob") to fake; to feign; to pretend', 'Synonyms: vortuschen, tuschen, vorgeben Er hat nur', 'so getan.  He just faked it. Er tut, als ob er', 'nichts wsste.  He pretends to know nothing.', '(chiefly colloquial) to put, to place, to add', 'Synonyms: setzen, legen, stellen, platzieren,', 'hinzufgen 2017, Simone Meier, Fleisch, Kein  Aber,', 'p. 27: Ich finds eklig, wenn du die Butter am', 'Morgen nicht direkt aufs Brot streichst, sondern', 'immer zuerst auf einen Teller tust. I find it', "disgusting when you don't spread your butter", 'straight on to your bread in the morning, but', 'always put it on the plate first. Tu das hier', 'rein.  Put it in here. Ich wrd noch was Salz an', 'die Kartoffeln tun.  I would add some more salt to', 'the potatoes. (chiefly colloquial, with "es") to', 'work, to function Synonym: funktionieren Die Uhr', "tut's nicht mehr.  The clock doesn't work anymore.", '(chiefly colloquial, but acceptable in writing)', 'Used with the preceding infinitive of another verb', 'to emphasise this verb Er singt immer noch gern,', 'aber tanzen tut er gar nicht mehr.He still loves', "singing, but as to dancing, he doesn't do that", 'anymore at all. (colloquial, nonstandard) Used', 'with the following infinitive of another verb,', 'often to emphasise the statement Ich tu doch', 'zuhren!  I am listening! (as a response to the', 'reproach that one is not) Ich tu das jetzt mal', "aufrumen.  I'm cleaning this up now. (colloquial,", 'nonstandard) Used in the past subjunctive with the', 'infinitive of another verb to form the conditional', 'tense (instead of standard wrde) Ich tt mir das', 'noch mal berlegen.  I would think about that', 'again.'] }\\
\end{tabular}
}
%===überholen===
\card{\normalfont \Huge überholen}{
\begin{tabular}{lll}
\parbox[t][][t]{2.0 cm}{\normalfont \raggedleft ich\\du\\er/sie/es\\wir\\ihr\\sie} &    
\parbox[t][][t]{2cm}{\normalfont überhole\\überholst\\überholt\\überholen\\überholt\\überholen} &
\parbox[t][][t]{2cm}{\normalfont überholte\\überholtest\\überholte\\überholten\\überholtet\\überholten}\\
\end{tabular}
\begin{tabular}{l}
\parbox[t][][t]{8cm}{}\\
\parbox[t][][t]{8cm}{\normalfont \small ['(transitive or intransitive) to overtake; to pass', '(e.g. a vehicle, a runner) (transitive,', 'figuratively) to excel; to become more or better', 'than'] }\\
\end{tabular}
}
%===überlegen===
\card{\normalfont \Huge überlegen}{
\begin{tabular}{lll}
\parbox[t][][t]{2.0 cm}{\normalfont \raggedleft ich\\du\\er/sie/es\\wir\\ihr\\sie} &    
\parbox[t][][t]{2cm}{\normalfont überlege\\überlegst\\überlegt\\überlegen\\überlegt\\überlegen} &
\parbox[t][][t]{2cm}{\normalfont überlegte\\überlegtest\\überlegte\\überlegten\\überlegtet\\überlegten}\\
\end{tabular}
\begin{tabular}{l}
\parbox[t][][t]{8cm}{}\\
\parbox[t][][t]{8cm}{\normalfont \small ["to think, to think over Ich berlege es mir.  I'll", 'think about it. Ich habe es mir anders uberlegt.', "I've changed my mind."] }\\
\end{tabular}
}
%===übernehmen===
\card{\normalfont \Huge übernehmen}{
\begin{tabular}{lll}
\parbox[t][][t]{2.0 cm}{\normalfont \raggedleft ich\\du\\er/sie/es\\wir\\ihr\\sie} &    
\parbox[t][][t]{2cm}{\normalfont übernehme\\übernimmst\\übernimmt\\übernehmen\\übernehmt\\übernehmen} &
\parbox[t][][t]{2cm}{\normalfont übernahm\\übernahmst\\übernahm\\übernahmen\\übernahmt\\übernahmen}\\
\end{tabular}
\begin{tabular}{l}
\parbox[t][][t]{8cm}{}\\
\parbox[t][][t]{8cm}{\normalfont \small ['to take over, replace'] }\\
\end{tabular}
}
%===überraschen===
\card{\normalfont \Huge überraschen}{
\begin{tabular}{lll}
\parbox[t][][t]{2.0 cm}{\normalfont \raggedleft ich\\du\\er/sie/es\\wir\\ihr\\sie} &    
\parbox[t][][t]{2cm}{\normalfont überrasche\\überraschst\\überrascht\\überraschen\\überrascht\\überraschen} &
\parbox[t][][t]{2cm}{\normalfont überraschte\\überraschtest\\überraschte\\überraschten\\überraschtet\\überraschten}\\
\end{tabular}
\begin{tabular}{l}
\parbox[t][][t]{8cm}{}\\
\parbox[t][][t]{8cm}{\normalfont \small ['(transitive) to surprise'] }\\
\end{tabular}
}
%===übersetzen===
\card{\normalfont \Huge übersetzen}{
\begin{tabular}{lll}
\parbox[t][][t]{2.0 cm}{\normalfont \raggedleft ich\\du\\er/sie/es\\wir\\ihr\\sie} &    
\parbox[t][][t]{2cm}{\normalfont übersetze\\übersetzt\\übersetzt\\übersetzen\\übersetzt\\übersetzen} &
\parbox[t][][t]{2cm}{\normalfont übersetzte\\übersetztest\\übersetzte\\übersetzten\\übersetztet\\übersetzten}\\
\end{tabular}
\begin{tabular}{l}
\parbox[t][][t]{8cm}{}\\
\parbox[t][][t]{8cm}{\normalfont \small ['to translate, to interpret 1836, Heinrich Heine,', 'Die romantische Schule, In: Heinrich Heine: Werke', 'und Briefe in zehn Bnden, Aufbau-Verlag (1972),', 'volume 5, page 38, [...] jetzt bersetzte er, mit', 'unerhrtem Flei, auch die brigen heidnischen', 'Dichter des Altertums, [...] now he translated,', 'with unheard-of effort, also the remaining pagan', 'poets of the antiquity, (computing) to compile'] }\\
\end{tabular}
}
%===überweisen===
\card{\normalfont \Huge überweisen}{
\begin{tabular}{lll}
\parbox[t][][t]{2.0 cm}{\normalfont \raggedleft ich\\du\\er/sie/es\\wir\\ihr\\sie} &    
\parbox[t][][t]{2cm}{\normalfont überweise\\überweist\\überweist\\überweisen\\überweist\\überweisen} &
\parbox[t][][t]{2cm}{\normalfont überwies\\überwiest\\überwies\\überwiesen\\überwiest\\überwiesen}\\
\end{tabular}
\begin{tabular}{l}
\parbox[t][][t]{8cm}{}\\
\parbox[t][][t]{8cm}{\normalfont \small ['to transfer Sie mssen den Betrag sofort', 'berweisen.You have to transfer the amount', 'immediately.'] }\\
\end{tabular}
}
%===überwinden===
\card{\normalfont \Huge überwinden}{
\begin{tabular}{lll}
\parbox[t][][t]{2.0 cm}{\normalfont \raggedleft ich\\du\\er/sie/es\\wir\\ihr\\sie} &    
\parbox[t][][t]{2cm}{\normalfont überwinde\\überwindest\\überwindet\\überwinden\\überwindet\\überwinden} &
\parbox[t][][t]{2cm}{\normalfont überwand\\überwandest\\überwand\\überwanden\\überwandet\\überwanden}\\
\end{tabular}
\begin{tabular}{l}
\parbox[t][][t]{8cm}{}\\
\parbox[t][][t]{8cm}{\normalfont \small ['(transitive) to overcome, to get over'] }\\
\end{tabular}
}
%===überzeugen===
\card{\normalfont \Huge überzeugen}{
\begin{tabular}{lll}
\parbox[t][][t]{2.0 cm}{\normalfont \raggedleft ich\\du\\er/sie/es\\wir\\ihr\\sie} &    
\parbox[t][][t]{2cm}{\normalfont überzeuge\\überzeugst\\überzeugt\\überzeugen\\überzeugt\\überzeugen} &
\parbox[t][][t]{2cm}{\normalfont überzeugte\\überzeugtest\\überzeugte\\überzeugten\\überzeugtet\\überzeugten}\\
\end{tabular}
\begin{tabular}{l}
\parbox[t][][t]{8cm}{}\\
\parbox[t][][t]{8cm}{\normalfont \small ['(transitive) to convince Lothar Berg, Ohne', 'Kompromiss. Das Leben des Ausbrecherknigs Eckehard', '"Ekke" Lehman, p. 154: Ekke prfte in seiner', 'direkten Art, ob sie nicht nur Ballast fr ihn sei,', 'aber sie berzeugte ihn eines Besseren: (reflexive)', 'to convince oneself'] }\\
\end{tabular}
}
%===unterbrechen===
\card{\normalfont \Huge unterbrechen}{
\begin{tabular}{lll}
\parbox[t][][t]{2.0 cm}{\normalfont \raggedleft ich\\du\\er/sie/es\\wir\\ihr\\sie} &    
\parbox[t][][t]{2cm}{\normalfont unterbreche\\unterbrichst\\unterbricht\\unterbrechen\\unterbrecht\\unterbrechen} &
\parbox[t][][t]{2cm}{\normalfont unterbrach\\unterbrachst\\unterbrach\\unterbrachen\\unterbracht\\unterbrachen}\\
\end{tabular}
\begin{tabular}{l}
\parbox[t][][t]{8cm}{}\\
\parbox[t][][t]{8cm}{\normalfont \small ['to interrupt'] }\\
\end{tabular}
}
%===unterhalten===
\card{\normalfont \Huge unterhalten}{
\begin{tabular}{lll}
\parbox[t][][t]{2.0 cm}{\normalfont \raggedleft ich\\du\\er/sie/es\\wir\\ihr\\sie} &    
\parbox[t][][t]{2cm}{\normalfont unterhalte\\unterhältst\\unterhält\\unterhalten\\unterhaltet\\unterhalten} &
\parbox[t][][t]{2cm}{\normalfont unterhielt\\unterhieltest\\unterhielt\\unterhielten\\unterhieltet\\unterhielten}\\
\end{tabular}
\begin{tabular}{l}
\parbox[t][][t]{8cm}{}\\
\parbox[t][][t]{8cm}{\normalfont \small ['(transitive) to maintain (transitive) to entertain', '(reflexive) to converse Er hat sich mit einem', 'Mitarbeiter ber die Arbeit unterhalten.He', 'conversed with an employee about the work'] }\\
\end{tabular}
}
%===unternehmen===
\card{\normalfont \Huge unternehmen}{
\begin{tabular}{lll}
\parbox[t][][t]{2.0 cm}{\normalfont \raggedleft ich\\du\\er/sie/es\\wir\\ihr\\sie} &    
\parbox[t][][t]{2cm}{\normalfont unternehme\\unternimmst\\unternimmt\\unternehmen\\unternehmt\\unternehmen} &
\parbox[t][][t]{2cm}{\normalfont unternahm\\unternahmst\\unternahm\\unternahmen\\unternahmt\\unternahmen}\\
\end{tabular}
\begin{tabular}{l}
\parbox[t][][t]{8cm}{}\\
\parbox[t][][t]{8cm}{\normalfont \small ['(transitive) to undertake'] }\\
\end{tabular}
}
%===unterrichten===
\card{\normalfont \Huge unterrichten}{
\begin{tabular}{lll}
\parbox[t][][t]{2.0 cm}{\normalfont \raggedleft ich\\du\\er/sie/es\\wir\\ihr\\sie} &    
\parbox[t][][t]{2cm}{\normalfont unterrichte\\unterrichtest\\unterrichtet\\unterrichten\\unterrichtet\\unterrichten} &
\parbox[t][][t]{2cm}{\normalfont unterrichtete\\unterrichtetest\\unterrichtete\\unterrichteten\\unterrichtetet\\unterrichteten}\\
\end{tabular}
\begin{tabular}{l}
\parbox[t][][t]{8cm}{}\\
\parbox[t][][t]{8cm}{\normalfont \small ['to teach, to school to inform'] }\\
\end{tabular}
}
%===unterscheiden===
\card{\normalfont \Huge unterscheiden}{
\begin{tabular}{lll}
\parbox[t][][t]{2.0 cm}{\normalfont \raggedleft ich\\du\\er/sie/es\\wir\\ihr\\sie} &    
\parbox[t][][t]{2cm}{\normalfont unterscheide\\unterscheidest\\unterscheidet\\unterscheiden\\unterscheidet\\unterscheiden} &
\parbox[t][][t]{2cm}{\normalfont unterschied\\unterschiedest\\unterschied\\unterschieden\\unterschiedet\\unterschieden}\\
\end{tabular}
\begin{tabular}{l}
\parbox[t][][t]{8cm}{}\\
\parbox[t][][t]{8cm}{\normalfont \small ['(transitive) to distinguish, differentiate', '(reflexive) to differ'] }\\
\end{tabular}
}
%===unterschreiben===
\card{\normalfont \Huge unterschreiben}{
\begin{tabular}{lll}
\parbox[t][][t]{2.0 cm}{\normalfont \raggedleft ich\\du\\er/sie/es\\wir\\ihr\\sie} &    
\parbox[t][][t]{2cm}{\normalfont unterschreibe\\unterschreibst\\unterschreibt\\unterschreiben\\unterschreibt\\unterschreiben} &
\parbox[t][][t]{2cm}{\normalfont unterschrieb\\unterschriebst\\unterschrieb\\unterschrieben\\unterschriebt\\unterschrieben}\\
\end{tabular}
\begin{tabular}{l}
\parbox[t][][t]{8cm}{}\\
\parbox[t][][t]{8cm}{\normalfont \small ["(transitive or intransitive) to sign (write one's", 'signature) Unterschreiben Sie bitte auf der', 'punktierten Linie.Please sign on the dotted line.'] }\\
\end{tabular}
}
%===unterstützen===
\card{\normalfont \Huge unterstützen}{
\begin{tabular}{lll}
\parbox[t][][t]{2.0 cm}{\normalfont \raggedleft ich\\du\\er/sie/es\\wir\\ihr\\sie} &    
\parbox[t][][t]{2cm}{\normalfont unterstütze\\unterstützt\\unterstützt\\unterstützen\\unterstützt\\unterstützen} &
\parbox[t][][t]{2cm}{\normalfont unterstützte\\unterstütztest\\unterstützte\\unterstützten\\unterstütztet\\unterstützten}\\
\end{tabular}
\begin{tabular}{l}
\parbox[t][][t]{8cm}{}\\
\parbox[t][][t]{8cm}{\normalfont \small ['to support Sein Onkel untersttzt ihn bis heute', 'finanziell.His uncle supports him financially', 'until today.'] }\\
\end{tabular}
}
%===verabschieden===
\card{\normalfont \Huge verabschieden}{
\begin{tabular}{lll}
\parbox[t][][t]{2.0 cm}{\normalfont \raggedleft ich\\du\\er/sie/es\\wir\\ihr\\sie} &    
\parbox[t][][t]{2cm}{\normalfont verabschiede\\verabschiedest\\verabschiedet\\verabschieden\\verabschiedet\\verabschieden} &
\parbox[t][][t]{2cm}{\normalfont verabschiedete\\verabschiedetest\\verabschiedete\\verabschiedeten\\verabschiedetet\\verabschiedeten}\\
\end{tabular}
\begin{tabular}{l}
\parbox[t][][t]{8cm}{}\\
\parbox[t][][t]{8cm}{\normalfont \small ['(reflexive) to say goodbye; to take leave', '(reflexive  with von or transitive) to say goodbye', 'to someone; to take leave of someone (transitive)', 'to see off; to dismiss (transitive, politics) to', 'adopt; to pass (a law, resolution, etc.)'] }\\
\end{tabular}
}
%===verändern===
\card{\normalfont \Huge verändern}{
\begin{tabular}{lll}
\parbox[t][][t]{2.0 cm}{\normalfont \raggedleft ich\\du\\er/sie/es\\wir\\ihr\\sie} &    
\parbox[t][][t]{2cm}{\normalfont verändreich verändereich veränder\\veränderst\\verändert\\verändern\\verändert\\verändern} &
\parbox[t][][t]{2cm}{\normalfont veränderte\\verändertest\\veränderte\\veränderten\\verändertet\\veränderten}\\
\end{tabular}
\begin{tabular}{l}
\parbox[t][][t]{8cm}{}\\
\parbox[t][][t]{8cm}{\normalfont \small ['(transitive) to alter, to change Er vernderte die', 'Einstellungen. He changed the settings.', '(reflexive) to change Ich habe mich verndert, aber', "er bemerkt es nicht. I have changed, but he didn't", 'notice it.'] }\\
\end{tabular}
}
%===verbessern===
\card{\normalfont \Huge verbessern}{
\begin{tabular}{lll}
\parbox[t][][t]{2.0 cm}{\normalfont \raggedleft ich\\du\\er/sie/es\\wir\\ihr\\sie} &    
\parbox[t][][t]{2cm}{\normalfont verbessreich verbessereich verbesser\\verbesserst\\verbessert\\verbessern\\verbessert\\verbessern} &
\parbox[t][][t]{2cm}{\normalfont verbesserte\\verbessertest\\verbesserte\\verbesserten\\verbessertet\\verbesserten}\\
\end{tabular}
\begin{tabular}{l}
\parbox[t][][t]{8cm}{}\\
\parbox[t][][t]{8cm}{\normalfont \small ['(transitive) to improve (to make better)', '(reflexive) to improve (to become better)', '(transitive) to correct'] }\\
\end{tabular}
}
%===verbieten===
\card{\normalfont \Huge verbieten}{
\begin{tabular}{lll}
\parbox[t][][t]{2.0 cm}{\normalfont \raggedleft ich\\du\\er/sie/es\\wir\\ihr\\sie} &    
\parbox[t][][t]{2cm}{\normalfont verbiete\\verbietest\\verbietet\\verbieten\\verbietet\\verbieten} &
\parbox[t][][t]{2cm}{\normalfont verbot\\verbotest\\verbot\\verboten\\verbotet\\verboten}\\
\end{tabular}
\begin{tabular}{l}
\parbox[t][][t]{8cm}{}\\
\parbox[t][][t]{8cm}{\normalfont \small ['(with dative) to forbid, prohibit Ich verbiete', 'Ihnen in diesem Ton mit meiner Frau zu sprechen.I', 'forbid you to talk in that tone of voice with my', 'wife.'] }\\
\end{tabular}
}
%===verbinden===
\card{\normalfont \Huge verbinden}{
\begin{tabular}{lll}
\parbox[t][][t]{2.0 cm}{\normalfont \raggedleft ich\\du\\er/sie/es\\wir\\ihr\\sie} &    
\parbox[t][][t]{2cm}{\normalfont verbinde\\verbindest\\verbindet\\verbinden\\verbindet\\verbinden} &
\parbox[t][][t]{2cm}{\normalfont verband\\verbandest\\verband\\verbanden\\verbandet\\verbanden}\\
\end{tabular}
\begin{tabular}{l}
\parbox[t][][t]{8cm}{}\\
\parbox[t][][t]{8cm}{\normalfont \small ['to join, to combine, to connect 2010, Der Spiegel,', 'issue 46/2010, page 103: In nur 28 Minuten soll', 'ein neuer Schnellzug knftig Tel Aviv und Jerusalem', 'verbinden. In only 28 minutes, a new express train', 'is to connect Tel Aviv and Jerusalem in the', 'future. (medicine) to bandage (telephony) to put', 'through'] }\\
\end{tabular}
}
%===verbrauchen===
\card{\normalfont \Huge verbrauchen}{
\begin{tabular}{lll}
\parbox[t][][t]{2.0 cm}{\normalfont \raggedleft ich\\du\\er/sie/es\\wir\\ihr\\sie} &    
\parbox[t][][t]{2cm}{\normalfont verbrauche\\verbrauchst\\verbraucht\\verbrauchen\\verbraucht\\verbrauchen} &
\parbox[t][][t]{2cm}{\normalfont verbrauchte\\verbrauchtest\\verbrauchte\\verbrauchten\\verbrauchtet\\verbrauchten}\\
\end{tabular}
\begin{tabular}{l}
\parbox[t][][t]{8cm}{}\\
\parbox[t][][t]{8cm}{\normalfont \small ['to consume 1932, Erich Mhsam, Die Befreiung der', 'Gesellschaft vom Staat, in: Erich Mhsam:', 'Prosaschriften II, Verlag europische ideen Berlin', '(1978), page 255: Wir verstehen unter Kommunismus', 'die auf Gtergemeinschaft beruhende', 'Gesellschaftsbeziehung, die jedem nach seinen', 'Fhigkeiten zu arbeiten, jedem nach seinen', 'Bedrfnissen zu verbrauchen erlaubt. We understand', 'by communism the relationship of society that is', 'based on public ownership, that allows everyone to', 'work according to his capabilities, everyone to', 'consume according to his needs.'] }\\
\end{tabular}
}
%===verderben===
\card{\normalfont \Huge verderben}{
\begin{tabular}{lll}
\parbox[t][][t]{2.0 cm}{\normalfont \raggedleft ich\\du\\er/sie/es\\wir\\ihr\\sie} &    
\parbox[t][][t]{2cm}{\normalfont verderbe\\verdirbst\\verdirbt\\verderben\\verderbt\\verderben} &
\parbox[t][][t]{2cm}{\normalfont verdarb\\verdarbst\\verdarb\\verdarben\\verdarbt\\verdarben}\\
\end{tabular}
\begin{tabular}{l}
\parbox[t][][t]{8cm}{}\\
\parbox[t][][t]{8cm}{\normalfont \small ['(transitive, with haben) to deprive (someone) of', '(something); to rob (someone) of (some feeling)', '(transitive, with haben) to ruin; to render', '(something) useless; to corrupt; to spoil', '(intransitive, usually of food, with sein) to', 'spoil; to rot; to perish (intransitive, with sein)', 'to be offensive; to live sinfully'] }\\
\end{tabular}
}
%===verdienen===
\card{\normalfont \Huge verdienen}{
\begin{tabular}{lll}
\parbox[t][][t]{2.0 cm}{\normalfont \raggedleft ich\\du\\er/sie/es\\wir\\ihr\\sie} &    
\parbox[t][][t]{2cm}{\normalfont verdiene\\verdienst\\verdient\\verdienen\\verdient\\verdienen} &
\parbox[t][][t]{2cm}{\normalfont verdiente\\verdientest\\verdiente\\verdienten\\verdientet\\verdienten}\\
\end{tabular}
\begin{tabular}{l}
\parbox[t][][t]{8cm}{}\\
\parbox[t][][t]{8cm}{\normalfont \small ['(transitive, intransitive) to earn (transitive,', 'intransitive) to make something (on something)', '(transitive) to deserve something (for something)'] }\\
\end{tabular}
}
%===verdrießen===
\card{\normalfont \Huge verdrießen}{
\begin{tabular}{lll}
\parbox[t][][t]{2.0 cm}{\normalfont \raggedleft ich\\du\\er/sie/es\\wir\\ihr\\sie} &    
\parbox[t][][t]{2cm}{\normalfont verdrieße\\verdrießt\\verdrießt\\verdrießen\\verdrießt\\verdrießen} &
\parbox[t][][t]{2cm}{\normalfont verdross\\verdrosst\\verdross\\verdrossen\\verdrosst\\verdrossen}\\
\end{tabular}
\begin{tabular}{l}
\parbox[t][][t]{8cm}{}\\
\parbox[t][][t]{8cm}{\normalfont \small ['to chagrin'] }\\
\end{tabular}
}
%===vergessen===
\card{\normalfont \Huge vergessen}{
\begin{tabular}{lll}
\parbox[t][][t]{2.0 cm}{\normalfont \raggedleft ich\\du\\er/sie/es\\wir\\ihr\\sie} &    
\parbox[t][][t]{2cm}{\normalfont vergesse\\vergisst\\vergisst\\vergessen\\vergesst\\vergessen} &
\parbox[t][][t]{2cm}{\normalfont vergaß\\vergaßt\\vergaß\\vergaßen\\vergaßt\\vergaßen}\\
\end{tabular}
\begin{tabular}{l}
\parbox[t][][t]{8cm}{}\\
\parbox[t][][t]{8cm}{\normalfont \small ['(transitive) to forget (lose remembrance of) Diese', 'ganzen Geschichten hatte ich ja ganz vergessen.', "I'd completely forgotten all these stories.", '(transitive) to forget (fail to do something out', 'of forgetfulness) Ich habe vergessen, den Brief', 'abzuschicken. I forgot to send off the letter.', '(transitive) to leave (forget to take) Ich habe', 'meinen Schlssel bei dir vergessen. I left my key', 'at your place.'] }\\
\end{tabular}
}
%===vergleichen===
\card{\normalfont \Huge vergleichen}{
\begin{tabular}{lll}
\parbox[t][][t]{2.0 cm}{\normalfont \raggedleft ich\\du\\er/sie/es\\wir\\ihr\\sie} &    
\parbox[t][][t]{2cm}{\normalfont vergleiche\\vergleichst\\vergleicht\\vergleichen\\vergleicht\\vergleichen} &
\parbox[t][][t]{2cm}{\normalfont verglich\\verglichst\\verglich\\verglichen\\verglicht\\verglichen}\\
\end{tabular}
\begin{tabular}{l}
\parbox[t][][t]{8cm}{}\\
\parbox[t][][t]{8cm}{\normalfont \small ['to compare'] }\\
\end{tabular}
}
%===verhaften===
\card{\normalfont \Huge verhaften}{
\begin{tabular}{lll}
\parbox[t][][t]{2.0 cm}{\normalfont \raggedleft ich\\du\\er/sie/es\\wir\\ihr\\sie} &    
\parbox[t][][t]{2cm}{\normalfont verhafte\\verhaftest\\verhaftet\\verhaften\\verhaftet\\verhaften} &
\parbox[t][][t]{2cm}{\normalfont verhaftete\\verhaftetest\\verhaftete\\verhafteten\\verhaftetet\\verhafteten}\\
\end{tabular}
\begin{tabular}{l}
\parbox[t][][t]{8cm}{}\\
\parbox[t][][t]{8cm}{\normalfont \small ['to imprison, to put into confinementafter judicial', 'writ (the state of which is called Haft)'] }\\
\end{tabular}
}
%===verhalten===
\card{\normalfont \Huge verhalten}{
\begin{tabular}{lll}
\parbox[t][][t]{2.0 cm}{\normalfont \raggedleft ich\\du\\er/sie/es\\wir\\ihr\\sie} &    
\parbox[t][][t]{2cm}{\normalfont verhalte\\verhältst\\verhält\\verhalten\\verhaltet\\verhalten} &
\parbox[t][][t]{2cm}{\normalfont verhielt\\verhieltest\\verhielt\\verhielten\\verhieltet\\verhielten}\\
\end{tabular}
\begin{tabular}{l}
\parbox[t][][t]{8cm}{}\\
\parbox[t][][t]{8cm}{\normalfont \small ['(reflexive) to behave (reflexive, impersonal) to', 'be 2010, Der Spiegel, issue 24/2010, page 87:', 'Politisch war Frankreich lange Zeit ein Riese,', 'wirtschaftlich aber ein Zwerg, bei den Deutschen', 'verhielt es sich genau umgekehrt. Politically', 'France was a giant for a long time, but', 'economically a dwarf, with the Germans it was', 'exactly the other way round. (reflexive, informal)', 'to repress (reflexive, informal) to go more slowly', '(reflexive, sports, riding) to parry (reflexive,', 'regional) to have a good attitude towards oneself', '(reflexive, Austria, Switzerland) to undertake', '(reflexive, archaic outside Switzerland) to close', 'with the hand'] }\\
\end{tabular}
}
%===verhandeln===
\card{\normalfont \Huge verhandeln}{
\begin{tabular}{lll}
\parbox[t][][t]{2.0 cm}{\normalfont \raggedleft ich\\du\\er/sie/es\\wir\\ihr\\sie} &    
\parbox[t][][t]{2cm}{\normalfont verhandleich verhandeleich verhandel\\verhandelst\\verhandelt\\verhandeln\\verhandelt\\verhandeln} &
\parbox[t][][t]{2cm}{\normalfont verhandelte\\verhandeltest\\verhandelte\\verhandelten\\verhandeltet\\verhandelten}\\
\end{tabular}
\begin{tabular}{l}
\parbox[t][][t]{8cm}{}\\
\parbox[t][][t]{8cm}{\normalfont \small ['to negotiate'] }\\
\end{tabular}
}
%===verhindern===
\card{\normalfont \Huge verhindern}{
\begin{tabular}{lll}
\parbox[t][][t]{2.0 cm}{\normalfont \raggedleft ich\\du\\er/sie/es\\wir\\ihr\\sie} &    
\parbox[t][][t]{2cm}{\normalfont verhindreich verhindereich verhinder\\verhinderst\\verhindert\\verhindern\\verhindert\\verhindern} &
\parbox[t][][t]{2cm}{\normalfont verhinderte\\verhindertest\\verhinderte\\verhinderten\\verhindertet\\verhinderten}\\
\end{tabular}
\begin{tabular}{l}
\parbox[t][][t]{8cm}{}\\
\parbox[t][][t]{8cm}{\normalfont \small ['to prevent, to inhibit'] }\\
\end{tabular}
}
%===verkaufen===
\card{\normalfont \Huge verkaufen}{
\begin{tabular}{lll}
\parbox[t][][t]{2.0 cm}{\normalfont \raggedleft ich\\du\\er/sie/es\\wir\\ihr\\sie} &    
\parbox[t][][t]{2cm}{\normalfont verkaufe\\verkaufst\\verkauft\\verkaufen\\verkauft\\verkaufen} &
\parbox[t][][t]{2cm}{\normalfont verkaufte\\verkauftest\\verkaufte\\verkauften\\verkauftet\\verkauften}\\
\end{tabular}
\begin{tabular}{l}
\parbox[t][][t]{8cm}{}\\
\parbox[t][][t]{8cm}{\normalfont \small ['to sell'] }\\
\end{tabular}
}
%===verlangen===
\card{\normalfont \Huge verlangen}{
\begin{tabular}{lll}
\parbox[t][][t]{2.0 cm}{\normalfont \raggedleft ich\\du\\er/sie/es\\wir\\ihr\\sie} &    
\parbox[t][][t]{2cm}{\normalfont verlange\\verlangst\\verlangt\\verlangen\\verlangt\\verlangen} &
\parbox[t][][t]{2cm}{\normalfont verlangte\\verlangtest\\verlangte\\verlangten\\verlangtet\\verlangten}\\
\end{tabular}
\begin{tabular}{l}
\parbox[t][][t]{8cm}{}\\
\parbox[t][][t]{8cm}{\normalfont \small ['to ask for, demand'] }\\
\end{tabular}
}
%===verlassen===
\card{\normalfont \Huge verlassen}{
\begin{tabular}{lll}
\parbox[t][][t]{2.0 cm}{\normalfont \raggedleft ich\\du\\er/sie/es\\wir\\ihr\\sie} &    
\parbox[t][][t]{2cm}{\normalfont verlasse\\verlässt\\verlässt\\verlassen\\verlasst\\verlassen} &
\parbox[t][][t]{2cm}{\normalfont verließ\\verließt\\verließ\\verließen\\verließt\\verließen}\\
\end{tabular}
\begin{tabular}{l}
\parbox[t][][t]{8cm}{}\\
\parbox[t][][t]{8cm}{\normalfont \small ['(transitive) to leave, to abandon; to depart, to', 'forsake Die Ratten verlassen das sinkende', 'Schiff.The rats are abandoning the sinking ship.', 'Er verlie das Haus um 7 Uhr.He left the house at 7', "o'clock. (transitive, reflexive, with auf) to", 'trust; to rely on Ich verlasse mich darauf, dass', 'du morgen da bist.I trust you that you will be', 'there tomorrow. Ich habe mich sehr auf deine', 'Informationen verlassen!I did rely a lot upon your', 'information! (transitive, figuratively) to pass', 'away; to cease; to die (with direct object uns) Es', 'tut mir leid, aber deine Mutter hat uns gestern', 'Nacht verlassen.I am sorry, but your mother passed', 'away last night. (transitive) to desert, to dump', '(e.g. a partner in a romantic relationship) Nach', 'einer dreijhrigen Beziehung verlie sie ihn.After', 'three years of an affectionate relationship, she', 'dumped him. (transitive) to exit or close a', 'computer program or app.'] }\\
\end{tabular}
}
%===verletzen===
\card{\normalfont \Huge verletzen}{
\begin{tabular}{lll}
\parbox[t][][t]{2.0 cm}{\normalfont \raggedleft ich\\du\\er/sie/es\\wir\\ihr\\sie} &    
\parbox[t][][t]{2cm}{\normalfont verletze\\verletzt\\verletzt\\verletzen\\verletzt\\verletzen} &
\parbox[t][][t]{2cm}{\normalfont verletzte\\verletztest\\verletzte\\verletzten\\verletztet\\verletzten}\\
\end{tabular}
\begin{tabular}{l}
\parbox[t][][t]{8cm}{}\\
\parbox[t][][t]{8cm}{\normalfont \small ['to hurt, to injure to violate (rules, laws, etc.)', '2010, Der Spiegel, issue 46/2010, page 89:', 'Unternehmen und Manager, die bei ihren Geschften', 'im Ausland Menschenrechte verletzen, sollen knftig', 'auch nach deutschem Zivil- und Wirtschaftsrecht', 'haftbar gemacht werden. Entrepreneurs and managers', 'that violate human rights during their foreign', 'business activities are to be held liable', 'according to German civil and commercial law in', 'the future.'] }\\
\end{tabular}
}
%===vermuten===
\card{\normalfont \Huge vermuten}{
\begin{tabular}{lll}
\parbox[t][][t]{2.0 cm}{\normalfont \raggedleft ich\\du\\er/sie/es\\wir\\ihr\\sie} &    
\parbox[t][][t]{2cm}{\normalfont vermute\\vermutest\\vermutet\\vermuten\\vermutet\\vermuten} &
\parbox[t][][t]{2cm}{\normalfont vermutete\\vermutetest\\vermutete\\vermuteten\\vermutetet\\vermuteten}\\
\end{tabular}
\begin{tabular}{l}
\parbox[t][][t]{8cm}{}\\
\parbox[t][][t]{8cm}{\normalfont \small ['to assume, suppose, presume, suspect, (US) guess', '(transitive, a person, at a location) to suppose', 'someone to be 1931, Arthur Schnitzler, Flucht in', 'die Finsternis, S. Fischer Verlag, page 125:', 'Niemand konnte ihn hier vermuten, er hatte das', 'Gefhl vollkommener Sicherheit, von keiner Seite', 'drohte irgendwelche Gefahr. Nobody could suppose', 'him to be here, he had the feeling of complete', 'security, from no side did any danger threaten', 'him.'] }\\
\end{tabular}
}
%===versichern===
\card{\normalfont \Huge versichern}{
\begin{tabular}{lll}
\parbox[t][][t]{2.0 cm}{\normalfont \raggedleft ich\\du\\er/sie/es\\wir\\ihr\\sie} &    
\parbox[t][][t]{2cm}{\normalfont versichreich versichereich versicher\\versicherst\\versichert\\versichern\\versichert\\versichern} &
\parbox[t][][t]{2cm}{\normalfont versicherte\\versichertest\\versicherte\\versicherten\\versichertet\\versicherten}\\
\end{tabular}
\begin{tabular}{l}
\parbox[t][][t]{8cm}{}\\
\parbox[t][][t]{8cm}{\normalfont \small ['to insure (to provide for compensation if some', 'specified risk occurs) Bei welcher Krankenkasse', 'bist du versichert? to reassure, assure Ich', 'versichere Ihnen, dass ...'] }\\
\end{tabular}
}
%===versprechen===
\card{\normalfont \Huge versprechen}{
\begin{tabular}{lll}
\parbox[t][][t]{2.0 cm}{\normalfont \raggedleft ich\\du\\er/sie/es\\wir\\ihr\\sie} &    
\parbox[t][][t]{2cm}{\normalfont verspreche\\versprichst\\verspricht\\versprechen\\versprecht\\versprechen} &
\parbox[t][][t]{2cm}{\normalfont versprach\\versprachst\\versprach\\versprachen\\verspracht\\versprachen}\\
\end{tabular}
\begin{tabular}{l}
\parbox[t][][t]{8cm}{}\\
\parbox[t][][t]{8cm}{\normalfont \small ['(transitive) to promise Ich verspreche es dir. I', 'promise you. (Literally: I promise it to you.) Er', "hat versprochen, Kuchen mitzubringen. He's", 'promised to bring along cake. 2010, Der Spiegel,', 'issue 52/2010, page 26: Wirtschaft und Regierung', 'blicken mit Zuversicht ins nchste Jahr. 2011', 'verspricht eine Fortsetzung des Aufschwungs.', 'Industry and government look with confidence into', 'the next year. 2011 promises a continuation of the', 'economic upswing. (transitive, with reflexive', 'dative) to expect (something positive); to hope', 'for Ich verspreche mir viel davon. I expect a lot', 'from it. (Literally: I promise myself a lot...)', '(reflexive, with dative object) to promise oneself', '(to) Er hatte sich einer Frau versprochen und', 'heiratete dann eine andere. He had promised', 'himself to one woman and then married another.', '(reflexive) to make a verbal slip; to misspeak', 'Entschuldigung, ich hatte mich versprochen. Ich', 'meinte "hundert", nicht "tausend". Sorry, I', 'misspoke. I meant "a hundred", not "a thousand".'] }\\
\end{tabular}
}
%===verstehen===
\card{\normalfont \Huge verstehen}{
\begin{tabular}{lll}
\parbox[t][][t]{2.0 cm}{\normalfont \raggedleft ich\\du\\er/sie/es\\wir\\ihr\\sie} &    
\parbox[t][][t]{2cm}{\normalfont verstehe\\verstehst\\versteht\\verstehen\\versteht\\verstehen} &
\parbox[t][][t]{2cm}{\normalfont verstand\\verstandest\\verstand\\verstanden\\verstandet\\verstanden}\\
\end{tabular}
\begin{tabular}{l}
\parbox[t][][t]{8cm}{}\\
\parbox[t][][t]{8cm}{\normalfont \small ['(transitive or intransitive) to understand,', 'comprehend (to be aware of the meaning of) Ich', 'verstehe, wie diese Maschine funktioniertI', 'understand how this machine works. (transitive) to', 'know, know how to, be good (with) (to understand', 'or have a grasp of through experience or study)', '(transitive) to understand, take, see (to impute', 'meaning, etc., that is not explicitly stated)', '(reflexive) to get along well (with=mit), to like', 'Sie verstehen sich miteinander.They get along well', 'with each other. (reflexive) to understand each', 'other, hear each other (to be able to communicate)', '(reflexive) to go without saying, to be obvious', '(informal) Versteht sich!  Of course! (reflexive)', 'to think of oneself (as something) (reflexive) to', 'be an expert (at something)'] }\\
\end{tabular}
}
%===versuchen===
\card{\normalfont \Huge versuchen}{
\begin{tabular}{lll}
\parbox[t][][t]{2.0 cm}{\normalfont \raggedleft ich\\du\\er/sie/es\\wir\\ihr\\sie} &    
\parbox[t][][t]{2cm}{\normalfont versuche\\versuchst\\versucht\\versuchen\\versucht\\versuchen} &
\parbox[t][][t]{2cm}{\normalfont versuchte\\versuchtest\\versuchte\\versuchten\\versuchtet\\versuchten}\\
\end{tabular}
\begin{tabular}{l}
\parbox[t][][t]{8cm}{}\\
\parbox[t][][t]{8cm}{\normalfont \small ['to try, attempt Synonym: probieren 2019, 20', 'Minuten, 28 January: Die Pillen seien fr ihren', 'gyptischen Ehemann bestimmt gewesen, da dieser', 'unter starken Rckenschmerzen leide, versuchte sie', 'sich zu erklren. The pills were meant for her', 'Egyptian husband, since he suffered from extreme', 'back pain, she tried to explain. to try, taste', 'Synonyms: kosten, probieren, verkosten to tempt', 'Synonyms: in Versuchung fhren, verfhren,', 'verlocken, verleiten'] }\\
\end{tabular}
}
%===verteilen===
\card{\normalfont \Huge verteilen}{
\begin{tabular}{lll}
\parbox[t][][t]{2.0 cm}{\normalfont \raggedleft ich\\du\\er/sie/es\\wir\\ihr\\sie} &    
\parbox[t][][t]{2cm}{\normalfont verteile\\verteilst\\verteilt\\verteilen\\verteilt\\verteilen} &
\parbox[t][][t]{2cm}{\normalfont verteilte\\verteiltest\\verteilte\\verteilten\\verteiltet\\verteilten}\\
\end{tabular}
\begin{tabular}{l}
\parbox[t][][t]{8cm}{}\\
\parbox[t][][t]{8cm}{\normalfont \small ['(transitive) to distribute (transitive) to spread', '(reflexive) to spread out'] }\\
\end{tabular}
}
%===vertrauen===
\card{\normalfont \Huge vertrauen}{
\begin{tabular}{lll}
\parbox[t][][t]{2.0 cm}{\normalfont \raggedleft ich\\du\\er/sie/es\\wir\\ihr\\sie} &    
\parbox[t][][t]{2cm}{\normalfont vertraue\\vertraust\\vertraut\\vertrauen\\vertraut\\vertrauen} &
\parbox[t][][t]{2cm}{\normalfont vertraute\\vertrautest\\vertraute\\vertrauten\\vertrautet\\vertrauten}\\
\end{tabular}
\begin{tabular}{l}
\parbox[t][][t]{8cm}{}\\
\parbox[t][][t]{8cm}{\normalfont \small ['to trust (to place confidence in)'] }\\
\end{tabular}
}
%===verwenden===
\card{\normalfont \Huge verwenden}{
\begin{tabular}{lll}
\parbox[t][][t]{2.0 cm}{\normalfont \raggedleft ich\\du\\er/sie/es\\wir\\ihr\\sie} &    
\parbox[t][][t]{2cm}{\normalfont verwende\\verwendest\\verwendet\\verwenden\\verwendet\\verwenden} &
\parbox[t][][t]{2cm}{\normalfont verwendete\\verwendetest\\verwendete\\verwendeten\\verwendetet\\verwendeten}\\
\end{tabular}
\begin{tabular}{l}
\parbox[t][][t]{8cm}{}\\
\parbox[t][][t]{8cm}{\normalfont \small ['to use (employ, apply) 1931, Gebhard Mehring,', 'Schrift und Schrifttum, Silberburg-Verlag, page', '21: Das rmische Zahlensystem [] besteht aus 7', 'Buchstaben, die zur Bezeichnung von Zahlenwerten', 'verwendet werden: M D C L X V I The Roman numeral', 'system [] consists of 7 letters, which are used', 'for the representation of numerical values: M D C', 'L X V I (formal, reflexive) to intercede Ich', 'verwende mich bei ihm fr dich. I approach him on', 'your behalf.'] }\\
\end{tabular}
}
%===verwirren===
\card{\normalfont \Huge verwirren}{
\begin{tabular}{lll}
\parbox[t][][t]{2.0 cm}{\normalfont \raggedleft ich\\du\\er/sie/es\\wir\\ihr\\sie} &    
\parbox[t][][t]{2cm}{\normalfont verwirre\\verwirrst\\verwirrt\\verwirren\\verwirrt\\verwirren} &
\parbox[t][][t]{2cm}{\normalfont verwirrte\\verwirrtest\\verwirrte\\verwirrten\\verwirrtet\\verwirrten}\\
\end{tabular}
\begin{tabular}{l}
\parbox[t][][t]{8cm}{}\\
\parbox[t][][t]{8cm}{\normalfont \small ['to confuse'] }\\
\end{tabular}
}
%===verzeihen===
\card{\normalfont \Huge verzeihen}{
\begin{tabular}{lll}
\parbox[t][][t]{2.0 cm}{\normalfont \raggedleft ich\\du\\er/sie/es\\wir\\ihr\\sie} &    
\parbox[t][][t]{2cm}{\normalfont verzeihe\\verzeihst\\verzeiht\\verzeihen\\verzeiht\\verzeihen} &
\parbox[t][][t]{2cm}{\normalfont verzieh\\verziehst\\verzieh\\verziehen\\verzieht\\verziehen}\\
\end{tabular}
\begin{tabular}{l}
\parbox[t][][t]{8cm}{}\\
\parbox[t][][t]{8cm}{\normalfont \small ['to forgive; to pardon; to excuse'] }\\
\end{tabular}
}
%===vorbereiten===
\card{\normalfont \Huge vorbereiten}{
\begin{tabular}{lll}
\parbox[t][][t]{2.0 cm}{\normalfont \raggedleft ich\\du\\er/sie/es\\wir\\ihr\\sie} &    
\parbox[t][][t]{2cm}{\normalfont bereite vor\\bereitest vor\\bereitet vor\\bereiten vor\\bereitet vor\\bereiten vor} &
\parbox[t][][t]{2cm}{\normalfont bereitete vor\\bereitetest vor\\bereitete vor\\bereiteten vor\\bereitetet vor\\bereiteten vor}\\
\end{tabular}
\begin{tabular}{l}
\parbox[t][][t]{8cm}{}\\
\parbox[t][][t]{8cm}{\normalfont \small ['(transitive) to prepare (reflexive) to prepare, to', 'get ready (auf for)'] }\\
\end{tabular}
}
%===vorkommen===
\card{\normalfont \Huge vorkommen}{
\begin{tabular}{lll}
\parbox[t][][t]{2.0 cm}{\normalfont \raggedleft ich\\du\\er/sie/es\\wir\\ihr\\sie} &    
\parbox[t][][t]{2cm}{\normalfont komme vor\\kommst vor\\kommt vor\\kommen vor\\kommt vor\\kommen vor} &
\parbox[t][][t]{2cm}{\normalfont kam vor\\kamst vor\\kam vor\\kamen vor\\kamt vor\\kamen vor}\\
\end{tabular}
\begin{tabular}{l}
\parbox[t][][t]{8cm}{}\\
\parbox[t][][t]{8cm}{\normalfont \small ['to occur, happen 2010, Der Spiegel, issue 22/2010,', 'page 126: Pasta, Kuchen, Msli, Brot  wer an', 'Zliakie leidet, muss viele gngige Lebensmittel', 'meiden: Das Eiwei Gluten, das bei den Betroffenen', 'zu chronischer Darmentzndung fhrt, kommt in den', 'meisten Getreidearten vor. Pasta, cakes, muesli,', 'bread  someone who suffers from celiac disease has', 'to avoid many common foods: the protein gluten,', 'which leads to chronic intestinal inflammation for', 'the sufferers, occurs in most types of grain. to', 'seem, appear 1931, Arthur Schnitzler, Flucht in', 'die Finsternis, S. Fischer Verlag, page 38: Er', 'ging rasch und sicher, trllerte vor sich hin,', 'endlich begann er sogar zu singen mit einer schnen', 'dunklen Stimme, die ihm selber fremd vorkam. He', 'walked fast and firmly, trilled to himself,', 'finally he even started to sing in a beautiful', 'dark voice, which seemed unfamiliar to himself.', '(reflexive) to feel 1913, Fanny zu Reventlow,', 'Herrn Dames Aufzeichnungen, Albert Langen, page', '102: In meiner Wohnung kam ich mir zuerst beinah', 'wie ein Fremder vor [] In my apartment, I almost', 'felt like a stranger at first []'] }\\
\end{tabular}
}
%===vorschlagen===
\card{\normalfont \Huge vorschlagen}{
\begin{tabular}{lll}
\parbox[t][][t]{2.0 cm}{\normalfont \raggedleft ich\\du\\er/sie/es\\wir\\ihr\\sie} &    
\parbox[t][][t]{2cm}{\normalfont schlage vor\\schlägst vor\\schlägt vor\\schlagen vor\\schlagt vor\\schlagen vor} &
\parbox[t][][t]{2cm}{\normalfont schlug vor\\schlugst vor\\schlug vor\\schlugen vor\\schlugt vor\\schlugen vor}\\
\end{tabular}
\begin{tabular}{l}
\parbox[t][][t]{8cm}{}\\
\parbox[t][][t]{8cm}{\normalfont \small ['to propose, to suggest'] }\\
\end{tabular}
}
%===vorstellen===
\card{\normalfont \Huge vorstellen}{
\begin{tabular}{lll}
\parbox[t][][t]{2.0 cm}{\normalfont \raggedleft ich\\du\\er/sie/es\\wir\\ihr\\sie} &    
\parbox[t][][t]{2cm}{\normalfont stelle vor\\stellst vor\\stellt vor\\stellen vor\\stellt vor\\stellen vor} &
\parbox[t][][t]{2cm}{\normalfont stellte vor\\stelltest vor\\stellte vor\\stellten vor\\stelltet vor\\stellten vor}\\
\end{tabular}
\begin{tabular}{l}
\parbox[t][][t]{8cm}{}\\
\parbox[t][][t]{8cm}{\normalfont \small ['(transitive) to move (something) forward', '(transitive) to put (the clocks) forward', '(transitive) to represent; to mean (transitive) to', 'introduce (someone), to present (reflexive) to', 'move forward (reflexive) to introduce oneself', '1931, Arthur Schnitzler, Flucht in die Finsternis,', 'S. Fischer Verlag, page 51: Und da er keinen Grund', 'hatte, ihr seinen Namen zu verhehlen, so stellte', 'er sich in aller Form vor. And because he had no', 'reason to conceal his name from her, he introduced', 'himself in all due form. (reflexive  dative) to', "imagine Ich kann's mir nicht vorstellen!I can't", 'imagine this! 2017, Simone Meier, Fleisch, Kein', 'Aber 2018, p. 41: Er war noch nie im Irrenhaus', 'gewesen, er kannte keine Leute dort, aber er', 'stellte sich hinter der schlosshnlichen Fassade', 'eine Reihe von Gummizellen mit tobschtigen', 'Patienten drin vor []. So far he had never been to', "the asylum and didn't know anyone there, but", 'behind the palatial faade he imagined a row of', 'padded cells with raving mad patients in them.'] }\\
\end{tabular}
}
%===wachsen===
\card{\normalfont \Huge wachsen}{
\begin{tabular}{lll}
\parbox[t][][t]{2.0 cm}{\normalfont \raggedleft ich\\du\\er/sie/es\\wir\\ihr\\sie} &    
\parbox[t][][t]{2cm}{\normalfont wachse\\wächst\\wächst\\wachsen\\wachst\\wachsen} &
\parbox[t][][t]{2cm}{\normalfont wuchs\\wuchst\\wuchs\\wuchsen\\wuchst\\wuchsen}\\
\end{tabular}
\begin{tabular}{l}
\parbox[t][][t]{8cm}{}\\
\parbox[t][][t]{8cm}{\normalfont \small ['(intransitive) to grow (intransitive,', 'figuratively) to grow, to increase 2010, Der', 'Spiegel, issue 30/2010, page 57: Zwar ist der', 'Anteil der Frauen in Betriebsratsmtern mit knapp', '25 Prozent noch vergleichsweise gering, aber er', 'wchst. It is true that the portion of women in', 'works council positions is still comparatively low', 'with 25 percent, but it is growing.'] }\\
\end{tabular}
}
%===wagen===
\card{\normalfont \Huge wagen}{
\begin{tabular}{lll}
\parbox[t][][t]{2.0 cm}{\normalfont \raggedleft ich\\du\\er/sie/es\\wir\\ihr\\sie} &    
\parbox[t][][t]{2cm}{\normalfont wage\\wagst\\wagt\\wagen\\wagt\\wagen} &
\parbox[t][][t]{2cm}{\normalfont wagte\\wagtest\\wagte\\wagten\\wagtet\\wagten}\\
\end{tabular}
\begin{tabular}{l}
\parbox[t][][t]{8cm}{}\\
\parbox[t][][t]{8cm}{\normalfont \small ['to venture, dare to risk, jeopardize'] }\\
\end{tabular}
}
%===wägen===
\card{\normalfont \Huge wägen}{
\begin{tabular}{lll}
\parbox[t][][t]{2.0 cm}{\normalfont \raggedleft ich\\du\\er/sie/es\\wir\\ihr\\sie} &    
\parbox[t][][t]{2cm}{\normalfont wäge\\wägst\\wägt\\wägen\\wägt\\wägen} &
\parbox[t][][t]{2cm}{\normalfont wog\\wogst\\wog\\wogen\\wogt\\wogen}\\
\end{tabular}
\begin{tabular}{l}
\parbox[t][][t]{8cm}{}\\
\parbox[t][][t]{8cm}{\normalfont \small ['to weigh something'] }\\
\end{tabular}
}
%===wählen===
\card{\normalfont \Huge wählen}{
\begin{tabular}{lll}
\parbox[t][][t]{2.0 cm}{\normalfont \raggedleft ich\\du\\er/sie/es\\wir\\ihr\\sie} &    
\parbox[t][][t]{2cm}{\normalfont wähle\\wählst\\wählt\\wählen\\wählt\\wählen} &
\parbox[t][][t]{2cm}{\normalfont wählte\\wähltest\\wählte\\wählten\\wähltet\\wählten}\\
\end{tabular}
\begin{tabular}{l}
\parbox[t][][t]{8cm}{}\\
\parbox[t][][t]{8cm}{\normalfont \small ['to choose to dial (a telephone number, etc.) to', 'vote Wer whlen darf, kann whlen.Those who are', 'allowed to vote can vote.'] }\\
\end{tabular}
}
%===wandern===
\card{\normalfont \Huge wandern}{
\begin{tabular}{lll}
\parbox[t][][t]{2.0 cm}{\normalfont \raggedleft ich\\du\\er/sie/es\\wir\\ihr\\sie} &    
\parbox[t][][t]{2cm}{\normalfont wandreich wandereich wander\\wanderst\\wandert\\wandern\\wandert\\wandern} &
\parbox[t][][t]{2cm}{\normalfont wanderte\\wandertest\\wanderte\\wanderten\\wandertet\\wanderten}\\
\end{tabular}
\begin{tabular}{l}
\parbox[t][][t]{8cm}{}\\
\parbox[t][][t]{8cm}{\normalfont \small ['to hike to wander'] }\\
\end{tabular}
}
%===warnen===
\card{\normalfont \Huge warnen}{
\begin{tabular}{lll}
\parbox[t][][t]{2.0 cm}{\normalfont \raggedleft ich\\du\\er/sie/es\\wir\\ihr\\sie} &    
\parbox[t][][t]{2cm}{\normalfont warne\\warnst\\warnt\\warnen\\warnt\\warnen} &
\parbox[t][][t]{2cm}{\normalfont warnte\\warntest\\warnte\\warnten\\warntet\\warnten}\\
\end{tabular}
\begin{tabular}{l}
\parbox[t][][t]{8cm}{}\\
\parbox[t][][t]{8cm}{\normalfont \small ['to warn (vor of/against)'] }\\
\end{tabular}
}
%===warten===
\card{\normalfont \Huge warten}{
\begin{tabular}{lll}
\parbox[t][][t]{2.0 cm}{\normalfont \raggedleft ich\\du\\er/sie/es\\wir\\ihr\\sie} &    
\parbox[t][][t]{2cm}{\normalfont warte\\wartest\\wartet\\warten\\wartet\\warten} &
\parbox[t][][t]{2cm}{\normalfont wartete\\wartetest\\wartete\\warteten\\wartetet\\warteten}\\
\end{tabular}
\begin{tabular}{l}
\parbox[t][][t]{8cm}{}\\
\parbox[t][][t]{8cm}{\normalfont \small ['to wait to maintain (e.g. a car)'] }\\
\end{tabular}
}
%===waschen===
\card{\normalfont \Huge waschen}{
\begin{tabular}{lll}
\parbox[t][][t]{2.0 cm}{\normalfont \raggedleft ich\\du\\er/sie/es\\wir\\ihr\\sie} &    
\parbox[t][][t]{2cm}{\normalfont wasche\\wäschst\\wäscht\\waschen\\wascht\\waschen} &
\parbox[t][][t]{2cm}{\normalfont wusch\\wuschst\\wusch\\wuschen\\wuscht\\wuschen}\\
\end{tabular}
\begin{tabular}{l}
\parbox[t][][t]{8cm}{}\\
\parbox[t][][t]{8cm}{\normalfont \small ['to wash Wasch dir die Hnde, bevor du zu Tisch', 'gehst.Wash your hands before you sit down to eat.'] }\\
\end{tabular}
}
%===wechseln===
\card{\normalfont \Huge wechseln}{
\begin{tabular}{lll}
\parbox[t][][t]{2.0 cm}{\normalfont \raggedleft ich\\du\\er/sie/es\\wir\\ihr\\sie} &    
\parbox[t][][t]{2cm}{\normalfont wechsleich wechseleich wechsel\\wechselst\\wechselt\\wechseln\\wechselt\\wechseln} &
\parbox[t][][t]{2cm}{\normalfont wechselte\\wechseltest\\wechselte\\wechselten\\wechseltet\\wechselten}\\
\end{tabular}
\begin{tabular}{l}
\parbox[t][][t]{8cm}{}\\
\parbox[t][][t]{8cm}{\normalfont \small ['change, exchange'] }\\
\end{tabular}
}
%===wecken===
\card{\normalfont \Huge wecken}{
\begin{tabular}{lll}
\parbox[t][][t]{2.0 cm}{\normalfont \raggedleft ich\\du\\er/sie/es\\wir\\ihr\\sie} &    
\parbox[t][][t]{2cm}{\normalfont wecke\\weckst\\weckt\\wecken\\weckt\\wecken} &
\parbox[t][][t]{2cm}{\normalfont weckte\\wecktest\\weckte\\weckten\\wecktet\\weckten}\\
\end{tabular}
\begin{tabular}{l}
\parbox[t][][t]{8cm}{}\\
\parbox[t][][t]{8cm}{\normalfont \small ['(transitive) to wake, to wake up 1919, Walther', 'Kabel, Irrende Seelen, Werner Dietsch Verlag, page', '16: Erst die helle Sonne, die mein Zimmer in eine', 'Flle von Licht tauchte, weckte mich.Only the', 'bright sun, which bathed my room in an abundance', 'of light, woke me up.'] }\\
\end{tabular}
}
%===wehren===
\card{\normalfont \Huge wehren}{
\begin{tabular}{lll}
\parbox[t][][t]{2.0 cm}{\normalfont \raggedleft ich\\du\\er/sie/es\\wir\\ihr\\sie} &    
\parbox[t][][t]{2cm}{\normalfont wehre\\wehrst\\wehrt\\wehren\\wehrt\\wehren} &
\parbox[t][][t]{2cm}{\normalfont wehrte\\wehrtest\\wehrte\\wehrten\\wehrtet\\wehrten}\\
\end{tabular}
\begin{tabular}{l}
\parbox[t][][t]{8cm}{}\\
\parbox[t][][t]{8cm}{\normalfont \small ['to fight (reflexive) to defend Ich wollte mich', 'doch nur wehren. - I only tried to defend myself.'] }\\
\end{tabular}
}
%===weichen===
\card{\normalfont \Huge weichen}{
\begin{tabular}{lll}
\parbox[t][][t]{2.0 cm}{\normalfont \raggedleft ich\\du\\er/sie/es\\wir\\ihr\\sie} &    
\parbox[t][][t]{2cm}{\normalfont weiche\\weichst\\weicht\\weichen\\weicht\\weichen} &
\parbox[t][][t]{2cm}{\normalfont wich\\wichst\\wich\\wichen\\wicht\\wichen}\\
\end{tabular}
\begin{tabular}{l}
\parbox[t][][t]{8cm}{}\\
\parbox[t][][t]{8cm}{\normalfont \small ['(intransitive) to wane (transitive, intransitive,', 'with dative or vor) to yield'] }\\
\end{tabular}
}
%===weinen===
\card{\normalfont \Huge weinen}{
\begin{tabular}{lll}
\parbox[t][][t]{2.0 cm}{\normalfont \raggedleft ich\\du\\er/sie/es\\wir\\ihr\\sie} &    
\parbox[t][][t]{2cm}{\normalfont weine\\weinst\\weint\\weinen\\weint\\weinen} &
\parbox[t][][t]{2cm}{\normalfont weinte\\weintest\\weinte\\weinten\\weintet\\weinten}\\
\end{tabular}
\begin{tabular}{l}
\parbox[t][][t]{8cm}{}\\
\parbox[t][][t]{8cm}{\normalfont \small ['to weep, cry'] }\\
\end{tabular}
}
%===weisen===
\card{\normalfont \Huge weisen}{
\begin{tabular}{lll}
\parbox[t][][t]{2.0 cm}{\normalfont \raggedleft ich\\du\\er/sie/es\\wir\\ihr\\sie} &    
\parbox[t][][t]{2cm}{\normalfont weise\\weist\\weist\\weisen\\weist\\weisen} &
\parbox[t][][t]{2cm}{\normalfont wies\\wiest\\wies\\wiesen\\wiest\\wiesen}\\
\end{tabular}
\begin{tabular}{l}
\parbox[t][][t]{8cm}{}\\
\parbox[t][][t]{8cm}{\normalfont \small ['(intransitive, rare) to point auf etwas weisen  to', 'point at something (transitive) to indicate; to', 'show'] }\\
\end{tabular}
}
%===wenden===
\card{\normalfont \Huge wenden}{
\begin{tabular}{lll}
\parbox[t][][t]{2.0 cm}{\normalfont \raggedleft ich\\du\\er/sie/es\\wir\\ihr\\sie} &    
\parbox[t][][t]{2cm}{\normalfont wende\\wendest\\wendet\\wenden\\wendet\\wenden} &
\parbox[t][][t]{2cm}{\normalfont wendete\\wendetest\\wendete\\wendeten\\wendetet\\wendeten}\\
\end{tabular}
\begin{tabular}{l}
\parbox[t][][t]{8cm}{}\\
\parbox[t][][t]{8cm}{\normalfont \small ['(transitive) to turn something so as to cook or', 'roast it from both sides (transitive, chiefly', 'literary) to turn something (in general)', '(transitive, literary, dated) to avert; to curb', '1545, Martin Luther, Biblia, Hans Lufft, Psalm 33:', 'Der HERR macht zunicht der Heiden Rat / Vnd wendet', 'die gedancken der Vlcker. Aber der Rat des HERRN', 'bleibet ewiglich / Seines hertzen gedancken fur', "vnd fur. The LORD makes to naught the heathens'", 'council; and curbs the cogitations of the peoples.', 'But the council of the LORD abides eternally; his', "heart's cogitations forever and ever.", '(intransitive) to make a u-turn; to turn around', "one's car or vehicle (reflexive, chiefly literary)", 'to turn around (reflexive, with an + accusative)', 'to turn to; to consult'] }\\
\end{tabular}
}
%===werben===
\card{\normalfont \Huge werben}{
\begin{tabular}{lll}
\parbox[t][][t]{2.0 cm}{\normalfont \raggedleft ich\\du\\er/sie/es\\wir\\ihr\\sie} &    
\parbox[t][][t]{2cm}{\normalfont werbe\\wirbst\\wirbt\\werben\\werbt\\werben} &
\parbox[t][][t]{2cm}{\normalfont warb\\warbst\\warb\\warben\\warbt\\warben}\\
\end{tabular}
\begin{tabular}{l}
\parbox[t][][t]{8cm}{}\\
\parbox[t][][t]{8cm}{\normalfont \small ['(transitive) to recruit; to enlist (intransitive,', 'with fr) to advertise'] }\\
\end{tabular}
}
%===werden===
\card{\normalfont \Huge werden}{
\begin{tabular}{lll}
\parbox[t][][t]{2.0 cm}{\normalfont \raggedleft ich\\du\\er/sie/es\\wir\\ihr\\sie} &    
\parbox[t][][t]{2cm}{\normalfont werde,werd', werd\\wirst\\wird\\werden\\werdet\\werden} &
\parbox[t][][t]{2cm}{\normalfont wurde;ward (archaic)\\wurdest;wardest, wardst (archaic)\\wurde;ward (archaic)\\wurden;warden (archaic)\\wurdet;wardet (archaic)\\wurden;warden (archaic)}\\
\end{tabular}
\begin{tabular}{l}
\parbox[t][][t]{8cm}{}\\
\parbox[t][][t]{8cm}{\normalfont \small ['(auxiliary, with an infinitive, past participle', 'geworden) will; to be going (to do something);', 'forms the future tense Ich werde nach Hause', 'gehen.I will go home. (auxiliary, with a past', 'participle, past participle worden) to be done;', 'forms the passive voice Das Buch wird gerade', 'gelesen. (present tense) The book is being read.', 'Er war geschlagen worden. (past perfect tense) He', 'had been beaten. (intransitive, past participle:', '"geworden") to become; to get; to grow; to turn Es', "wird heier.It's getting hotter."] }\\
\end{tabular}
}
%===werfen===
\card{\normalfont \Huge werfen}{
\begin{tabular}{lll}
\parbox[t][][t]{2.0 cm}{\normalfont \raggedleft ich\\du\\er/sie/es\\wir\\ihr\\sie} &    
\parbox[t][][t]{2cm}{\normalfont werfe\\wirfst\\wirft\\werfen\\werft\\werfen} &
\parbox[t][][t]{2cm}{\normalfont warf\\warfst\\warf\\warfen\\warft\\warfen}\\
\end{tabular}
\begin{tabular}{l}
\parbox[t][][t]{8cm}{}\\
\parbox[t][][t]{8cm}{\normalfont \small ['(transitive, intransitive) to throw (transitive,', 'computing, exception handling) to throw Der', 'Konstruktor mit dem einen Parameter wirft in', 'diesem Falle eine Ausnahme.The constructor with', 'this one parameter throws an exception in this', 'case. (transitive) To cast; to project einen', 'Schatten werfen  to cast a shadow ein Bild an/auf', 'die Wand werfen  to project an image on the wall', '(transitive, intransitive) to give birth (of some', 'animals) Die Sau wirft ihre Jungen.The sow is', 'birthing her young. (reflexive) to throw oneself', '(on a bed etc.)'] }\\
\end{tabular}
}
%===widersprechen===
\card{\normalfont \Huge widersprechen}{
\begin{tabular}{lll}
\parbox[t][][t]{2.0 cm}{\normalfont \raggedleft ich\\du\\er/sie/es\\wir\\ihr\\sie} &    
\parbox[t][][t]{2cm}{\normalfont widerspreche\\widersprichst\\widerspricht\\widersprechen\\widersprecht\\widersprechen} &
\parbox[t][][t]{2cm}{\normalfont widersprach\\widersprachst\\widersprach\\widersprachen\\widerspracht\\widersprachen}\\
\end{tabular}
\begin{tabular}{l}
\parbox[t][][t]{8cm}{}\\
\parbox[t][][t]{8cm}{\normalfont \small ['to object to contradict (reflexive) to be', 'contradictory'] }\\
\end{tabular}
}
%===widmen===
\card{\normalfont \Huge widmen}{
\begin{tabular}{lll}
\parbox[t][][t]{2.0 cm}{\normalfont \raggedleft ich\\du\\er/sie/es\\wir\\ihr\\sie} &    
\parbox[t][][t]{2cm}{\normalfont widme\\widmest\\widmet\\widmen\\widmet\\widmen} &
\parbox[t][][t]{2cm}{\normalfont widmete\\widmetest\\widmete\\widmeten\\widmetet\\widmeten}\\
\end{tabular}
\begin{tabular}{l}
\parbox[t][][t]{8cm}{}\\
\parbox[t][][t]{8cm}{\normalfont \small ['(transitive) to dedicate (reflexive, +dative) to', 'attend to'] }\\
\end{tabular}
}
%===wiederholen===
\card{\normalfont \Huge wiederholen}{
\begin{tabular}{lll}
\parbox[t][][t]{2.0 cm}{\normalfont \raggedleft ich\\du\\er/sie/es\\wir\\ihr\\sie} &    
\parbox[t][][t]{2cm}{\normalfont wiederhole\\wiederholst\\wiederholt\\wiederholen\\wiederholt\\wiederholen} &
\parbox[t][][t]{2cm}{\normalfont wiederholte\\wiederholtest\\wiederholte\\wiederholten\\wiederholtet\\wiederholten}\\
\end{tabular}
\begin{tabular}{l}
\parbox[t][][t]{8cm}{}\\
\parbox[t][][t]{8cm}{\normalfont \small ['(transitive or intransitive) to repeat (to do or', 'say again) (transitive or intransitive) to', 'recapitulate (to summarize or repeat in concise', 'form) (transitive or intransitive) to revise,', 'review, study (look over again (something', 'previously written or learned)) (transitive or', 'intransitive) to play again, replay (to perform in', '(a sport) or participate in (a game) again)', '(transitive) to retake (to take or do (an exam,', 'penalty shot, etc.) again; to photograph or film', 'again) (reflexive) to repeat oneself (to say again', 'what one has said) (reflexive) to recur, repeat', 'itself (to happen again)'] }\\
\end{tabular}
}
%===wiegen===
\card{\normalfont \Huge wiegen}{
\begin{tabular}{lll}
\parbox[t][][t]{2.0 cm}{\normalfont \raggedleft ich\\du\\er/sie/es\\wir\\ihr\\sie} &    
\parbox[t][][t]{2cm}{\normalfont wiege\\wiegst\\wiegt\\wiegen\\wiegt\\wiegen} &
\parbox[t][][t]{2cm}{\normalfont wog\\wogst\\wog\\wogen\\wogt\\wogen}\\
\end{tabular}
\begin{tabular}{l}
\parbox[t][][t]{8cm}{}\\
\parbox[t][][t]{8cm}{\normalfont \small ['(transitive) to weigh; to measure the weight of;', 'to balance (intransitive or reflexive) to weigh;', 'to be of a certain weight'] }\\
\end{tabular}
}
%===winden===
\card{\normalfont \Huge winden}{
\begin{tabular}{lll}
\parbox[t][][t]{2.0 cm}{\normalfont \raggedleft ich\\du\\er/sie/es\\wir\\ihr\\sie} &    
\parbox[t][][t]{2cm}{\normalfont winde\\windest\\windet\\winden\\windet\\winden} &
\parbox[t][][t]{2cm}{\normalfont wand\\wandest\\wand\\wanden\\wandet\\wanden}\\
\end{tabular}
\begin{tabular}{l}
\parbox[t][][t]{8cm}{}\\
\parbox[t][][t]{8cm}{\normalfont \small ['(transitive) to wind (transitive) to twist; to', 'twirl (reflexive) to squirm; to writhe'] }\\
\end{tabular}
}
%===wirken===
\card{\normalfont \Huge wirken}{
\begin{tabular}{lll}
\parbox[t][][t]{2.0 cm}{\normalfont \raggedleft ich\\du\\er/sie/es\\wir\\ihr\\sie} &    
\parbox[t][][t]{2cm}{\normalfont wirke\\wirkst\\wirkt\\wirken\\wirkt\\wirken} &
\parbox[t][][t]{2cm}{\normalfont wirkte\\wirktest\\wirkte\\wirkten\\wirktet\\wirkten}\\
\end{tabular}
\begin{tabular}{l}
\parbox[t][][t]{8cm}{}\\
\parbox[t][][t]{8cm}{\normalfont \small ['to work (to function correctly) to be effective to', 'take effect to act, behave as'] }\\
\end{tabular}
}
%===wissen===
\card{\normalfont \Huge wissen}{
\begin{tabular}{lll}
\parbox[t][][t]{2.0 cm}{\normalfont \raggedleft ich\\du\\er/sie/es\\wir\\ihr\\sie} &    
\parbox[t][][t]{2cm}{\normalfont weiß\\weißt\\weiß\\wissen\\wisst\\wissen} &
\parbox[t][][t]{2cm}{\normalfont wusste\\wusstest\\wusste\\wussten\\wusstet\\wussten}\\
\end{tabular}
\begin{tabular}{l}
\parbox[t][][t]{8cm}{}\\
\parbox[t][][t]{8cm}{\normalfont \small ['(transitive or intransitive) to know; to be aware', 'of (a fact) Ich wei, wo du bist.  "I know where', 'you are." von etwas wissen  "to know about', 'something" to remember 1960, Marie Luise', "Kaschnitz, 'Schneeschmelze': Als er neun Jahre alt", 'war, sagte die Frau, hat er mich zum ersten Mal', 'geschlagen. Weit du noch? "When he was nine," said', 'the woman, "he hit me for the first time. Do you', 'remember?"'] }\\
\end{tabular}
}
%===wohnen===
\card{\normalfont \Huge wohnen}{
\begin{tabular}{lll}
\parbox[t][][t]{2.0 cm}{\normalfont \raggedleft ich\\du\\er/sie/es\\wir\\ihr\\sie} &    
\parbox[t][][t]{2cm}{\normalfont wohne\\wohnst\\wohnt\\wohnen\\wohnt\\wohnen} &
\parbox[t][][t]{2cm}{\normalfont wohnte\\wohntest\\wohnte\\wohnten\\wohntet\\wohnten}\\
\end{tabular}
\begin{tabular}{l}
\parbox[t][][t]{8cm}{}\\
\parbox[t][][t]{8cm}{\normalfont \small ['(intransitive) to live, reside, dwell (to remain', 'or be settled permanently, or for a considerable', 'time) (intransitive) to stay (to remain in a place', 'for a definite or short period of time)'] }\\
\end{tabular}
}
%===wollen===
\card{\normalfont \Huge wollen}{
\begin{tabular}{lll}
\parbox[t][][t]{2.0 cm}{\normalfont \raggedleft ich\\du\\er/sie/es\\wir\\ihr\\sie} &    
\parbox[t][][t]{2cm}{\normalfont will\\willst\\will\\wollen\\wollt\\wollen} &
\parbox[t][][t]{2cm}{\normalfont wollte\\wolltest\\wollte\\wollten\\wolltet\\wollten}\\
\end{tabular}
\begin{tabular}{l}
\parbox[t][][t]{8cm}{}\\
\parbox[t][][t]{8cm}{\normalfont \small ['(transitive or intransitive, past participle:', '"gewollt") To want; to wish; to desire; to demand.', 'Ich will doch nur das Beste.I want only the best.', '(auxiliary, past participle: "wollen") To want (to', 'do something). Ich will gehen.  I want to go.'] }\\
\end{tabular}
}
%===wühlen===
\card{\normalfont \Huge wühlen}{
\begin{tabular}{lll}
\parbox[t][][t]{2.0 cm}{\normalfont \raggedleft ich\\du\\er/sie/es\\wir\\ihr\\sie} &    
\parbox[t][][t]{2cm}{\normalfont wühle\\wühlst\\wühlt\\wühlen\\wühlt\\wühlen} &
\parbox[t][][t]{2cm}{\normalfont wühlte\\wühltest\\wühlte\\wühlten\\wühltet\\wühlten}\\
\end{tabular}
\begin{tabular}{l}
\parbox[t][][t]{8cm}{}\\
\parbox[t][][t]{8cm}{\normalfont \small ['to rummage, to grub to dig, to burrow, to tunnel', '(figuratively, politics) to agitate'] }\\
\end{tabular}
}
%===wundern===
\card{\normalfont \Huge wundern}{
\begin{tabular}{lll}
\parbox[t][][t]{2.0 cm}{\normalfont \raggedleft ich\\du\\er/sie/es\\wir\\ihr\\sie} &    
\parbox[t][][t]{2cm}{\normalfont wundreich wundereich wunder\\wunderst\\wundert\\wundern\\wundert\\wundern} &
\parbox[t][][t]{2cm}{\normalfont wunderte\\wundertest\\wunderte\\wunderten\\wundertet\\wunderten}\\
\end{tabular}
\begin{tabular}{l}
\parbox[t][][t]{8cm}{}\\
\parbox[t][][t]{8cm}{\normalfont \small ['(reflexive) to wonder'] }\\
\end{tabular}
}
%===wünschen===
\card{\normalfont \Huge wünschen}{
\begin{tabular}{lll}
\parbox[t][][t]{2.0 cm}{\normalfont \raggedleft ich\\du\\er/sie/es\\wir\\ihr\\sie} &    
\parbox[t][][t]{2cm}{\normalfont wünsche\\wünschst\\wünscht\\wünschen\\wünscht\\wünschen} &
\parbox[t][][t]{2cm}{\normalfont wünschte\\wünschtest\\wünschte\\wünschten\\wünschtet\\wünschten}\\
\end{tabular}
\begin{tabular}{l}
\parbox[t][][t]{8cm}{}\\
\parbox[t][][t]{8cm}{\normalfont \small ['(transitive, with reflexive dative) to wish for;', 'to make a wish for; to want; to desire Ich wnsche', 'mir ein Meerschweinchen.I want a guinea pig.', '(transitive, with non-reflexive dative) to wish', 'someone something Ich wnsche dir alles Gute.I wish', 'you all the best. (transitive, without dative,', 'formal) to demand; to order; in negation: not to', 'tolerate Ich wnsche eine Erklrung!I demand an', 'explanation! Ich wnsche ein solches Verhalten', "nicht.I won't tolerate such behaviour."] }\\
\end{tabular}
}
%===zahlen===
\card{\normalfont \Huge zahlen}{
\begin{tabular}{lll}
\parbox[t][][t]{2.0 cm}{\normalfont \raggedleft ich\\du\\er/sie/es\\wir\\ihr\\sie} &    
\parbox[t][][t]{2cm}{\normalfont zahle\\zahlst\\zahlt\\zahlen\\zahlt\\zahlen} &
\parbox[t][][t]{2cm}{\normalfont zahlte\\zahltest\\zahlte\\zahlten\\zahltet\\zahlten}\\
\end{tabular}
\begin{tabular}{l}
\parbox[t][][t]{8cm}{}\\
\parbox[t][][t]{8cm}{\normalfont \small ['to pay (for something). Kellner, zahlen bitte!', 'Waiter, the bill please! to pay for, to atone for.'] }\\
\end{tabular}
}
%===zählen===
\card{\normalfont \Huge zählen}{
\begin{tabular}{lll}
\parbox[t][][t]{2.0 cm}{\normalfont \raggedleft ich\\du\\er/sie/es\\wir\\ihr\\sie} &    
\parbox[t][][t]{2cm}{\normalfont zähle\\zählst\\zählt\\zählen\\zählt\\zählen} &
\parbox[t][][t]{2cm}{\normalfont zählte\\zähltest\\zählte\\zählten\\zähltet\\zählten}\\
\end{tabular}
\begin{tabular}{l}
\parbox[t][][t]{8cm}{}\\
\parbox[t][][t]{8cm}{\normalfont \small ['(archaic) to tell, to recount (now erzhlen) 2019,', 'CaroSell (lyrics),  "Opa, ich vermisse dich", in', 'Intolerance:Ihre Trnen sind nicht zu bersehenWenn', 'sie von Opa zhlenTheir tears are not missWhen they', 'tell about grandpa to reckon, to consider, to deem', '(later used in the active voice with a passive', 'meaning) Der Wal wird nicht zu den Fischen gezhlt.', 'The whale is not considered a fish. Der Wal zhlt', 'nicht zu den Fischen.  The whale is not to be', 'considered a fish. Man zhlt mich zu den Gegnern', 'dieser Politik.  One considers me an opponent of', 'such politics. Ich zhle mich glcklich,', 'dazuzugehren.  I consider myself fortunate to be a', 'part. 2010,  Der Spiegel, number 34/2010, page', '131:Der Autosalon in Moskau zhlt zu den', 'internationalen Schaubhnen fr', 'Fahrzeuginteressierte mit unbegrenzten', 'Ansprchen.(please add an English translation of', 'this quote) to be of significance, to matter, to', 'count Jede Stimme zhlt.  Every vote counts. Was', 'zhlt, ist dass du heile aus der Sache', 'herauskommst.  What is the most important is that', 'you come out intact out of the affair. to count', '(to determine the number of objects in a group),', 'to tell Ich habe gerade gezhlt, dass wir nur noch', 'zwlf Tomaten haben.  I have just counted and seen', 'us to have only twelve tomatoes in total. Wieviele', 'Flaschen zhlst du? Ich zhle dreizehn Flaschen.', 'How many bottles do you count?  I count thirteen', 'bottles. to count (to enumerate the digits of', "one's numeral system) Unsere Tochter zhlt schon.", 'Our daughter is able to count already. To have the', 'quantity specified by the grammatical object as', 'the total amount of the subject, synonym of haben', 'Die Stadt zhlt etwa 13.000 Einwohner.The town has', 'approximately 13,000 inhabitants.'] }\\
\end{tabular}
}
%===zeichnen===
\card{\normalfont \Huge zeichnen}{
\begin{tabular}{lll}
\parbox[t][][t]{2.0 cm}{\normalfont \raggedleft ich\\du\\er/sie/es\\wir\\ihr\\sie} &    
\parbox[t][][t]{2cm}{\normalfont zeichne\\zeichnest\\zeichnet\\zeichnen\\zeichnet\\zeichnen} &
\parbox[t][][t]{2cm}{\normalfont zeichnete\\zeichnetest\\zeichnete\\zeichneten\\zeichnetet\\zeichneten}\\
\end{tabular}
\begin{tabular}{l}
\parbox[t][][t]{8cm}{}\\
\parbox[t][][t]{8cm}{\normalfont \small ['to draw'] }\\
\end{tabular}
}
%===zeigen===
\card{\normalfont \Huge zeigen}{
\begin{tabular}{lll}
\parbox[t][][t]{2.0 cm}{\normalfont \raggedleft ich\\du\\er/sie/es\\wir\\ihr\\sie} &    
\parbox[t][][t]{2cm}{\normalfont zeige\\zeigst\\zeigt\\zeigen\\zeigt\\zeigen} &
\parbox[t][][t]{2cm}{\normalfont zeigte\\zeigtest\\zeigte\\zeigten\\zeigtet\\zeigten}\\
\end{tabular}
\begin{tabular}{l}
\parbox[t][][t]{8cm}{}\\
\parbox[t][][t]{8cm}{\normalfont \small ['to point at to demonstrate, show Synonyms:', 'demonstrieren, vorfhren (reflexive) to appear,', 'become apparent, come out, turn out, show up, to', 'be manifested'] }\\
\end{tabular}
}
%===zerstieben===
\card{\normalfont \Huge zerstieben}{
\begin{tabular}{lll}
\parbox[t][][t]{2.0 cm}{\normalfont \raggedleft ich\\du\\er/sie/es\\wir\\ihr\\sie} &    
\parbox[t][][t]{2cm}{\normalfont zerstiebe\\zerstiebst\\zerstiebt\\zerstieben\\zerstiebt\\zerstieben} &
\parbox[t][][t]{2cm}{\normalfont zerstob\\zerstobst\\zerstob\\zerstoben\\zerstobt\\zerstoben}\\
\end{tabular}
\begin{tabular}{l}
\parbox[t][][t]{8cm}{}\\
\parbox[t][][t]{8cm}{\normalfont \small ['(formal) to disperse'] }\\
\end{tabular}
}
%===zerstören===
\card{\normalfont \Huge zerstören}{
\begin{tabular}{lll}
\parbox[t][][t]{2.0 cm}{\normalfont \raggedleft ich\\du\\er/sie/es\\wir\\ihr\\sie} &    
\parbox[t][][t]{2cm}{\normalfont zerstöre\\zerstörst\\zerstört\\zerstören\\zerstört\\zerstören} &
\parbox[t][][t]{2cm}{\normalfont zerstörte\\zerstörtest\\zerstörte\\zerstörten\\zerstörtet\\zerstörten}\\
\end{tabular}
\begin{tabular}{l}
\parbox[t][][t]{8cm}{}\\
\parbox[t][][t]{8cm}{\normalfont \small ['to destroy, demolish, devastate, eliminate Die', 'Alkohol hat seine Gesundheit zerstrt.Alcohol has', 'destroyed his health. to dispel die Illusion der', 'Unverwundbarkeit zerstren  to dispel the illusion', 'of invulnerability'] }\\
\end{tabular}
}
%===ziehen===
\card{\normalfont \Huge ziehen}{
\begin{tabular}{lll}
\parbox[t][][t]{2.0 cm}{\normalfont \raggedleft ich\\du\\er/sie/es\\wir\\ihr\\sie} &    
\parbox[t][][t]{2cm}{\normalfont ziehe\\ziehst\\zieht\\ziehen\\zieht\\ziehen} &
\parbox[t][][t]{2cm}{\normalfont zog\\zogst\\zog\\zogen\\zogt\\zogen}\\
\end{tabular}
\begin{tabular}{l}
\parbox[t][][t]{8cm}{}\\
\parbox[t][][t]{8cm}{\normalfont \small ['(transitive or intransitive, auxiliary: "haben")', 'to pull (e.g., a door handle); to drag', '(transitive, auxiliary: "haben") to draw (e.g. a', 'weapon); to extract; to puff (transitive,', 'auxiliary: "haben") to draw (a conclusion, lesson,', 'etc.) 2010, Der Spiegel, issue 25/2010, page 77:', 'Es gilt deshalb, die richtigen Lehren aus der', 'Krise zu ziehen, aus den Fehlern der Vergangenheit', 'zu lernen, um die Zukunft zu sichern.Therefore it', 'is necessary to draw the right lessons from the', 'crisis, to learn from the mistakes of the past for', 'securing the future. (impersonal, intransitive) to', "be drafty Es zieht.  It's drafty. (intransitive,", 'auxiliary: "sein") to move; to migrate Ich ziehe', 'nach Hamburg, aber mein Bruder zieht in eine', "andere Stadt.I'm moving to Hamburg, but my brother", 'is moving to another city. (intransitive,', 'auxiliary: "sein") to roam; to head (reflexive,', 'auxiliary: "haben") to stretch; to warp'] }\\
\end{tabular}
}
%===zuhören===
\card{\normalfont \Huge zuhören}{
\begin{tabular}{lll}
\parbox[t][][t]{2.0 cm}{\normalfont \raggedleft ich\\du\\er/sie/es\\wir\\ihr\\sie} &    
\parbox[t][][t]{2cm}{\normalfont höre zu\\hörst zu\\hört zu\\hören zu\\hört zu\\hören zu} &
\parbox[t][][t]{2cm}{\normalfont hörte zu\\hörtest zu\\hörte zu\\hörten zu\\hörtet zu\\hörten zu}\\
\end{tabular}
\begin{tabular}{l}
\parbox[t][][t]{8cm}{}\\
\parbox[t][][t]{8cm}{\normalfont \small ['to listen ([+dative = to someone or something])'] }\\
\end{tabular}
}
%===zunehmen===
\card{\normalfont \Huge zunehmen}{
\begin{tabular}{lll}
\parbox[t][][t]{2.0 cm}{\normalfont \raggedleft ich\\du\\er/sie/es\\wir\\ihr\\sie} &    
\parbox[t][][t]{2cm}{\normalfont nehme zu\\nimmst zu\\nimmt zu\\nehmen zu\\nehmt zu\\nehmen zu} &
\parbox[t][][t]{2cm}{\normalfont nahm zu\\nahmst zu\\nahm zu\\nahmen zu\\nahmt zu\\nahmen zu}\\
\end{tabular}
\begin{tabular}{l}
\parbox[t][][t]{8cm}{}\\
\parbox[t][][t]{8cm}{\normalfont \small ['to increase to gain weight'] }\\
\end{tabular}
}
%===zusammenarbeiten===
\card{\normalfont \Huge zusammenarbeiten}{
\begin{tabular}{lll}
\parbox[t][][t]{2.0 cm}{\normalfont \raggedleft ich\\du\\er/sie/es\\wir\\ihr\\sie} &    
\parbox[t][][t]{2cm}{\normalfont arbeite zusammen\\arbeitest zusammen\\arbeitet zusammen\\arbeiten zusammen\\arbeitet zusammen\\arbeiten zusammen} &
\parbox[t][][t]{2cm}{\normalfont arbeitete zusammen\\arbeitetest zusammen\\arbeitete zusammen\\arbeiteten zusammen\\arbeitetet zusammen\\arbeiteten zusammen}\\
\end{tabular}
\begin{tabular}{l}
\parbox[t][][t]{8cm}{}\\
\parbox[t][][t]{8cm}{\normalfont \small ['to work together, to cooperate'] }\\
\end{tabular}
}
%===zusammenfassen===
\card{\normalfont \Huge zusammenfassen}{
\begin{tabular}{lll}
\parbox[t][][t]{2.0 cm}{\normalfont \raggedleft ich\\du\\er/sie/es\\wir\\ihr\\sie} &    
\parbox[t][][t]{2cm}{\normalfont fasse zusammen\\fasst zusammen\\fasst zusammen\\fassen zusammen\\fasst zusammen\\fassen zusammen} &
\parbox[t][][t]{2cm}{\normalfont fasste zusammen\\fasstest zusammen\\fasste zusammen\\fassten zusammen\\fasstet zusammen\\fassten zusammen}\\
\end{tabular}
\begin{tabular}{l}
\parbox[t][][t]{8cm}{}\\
\parbox[t][][t]{8cm}{\normalfont \small ['to summarize, to sum up to consolidate'] }\\
\end{tabular}
}
%===zweifeln===
\card{\normalfont \Huge zweifeln}{
\begin{tabular}{lll}
\parbox[t][][t]{2.0 cm}{\normalfont \raggedleft ich\\du\\er/sie/es\\wir\\ihr\\sie} &    
\parbox[t][][t]{2cm}{\normalfont zweifleich zweifeleich zweifel\\zweifelst\\zweifelt\\zweifeln\\zweifelt\\zweifeln} &
\parbox[t][][t]{2cm}{\normalfont zweifelte\\zweifeltest\\zweifelte\\zweifelten\\zweifeltet\\zweifelten}\\
\end{tabular}
\begin{tabular}{l}
\parbox[t][][t]{8cm}{}\\
\parbox[t][][t]{8cm}{\normalfont \small ['to doubt, to be doubtful (an etwas = of something)', '1931, Arthur Schnitzler, Flucht in die Finsternis,', 'S. Fischer Verlag, page 136: Er ist der berhmte', 'Arzt, niemand wird an der Richtigkeit seiner', 'Diagnose zweifeln. He is the famous doctor, nobody', 'will be doubtful about the correctness of his', 'diagnosis.'] }\\
\end{tabular}
}
%===zwingen===
\card{\normalfont \Huge zwingen}{
\begin{tabular}{lll}
\parbox[t][][t]{2.0 cm}{\normalfont \raggedleft ich\\du\\er/sie/es\\wir\\ihr\\sie} &    
\parbox[t][][t]{2cm}{\normalfont zwinge\\zwingst\\zwingt\\zwingen\\zwingt\\zwingen} &
\parbox[t][][t]{2cm}{\normalfont zwang\\zwangst\\zwang\\zwangen\\zwangt\\zwangen}\\
\end{tabular}
\begin{tabular}{l}
\parbox[t][][t]{8cm}{}\\
\parbox[t][][t]{8cm}{\normalfont \small ['(transitive or reflexive) to force; to compel; to', 'make (someone do something) 1908, Walther Kabel,', 'Das Tagebuch eines Irren, in: Bibliothek der', 'Unterhaltung und des Wissens, vol. 9, Union', 'Deutsche Verlagsgesellschaft, p. 132: Nur mit', 'Aufbietung seiner ganzen Energie zwang er sich zur', 'Ruhe. Only with the mobilization of all his energy', 'did he force himself to be calm. (intransitive,', 'with "zu ...") to necessitate; to call for'] }\\
\end{tabular}
}\end{document}