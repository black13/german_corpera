\documentclass[a4paper,backgrid,frontgrid]{flacards}
\usepackage{array,booktabs,tabularx}
\usepackage[condensed,sfdefault]{universalis}
\usepackage[T1]{fontenc}
\begin{document}
\pagesetup{2}{4}

%===beleidigen===
\card{\normalfont \Huge beleidigen}{
\begin{tabular}{lll}
\parbox[t][][t]{2.0 cm}{\normalfont \raggedleft ich\\du\\er/sie/es\\wir\\ihr\\sie} &    
\parbox[t][][t]{2cm}{\normalfont beleidige\\beleidigst\\beleidigt\\beleidigen\\beleidigt\\beleidigen} &
\parbox[t][][t]{2cm}{\normalfont beleidigte\\beleidigtest\\beleidigte\\beleidigten\\beleidigtet\\beleidigten}\\
\end{tabular}
\begin{tabular}{l}
\parbox[t][][t]{8cm}{}\\
\parbox[t][][t]{8cm}{\normalfont \small ['To offend'] }\\
\end{tabular}
}
%===behaupten===
\card{\normalfont \Huge behaupten}{
\begin{tabular}{lll}
\parbox[t][][t]{2.0 cm}{\normalfont \raggedleft ich\\du\\er/sie/es\\wir\\ihr\\sie} &    
\parbox[t][][t]{2cm}{\normalfont behaupte\\behauptest\\behauptet\\behaupten\\behauptet\\behaupten} &
\parbox[t][][t]{2cm}{\normalfont behauptete\\behauptetest\\behauptete\\behaupteten\\behauptetet\\behaupteten}\\
\end{tabular}
\begin{tabular}{l}
\parbox[t][][t]{8cm}{}\\
\parbox[t][][t]{8cm}{\normalfont \small ['to claim, maintain, assert (reflexive) to stand', "one's ground"] }\\
\end{tabular}
}
%===zwingen===
\card{\normalfont \Huge zwingen}{
\begin{tabular}{lll}
\parbox[t][][t]{2.0 cm}{\normalfont \raggedleft ich\\du\\er/sie/es\\wir\\ihr\\sie} &    
\parbox[t][][t]{2cm}{\normalfont zwinge\\zwingst\\zwingt\\zwingen\\zwingt\\zwingen} &
\parbox[t][][t]{2cm}{\normalfont zwang\\zwangst\\zwang\\zwangen\\zwangt\\zwangen}\\
\end{tabular}
\begin{tabular}{l}
\parbox[t][][t]{8cm}{}\\
\parbox[t][][t]{8cm}{\normalfont \small ['(transitive or reflexive) to force; to compel; to', 'make (someone do something) 1908, Walther Kabel,', 'Das Tagebuch eines Irren, in: Bibliothek der', 'Unterhaltung und des Wissens, vol. 9, Union', 'Deutsche Verlagsgesellschaft, p. 132: Nur mit', 'Aufbietung seiner ganzen Energie zwang er sich zur', 'Ruhe. Only with the mobilization of all his energy', 'did he force himself to be calm. (intransitive,', 'with "zu ...") to necessitate; to call for'] }\\
\end{tabular}
}
%===erfinden===
\card{\normalfont \Huge erfinden}{
\begin{tabular}{lll}
\parbox[t][][t]{2.0 cm}{\normalfont \raggedleft ich\\du\\er/sie/es\\wir\\ihr\\sie} &    
\parbox[t][][t]{2cm}{\normalfont erfinde\\erfindest\\erfindet\\erfinden\\erfindet\\erfinden} &
\parbox[t][][t]{2cm}{\normalfont erfand\\erfandest\\erfand\\erfanden\\erfandet\\erfanden}\\
\end{tabular}
\begin{tabular}{l}
\parbox[t][][t]{8cm}{}\\
\parbox[t][][t]{8cm}{\normalfont \small ['(transitive) to invent (transitive) to fabricate', '(something untrue); to make up (a story)', '(transitive) to find out, to discover'] }\\
\end{tabular}
}
%===kaufen===
\card{\normalfont \Huge kaufen}{
\begin{tabular}{lll}
\parbox[t][][t]{2.0 cm}{\normalfont \raggedleft ich\\du\\er/sie/es\\wir\\ihr\\sie} &    
\parbox[t][][t]{2cm}{\normalfont kaufe\\kaufst\\kauft\\kaufen\\kauft\\kaufen} &
\parbox[t][][t]{2cm}{\normalfont kaufte\\kauftest\\kaufte\\kauften\\kauftet\\kauften}\\
\end{tabular}
\begin{tabular}{l}
\parbox[t][][t]{8cm}{}\\
\parbox[t][][t]{8cm}{\normalfont \small ['to buy Sie kauft ein Auto.She is buying a car. Ich', 'glaube, wir haben zu viel gekauft.I think we', 'bought too much.'] }\\
\end{tabular}
}
%===verletzen===
\card{\normalfont \Huge verletzen}{
\begin{tabular}{lll}
\parbox[t][][t]{2.0 cm}{\normalfont \raggedleft ich\\du\\er/sie/es\\wir\\ihr\\sie} &    
\parbox[t][][t]{2cm}{\normalfont verletze\\verletzt\\verletzt\\verletzen\\verletzt\\verletzen} &
\parbox[t][][t]{2cm}{\normalfont verletzte\\verletztest\\verletzte\\verletzten\\verletztet\\verletzten}\\
\end{tabular}
\begin{tabular}{l}
\parbox[t][][t]{8cm}{}\\
\parbox[t][][t]{8cm}{\normalfont \small ['to hurt, to injure to violate (rules, laws, etc.)', '2010, Der Spiegel, issue 46/2010, page 89:', 'Unternehmen und Manager, die bei ihren Geschften', 'im Ausland Menschenrechte verletzen, sollen knftig', 'auch nach deutschem Zivil- und Wirtschaftsrecht', 'haftbar gemacht werden. Entrepreneurs and managers', 'that violate human rights during their foreign', 'business activities are to be held liable', 'according to German civil and commercial law in', 'the future.'] }\\
\end{tabular}
}
%===trinken===
\card{\normalfont \Huge trinken}{
\begin{tabular}{lll}
\parbox[t][][t]{2.0 cm}{\normalfont \raggedleft ich\\du\\er/sie/es\\wir\\ihr\\sie} &    
\parbox[t][][t]{2cm}{\normalfont trinke\\trinkst\\trinkt\\trinken\\trinkt\\trinken} &
\parbox[t][][t]{2cm}{\normalfont trank\\trankst\\trank\\tranken\\trankt\\tranken}\\
\end{tabular}
\begin{tabular}{l}
\parbox[t][][t]{8cm}{}\\
\parbox[t][][t]{8cm}{\normalfont \small ['(transitive) to drink (to consume (a liquid)', 'through the mouth or the liquid contained within', '(a bottle, glass, etc.)) (intransitive) to drink', '(to consume alcoholic beverages) (intransitive) to', 'drink; to toast (engage in a salutation (of', 'someone), accompanying the raising of glasses', "while drinking alcohol) (reflexive) to drink one's", 'fill; to drink to satiety'] }\\
\end{tabular}
}
%===schauen===
\card{\normalfont \Huge schauen}{
\begin{tabular}{lll}
\parbox[t][][t]{2.0 cm}{\normalfont \raggedleft ich\\du\\er/sie/es\\wir\\ihr\\sie} &    
\parbox[t][][t]{2cm}{\normalfont schaue\\schaust\\schaut\\schauen\\schaut\\schauen} &
\parbox[t][][t]{2cm}{\normalfont schaute\\schautest\\schaute\\schauten\\schautet\\schauten}\\
\end{tabular}
\begin{tabular}{l}
\parbox[t][][t]{8cm}{}\\
\parbox[t][][t]{8cm}{\normalfont \small ['to look (at something)'] }\\
\end{tabular}
}
%===diskutieren===
\card{\normalfont \Huge diskutieren}{
\begin{tabular}{lll}
\parbox[t][][t]{2.0 cm}{\normalfont \raggedleft ich\\du\\er/sie/es\\wir\\ihr\\sie} &    
\parbox[t][][t]{2cm}{\normalfont diskutiere\\diskutierst\\diskutiert\\diskutieren\\diskutiert\\diskutieren} &
\parbox[t][][t]{2cm}{\normalfont diskutierte\\diskutiertest\\diskutierte\\diskutierten\\diskutiertet\\diskutierten}\\
\end{tabular}
\begin{tabular}{l}
\parbox[t][][t]{8cm}{}\\
\parbox[t][][t]{8cm}{\normalfont \small ['to discuss ber dieses Thema mssen wir diskutieren.', 'We need to discuss this issue.'] }\\
\end{tabular}
}
%===vorkommen===
\card{\normalfont \Huge vorkommen}{
\begin{tabular}{lll}
\parbox[t][][t]{2.0 cm}{\normalfont \raggedleft ich\\du\\er/sie/es\\wir\\ihr\\sie} &    
\parbox[t][][t]{2cm}{\normalfont komme vor\\kommst vor\\kommt vor\\kommen vor\\kommt vor\\kommen vor} &
\parbox[t][][t]{2cm}{\normalfont kam vor\\kamst vor\\kam vor\\kamen vor\\kamt vor\\kamen vor}\\
\end{tabular}
\begin{tabular}{l}
\parbox[t][][t]{8cm}{}\\
\parbox[t][][t]{8cm}{\normalfont \small ['to occur, happen 2010, Der Spiegel, issue 22/2010,', 'page 126: Pasta, Kuchen, Msli, Brot  wer an', 'Zliakie leidet, muss viele gngige Lebensmittel', 'meiden: Das Eiwei Gluten, das bei den Betroffenen', 'zu chronischer Darmentzndung fhrt, kommt in den', 'meisten Getreidearten vor. Pasta, cakes, muesli,', 'bread  someone who suffers from celiac disease has', 'to avoid many common foods: the protein gluten,', 'which leads to chronic intestinal inflammation for', 'the sufferers, occurs in most types of grain. to', 'seem, appear 1931, Arthur Schnitzler, Flucht in', 'die Finsternis, S. Fischer Verlag, page 38: Er', 'ging rasch und sicher, trllerte vor sich hin,', 'endlich begann er sogar zu singen mit einer schnen', 'dunklen Stimme, die ihm selber fremd vorkam. He', 'walked fast and firmly, trilled to himself,', 'finally he even started to sing in a beautiful', 'dark voice, which seemed unfamiliar to himself.', '(reflexive) to feel 1913, Fanny zu Reventlow,', 'Herrn Dames Aufzeichnungen, Albert Langen, page', '102: In meiner Wohnung kam ich mir zuerst beinah', 'wie ein Fremder vor [] In my apartment, I almost', 'felt like a stranger at first []'] }\\
\end{tabular}
}
%===feixen===
\card{\normalfont \Huge feixen}{
\begin{tabular}{lll}
\parbox[t][][t]{2.0 cm}{\normalfont \raggedleft ich\\du\\er/sie/es\\wir\\ihr\\sie} &    
\parbox[t][][t]{2cm}{\normalfont feixe\\feixt\\feixt\\feixen\\feixt\\feixen} &
\parbox[t][][t]{2cm}{\normalfont feixte\\feixtest\\feixte\\feixten\\feixtet\\feixten}\\
\end{tabular}
\begin{tabular}{l}
\parbox[t][][t]{8cm}{}\\
\parbox[t][][t]{8cm}{\normalfont \small ['(colloquial) to smirk'] }\\
\end{tabular}
}
%===erhöhen===
\card{\normalfont \Huge erhöhen}{
\begin{tabular}{lll}
\parbox[t][][t]{2.0 cm}{\normalfont \raggedleft ich\\du\\er/sie/es\\wir\\ihr\\sie} &    
\parbox[t][][t]{2cm}{\normalfont erhöhe\\erhöhst\\erhöht\\erhöhen\\erhöht\\erhöhen} &
\parbox[t][][t]{2cm}{\normalfont erhöhte\\erhöhtest\\erhöhte\\erhöhten\\erhöhtet\\erhöhten}\\
\end{tabular}
\begin{tabular}{l}
\parbox[t][][t]{8cm}{}\\
\parbox[t][][t]{8cm}{\normalfont \small ['to heighten, raise unsere Moral erhhento heighten', 'our morality to increase to exalt Wer sich selbst', 'erhhet, der soll erniedriget werden.[1] The one', 'who exalts oneself, that one shall be humbled.'] }\\
\end{tabular}
}
%===stoßen===
\card{\normalfont \Huge stoßen}{
\begin{tabular}{lll}
\parbox[t][][t]{2.0 cm}{\normalfont \raggedleft ich\\du\\er/sie/es\\wir\\ihr\\sie} &    
\parbox[t][][t]{2cm}{\normalfont stoße\\stößt\\stößt\\stoßen\\stoßt\\stoßen} &
\parbox[t][][t]{2cm}{\normalfont stieß\\stießt\\stieß\\stießen\\stießt\\stießen}\\
\end{tabular}
\begin{tabular}{l}
\parbox[t][][t]{8cm}{}\\
\parbox[t][][t]{8cm}{\normalfont \small ['(transitive) to push; to shove; to thrust', '(transitive or reflexive) to bump; to knock; to', 'strike; to hurt (reflexive, figuratively, with an)', 'to take exception (to something) (intransitive) to', 'jolt; to kick; to thrust'] }\\
\end{tabular}
}
%===spalten===
\card{\normalfont \Huge spalten}{
\begin{tabular}{lll}
\parbox[t][][t]{2.0 cm}{\normalfont \raggedleft ich\\du\\er/sie/es\\wir\\ihr\\sie} &    
\parbox[t][][t]{2cm}{\normalfont spalte\\spaltest\\spaltet\\spalten\\spaltet\\spalten} &
\parbox[t][][t]{2cm}{\normalfont spaltete\\spaltetest\\spaltete\\spalteten\\spaltetet\\spalteten}\\
\end{tabular}
\begin{tabular}{l}
\parbox[t][][t]{8cm}{}\\
\parbox[t][][t]{8cm}{\normalfont \small ['(transitive) to split (something); to cleave; to', 'chop (reflexive) to split up; to become divided'] }\\
\end{tabular}
}
%===verkaufen===
\card{\normalfont \Huge verkaufen}{
\begin{tabular}{lll}
\parbox[t][][t]{2.0 cm}{\normalfont \raggedleft ich\\du\\er/sie/es\\wir\\ihr\\sie} &    
\parbox[t][][t]{2cm}{\normalfont verkaufe\\verkaufst\\verkauft\\verkaufen\\verkauft\\verkaufen} &
\parbox[t][][t]{2cm}{\normalfont verkaufte\\verkauftest\\verkaufte\\verkauften\\verkauftet\\verkauften}\\
\end{tabular}
\begin{tabular}{l}
\parbox[t][][t]{8cm}{}\\
\parbox[t][][t]{8cm}{\normalfont \small ['to sell'] }\\
\end{tabular}
}
%===werben===
\card{\normalfont \Huge werben}{
\begin{tabular}{lll}
\parbox[t][][t]{2.0 cm}{\normalfont \raggedleft ich\\du\\er/sie/es\\wir\\ihr\\sie} &    
\parbox[t][][t]{2cm}{\normalfont werbe\\wirbst\\wirbt\\werben\\werbt\\werben} &
\parbox[t][][t]{2cm}{\normalfont warb\\warbst\\warb\\warben\\warbt\\warben}\\
\end{tabular}
\begin{tabular}{l}
\parbox[t][][t]{8cm}{}\\
\parbox[t][][t]{8cm}{\normalfont \small ['(transitive) to recruit; to enlist (intransitive,', 'with fr) to advertise'] }\\
\end{tabular}
}\end{document}