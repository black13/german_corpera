\documentclass[a4paper,backgrid,frontgrid]{flacards}
\usepackage{array,booktabs,tabularx}
\usepackage[condensed,sfdefault]{universalis}
\usepackage[T1]{fontenc}
\begin{document}
\pagesetup{2}{4}


%===ändern===
\card{\normalfont \Huge ändern}{
\begin{tabular}{lll}
\parbox[t][][t]{2.0 cm}{\normalfont \raggedleft ich\\du\\er/sie/es\\wir\\ihr\\sie} &    
\parbox[t][][t]{2cm}{\normalfont ändere\\änderst\\ändert\\ändern\\ändert\\ändern} &
\parbox[t][][t]{2cm}{\normalfont änderte\\ändertest\\änderte\\änderten\\ändertet\\änderten}\\
\end{tabular}
\begin{tabular}{l}
\parbox[t][][t]{8cm}{}\\
\parbox[t][][t]{8cm}{\normalfont \footnotesize 
etw ändern - 
to change (or alter) sth - 
ich kann es nicht ändern - 
I can't do anything about it - 
(s)eine Meinung ändern - 
to change one's mind - 
den Namen ändern - 
to change one's name - 
das ändert nichts daran, dass ... - 
that doesn't change (or alter) the fact that 
}\\
\end{tabular}
}
%===abbiegen===
\card{\normalfont \Huge abbiegen}{
\begin{tabular}{lll}
\parbox[t][][t]{2.0 cm}{\normalfont \raggedleft ich\\du\\er/sie/es\\wir\\ihr\\sie} &    
\parbox[t][][t]{2cm}{\normalfont biege ab\\biegst ab\\biegt ab\\biegen ab\\biegt ab\\biegen ab} &
\parbox[t][][t]{2cm}{\normalfont bog ab\\bogst ab\\bog ab\\bogen ab\\bogt ab\\bogen ab}\\
\end{tabular}
\begin{tabular}{l}
\parbox[t][][t]{8cm}{}\\
\parbox[t][][t]{8cm}{\normalfont \footnotesize 
etw abbiegen - 
to get out of sth fam - 
ich sollte eine Rede halten, aber zum Glück konnte ich das abbiegen - 
I was supposed to give a speech but luckily I managed to get out of it - 
einen Plan abbiegen - 
to forestall a plan - 
einen Finger abbiegen - 
to bend a finger - 
abbiegen - 
to turn (off) 
}\\
\end{tabular}
}
%===abfahren===
\card{\normalfont \Huge abfahren}{
\begin{tabular}{lll}
\parbox[t][][t]{2.0 cm}{\normalfont \raggedleft ich\\du\\er/sie/es\\wir\\ihr\\sie} &    
\parbox[t][][t]{2cm}{\normalfont fahre ab\\fährst ab\\fährt ab\\fahren ab\\fahrt ab\\fahren ab} &
\parbox[t][][t]{2cm}{\normalfont fuhr ab\\fuhrst ab\\fuhr ab\\fuhren ab\\fuhrt ab\\fuhren ab}\\
\end{tabular}
\begin{tabular}{l}
\parbox[t][][t]{8cm}{}\\
\parbox[t][][t]{8cm}{\normalfont \footnotesize 
abfahren - 
to depart - 
abfahren - 
to leave - 
abfahren Auto, Fahrer a. - 
to drive off fam - 
ich fahre in ein paar Tagen ab - 
I'll be leaving in a couple of days - 
abfahren - 
to ski down 
}\\
\end{tabular}
}
%===abgeben===
\card{\normalfont \Huge abgeben}{
\begin{tabular}{lll}
\parbox[t][][t]{2.0 cm}{\normalfont \raggedleft ich\\du\\er/sie/es\\wir\\ihr\\sie} &    
\parbox[t][][t]{2cm}{\normalfont gebe ab\\gibst ab\\gibt ab\\geben ab\\gebt ab\\geben ab} &
\parbox[t][][t]{2cm}{\normalfont gab ab\\gabst ab\\gab ab\\gaben ab\\gabt ab\\gaben ab}\\
\end{tabular}
\begin{tabular}{l}
\parbox[t][][t]{8cm}{}\\
\parbox[t][][t]{8cm}{\normalfont \footnotesize 
etw an jdn abgeben - 
to hand over sth sep to sb - 
abgeben Doktorarbeit, Hausarbeit - 
to submit sth (or hand in sep sth) to sb - 
etw (bei jdm) abgeben - 
to leave sth (with sb) - 
das Gepäck abgeben - 
to check (in) one's luggage (or baggage) - 
einen Koffer an der Gepäckaufbewahrung abgeben - 
to leave a case in the left-luggage office (or 
}\\
\end{tabular}
}
%===abholen===
\card{\normalfont \Huge abholen}{
\begin{tabular}{lll}
\parbox[t][][t]{2.0 cm}{\normalfont \raggedleft ich\\du\\er/sie/es\\wir\\ihr\\sie} &    
\parbox[t][][t]{2cm}{\normalfont hole ab\\holst ab\\holt ab\\holen ab\\holt ab\\holen ab} &
\parbox[t][][t]{2cm}{\normalfont holte ab\\holtest ab\\holte ab\\holten ab\\holtet ab\\holten ab}\\
\end{tabular}
\begin{tabular}{l}
\parbox[t][][t]{8cm}{}\\
\parbox[t][][t]{8cm}{\normalfont \footnotesize 
etw (bei jdm) abholen - 
to collect sth (from sb) - 
etw abholen lassen - 
to have sth collected - 
jdn (irgendwo/bei jdm) abholen - 
to pick up sep - 
(or collect) sb (from somewhere/from sb's (place)) - 
sich akk (von jdm) abholen lassen - 
to be picked up (or collected)  (by somebody) - 
jdn abholen 
}\\
\end{tabular}
}
%===ablehnen===
\card{\normalfont \Huge ablehnen}{
\begin{tabular}{lll}
\parbox[t][][t]{2.0 cm}{\normalfont \raggedleft ich\\du\\er/sie/es\\wir\\ihr\\sie} &    
\parbox[t][][t]{2cm}{\normalfont lehne ab\\lehnst ab\\lehnt ab\\lehnen ab\\lehnt ab\\lehnen ab} &
\parbox[t][][t]{2cm}{\normalfont lehnte ab\\lehntest ab\\lehnte ab\\lehnten ab\\lehntet ab\\lehnten ab}\\
\end{tabular}
\begin{tabular}{l}
\parbox[t][][t]{8cm}{}\\
\parbox[t][][t]{8cm}{\normalfont \footnotesize 
etw ablehnen - 
to turn down sth sep - 
etw ablehnen - 
to refuse (or reject) sth - 
einen Antrag ablehnen - 
(Vorschlag) - 
to reject (or defeat)  a proposal - 
einen Antrag ablehnen - 
(Gesuch) - 
to reject an application 
}\\
\end{tabular}
}
%===abnehmen===
\card{\normalfont \Huge abnehmen}{
\begin{tabular}{lll}
\parbox[t][][t]{2.0 cm}{\normalfont \raggedleft ich\\du\\er/sie/es\\wir\\ihr\\sie} &    
\parbox[t][][t]{2cm}{\normalfont nehme ab\\nimmst ab\\nimmt ab\\nehmen ab\\nehmt ab\\nehmen ab} &
\parbox[t][][t]{2cm}{\normalfont nahm ab\\nahmst ab\\nahm ab\\nahmen ab\\nahmt ab\\nahmen ab}\\
\end{tabular}
\begin{tabular}{l}
\parbox[t][][t]{8cm}{}\\
\parbox[t][][t]{8cm}{\normalfont \footnotesize 
etw abnehmen - 
to take off sth sep - 
etw abnehmen - 
(herunternehmen) - 
to take down sth sep - 
nehmen Sie bitte den Hut ab! - 
please take your hat off! - 
bei der Jacke kann man die Kapuze abnehmen - 
the jacket has a detachable hood - 
sich dat den Bart abnehmen 
}\\
\end{tabular}
}
%===achten===
\card{\normalfont \Huge achten}{
\begin{tabular}{lll}
\parbox[t][][t]{2.0 cm}{\normalfont \raggedleft ich\\du\\er/sie/es\\wir\\ihr\\sie} &    
\parbox[t][][t]{2cm}{\normalfont achte\\achtest\\achtet\\achten\\achtet\\achten} &
\parbox[t][][t]{2cm}{\normalfont achtete\\achtetest\\achtete\\achteten\\achtetet\\achteten}\\
\end{tabular}
\begin{tabular}{l}
\parbox[t][][t]{8cm}{}\\
\parbox[t][][t]{8cm}{\normalfont \footnotesize 
jdn achten - 
to respect sb - 
jdn als etw achten - 
to respect sb as sth - 
auf jdn/etw achten - 
to look after (or - 
fam keep an eye on) sb/sth - 
auf jdn/etw achten - 
to pay attention to sb/sth - 
auf das Kleingedruckte achten 
}\\
\end{tabular}
}
%===anbieten===
\card{\normalfont \Huge anbieten}{
\begin{tabular}{lll}
\parbox[t][][t]{2.0 cm}{\normalfont \raggedleft ich\\du\\er/sie/es\\wir\\ihr\\sie} &    
\parbox[t][][t]{2cm}{\normalfont biete an\\bietest an\\bietet an\\bieten an\\bietet an\\bieten an} &
\parbox[t][][t]{2cm}{\normalfont bot an\\botest an\\bot an\\boten an\\botet an\\boten an}\\
\end{tabular}
\begin{tabular}{l}
\parbox[t][][t]{8cm}{}\\
\parbox[t][][t]{8cm}{\normalfont \footnotesize 
(jdm) etw anbieten - 
to offer (sb) sth - 
(jdm) etw anbieten - 
to offer sth (to sb) - 
darf ich Ihnen noch ein Stück Kuchen anbieten? - 
would you like another piece of cake? - 
na, was bietet die Speisekarte denn heute an? - 
well, what's on the menu today? - 
dieser Laden bietet regelmäßig verschiedene Südfrüchte an - 
this shop often has exotic fruit for sale 
}\\
\end{tabular}
}
%===anfangen===
\card{\normalfont \Huge anfangen}{
\begin{tabular}{lll}
\parbox[t][][t]{2.0 cm}{\normalfont \raggedleft ich\\du\\er/sie/es\\wir\\ihr\\sie} &    
\parbox[t][][t]{2cm}{\normalfont fange an\\fängst an\\fängt an\\fangen an\\fangt an\\fangen an} &
\parbox[t][][t]{2cm}{\normalfont fing an\\fingst an\\fing an\\fingen an\\fingt an\\fingen an}\\
\end{tabular}
\begin{tabular}{l}
\parbox[t][][t]{8cm}{}\\
\parbox[t][][t]{8cm}{\normalfont \footnotesize 
etw anfangen - 
to begin (or start) sth - 
etw (mit jdm) anfangen - 
to start (up) sth (with sb) - 
er fing ein Gespräch mit ihr an - 
he started (or struck up)  a conversation with her - 
er fing ein Gespräch mit ihr an - 
he started talking to her - 
etw mit etw dat - 
anfangen 
}\\
\end{tabular}
}
%===anfassen===
\card{\normalfont \Huge anfassen}{
\begin{tabular}{lll}
\parbox[t][][t]{2.0 cm}{\normalfont \raggedleft ich\\du\\er/sie/es\\wir\\ihr\\sie} &    
\parbox[t][][t]{2cm}{\normalfont fasse an\\fasst an\\fasst an\\fassen an\\fasst an\\fassen an} &
\parbox[t][][t]{2cm}{\normalfont fasste an\\fasstest an\\fasste an\\fassten an\\fasstet an\\fassten an}\\
\end{tabular}
\begin{tabular}{l}
\parbox[t][][t]{8cm}{}\\
\parbox[t][][t]{8cm}{\normalfont \footnotesize 
etw anfassen - 
to touch sth - 
die Lebensmittel bitte nicht anfassen - 
please do not handle the groceries - 
fass mal ihre Stirn an, wie heiß die ist! - 
feel how hot her forehead is! - 
fass mich nicht an! - 
don't (you) touch me! - 
jdn anfassen - 
to take hold of sb 
}\\
\end{tabular}
}
%===ankommen===
\card{\normalfont \Huge ankommen}{
\begin{tabular}{lll}
\parbox[t][][t]{2.0 cm}{\normalfont \raggedleft ich\\du\\er/sie/es\\wir\\ihr\\sie} &    
\parbox[t][][t]{2cm}{\normalfont komme an\\kommst an\\kommt an\\kommen an\\kommt an\\kommen an} &
\parbox[t][][t]{2cm}{\normalfont kam an\\kamst an\\kam an\\kamen an\\kamt an\\kamen an}\\
\end{tabular}
\begin{tabular}{l}
\parbox[t][][t]{8cm}{}\\
\parbox[t][][t]{8cm}{\normalfont \footnotesize 
ankommen - 
to arrive - 
seid ihr auch gut angekommen? - 
did you arrive safely? - 
(bei jdm) ankommen - 
to be delivered (to sb) - 
bei etw dat - 
ankommen - 
to reach sth - 
ankommen 
}\\
\end{tabular}
}
%===anmachen===
\card{\normalfont \Huge anmachen}{
\begin{tabular}{lll}
\parbox[t][][t]{2.0 cm}{\normalfont \raggedleft ich\\du\\er/sie/es\\wir\\ihr\\sie} &    
\parbox[t][][t]{2cm}{\normalfont mache an\\machst an\\macht an\\machen an\\macht an\\machen an} &
\parbox[t][][t]{2cm}{\normalfont machte an\\machtest an\\machte an\\machten an\\machtet an\\machten an}\\
\end{tabular}
\begin{tabular}{l}
\parbox[t][][t]{8cm}{}\\
\parbox[t][][t]{8cm}{\normalfont \footnotesize 
etw (an etw akk o dat) anmachen - 
Brosche, Gardinen etc. - 
to put sth on sep  (to sth) - 
etw anmachen - 
to turn (or put) sth on sep - 
etw anmachen - 
to light sth - 
etw (mit etw dat) anmachen - 
to dress sth (with sth) - 
jdn anmachen 
}\\
\end{tabular}
}
%===anmelden===
\card{\normalfont \Huge anmelden}{
\begin{tabular}{lll}
\parbox[t][][t]{2.0 cm}{\normalfont \raggedleft ich\\du\\er/sie/es\\wir\\ihr\\sie} &    
\parbox[t][][t]{2cm}{\normalfont melde an\\meldest an\\meldet an\\melden an\\meldet an\\melden an} &
\parbox[t][][t]{2cm}{\normalfont meldete an\\meldetest an\\meldete an\\meldeten an\\meldetet an\\meldeten an}\\
\end{tabular}
\begin{tabular}{l}
\parbox[t][][t]{8cm}{}\\
\parbox[t][][t]{8cm}{\normalfont \footnotesize 
jdn/etw (bei jdm) anmelden - 
to announce sb/sth (to sb) - 
einen Besuch anmelden - 
to announce a visit - 
wen darf ich anmelden? - 
who shall I say is calling? - 
ich bin angemeldet - 
I have an appointment - 
angemeldet - 
announced 
}\\
\end{tabular}
}
%===annehmen===
\card{\normalfont \Huge annehmen}{
\begin{tabular}{lll}
\parbox[t][][t]{2.0 cm}{\normalfont \raggedleft ich\\du\\er/sie/es\\wir\\ihr\\sie} &    
\parbox[t][][t]{2cm}{\normalfont nehme an\\nimmst an\\nimmt an\\nehmen an\\nehmt an\\nehmen an} &
\parbox[t][][t]{2cm}{\normalfont nahm an\\nahmst an\\nahm an\\nahmen an\\nahmt an\\nahmen an}\\
\end{tabular}
\begin{tabular}{l}
\parbox[t][][t]{8cm}{}\\
\parbox[t][][t]{8cm}{\normalfont \footnotesize 
etw (von jdm) annehmen - 
to accept sth (from sb) - 
nehmen Sie das Gespräch an? - 
will you take the call? - 
etw annehmen - 
to take sth (on) - 
etw annehmen - 
to accept sth - 
eine Herausforderung annehmen - 
to accept (or take up)  a challenge 
}\\
\end{tabular}
}
%===anrufen===
\card{\normalfont \Huge anrufen}{
\begin{tabular}{lll}
\parbox[t][][t]{2.0 cm}{\normalfont \raggedleft ich\\du\\er/sie/es\\wir\\ihr\\sie} &    
\parbox[t][][t]{2cm}{\normalfont rufe an\\rufst an\\ruft an\\rufen an\\ruft an\\rufen an} &
\parbox[t][][t]{2cm}{\normalfont rief an\\riefst an\\rief an\\riefen an\\rieft an\\riefen an}\\
\end{tabular}
\begin{tabular}{l}
\parbox[t][][t]{8cm}{}\\
\parbox[t][][t]{8cm}{\normalfont \footnotesize 
jdn anrufen - 
to call sb (on the telephone) - 
jdn anrufen - 
to phone sb - 
jdn anrufen - 
to give sb a ring (or - 
fam call) - 
angerufen werden - 
to get a telephone call - 
jdn anrufen 
}\\
\end{tabular}
}
%===anschauen===
\card{\normalfont \Huge anschauen}{
\begin{tabular}{lll}
\parbox[t][][t]{2.0 cm}{\normalfont \raggedleft ich\\du\\er/sie/es\\wir\\ihr\\sie} &    
\parbox[t][][t]{2cm}{\normalfont schaue an\\schaust an\\schaut an\\schauen an\\schaut an\\schauen an} &
\parbox[t][][t]{2cm}{\normalfont schaute an\\schautest an\\schaute an\\schauten an\\schautet an\\schauten an}\\
\end{tabular}
\begin{tabular}{l}
\parbox[t][][t]{8cm}{}\\
\parbox[t][][t]{8cm}{\normalfont \footnotesize 
jdn/etw anschauen - 
to look at sb/sth - 
wie schaust du mich denn an! - 
what are you looking at me like that for? - 
jdn/etw genauer anschauen - 
to look more closely at (or examine) sb/sth - 
lass mich das Foto mal anschauen - 
let me have a look at the photo - 
einen Film/die Nachrichten anschauen - 
to watch a film/the news 
}\\
\end{tabular}
}
%===ansehen===
\card{\normalfont \Huge ansehen}{
\begin{tabular}{lll}
\parbox[t][][t]{2.0 cm}{\normalfont \raggedleft ich\\du\\er/sie/es\\wir\\ihr\\sie} &    
\parbox[t][][t]{2cm}{\normalfont sehe an\\siehst an\\sieht an\\sehen an\\seht an\\sehen an} &
\parbox[t][][t]{2cm}{\normalfont sah an\\sahst an\\sah an\\sahen an\\saht an\\sahen an}\\
\end{tabular}
\begin{tabular}{l}
\parbox[t][][t]{8cm}{}\\
\parbox[t][][t]{8cm}{\normalfont \footnotesize 
jdn ansehen - 
to look at sb - 
jdn ärgerlich/böse/unschuldig ansehen - 
to give sb an irritated/angry/innocent look - 
jdn groß ansehen - 
to stare at sb (with surprise) - 
jdn nicht mehr ansehen - 
fam - 
to not look at (or want to know) sb any more - 
jdn verdutzt/verwundert ansehen 
}\\
\end{tabular}
}
%===antworten===
\card{\normalfont \Huge antworten}{
\begin{tabular}{lll}
\parbox[t][][t]{2.0 cm}{\normalfont \raggedleft ich\\du\\er/sie/es\\wir\\ihr\\sie} &    
\parbox[t][][t]{2cm}{\normalfont antworte\\antwortest\\antwortet\\antworten\\antwortet\\antworten} &
\parbox[t][][t]{2cm}{\normalfont antwortete\\antwortetest\\antwortete\\antworteten\\antwortetet\\antworteten}\\
\end{tabular}
\begin{tabular}{l}
\parbox[t][][t]{8cm}{}\\
\parbox[t][][t]{8cm}{\normalfont \footnotesize 
(jdm) antworten - 
to answer (sb) - 
(jdm) antworten - 
to reply (to sb) - 
ich kann Ihnen darauf leider nichts antworten - 
unfortunately I cannot give you an answer to that - 
was soll man darauf noch antworten?! - 
what kind of answer can you give to that? - 
auf jds Frage antworten - 
to answer sb's (or reply to sb's) question 
}\\
\end{tabular}
}
%===anziehen===
\card{\normalfont \Huge anziehen}{
\begin{tabular}{lll}
\parbox[t][][t]{2.0 cm}{\normalfont \raggedleft ich\\du\\er/sie/es\\wir\\ihr\\sie} &    
\parbox[t][][t]{2cm}{\normalfont ziehe an\\ziehst an\\zieht an\\ziehen an\\zieht an\\ziehen an} &
\parbox[t][][t]{2cm}{\normalfont zog an\\zogst an\\zog an\\zogen an\\zogt an\\zogen an}\\
\end{tabular}
\begin{tabular}{l}
\parbox[t][][t]{8cm}{}\\
\parbox[t][][t]{8cm}{\normalfont \footnotesize 
(sich dat) etw anziehen - 
to put on sth sep - 
(sich dat) etw anziehen - 
to don sth form or liter - 
(sich dat) die Schuhe anziehen - 
to put on (or slip into) one's shoes - 
jdn anziehen - 
to dress sb - 
jdm etw anziehen - 
to put sth on sb 
}\\
\end{tabular}
}
%===anzünden===
\card{\normalfont \Huge anzünden}{
\begin{tabular}{lll}
\parbox[t][][t]{2.0 cm}{\normalfont \raggedleft ich\\du\\er/sie/es\\wir\\ihr\\sie} &    
\parbox[t][][t]{2cm}{\normalfont zünde an\\zündest an\\zündet an\\zünden an\\zündet an\\zünden an} &
\parbox[t][][t]{2cm}{\normalfont zündete an\\zündetest an\\zündete an\\zündeten an\\zündetet an\\zündeten an}\\
\end{tabular}
\begin{tabular}{l}
\parbox[t][][t]{8cm}{}\\
\parbox[t][][t]{8cm}{\normalfont \footnotesize 
etw anzünden - 
to light sth - 
etw anzünden - 
to set sth on fire - 
etw anzünden - 
to set fire (or light) to sth - 
etw anzünden - 
to light sth 
}\\
\end{tabular}
}
%===ärgern===
\card{\normalfont \Huge ärgern}{
\begin{tabular}{lll}
\parbox[t][][t]{2.0 cm}{\normalfont \raggedleft ich\\du\\er/sie/es\\wir\\ihr\\sie} &    
\parbox[t][][t]{2cm}{\normalfont ärgere\\ärgerst\\ärgert\\ärgern\\ärgert\\ärgern} &
\parbox[t][][t]{2cm}{\normalfont ärgerte\\ärgertest\\ärgerte\\ärgerten\\ärgertet\\ärgerten}\\
\end{tabular}
\begin{tabular}{l}
\parbox[t][][t]{8cm}{}\\
\parbox[t][][t]{8cm}{\normalfont \footnotesize 
jdn (mit etw dat) ärgern - 
to annoy (or irritate) sb (with sth) - 
du willst mich wohl ärgern? - 
are you trying to annoy me? - 
das kann einen wirklich ärgern! - 
that is really annoying! - 
ich ärgere mich, dass ich nicht hingegangen bin - 
I'm annoyed with myself for not having gone - 
ich ärgere mich, weil er immer zu spät kommt - 
I'm fed up (or annoyed) because he's always late 
}\\
\end{tabular}
}
%===atmen===
\card{\normalfont \Huge atmen}{
\begin{tabular}{lll}
\parbox[t][][t]{2.0 cm}{\normalfont \raggedleft ich\\du\\er/sie/es\\wir\\ihr\\sie} &    
\parbox[t][][t]{2cm}{\normalfont atme\\atmest\\atmet\\atmen\\atmet\\atmen} &
\parbox[t][][t]{2cm}{\normalfont atmete\\atmetest\\atmete\\atmeten\\atmetet\\atmeten}\\
\end{tabular}
\begin{tabular}{l}
\parbox[t][][t]{8cm}{}\\
\parbox[t][][t]{8cm}{\normalfont \footnotesize 
atmen - 
to breathe - 
frei atmen - 
fig - 
to breathe freely - 
atmen - 
to breathe - 
etw atmen - 
to breathe sth (in) - 
Atmen 
}\\
\end{tabular}
}
%===auffordern===
\card{\normalfont \Huge auffordern}{
\begin{tabular}{lll}
\parbox[t][][t]{2.0 cm}{\normalfont \raggedleft ich\\du\\er/sie/es\\wir\\ihr\\sie} &    
\parbox[t][][t]{2cm}{\normalfont fordere auf\\forderst auf\\fordert auf\\fordern auf\\fordert auf\\fordern auf} &
\parbox[t][][t]{2cm}{\normalfont forderte auf\\fordertest auf\\forderte auf\\forderten auf\\fordertet auf\\forderten auf}\\
\end{tabular}
\begin{tabular}{l}
\parbox[t][][t]{8cm}{}\\
\parbox[t][][t]{8cm}{\normalfont \footnotesize 
jdn auffordern, etw zu tun - 
to ask (or - 
form request) sb to do sth - 
wir fordern Sie auf, ... - 
you are requested ... - 
jdn zum Bleiben auffordern - 
to ask (or - 
form call upon) sb to stay - 
jdn zum Gehen/Schweigen auffordern - 
to ask (or tell) sb to go/to be quiet 
}\\
\end{tabular}
}
%===aufgeben===
\card{\normalfont \Huge aufgeben}{
\begin{tabular}{lll}
\parbox[t][][t]{2.0 cm}{\normalfont \raggedleft ich\\du\\er/sie/es\\wir\\ihr\\sie} &    
\parbox[t][][t]{2cm}{\normalfont gebe auf\\gibst auf\\gibt auf\\geben auf\\gebt auf\\geben auf} &
\parbox[t][][t]{2cm}{\normalfont gab auf\\gabst auf\\gab auf\\gaben auf\\gabt auf\\gaben auf}\\
\end{tabular}
\begin{tabular}{l}
\parbox[t][][t]{8cm}{}\\
\parbox[t][][t]{8cm}{\normalfont \footnotesize 
jdm etw aufgeben - 
to pose sth for sb - 
(jdm) etw aufgeben - 
to give (or set)  (sb) sth - 
Gepäck aufgeben - 
to register luggage - 
Gepäck aufgeben - 
LUFT - 
to check in luggage - 
aufgeben 
}\\
\end{tabular}
}
%===aufheben===
\card{\normalfont \Huge aufheben}{
\begin{tabular}{lll}
\parbox[t][][t]{2.0 cm}{\normalfont \raggedleft ich\\du\\er/sie/es\\wir\\ihr\\sie} &    
\parbox[t][][t]{2cm}{\normalfont hebe auf\\hebst auf\\hebt auf\\heben auf\\hebt auf\\heben auf} &
\parbox[t][][t]{2cm}{\normalfont hob auf\\hobst auf\\hob auf\\hoben auf\\hobt auf\\hoben auf}\\
\end{tabular}
\begin{tabular}{l}
\parbox[t][][t]{8cm}{}\\
\parbox[t][][t]{8cm}{\normalfont \footnotesize 
etw (von etw dat) aufheben - 
to pick up sth sep  (off sth) - 
jdn/etw aufheben - 
to help sb (to) get up - 
jdn/etw aufheben - 
to lift up sth sep - 
(jdm) etw aufheben - 
to put aside sth sep for sb - 
(jdm) etw aufheben - 
to keep (back sep ) sth for sb 
}\\
\end{tabular}
}
%===aufhören===
\card{\normalfont \Huge aufhören}{
\begin{tabular}{lll}
\parbox[t][][t]{2.0 cm}{\normalfont \raggedleft ich\\du\\er/sie/es\\wir\\ihr\\sie} &    
\parbox[t][][t]{2cm}{\normalfont höre auf\\hörst auf\\hört auf\\hören auf\\hört auf\\hören auf} &
\parbox[t][][t]{2cm}{\normalfont hörte auf\\hörtest auf\\hörte auf\\hörten auf\\hörtet auf\\hörten auf}\\
\end{tabular}
\begin{tabular}{l}
\parbox[t][][t]{8cm}{}\\
\parbox[t][][t]{8cm}{\normalfont \footnotesize 
(mit etw dat) aufhören - 
to stop (or leave off)  (sth) - 
hör endlich auf! - 
(will you) stop it! (or leave off!) - 
mit dem Lamentieren aufhören - 
to stop complaining - 
plötzlich aufhören - 
to stop dead - 
aufhören, etw zu tun - 
to stop (or leave off) doing sth 
}\\
\end{tabular}
}
%===aufpassen===
\card{\normalfont \Huge aufpassen}{
\begin{tabular}{lll}
\parbox[t][][t]{2.0 cm}{\normalfont \raggedleft ich\\du\\er/sie/es\\wir\\ihr\\sie} &    
\parbox[t][][t]{2cm}{\normalfont passe auf\\passt auf\\passt auf\\passen auf\\passt auf\\passen auf} &
\parbox[t][][t]{2cm}{\normalfont passte auf\\passtest auf\\passte auf\\passten auf\\passtet auf\\passten auf}\\
\end{tabular}
\begin{tabular}{l}
\parbox[t][][t]{8cm}{}\\
\parbox[t][][t]{8cm}{\normalfont \footnotesize 
aufpassen - 
to pay attention - 
genau aufpassen - 
to pay close attention - 
kannst du nicht aufpassen, was man dir sagt? - 
can't you listen to what is being said to you? - 
aufpassen, dass ... - 
to take care that ... - 
pass auf! aufgepasst! (sei aufmerksam) - 
(be) careful! 
}\\
\end{tabular}
}
%===aufräumen===
\card{\normalfont \Huge aufräumen}{
\begin{tabular}{lll}
\parbox[t][][t]{2.0 cm}{\normalfont \raggedleft ich\\du\\er/sie/es\\wir\\ihr\\sie} &    
\parbox[t][][t]{2cm}{\normalfont räume auf\\räumst auf\\räumt auf\\räumen auf\\räumt auf\\räumen auf} &
\parbox[t][][t]{2cm}{\normalfont räumte auf\\räumtest auf\\räumte auf\\räumten auf\\räumtet auf\\räumten auf}\\
\end{tabular}
\begin{tabular}{l}
\parbox[t][][t]{8cm}{}\\
\parbox[t][][t]{8cm}{\normalfont \footnotesize 
etw aufräumen - 
to tidy (or clear) up sth sep - 
einen Schrank aufräumen - 
to clear (or tidy) out a cupboard sep - 
einen Schreibtisch aufräumen - 
to clear (up) a desk sep - 
Spielsachen aufräumen - 
to clear (or tidy) away toys sep - 
aufgeräumt sein - 
to be (neat and) tidy 
}\\
\end{tabular}
}
%===aufregen===
\card{\normalfont \Huge aufregen}{
\begin{tabular}{lll}
\parbox[t][][t]{2.0 cm}{\normalfont \raggedleft ich\\du\\er/sie/es\\wir\\ihr\\sie} &    
\parbox[t][][t]{2cm}{\normalfont rege auf\\regst auf\\regt auf\\regen auf\\regt auf\\regen auf} &
\parbox[t][][t]{2cm}{\normalfont regte auf\\regtest auf\\regte auf\\regten auf\\regtet auf\\regten auf}\\
\end{tabular}
\begin{tabular}{l}
\parbox[t][][t]{8cm}{}\\
\parbox[t][][t]{8cm}{\normalfont \footnotesize 
jdn aufregen - 
(erregen) - 
to excite sb - 
jdn aufregen - 
(verärgern) - 
to annoy (or irritate) sb - 
jdn aufregen - 
(nervös machen) - 
to make sb nervous - 
jdn aufregen 
}\\
\end{tabular}
}
%===aufstehen===
\card{\normalfont \Huge aufstehen}{
\begin{tabular}{lll}
\parbox[t][][t]{2.0 cm}{\normalfont \raggedleft ich\\du\\er/sie/es\\wir\\ihr\\sie} &    
\parbox[t][][t]{2cm}{\normalfont stehe auf\\stehst auf\\steht auf\\stehen auf\\steht auf\\stehen auf} &
\parbox[t][][t]{2cm}{\normalfont stand auf\\standest auf\\stand auf\\standen auf\\standet auf\\standen auf}\\
\end{tabular}
\begin{tabular}{l}
\parbox[t][][t]{8cm}{}\\
\parbox[t][][t]{8cm}{\normalfont \footnotesize 
(von etw dat) aufstehen - 
to get (or stand) up (from sth) - 
(von etw dat) aufstehen - 
to arise (or - 
form rise)  (from sth) - 
vor jdm/für jdn aufstehen - 
to get (or stand) up for (or - 
form before) sb - 
vor jdm/für jdn aufstehen - 
(aus Achtung) 
}\\
\end{tabular}
}
%===aufwachen===
\card{\normalfont \Huge aufwachen}{
\begin{tabular}{lll}
\parbox[t][][t]{2.0 cm}{\normalfont \raggedleft ich\\du\\er/sie/es\\wir\\ihr\\sie} &    
\parbox[t][][t]{2cm}{\normalfont wache auf\\wachst auf\\wacht auf\\wachen auf\\wacht auf\\wachen auf} &
\parbox[t][][t]{2cm}{\normalfont wachte auf\\wachtest auf\\wachte auf\\wachten auf\\wachtet auf\\wachten auf}\\
\end{tabular}
\begin{tabular}{l}
\parbox[t][][t]{8cm}{}\\
\parbox[t][][t]{8cm}{\normalfont \footnotesize 
aufwachen - 
to wake (up) - 
aufwachen - 
to awake(n) liter - 
aus einem Alptraum/einer Narkose aufwachen - 
to start up from a nightmare/to come round from an anaesthetic (or - 
Am anesthetic) 
}\\
\end{tabular}
}
%===ausgeben===
\card{\normalfont \Huge ausgeben}{
\begin{tabular}{lll}
\parbox[t][][t]{2.0 cm}{\normalfont \raggedleft ich\\du\\er/sie/es\\wir\\ihr\\sie} &    
\parbox[t][][t]{2cm}{\normalfont gebe aus\\gibst aus\\gibt aus\\geben aus\\gebt aus\\geben aus} &
\parbox[t][][t]{2cm}{\normalfont gab aus\\gabst aus\\gab aus\\gaben aus\\gabt aus\\gaben aus}\\
\end{tabular}
\begin{tabular}{l}
\parbox[t][][t]{8cm}{}\\
\parbox[t][][t]{8cm}{\normalfont \footnotesize 
etw (für etw akk) ausgeben - 
to spend sth (on sth) - 
einen Teil seines Gehalts für etw akk - 
ausgeben - 
to invest (or spend) part of one's salary on sth - 
etw (an jdn) ausgeben - 
to distribute (or - 
sep give out) sth (to sb) - 
etw (an jdn) ausgeben - 
(aushändigen a.) 
}\\
\end{tabular}
}
%===ausgehen===
\card{\normalfont \Huge ausgehen}{
\begin{tabular}{lll}
\parbox[t][][t]{2.0 cm}{\normalfont \raggedleft ich\\du\\er/sie/es\\wir\\ihr\\sie} &    
\parbox[t][][t]{2cm}{\normalfont gehe aus\\gehst aus\\geht aus\\gehen aus\\geht aus\\gehen aus} &
\parbox[t][][t]{2cm}{\normalfont ging aus\\gingst aus\\ging aus\\gingen aus\\gingt aus\\gingen aus}\\
\end{tabular}
\begin{tabular}{l}
\parbox[t][][t]{8cm}{}\\
\parbox[t][][t]{8cm}{\normalfont \footnotesize 
ausgehen - 
to go out - 
er ging aus, um Einkäufe zu machen - 
he went out for shopping - 
ausgegangen sein - 
to have gone out - 
ausgegangen sein - 
to be out - 
ausgehen - 
to go out 
}\\
\end{tabular}
}
%===ausmachen===
\card{\normalfont \Huge ausmachen}{
\begin{tabular}{lll}
\parbox[t][][t]{2.0 cm}{\normalfont \raggedleft ich\\du\\er/sie/es\\wir\\ihr\\sie} &    
\parbox[t][][t]{2cm}{\normalfont mache aus\\machst aus\\macht aus\\machen aus\\macht aus\\machen aus} &
\parbox[t][][t]{2cm}{\normalfont machte aus\\machtest aus\\machte aus\\machten aus\\machtet aus\\machten aus}\\
\end{tabular}
\begin{tabular}{l}
\parbox[t][][t]{8cm}{}\\
\parbox[t][][t]{8cm}{\normalfont \footnotesize 
etw ausmachen - 
to put out sth sep - 
das Feuer/die Kerze/die Zigarette ausmachen - 
to put out the fire/candle/cigarette sep - 
etw ausmachen - 
to switch off sth sep - 
etw ausmachen - 
(abdrehen) - 
to turn off sth sep - 
den Fernseher/das Radio ausmachen 
}\\
\end{tabular}
}
%===ausschalten===
\card{\normalfont \Huge ausschalten}{
\begin{tabular}{lll}
\parbox[t][][t]{2.0 cm}{\normalfont \raggedleft ich\\du\\er/sie/es\\wir\\ihr\\sie} &    
\parbox[t][][t]{2cm}{\normalfont schalte aus\\schaltest aus\\schaltet aus\\schalten aus\\schaltet aus\\schalten aus} &
\parbox[t][][t]{2cm}{\normalfont schaltete aus\\schaltetest aus\\schaltete aus\\schalteten aus\\schaltetet aus\\schalteten aus}\\
\end{tabular}
\begin{tabular}{l}
\parbox[t][][t]{8cm}{}\\
\parbox[t][][t]{8cm}{\normalfont \footnotesize 
etw ausschalten - 
to turn (or switch) off sth sep - 
jdn/etw ausschalten - 
to eliminate sb/sth - 
jdn/etw ausschalten - 
to put sb out of the running - 
sich akk (automatisch) ausschalten - 
to switch (or turn)  (itself) off (automatically) 
}\\
\end{tabular}
}
%===ausschließen===
\card{\normalfont \Huge ausschließen}{
\begin{tabular}{lll}
\parbox[t][][t]{2.0 cm}{\normalfont \raggedleft ich\\du\\er/sie/es\\wir\\ihr\\sie} &    
\parbox[t][][t]{2cm}{\normalfont schließe aus\\schließt aus\\schließt aus\\schließen aus\\schließt aus\\schließen aus} &
\parbox[t][][t]{2cm}{\normalfont schloss aus\\schlossest aus\\schloss aus\\schlossen aus\\schlosst aus\\schlossen aus}\\
\end{tabular}
\begin{tabular}{l}
\parbox[t][][t]{8cm}{}\\
\parbox[t][][t]{8cm}{\normalfont \footnotesize 
jdn (aus etw dat/von etw dat) ausschließen - 
to exclude sb (from sth) - 
jdn (aus etw dat/von etw dat) ausschließen - 
(als Strafe a.) - 
to bar sb (from sth) - 
die Öffentlichkeit (von etw dat) ausschließen - 
JUR - 
to hold sth in camera spec - 
die Öffentlichkeit (von etw dat) ausschließen - 
JUR 
}\\
\end{tabular}
}
%===aussehen===
\card{\normalfont \Huge aussehen}{
\begin{tabular}{lll}
\parbox[t][][t]{2.0 cm}{\normalfont \raggedleft ich\\du\\er/sie/es\\wir\\ihr\\sie} &    
\parbox[t][][t]{2cm}{\normalfont sehe aus\\siehst aus\\sieht aus\\sehen aus\\seht aus\\sehen aus} &
\parbox[t][][t]{2cm}{\normalfont sah aus\\sahst aus\\sah aus\\sahen aus\\saht aus\\sahen aus}\\
\end{tabular}
\begin{tabular}{l}
\parbox[t][][t]{8cm}{}\\
\parbox[t][][t]{8cm}{\normalfont \footnotesize 
aussehen - 
to look - 
du siehst gut/gesund/schick aus - 
you look great/healthy/smart - 
aussehen wie ... - 
to look like ... - 
es sieht gut/schlecht aus - 
things are looking good/not looking too good - 
bei jdm sieht es gut/schlecht aus - 
things are looking good/not looking too good for sb 
}\\
\end{tabular}
}
%===aussprechen===
\card{\normalfont \Huge aussprechen}{
\begin{tabular}{lll}
\parbox[t][][t]{2.0 cm}{\normalfont \raggedleft ich\\du\\er/sie/es\\wir\\ihr\\sie} &    
\parbox[t][][t]{2cm}{\normalfont spreche aus\\sprichst aus\\spricht aus\\sprechen aus\\sprecht aus\\sprechen aus} &
\parbox[t][][t]{2cm}{\normalfont sprach aus\\sprachst aus\\sprach aus\\sprachen aus\\spracht aus\\sprachen aus}\\
\end{tabular}
\begin{tabular}{l}
\parbox[t][][t]{8cm}{}\\
\parbox[t][][t]{8cm}{\normalfont \footnotesize 
etw aussprechen - 
to pronounce sth - 
wie spricht man das (Wort) aus? - 
how do you pronounce (or say) that (word)? - 
etw aussprechen - 
to express sth - 
kaum hatte er den Satz ausgesprochen, ... - 
he had barely finished the sentence ... - 
ein Lob aussprechen - 
to give a word of praise 
}\\
\end{tabular}
}
%===ausstellen===
\card{\normalfont \Huge ausstellen}{
\begin{tabular}{lll}
\parbox[t][][t]{2.0 cm}{\normalfont \raggedleft ich\\du\\er/sie/es\\wir\\ihr\\sie} &    
\parbox[t][][t]{2cm}{\normalfont stelle aus\\stellst aus\\stellt aus\\stellen aus\\stellt aus\\stellen aus} &
\parbox[t][][t]{2cm}{\normalfont stellte aus\\stelltest aus\\stellte aus\\stellten aus\\stelltet aus\\stellten aus}\\
\end{tabular}
\begin{tabular}{l}
\parbox[t][][t]{8cm}{}\\
\parbox[t][][t]{8cm}{\normalfont \footnotesize 
etw ausstellen - 
to display sth - 
etw ausstellen - 
(auf Messe, in Museum) - 
to exhibit sth - 
(jdm) etw ausstellen - 
to write (out sep ) (sb) sth - 
(jdm) etw ausstellen - 
to make out sep sth (for sb) - 
(jdm) eine Rechnung ausstellen 
}\\
\end{tabular}
}
%===ausziehen===
\card{\normalfont \Huge ausziehen}{
\begin{tabular}{lll}
\parbox[t][][t]{2.0 cm}{\normalfont \raggedleft ich\\du\\er/sie/es\\wir\\ihr\\sie} &    
\parbox[t][][t]{2cm}{\normalfont ziehe aus\\ziehst aus\\zieht aus\\ziehen aus\\zieht aus\\ziehen aus} &
\parbox[t][][t]{2cm}{\normalfont zog aus\\zogst aus\\zog aus\\zogen aus\\zogt aus\\zogen aus}\\
\end{tabular}
\begin{tabular}{l}
\parbox[t][][t]{8cm}{}\\
\parbox[t][][t]{8cm}{\normalfont \footnotesize 
(sich dat) etw ausziehen - 
to take off sep sth - 
(sich dat) etw ausziehen - 
to remove sth - 
jdm etw ausziehen - 
to remove (or take off) sb's sth - 
jdn ausziehen - 
to undress sb - 
sich akk - 
ausziehen 
}\\
\end{tabular}
}
%===backen===
\card{\normalfont \Huge backen}{
\begin{tabular}{lll}
\parbox[t][][t]{2.0 cm}{\normalfont \raggedleft ich\\du\\er/sie/es\\wir\\ihr\\sie} &    
\parbox[t][][t]{2cm}{\normalfont backe\\backst\\backt\\backen\\backt\\backen} &
\parbox[t][][t]{2cm}{\normalfont backte\\backtest\\backte\\backten\\backtet\\backten}\\
\end{tabular}
\begin{tabular}{l}
\parbox[t][][t]{8cm}{}\\
\parbox[t][][t]{8cm}{\normalfont \footnotesize 
etw backen - 
(im Ofen) - 
to bake sth - 
etw backen - 
A, FRG (in Fett) - 
to deep-fry sth - 
etw im Ofen backen - 
to bake sth (in the oven) - 
backen (im Ofen) - 
to bake 
}\\
\end{tabular}
}
%===baden===
\card{\normalfont \Huge baden}{
\begin{tabular}{lll}
\parbox[t][][t]{2.0 cm}{\normalfont \raggedleft ich\\du\\er/sie/es\\wir\\ihr\\sie} &    
\parbox[t][][t]{2cm}{\normalfont bade\\badest\\badet\\baden\\badet\\baden} &
\parbox[t][][t]{2cm}{\normalfont badete\\badetest\\badete\\badeten\\badetet\\badeten}\\
\end{tabular}
\begin{tabular}{l}
\parbox[t][][t]{8cm}{}\\
\parbox[t][][t]{8cm}{\normalfont \footnotesize 
baden - 
to bathe - 
baden - 
to have a bath - 
baden - 
to be in the bath - 
warm baden - 
to have a warm bath - 
(in etw dat) baden - 
to swim (in sth) 
}\\
\end{tabular}
}
%===bauen===
\card{\normalfont \Huge bauen}{
\begin{tabular}{lll}
\parbox[t][][t]{2.0 cm}{\normalfont \raggedleft ich\\du\\er/sie/es\\wir\\ihr\\sie} &    
\parbox[t][][t]{2cm}{\normalfont baue\\baust\\baut\\bauen\\baut\\bauen} &
\parbox[t][][t]{2cm}{\normalfont baute\\bautest\\baute\\bauten\\bautet\\bauten}\\
\end{tabular}
\begin{tabular}{l}
\parbox[t][][t]{8cm}{}\\
\parbox[t][][t]{8cm}{\normalfont \footnotesize 
(jdm) etw bauen - 
to build (or construct) sth (for sb) - 
sich dat etw bauen - 
to build oneself sth - 
etw bauen - 
to construct (or make) sth - 
ein Auto/eine Bombe/ein Flugzeug/ein Schiff bauen - 
to build a car/bomb/an aircraft/ship - 
ein Gerät bauen - 
to construct a machine 
}\\
\end{tabular}
}
%===beachten===
\card{\normalfont \Huge beachten}{
\begin{tabular}{lll}
\parbox[t][][t]{2.0 cm}{\normalfont \raggedleft ich\\du\\er/sie/es\\wir\\ihr\\sie} &    
\parbox[t][][t]{2cm}{\normalfont beachte\\beachtest\\beachtet\\beachten\\beachtet\\beachten} &
\parbox[t][][t]{2cm}{\normalfont beachtete\\beachtetest\\beachtete\\beachteten\\beachtetet\\beachteten}\\
\end{tabular}
\begin{tabular}{l}
\parbox[t][][t]{8cm}{}\\
\parbox[t][][t]{8cm}{\normalfont \footnotesize 
etw beachten - 
to observe (or comply with) sth - 
eine Anweisung/einen Rat beachten - 
to follow (or heed) advice/an instruction - 
ein Verkehrszeichen beachten - 
to observe a traffic sign - 
die Vorfahrt beachten - 
to yield (right of way) - 
die Vorfahrt beachten - 
Brit a. to give way 
}\\
\end{tabular}
}
%===beantragen===
\card{\normalfont \Huge beantragen}{
\begin{tabular}{lll}
\parbox[t][][t]{2.0 cm}{\normalfont \raggedleft ich\\du\\er/sie/es\\wir\\ihr\\sie} &    
\parbox[t][][t]{2cm}{\normalfont beantrage\\beantragst\\beantragt\\beantragen\\beantragt\\beantragen} &
\parbox[t][][t]{2cm}{\normalfont beantragte\\beantragtest\\beantragte\\beantragten\\beantragtet\\beantragten}\\
\end{tabular}
\begin{tabular}{l}
\parbox[t][][t]{8cm}{}\\
\parbox[t][][t]{8cm}{\normalfont \footnotesize 
jdn/etw (bei jdm/etw) beantragen - 
to apply for sb/sth (from sb/sth) - 
etw beantragen - 
to propose sth - 
etw beantragen - 
to put forward sth sep - 
etw beantragen - 
to apply (or file an application) for sth - 
die Höchststrafe beantragen - 
to seek (or request) the maximum penalty 
}\\
\end{tabular}
}
%===bedeuten===
\card{\normalfont \Huge bedeuten}{
\begin{tabular}{lll}
\parbox[t][][t]{2.0 cm}{\normalfont \raggedleft ich\\du\\er/sie/es\\wir\\ihr\\sie} &    
\parbox[t][][t]{2cm}{\normalfont bedeute\\bedeutest\\bedeutet\\bedeuten\\bedeutet\\bedeuten} &
\parbox[t][][t]{2cm}{\normalfont bedeutete\\bedeutetest\\bedeutete\\bedeuteten\\bedeutetet\\bedeuteten}\\
\end{tabular}
\begin{tabular}{l}
\parbox[t][][t]{8cm}{}\\
\parbox[t][][t]{8cm}{\normalfont \footnotesize 
etw bedeuten - 
to signify (or mean) sth - 
etw bedeuten - 
to mean (or represent) sth - 
was bedeutet dieses Symbol? - 
what does this symbol signify? - 
ihr Schweigen dürfte wohl Desinteresse bedeuten - 
her silence seems to indicate a lack of interest - 
bedeuten, dass - 
to indicate that 
}\\
\end{tabular}
}
%===bedienen===
\card{\normalfont \Huge bedienen}{
\begin{tabular}{lll}
\parbox[t][][t]{2.0 cm}{\normalfont \raggedleft ich\\du\\er/sie/es\\wir\\ihr\\sie} &    
\parbox[t][][t]{2cm}{\normalfont bediene\\bedienst\\bedient\\bedienen\\bedient\\bedienen} &
\parbox[t][][t]{2cm}{\normalfont bediente\\bedientest\\bediente\\bedienten\\bedientet\\bedienten}\\
\end{tabular}
\begin{tabular}{l}
\parbox[t][][t]{8cm}{}\\
\parbox[t][][t]{8cm}{\normalfont \footnotesize 
jdn bedienen - 
to serve (or wait on) sb - 
sich akk (von jdm) bedienen lassen - 
to be waited on (by sb) - 
einen Kunden bedienen - 
to serve a customer - 
werden Sie schon bedient? - 
are you being served? - 
jdn bedienen - 
to serve sb 
}\\
\end{tabular}
}
%===bedingen===
\card{\normalfont \Huge bedingen}{
\begin{tabular}{lll}
\parbox[t][][t]{2.0 cm}{\normalfont \raggedleft ich\\du\\er/sie/es\\wir\\ihr\\sie} &    
\parbox[t][][t]{2cm}{\normalfont bedinge\\bedingst\\bedingt\\bedingen\\bedingt\\bedingen} &
\parbox[t][][t]{2cm}{\normalfont bedingte\\bedingtest\\bedingte\\bedingten\\bedingtet\\bedingten}\\
\end{tabular}
\begin{tabular}{l}
\parbox[t][][t]{8cm}{}\\
\parbox[t][][t]{8cm}{\normalfont \footnotesize 
etw bedingen - 
to cause sth - 
höhere Löhne bedingen höhere Preise - 
higher wages lead to higher prices - 
durch etw akk bedingt sein - 
to be a result of sth - 
etw bedingen - 
to require - 
etw bedingen - 
to demand 
}\\
\end{tabular}
}
%===befehlen===
\card{\normalfont \Huge befehlen}{
\begin{tabular}{lll}
\parbox[t][][t]{2.0 cm}{\normalfont \raggedleft ich\\du\\er/sie/es\\wir\\ihr\\sie} &    
\parbox[t][][t]{2cm}{\normalfont befehle\\befiehlst\\befiehlt\\befehlen\\befehlt\\befehlen} &
\parbox[t][][t]{2cm}{\normalfont befahl\\befahlst\\befahl\\befahlen\\befahlt\\befahlen}\\
\end{tabular}
\begin{tabular}{l}
\parbox[t][][t]{8cm}{}\\
\parbox[t][][t]{8cm}{\normalfont \footnotesize 
jdm befehlen, etw zu tun - 
to order (or command) sb to do sth - 
etw befehlen - 
to order sth - 
von dir lasse ich mir nichts befehlen! - 
I won't take orders from you! - 
was befehlen Sie, Herr Hauptmann? - 
what are your orders, Captain? - 
jdn irgendwohin befehlen - 
to order sb (to go) somewhere 
}\\
\end{tabular}
}
%===befinden===
\card{\normalfont \Huge befinden}{
\begin{tabular}{lll}
\parbox[t][][t]{2.0 cm}{\normalfont \raggedleft ich\\du\\er/sie/es\\wir\\ihr\\sie} &    
\parbox[t][][t]{2cm}{\normalfont befinde\\befindest\\befindet\\befinden\\befindet\\befinden} &
\parbox[t][][t]{2cm}{\normalfont befand\\befandest\\befand\\befanden\\befandet\\befanden}\\
\end{tabular}
\begin{tabular}{l}
\parbox[t][][t]{8cm}{}\\
\parbox[t][][t]{8cm}{\normalfont \footnotesize 
sich akk irgendwo befinden - 
to be somewhere - 
unter den Geiseln befinden sich zwei Deutsche - 
there are two Germans amongst the hostages - 
sich akk im Ausland/im Urlaub befinden - 
to be abroad/on holiday (or - 
Am vacation) - 
sich akk in bester/schlechter Laune befinden - 
to be in an excellent/a bad mood - 
sich akk in guten Händen befinden 
}\\
\end{tabular}
}
%===befreien===
\card{\normalfont \Huge befreien}{
\begin{tabular}{lll}
\parbox[t][][t]{2.0 cm}{\normalfont \raggedleft ich\\du\\er/sie/es\\wir\\ihr\\sie} &    
\parbox[t][][t]{2cm}{\normalfont befreie\\befreist\\befreit\\befreien\\befreit\\befreien} &
\parbox[t][][t]{2cm}{\normalfont befreite\\befreitest\\befreite\\befreiten\\befreitet\\befreiten}\\
\end{tabular}
\begin{tabular}{l}
\parbox[t][][t]{8cm}{}\\
\parbox[t][][t]{8cm}{\normalfont \footnotesize 
jdn/ein Tier (aus (o. von) etw) befreien - 
to free (or set free sep - 
) - 
(or release) sb/an animal (from sth) - 
jdn/etw (von jdm/etw) befreien - 
to liberate sb/sth (from sb/sth) - 
etw von etw dat - 
befreien - 
to clear sth of (or remove sth from) sth - 
seine Schuhe vom Dreck befreien 
}\\
\end{tabular}
}
%===begegnen===
\card{\normalfont \Huge begegnen}{
\begin{tabular}{lll}
\parbox[t][][t]{2.0 cm}{\normalfont \raggedleft ich\\du\\er/sie/es\\wir\\ihr\\sie} &    
\parbox[t][][t]{2cm}{\normalfont begegne\\begegnest\\begegnet\\begegnen\\begegnet\\begegnen} &
\parbox[t][][t]{2cm}{\normalfont begegnete\\begegnetest\\begegnete\\begegneten\\begegnetet\\begegneten}\\
\end{tabular}
\begin{tabular}{l}
\parbox[t][][t]{8cm}{}\\
\parbox[t][][t]{8cm}{\normalfont \footnotesize 
jdm begegnen - 
to meet sb - 
ich bin ihm die Tage im Supermarkt begegnet - 
I bumped (or ran) into (or met) him recently at the supermarket - 
jds Blick begegnen - 
to meet sb's gaze (or eye) - 
sich akk - 
begegnen - 
to meet - 
etw dat 
}\\
\end{tabular}
}
%===beginnen===
\card{\normalfont \Huge beginnen}{
\begin{tabular}{lll}
\parbox[t][][t]{2.0 cm}{\normalfont \raggedleft ich\\du\\er/sie/es\\wir\\ihr\\sie} &    
\parbox[t][][t]{2cm}{\normalfont beginne\\beginnst\\beginnt\\beginnen\\beginnt\\beginnen} &
\parbox[t][][t]{2cm}{\normalfont begann\\begannst\\begann\\begannen\\begannt\\begannen}\\
\end{tabular}
\begin{tabular}{l}
\parbox[t][][t]{8cm}{}\\
\parbox[t][][t]{8cm}{\normalfont \footnotesize 
(mit etw dat) beginnen - 
to start (or begin)  (sth) - 
beginnen, etw zu tun - 
to start (or begin) to do (or doing) sth - 
als etw beginnen - 
to start out (or off) as sth - 
etw (mit etw dat) beginnen - 
to start (or begin) sth (with sth) - 
ein Gespräch beginnen - 
to strike up (or begin)  a conversation 
}\\
\end{tabular}
}
%===begleiten===
\card{\normalfont \Huge begleiten}{
\begin{tabular}{lll}
\parbox[t][][t]{2.0 cm}{\normalfont \raggedleft ich\\du\\er/sie/es\\wir\\ihr\\sie} &    
\parbox[t][][t]{2cm}{\normalfont begleite\\begleitest\\begleitet\\begleiten\\begleitet\\begleiten} &
\parbox[t][][t]{2cm}{\normalfont begleitete\\begleitetest\\begleitete\\begleiteten\\begleitetet\\begleiteten}\\
\end{tabular}
\begin{tabular}{l}
\parbox[t][][t]{8cm}{}\\
\parbox[t][][t]{8cm}{\normalfont \footnotesize 
jdn begleiten - 
a. fig - 
to accompany sb - 
jdn irgendwohin begleiten - 
to accompany (or come/go with) sb somewhere - 
jdn nach Hause/zur Bushaltestelle begleiten - 
to accompany (or - 
form escort) sb home/to the bus stop - 
jdn zur Tür begleiten - 
to take (or show) 
}\\
\end{tabular}
}
%===begründen===
\card{\normalfont \Huge begründen}{
\begin{tabular}{lll}
\parbox[t][][t]{2.0 cm}{\normalfont \raggedleft ich\\du\\er/sie/es\\wir\\ihr\\sie} &    
\parbox[t][][t]{2cm}{\normalfont begründe\\begründest\\begründet\\begründen\\begründet\\begründen} &
\parbox[t][][t]{2cm}{\normalfont begründete\\begründetest\\begründete\\begründeten\\begründetet\\begründeten}\\
\end{tabular}
\begin{tabular}{l}
\parbox[t][][t]{8cm}{}\\
\parbox[t][][t]{8cm}{\normalfont \footnotesize 
etw (mit etw dat) begründen - 
to give reasons for sth - 
eine Ablehnung/Forderung begründen - 
to justify a refusal/demand - 
eine Behauptung/Klage/einen Verdacht begründen - 
to substantiate a claim/complaint/suspicion - 
sein Verhalten ist einfach durch nichts zu begründen - 
his behaviour simply cannot be accounted for - 
etw begründen - 
to found (or establish) sth 
}\\
\end{tabular}
}
%===begrüßen===
\card{\normalfont \Huge begrüßen}{
\begin{tabular}{lll}
\parbox[t][][t]{2.0 cm}{\normalfont \raggedleft ich\\du\\er/sie/es\\wir\\ihr\\sie} &    
\parbox[t][][t]{2cm}{\normalfont begrüße\\begrüßt\\begrüßt\\begrüßen\\begrüßt\\begrüßen} &
\parbox[t][][t]{2cm}{\normalfont begrüßte\\begrüßtest\\begrüßte\\begrüßten\\begrüßtet\\begrüßten}\\
\end{tabular}
\begin{tabular}{l}
\parbox[t][][t]{8cm}{}\\
\parbox[t][][t]{8cm}{\normalfont \footnotesize 
jdn (mit etw dat) begrüßen - 
to greet (or welcome) sb (with sth) - 
ich begrüße Sie! - 
welcome! - 
jdn als etw akk - 
begrüßen - 
to greet sb as sth - 
jdn bei sich dat zu Hause begrüßen dürfen geh - 
to have the pleasure of welcoming sb into one's home form - 
wir würden uns freuen, Sie demnächst wieder bei uns begrüßen zu dürfen 
}\\
\end{tabular}
}
%===behalten===
\card{\normalfont \Huge behalten}{
\begin{tabular}{lll}
\parbox[t][][t]{2.0 cm}{\normalfont \raggedleft ich\\du\\er/sie/es\\wir\\ihr\\sie} &    
\parbox[t][][t]{2cm}{\normalfont behalte\\behältst\\behält\\behalten\\behaltet\\behalten} &
\parbox[t][][t]{2cm}{\normalfont behielt\\behieltest \\behielt\\behielten\\behieltet\\behielten}\\
\end{tabular}
\begin{tabular}{l}
\parbox[t][][t]{8cm}{}\\
\parbox[t][][t]{8cm}{\normalfont \footnotesize 
etw behalten - 
to keep sth - 
wozu willst du das alles behalten! - 
why hang on to all this! - 
etw für sich akk - 
behalten - 
to keep sth to oneself - 
jdn (bei sich dat) behalten - 
to have sb stay on (with one) - 
ich hätte dich ja noch gerne länger (bei mir) behalten 
}\\
\end{tabular}
}
%===behandeln===
\card{\normalfont \Huge behandeln}{
\begin{tabular}{lll}
\parbox[t][][t]{2.0 cm}{\normalfont \raggedleft ich\\du\\er/sie/es\\wir\\ihr\\sie} &    
\parbox[t][][t]{2cm}{\normalfont behandle\\behandelst\\behandelt\\behandeln\\behandelt\\behandeln} &
\parbox[t][][t]{2cm}{\normalfont behandelte\\behandeltest\\behandelte\\behandelten\\behandeltet\\behandelten}\\
\end{tabular}
\begin{tabular}{l}
\parbox[t][][t]{8cm}{}\\
\parbox[t][][t]{8cm}{\normalfont \footnotesize 
jdn/etw behandeln - 
to treat (or attend to) sb/sth - 
wer ist Ihr behandelnder Arzt? - 
who is the doctor treating you? - 
jdn/etw/ein Tier behandeln - 
to treat sb/an animal - 
jdn gut behandeln - 
to treat sb well - 
jdn schlecht behandeln - 
to treat sb badly 
}\\
\end{tabular}
}
%===behaupten===
\card{\normalfont \Huge behaupten}{
\begin{tabular}{lll}
\parbox[t][][t]{2.0 cm}{\normalfont \raggedleft ich\\du\\er/sie/es\\wir\\ihr\\sie} &    
\parbox[t][][t]{2cm}{\normalfont behaupte\\behauptest\\behauptet\\behaupten\\behauptet\\behaupten} &
\parbox[t][][t]{2cm}{\normalfont behauptete\\behauptetest\\behauptete\\behaupteten\\behauptetet\\behaupteten}\\
\end{tabular}
\begin{tabular}{l}
\parbox[t][][t]{8cm}{}\\
\parbox[t][][t]{8cm}{\normalfont \footnotesize 
etw (von jdm/etw) - 
behaupten - 
to claim - 
(or maintain) - 
(or assert) - 
sth (about sb/sth) - 
wer das (von ihr) behauptet, lügt! - 
whoever says that (about her) is lying! - 
behaupten, dass ... - 
to claim that ... 
}\\
\end{tabular}
}
%===behindern===
\card{\normalfont \Huge behindern}{
\begin{tabular}{lll}
\parbox[t][][t]{2.0 cm}{\normalfont \raggedleft ich\\du\\er/sie/es\\wir\\ihr\\sie} &    
\parbox[t][][t]{2cm}{\normalfont behindere\\behinderst\\behindert\\behindern\\behindert\\behindern} &
\parbox[t][][t]{2cm}{\normalfont behinderte\\behindertest\\behinderte\\behinderten\\behindertet\\behinderten}\\
\end{tabular}
\begin{tabular}{l}
\parbox[t][][t]{8cm}{}\\
\parbox[t][][t]{8cm}{\normalfont \footnotesize 
jdn (bei etw dat) behindern - 
to obstruct (or hinder) sb (in sth) - 
etw (bei etw dat) behindern - 
to hinder sth (in sth) - 
die Bewegungsfreiheit behindern - 
to impede one's movement(s) - 
etw behindern - 
to hamper sth - 
die erneuten Terroranschläge behindern den Friedensprozess - 
the renewed terrorist attacks are threatening the peace process 
}\\
\end{tabular}
}
%===beißen===
\card{\normalfont \Huge beißen}{
\begin{tabular}{lll}
\parbox[t][][t]{2.0 cm}{\normalfont \raggedleft ich\\du\\er/sie/es\\wir\\ihr\\sie} &    
\parbox[t][][t]{2cm}{\normalfont beiße\\beißt\\beißt\\beißen\\beißt\\beißen} &
\parbox[t][][t]{2cm}{\normalfont biss\\bissest\\biss\\bissen\\bisst\\bissen}\\
\end{tabular}
\begin{tabular}{l}
\parbox[t][][t]{8cm}{}\\
\parbox[t][][t]{8cm}{\normalfont \footnotesize 
jdn (in etw akk) beißen - 
to bite sb('s sth) (or sb (in the sth)) - 
sich akk - 
beißen - 
to bite each other (or one another) - 
er wird dich schon nicht beißen! fig - 
he won't bite you - 
das Brot ist so hart, dass man es kaum mehr beißen kann! - 
this bread is so hard that you can hardly bite into it - 
etwas/nichts zu beißen haben fam 
}\\
\end{tabular}
}
%===bekommen===
\card{\normalfont \Huge bekommen}{
\begin{tabular}{lll}
\parbox[t][][t]{2.0 cm}{\normalfont \raggedleft ich\\du\\er/sie/es\\wir\\ihr\\sie} &    
\parbox[t][][t]{2cm}{\normalfont bekomme\\bekommst\\bekommt\\bekommen\\bekommt\\bekommen} &
\parbox[t][][t]{2cm}{\normalfont bekam\\bekamst\\bekam\\bekamen\\bekamt\\bekamen}\\
\end{tabular}
\begin{tabular}{l}
\parbox[t][][t]{8cm}{}\\
\parbox[t][][t]{8cm}{\normalfont \footnotesize 
etw (von jdm) bekommen - 
to get sth (from sb) - 
wir bekommen demnächst Kabelfernsehen - 
we're going to get cable TV soon - 
von dieser Schokolade kann ich einfach nicht genug bekommen! - 
I just can't get enough of that chocolate! - 
habe ich heute Post bekommen? - 
did I get any post today? - 
einen Anruf/Brief bekommen - 
to get (or have) 
}\\
\end{tabular}
}
%===beleidigen===
\card{\normalfont \Huge beleidigen}{
\begin{tabular}{lll}
\parbox[t][][t]{2.0 cm}{\normalfont \raggedleft ich\\du\\er/sie/es\\wir\\ihr\\sie} &    
\parbox[t][][t]{2cm}{\normalfont beleidige\\beleidigst\\beleidigt\\beleidigen\\beleidigt\\beleidigen} &
\parbox[t][][t]{2cm}{\normalfont beleidigte\\beleidigtest\\beleidigte\\beleidigten\\beleidigtet\\beleidigten}\\
\end{tabular}
\begin{tabular}{l}
\parbox[t][][t]{8cm}{}\\
\parbox[t][][t]{8cm}{\normalfont \footnotesize 
jdn/etw (durch etw akk) beleidigen - 
to insult sb (with sth) - 
jdn/etw (durch etw akk) beleidigen - 
to offend sb/sth (with sth) - 
jdn beleidigen - 
to offend (or be offensive to) sb 
}\\
\end{tabular}
}
%===bellen===
\card{\normalfont \Huge bellen}{
\begin{tabular}{lll}
\parbox[t][][t]{2.0 cm}{\normalfont \raggedleft ich\\du\\er/sie/es\\wir\\ihr\\sie} &    
\parbox[t][][t]{2cm}{\normalfont belle\\bellst\\bellt\\bellen\\bellt\\bellen} &
\parbox[t][][t]{2cm}{\normalfont bellte\\belltest\\bellte\\bellten\\belltet\\bellten}\\
\end{tabular}
\begin{tabular}{l}
\parbox[t][][t]{8cm}{}\\
\parbox[t][][t]{8cm}{\normalfont \footnotesize 
bellen - 
to bark - 
etw bellen - 
to bark (out sep ) sth 
}\\
\end{tabular}
}
%===benutzen===
\card{\normalfont \Huge benutzen}{
\begin{tabular}{lll}
\parbox[t][][t]{2.0 cm}{\normalfont \raggedleft ich\\du\\er/sie/es\\wir\\ihr\\sie} &    
\parbox[t][][t]{2cm}{\normalfont benutze\\benutzt\\benutzt\\benutzen\\benutzt\\benutzen} &
\parbox[t][][t]{2cm}{\normalfont benutzte\\benutztest\\benutzte\\benutzten\\benutztet\\benutzten}\\
\end{tabular}
\begin{tabular}{l}
\parbox[t][][t]{8cm}{}\\
\parbox[t][][t]{8cm}{\normalfont \footnotesize 
etw (als etw) benutzen - 
to use sth (as sth) - 
nach dem Benutzen - 
after use - 
benutzt - 
used - 
das benutzte Geschirr - 
the dirty dishes pl - 
den Aufzug/die Bahn/den Bus benutzen - 
to take the lift (or 
}\\
\end{tabular}
}
%===beobachten===
\card{\normalfont \Huge beobachten}{
\begin{tabular}{lll}
\parbox[t][][t]{2.0 cm}{\normalfont \raggedleft ich\\du\\er/sie/es\\wir\\ihr\\sie} &    
\parbox[t][][t]{2cm}{\normalfont beobachte\\beobachtest\\beobachtet\\beobachten\\beobachtet\\beobachten} &
\parbox[t][][t]{2cm}{\normalfont beobachtete\\beobachtetest\\beobachtete\\beobachteten\\beobachtetet\\beobachteten}\\
\end{tabular}
\begin{tabular}{l}
\parbox[t][][t]{8cm}{}\\
\parbox[t][][t]{8cm}{\normalfont \footnotesize 
jdn/etw beobachten - 
to observe sb/sth - 
jdn/etw genau beobachten - 
to watch sb/sth closely - 
jdn (bei etw dat) beobachten - 
to watch sb (doing sth) - 
gut beobachtet! - 
well spotted! - 
(durch jdn (o. von jdm)) beobachtet werden - 
to be kept under the surveillance (of sb) 
}\\
\end{tabular}
}
%===beraten===
\card{\normalfont \Huge beraten}{
\begin{tabular}{lll}
\parbox[t][][t]{2.0 cm}{\normalfont \raggedleft ich\\du\\er/sie/es\\wir\\ihr\\sie} &    
\parbox[t][][t]{2cm}{\normalfont berate\\berätst\\berät\\beraten\\beratet\\beraten} &
\parbox[t][][t]{2cm}{\normalfont beriet\\berietest\\beriet\\berieten\\berietet\\berieten}\\
\end{tabular}
\begin{tabular}{l}
\parbox[t][][t]{8cm}{}\\
\parbox[t][][t]{8cm}{\normalfont \footnotesize 
jdn (in etw dat) beraten - 
to advise sb (or give sb advice)  (on sth) - 
jdn finanziell/rechtlich beraten - 
to give sb financial/legal advice - 
sich akk (von jdm) beraten lassen (, ob/wie) - 
to ask sb's advice (as to whether/on how) - 
etw beraten - 
to discuss sth - 
etw beraten - 
POL 
}\\
\end{tabular}
}
%===bereiten===
\card{\normalfont \Huge bereiten}{
\begin{tabular}{lll}
\parbox[t][][t]{2.0 cm}{\normalfont \raggedleft ich\\du\\er/sie/es\\wir\\ihr\\sie} &    
\parbox[t][][t]{2cm}{\normalfont bereite\\bereitest\\bereitet\\bereiten\\bereitet\\bereiten} &
\parbox[t][][t]{2cm}{\normalfont bereitete\\bereitetest\\bereitete\\bereiteten\\bereitetet\\bereiteten}\\
\end{tabular}
\begin{tabular}{l}
\parbox[t][][t]{8cm}{}\\
\parbox[t][][t]{8cm}{\normalfont \footnotesize 
jdm etw bereiten - 
to cause sb sth - 
einen freundlichen Empfang/eine Freude/eine Überraschung bereiten - 
to give sb a warm welcome/pleasure/a surprise - 
jdm Kopfschmerzen bereiten - 
to give sb a headache - 
(jdm) etw bereiten - 
to prepare sth (for sb) - 
Medikamente/Essen/Kaffee bereiten - 
to make up medicines/to prepare food, coffee sep 
}\\
\end{tabular}
}
%===berichten===
\card{\normalfont \Huge berichten}{
\begin{tabular}{lll}
\parbox[t][][t]{2.0 cm}{\normalfont \raggedleft ich\\du\\er/sie/es\\wir\\ihr\\sie} &    
\parbox[t][][t]{2cm}{\normalfont berichte\\berichtest\\berichtet\\berichten\\berichtet\\berichten} &
\parbox[t][][t]{2cm}{\normalfont berichtete\\berichtetest\\berichtete\\berichteten\\berichtetet\\berichteten}\\
\end{tabular}
\begin{tabular}{l}
\parbox[t][][t]{8cm}{}\\
\parbox[t][][t]{8cm}{\normalfont \footnotesize 
(jdm) etw berichten - 
to tell sb (sth) - 
was gibt's denn zu berichten? - 
what have you to tell me? - 
es gibt einiges zu berichten - 
I/we have a number of things to tell you - 
falsch/recht berichtet CH - 
wrong/right (or correct) - 
bin ich falsch/recht berichtet, wenn ich annehme ...? - 
am I wrong/right (or 
}\\
\end{tabular}
}
%===bersten===
\card{\normalfont \Huge bersten}{
\begin{tabular}{lll}
\parbox[t][][t]{2.0 cm}{\normalfont \raggedleft ich\\du\\er/sie/es\\wir\\ihr\\sie} &    
\parbox[t][][t]{2cm}{\normalfont berste\\birst\\birst\\bersten\\berstet\\bersten} &
\parbox[t][][t]{2cm}{\normalfont barst\\barstest\\barst\\barsten\\barstet\\barsten}\\
\end{tabular}
\begin{tabular}{l}
\parbox[t][][t]{8cm}{}\\
\parbox[t][][t]{8cm}{\normalfont \footnotesize 
bersten - 
to explode - 
bersten Ballon - 
to burst - 
bersten Glas, Eis - 
to break - 
bersten Glas, Eis - 
to crack - 
bersten Erde - 
to burst open 
}\\
\end{tabular}
}
%===berufen===
\card{\normalfont \Huge berufen}{
\begin{tabular}{lll}
\parbox[t][][t]{2.0 cm}{\normalfont \raggedleft ich\\du\\er/sie/es\\wir\\ihr\\sie} &    
\parbox[t][][t]{2cm}{\normalfont berufe\\berüfst\\berüft\\berufen\\beruft\\berufen} &
\parbox[t][][t]{2cm}{\normalfont berief\\beriefst\\berief\\beriefen\\berieft\\beriefen}\\
\end{tabular}
\begin{tabular}{l}
\parbox[t][][t]{8cm}{}\\
\parbox[t][][t]{8cm}{\normalfont \footnotesize 
berufen - 
qualified - 
berufen - 
competent - 
zu etw dat - 
berufen sein - 
to have a vocation (or calling) for sth (or to do sth) - 
er ist zu Großem berufen - 
he's meant for greater things - 
sich akk 
}\\
\end{tabular}
}
%===beruhigen===
\card{\normalfont \Huge beruhigen}{
\begin{tabular}{lll}
\parbox[t][][t]{2.0 cm}{\normalfont \raggedleft ich\\du\\er/sie/es\\wir\\ihr\\sie} &    
\parbox[t][][t]{2cm}{\normalfont beruhige\\beruhigst\\beruhigt\\beruhigen\\beruhigt\\beruhigen} &
\parbox[t][][t]{2cm}{\normalfont beruhigte\\beruhigtest\\beruhigte\\beruhigten\\beruhigtet\\beruhigten}\\
\end{tabular}
\begin{tabular}{l}
\parbox[t][][t]{8cm}{}\\
\parbox[t][][t]{8cm}{\normalfont \footnotesize 
jdn beruhigen - 
to reassure (or comfort) sb - 
ihr herzlicher Empfang beruhigte ihn wieder - 
their warm welcome set (or put) him at ease again - 
jds Gewissen/Gedanken beruhigen - 
to ease sb's conscience/mind - 
jdn/etw beruhigen - 
to calm sb/sth (down) - 
jdn/etw beruhigen - 
to pacify sb 
}\\
\end{tabular}
}
%===beschädigen===
\card{\normalfont \Huge beschädigen}{
\begin{tabular}{lll}
\parbox[t][][t]{2.0 cm}{\normalfont \raggedleft ich\\du\\er/sie/es\\wir\\ihr\\sie} &    
\parbox[t][][t]{2cm}{\normalfont beschädige\\beschädigst\\beschädigt\\beschädigen\\beschädigt\\beschädigen} &
\parbox[t][][t]{2cm}{\normalfont beschädigte\\beschädigtest\\beschädigte\\beschädigten\\beschädigtet\\beschädigten}\\
\end{tabular}
\begin{tabular}{l}
\parbox[t][][t]{8cm}{}\\
\parbox[t][][t]{8cm}{\normalfont \footnotesize 
etw beschädigen - 
to damage sth 
}\\
\end{tabular}
}
%===beschließen===
\card{\normalfont \Huge beschließen}{
\begin{tabular}{lll}
\parbox[t][][t]{2.0 cm}{\normalfont \raggedleft ich\\du\\er/sie/es\\wir\\ihr\\sie} &    
\parbox[t][][t]{2cm}{\normalfont beschließe\\beschließt\\beschließt\\beschließen\\beschließt\\beschließen} &
\parbox[t][][t]{2cm}{\normalfont beschloss\\beschlossest\\beschloss\\beschlossen\\beschlosst\\beschlossen}\\
\end{tabular}
\begin{tabular}{l}
\parbox[t][][t]{8cm}{}\\
\parbox[t][][t]{8cm}{\normalfont \footnotesize 
etw beschließen - 
to decide sth - 
ein Gesetz beschließen - 
to vote through a new bill - 
ein Gesetz beschließen - 
to pass a motion - 
beschließen, etw zu tun - 
to decide to do sth - 
beschließen, etw zu tun (nach reiflicher Überlegung) - 
to make up one's mind to do sth 
}\\
\end{tabular}
}
%===beschreiben===
\card{\normalfont \Huge beschreiben}{
\begin{tabular}{lll}
\parbox[t][][t]{2.0 cm}{\normalfont \raggedleft ich\\du\\er/sie/es\\wir\\ihr\\sie} &    
\parbox[t][][t]{2cm}{\normalfont beschreibe\\beschreibst\\beschreibt\\beschreiben\\beschreibt\\beschreiben} &
\parbox[t][][t]{2cm}{\normalfont beschrieb\\beschriebst\\beschrieb\\beschrieben\\beschriebt\\beschrieben}\\
\end{tabular}
\begin{tabular}{l}
\parbox[t][][t]{8cm}{}\\
\parbox[t][][t]{8cm}{\normalfont \footnotesize 
(jdm) jdn/etw beschreiben - 
to describe sb/sth (to sb) - 
(jdm) jdn/etw beschreiben - 
to give (sb) a description of sb/sth - 
du musst mir das nachher in allen Einzelheiten beschreiben - 
you'll have to tell me all about it later - 
kaum/nicht zu beschreiben sein - 
to be almost/absolutely indescribable - 
(jdm) etw gar nicht beschreiben können - 
to not be able to describe sth (to sb) 
}\\
\end{tabular}
}
%===beschweren===
\card{\normalfont \Huge beschweren}{
\begin{tabular}{lll}
\parbox[t][][t]{2.0 cm}{\normalfont \raggedleft ich\\du\\er/sie/es\\wir\\ihr\\sie} &    
\parbox[t][][t]{2cm}{\normalfont beschwere\\beschwerst\\beschwert\\beschweren\\beschwert\\beschweren} &
\parbox[t][][t]{2cm}{\normalfont beschwerte\\beschwertest\\beschwerte\\beschwerten\\beschwertet\\beschwerten}\\
\end{tabular}
\begin{tabular}{l}
\parbox[t][][t]{8cm}{}\\
\parbox[t][][t]{8cm}{\normalfont \footnotesize 
sich akk (bei jdm) (über jdn/etw) beschweren - 
to complain (about sb/sth) (to sb) - 
ich kann mich nicht beschweren - 
I can't complain - 
sich akk (mit etw dat) beschweren - 
to encumber oneself (with sth) - 
jdn/etw (mit etw dat) beschweren - 
Briefe, Papiere - 
to weight sb/sth (down) (with sth) - 
jdn beschweren 
}\\
\end{tabular}
}
%===besichtigen===
\card{\normalfont \Huge besichtigen}{
\begin{tabular}{lll}
\parbox[t][][t]{2.0 cm}{\normalfont \raggedleft ich\\du\\er/sie/es\\wir\\ihr\\sie} &    
\parbox[t][][t]{2cm}{\normalfont besichtige\\besichtigst\\besichtigt\\besichtigen\\besichtigt\\besichtigen} &
\parbox[t][][t]{2cm}{\normalfont besichtigte\\besichtigtest\\besichtigte\\besichtigten\\besichtigtet\\besichtigten}\\
\end{tabular}
\begin{tabular}{l}
\parbox[t][][t]{8cm}{}\\
\parbox[t][][t]{8cm}{\normalfont \footnotesize 
etw besichtigen - 
to visit sth - 
etw besichtigen - 
Sehenswürdigkeit a. - 
to have a look at sth - 
jdn besichtigen - 
hum neues Baby, zukünftigen Schwiegersohn - 
to inspect sb - 
einen Betrieb besichtigen - 
to have a look round (or have a tour of)  a factory/plant 
}\\
\end{tabular}
}
%===besitzen===
\card{\normalfont \Huge besitzen}{
\begin{tabular}{lll}
\parbox[t][][t]{2.0 cm}{\normalfont \raggedleft ich\\du\\er/sie/es\\wir\\ihr\\sie} &    
\parbox[t][][t]{2cm}{\normalfont besitze\\besitzt\\besitzt\\besitzen\\besitzt\\besitzen} &
\parbox[t][][t]{2cm}{\normalfont besaß\\besaßest\\besaß\\besaßen\\besaßt\\besaßen}\\
\end{tabular}
\begin{tabular}{l}
\parbox[t][][t]{8cm}{}\\
\parbox[t][][t]{8cm}{\normalfont \footnotesize 
etw besitzen - 
to possess (or own) - 
(or have and to hold) sth - 
etw rechtmäßig besitzen - 
to be the rightful owner of sth - 
etw treuhänderisch besitzen - 
to hold sth as a trustee - 
etw besitzen - 
to own (or - 
form possess) 
}\\
\end{tabular}
}
%===besorgen===
\card{\normalfont \Huge besorgen}{
\begin{tabular}{lll}
\parbox[t][][t]{2.0 cm}{\normalfont \raggedleft ich\\du\\er/sie/es\\wir\\ihr\\sie} &    
\parbox[t][][t]{2cm}{\normalfont besorge\\besorgst\\besorgt\\besorgen\\besorgt\\besorgen} &
\parbox[t][][t]{2cm}{\normalfont besorgte\\besorgtest\\besorgte\\besorgten\\besorgtet\\besorgten}\\
\end{tabular}
\begin{tabular}{l}
\parbox[t][][t]{8cm}{}\\
\parbox[t][][t]{8cm}{\normalfont \footnotesize 
(jdm) etw besorgen - 
to buy (or get)  (sb) sth - 
(jdm) etw besorgen - 
(beschaffen) - 
to get (or - 
form obtain) sth for sb (or sb sth) - 
(jdm) etw besorgen - 
(beschaffen) - 
to procure sth for sb form - 
sich dat etw besorgen 
}\\
\end{tabular}
}
%===bestätigen===
\card{\normalfont \Huge bestätigen}{
\begin{tabular}{lll}
\parbox[t][][t]{2.0 cm}{\normalfont \raggedleft ich\\du\\er/sie/es\\wir\\ihr\\sie} &    
\parbox[t][][t]{2cm}{\normalfont bestätige\\bestätigst\\bestätigt\\bestätigen\\bestätigt\\bestätigen} &
\parbox[t][][t]{2cm}{\normalfont bestätigte\\bestätigtest\\bestätigte\\bestätigten\\bestätigtet\\bestätigten}\\
\end{tabular}
\begin{tabular}{l}
\parbox[t][][t]{8cm}{}\\
\parbox[t][][t]{8cm}{\normalfont \footnotesize 
(jdm) etw bestätigen - 
to confirm (sb's) sth - 
eine Theorie bestätigen - 
to confirm (or bear out)  a theory - 
ein Alibi bestätigen - 
to corroborate an alibi - 
die Richtigkeit einer S. gen - 
bestätigen - 
to testify to sth's correctness - 
die Richtigkeit einer S. gen 
}\\
\end{tabular}
}
%===bestehen===
\card{\normalfont \Huge bestehen}{
\begin{tabular}{lll}
\parbox[t][][t]{2.0 cm}{\normalfont \raggedleft ich\\du\\er/sie/es\\wir\\ihr\\sie} &    
\parbox[t][][t]{2cm}{\normalfont bestehe\\bestehst\\besteht\\bestehen\\besteht\\bestehen} &
\parbox[t][][t]{2cm}{\normalfont bestand\\bestandest \\bestand\\bestanden\\bestandet\\bestanden}\\
\end{tabular}
\begin{tabular}{l}
\parbox[t][][t]{8cm}{}\\
\parbox[t][][t]{8cm}{\normalfont \footnotesize 
etw (mit etw dat) bestehen - 
to pass sth (with sth) - 
sie bestand ihre Prüfung mit Auszeichnung - 
she got a distinction in her exam - 
sie bestand ihre Prüfung mit Auszeichnung - 
she passed her exam with distinction - 
etw nicht bestehen - 
to fail sth - 
eine Probe (o. Aufgabe) - 
bestehen 
}\\
\end{tabular}
}
%===bestellen===
\card{\normalfont \Huge bestellen}{
\begin{tabular}{lll}
\parbox[t][][t]{2.0 cm}{\normalfont \raggedleft ich\\du\\er/sie/es\\wir\\ihr\\sie} &    
\parbox[t][][t]{2cm}{\normalfont bestelle\\bestellst\\bestellt\\bestellen\\bestellt\\bestellen} &
\parbox[t][][t]{2cm}{\normalfont bestellte\\bestelltest\\bestellte\\bestellten\\bestelltet\\bestellten}\\
\end{tabular}
\begin{tabular}{l}
\parbox[t][][t]{8cm}{}\\
\parbox[t][][t]{8cm}{\normalfont \footnotesize 
etw (bei jdm) bestellen - 
to order sth (from sb) - 
(sich dat) etw bestellen - 
to order (oneself) sth (or (for oneself)) - 
etw bei einem Kellner bestellen - 
to order (or ask for) sth from a waiter - 
etw bei einem Geschäft bestellen - 
to place an order for sth (with a shop) - 
eine Zeitung bestellen - 
to subscribe to a paper 
}\\
\end{tabular}
}
%===bestimmen===
\card{\normalfont \Huge bestimmen}{
\begin{tabular}{lll}
\parbox[t][][t]{2.0 cm}{\normalfont \raggedleft ich\\du\\er/sie/es\\wir\\ihr\\sie} &    
\parbox[t][][t]{2cm}{\normalfont bestimme\\bestimmst\\bestimmt\\bestimmen\\bestimmt\\bestimmen} &
\parbox[t][][t]{2cm}{\normalfont bestimmte\\bestimmtest\\bestimmte\\bestimmten\\bestimmtet\\bestimmten}\\
\end{tabular}
\begin{tabular}{l}
\parbox[t][][t]{8cm}{}\\
\parbox[t][][t]{8cm}{\normalfont \footnotesize 
etw bestimmen - 
to decide on (or - 
form determine) sth - 
etw bestimmen - 
(entscheiden) - 
to decide sth - 
einen Preis bestimmen - 
to fix (or set)  a price - 
Ort und Zeit bestimmen - 
to fix (or appoint)  a place and time 
}\\
\end{tabular}
}
%===besuchen===
\card{\normalfont \Huge besuchen}{
\begin{tabular}{lll}
\parbox[t][][t]{2.0 cm}{\normalfont \raggedleft ich\\du\\er/sie/es\\wir\\ihr\\sie} &    
\parbox[t][][t]{2cm}{\normalfont besuche\\besuchst\\besucht\\besuchen\\besucht\\besuchen} &
\parbox[t][][t]{2cm}{\normalfont besuchte\\besuchtest\\besuchte\\besuchten\\besuchtet\\besuchten}\\
\end{tabular}
\begin{tabular}{l}
\parbox[t][][t]{8cm}{}\\
\parbox[t][][t]{8cm}{\normalfont \footnotesize 
jdn besuchen - 
to visit (or call (in) on) - 
(or drop in on) sb - 
er wird oft von Freunden besucht - 
he often gets visits from friends - 
besuch mich bald mal wieder! - 
come again soon! - 
einen Patienten besuchen - 
to make a house call on (or visit) sb - 
einen Arzt besuchen 
}\\
\end{tabular}
}
%===beteiligen===
\card{\normalfont \Huge beteiligen}{
\begin{tabular}{lll}
\parbox[t][][t]{2.0 cm}{\normalfont \raggedleft ich\\du\\er/sie/es\\wir\\ihr\\sie} &    
\parbox[t][][t]{2cm}{\normalfont beteilige\\beteiligst\\beteiligt\\beteiligen\\beteiligt\\beteiligen} &
\parbox[t][][t]{2cm}{\normalfont beteiligte\\beteiligtest\\beteiligte\\beteiligten\\beteiligtet\\beteiligten}\\
\end{tabular}
\begin{tabular}{l}
\parbox[t][][t]{8cm}{}\\
\parbox[t][][t]{8cm}{\normalfont \footnotesize 
jdn (an etw dat) beteiligen - 
to give sb a share (in sth) - 
er beteiligte seinen Sohn mit 15 % an seiner Firma - 
he gave his son a 15% stake (or (financial) interest) in his company - 
sich akk (mit etw dat) (an etw dat) beteiligen - 
to participate (or take part)  (in sth) (with sth) - 
beteiligt sich dein Mann eigentlich auch an der Hausarbeit? - 
does your husband help around the house? (or with the housework?) - 
sich akk an einem Unternehmen beteiligen - 
to have a stake (or (financial) interest) in a company 
}\\
\end{tabular}
}
%===beten===
\card{\normalfont \Huge beten}{
\begin{tabular}{lll}
\parbox[t][][t]{2.0 cm}{\normalfont \raggedleft ich\\du\\er/sie/es\\wir\\ihr\\sie} &    
\parbox[t][][t]{2cm}{\normalfont bete\\betest\\betet\\beten\\betet\\beten} &
\parbox[t][][t]{2cm}{\normalfont betete\\betetest\\betete\\beteten\\betetet\\beteten}\\
\end{tabular}
\begin{tabular}{l}
\parbox[t][][t]{8cm}{}\\
\parbox[t][][t]{8cm}{\normalfont \footnotesize 
beten - 
to pray - 
für jdn/etw beten - 
to pray for sb/sth - 
um etw akk - 
beten - 
to pray for sth - 
zu jdm beten - 
to pray to sb - 
etw beten 
}\\
\end{tabular}
}
%===betonen===
\card{\normalfont \Huge betonen}{
\begin{tabular}{lll}
\parbox[t][][t]{2.0 cm}{\normalfont \raggedleft ich\\du\\er/sie/es\\wir\\ihr\\sie} &    
\parbox[t][][t]{2cm}{\normalfont betone\\betonst\\betont\\betonen\\betont\\betonen} &
\parbox[t][][t]{2cm}{\normalfont betonte\\betontest\\betonte\\betonten\\betontet\\betonten}\\
\end{tabular}
\begin{tabular}{l}
\parbox[t][][t]{8cm}{}\\
\parbox[t][][t]{8cm}{\normalfont \footnotesize 
etw betonen - 
to accentuate sth - 
dieses Kleid betont ihre Figur - 
this dress accentuates her figure - 
etw betonen - 
to stress (or emphasize) sth - 
betonen, dass - 
to stress (or emphasize) that - 
etw betonen - 
to stress sth 
}\\
\end{tabular}
}
%===betragen===
\card{\normalfont \Huge betragen}{
\begin{tabular}{lll}
\parbox[t][][t]{2.0 cm}{\normalfont \raggedleft ich\\du\\er/sie/es\\wir\\ihr\\sie} &    
\parbox[t][][t]{2cm}{\normalfont betrage\\beträgst\\beträgt\\betragen\\betragt\\betragen} &
\parbox[t][][t]{2cm}{\normalfont betrug\\betrugst\\betrug\\betrugen\\betrugt\\betrugen}\\
\end{tabular}
\begin{tabular}{l}
\parbox[t][][t]{8cm}{}\\
\parbox[t][][t]{8cm}{\normalfont \footnotesize 
betragen - 
to be - 
die Rechnung beträgt Euro 10 - 
the bill comes (or amounts) to 10 euros - 
die Preisdifferenz beträgt Euro 378 - 
the difference in price is (or comes to) 378 euros - 
sich akk irgendwie betragen - 
to behave in a certain manner - 
Betragen - 
behaviour (or 
}\\
\end{tabular}
}
%===betrügen===
\card{\normalfont \Huge betrügen}{
\begin{tabular}{lll}
\parbox[t][][t]{2.0 cm}{\normalfont \raggedleft ich\\du\\er/sie/es\\wir\\ihr\\sie} &    
\parbox[t][][t]{2cm}{\normalfont betrüge\\betrügst\\betrügt\\betrügen\\betrügt\\betrügen} &
\parbox[t][][t]{2cm}{\normalfont betrog\\betrogst\\betrog\\betrogen\\betrogt\\betrogen}\\
\end{tabular}
\begin{tabular}{l}
\parbox[t][][t]{8cm}{}\\
\parbox[t][][t]{8cm}{\normalfont \footnotesize 
jdn betrügen - 
to cheat (or swindle) sb - 
jdn um etw akk - 
betrügen - 
to cheat sb out of sth - 
betrogen - 
cheated - 
betrogen - 
deceived - 
ich fühle mich betrogen! 
}\\
\end{tabular}
}
%===bewegen===
\card{\normalfont \Huge bewegen}{
\begin{tabular}{lll}
\parbox[t][][t]{2.0 cm}{\normalfont \raggedleft ich\\du\\er/sie/es\\wir\\ihr\\sie} &    
\parbox[t][][t]{2cm}{\normalfont bewege\\bewegst\\bewegt\\bewegen\\bewegt\\bewegen} &
\parbox[t][][t]{2cm}{\normalfont bewog\\bewogst\\bewog\\bewogen\\bewogt\\bewogen}\\
\end{tabular}
\begin{tabular}{l}
\parbox[t][][t]{8cm}{}\\
\parbox[t][][t]{8cm}{\normalfont \footnotesize 
etw bewegen - 
to move sth - 
etw von etw dat/zu etw dat - 
bewegen - 
to move sth from/to sth - 
jdn bewegen - 
to concern sb - 
jdn bewegen - 
(innerlich aufwühlen) - 
to move sb 
}\\
\end{tabular}
}
%===beweisen===
\card{\normalfont \Huge beweisen}{
\begin{tabular}{lll}
\parbox[t][][t]{2.0 cm}{\normalfont \raggedleft ich\\du\\er/sie/es\\wir\\ihr\\sie} &    
\parbox[t][][t]{2cm}{\normalfont beweise\\beweist\\beweist\\beweisen\\beweist\\beweisen} &
\parbox[t][][t]{2cm}{\normalfont bewies\\bewiesest\\bewies\\bewiesen\\bewiest\\bewiesen}\\
\end{tabular}
\begin{tabular}{l}
\parbox[t][][t]{8cm}{}\\
\parbox[t][][t]{8cm}{\normalfont \footnotesize 
(jdm) etw beweisen - 
to prove sth (to sb) - 
der Angeklagte ist unschuldig, bis das Gegenteil bewiesen wird - 
the defendant (or accused) is innocent until proven guilty - 
was zu beweisen war - 
which was (the thing) to be proved - 
was zu beweisen war - 
quod erat demonstrandum - 
was (noch) zu beweisen wäre - 
which remains to be proved 
}\\
\end{tabular}
}
%===bezahlen===
\card{\normalfont \Huge bezahlen}{
\begin{tabular}{lll}
\parbox[t][][t]{2.0 cm}{\normalfont \raggedleft ich\\du\\er/sie/es\\wir\\ihr\\sie} &    
\parbox[t][][t]{2cm}{\normalfont bezahle\\bezahlst\\bezahlt\\bezahlen\\bezahlt\\bezahlen} &
\parbox[t][][t]{2cm}{\normalfont bezahlte\\bezahltest\\bezahlte\\bezahlten\\bezahltet\\bezahlten}\\
\end{tabular}
\begin{tabular}{l}
\parbox[t][][t]{8cm}{}\\
\parbox[t][][t]{8cm}{\normalfont \footnotesize 
(jdm) etw bezahlen - 
to pay (sb) sth - 
wenn Sie mir 100 Euro bezahlen, verrate ich alles! - 
give me 100 euros and I'll tell you everything! - 
die Rechnung muss gleich bezahlt werden - 
the bill must be settled immediately - 
ich bezahle den Wein! - 
I'll pay for the wine! - 
jdn (für etw akk) bezahlen - 
to pay sb (for sth) 
}\\
\end{tabular}
}
%===bezeichnen===
\card{\normalfont \Huge bezeichnen}{
\begin{tabular}{lll}
\parbox[t][][t]{2.0 cm}{\normalfont \raggedleft ich\\du\\er/sie/es\\wir\\ihr\\sie} &    
\parbox[t][][t]{2cm}{\normalfont bezeichne\\bezeichnest\\bezeichnet\\bezeichnen\\bezeichnet\\bezeichnen} &
\parbox[t][][t]{2cm}{\normalfont bezeichnete\\bezeichnetest\\bezeichnete\\bezeichneten\\bezeichnetet\\bezeichneten}\\
\end{tabular}
\begin{tabular}{l}
\parbox[t][][t]{8cm}{}\\
\parbox[t][][t]{8cm}{\normalfont \footnotesize 
jdn/etw (als jdn/etw) bezeichnen - 
to call sb/sth (sb/sth) - 
dein Verhalten kann man nur als ungehörig bezeichnen! - 
your behaviour can only be described as impertinent! - 
etw bezeichnen - 
to denote sth - 
(jdm) etw bezeichnen - 
to describe sth (to sb) - 
etw (durch etw akk/mit etw dat) bezeichnen - 
to mark sth (with sth) 
}\\
\end{tabular}
}
%===beziehen===
\card{\normalfont \Huge beziehen}{
\begin{tabular}{lll}
\parbox[t][][t]{2.0 cm}{\normalfont \raggedleft ich\\du\\er/sie/es\\wir\\ihr\\sie} &    
\parbox[t][][t]{2cm}{\normalfont beziehe\\beziehst\\bezieht\\beziehen\\bezieht\\beziehen} &
\parbox[t][][t]{2cm}{\normalfont bezog\\bezogst\\bezog\\bezogen\\bezogt\\bezogen}\\
\end{tabular}
\begin{tabular}{l}
\parbox[t][][t]{8cm}{}\\
\parbox[t][][t]{8cm}{\normalfont \footnotesize 
etw (mit etw dat) beziehen - 
to cover sth (with sth) - 
die Bettwäsche neu beziehen - 
to change the bed(linen) (or sheets) - 
etw neu beziehen - 
to re-cover sth - 
etw mit (neuen) Saiten beziehen Instrument - 
to (re)string sth - 
etw beziehen - 
to move into sth 
}\\
\end{tabular}
}
%===biegen===
\card{\normalfont \Huge biegen}{
\begin{tabular}{lll}
\parbox[t][][t]{2.0 cm}{\normalfont \raggedleft ich\\du\\er/sie/es\\wir\\ihr\\sie} &    
\parbox[t][][t]{2cm}{\normalfont biege\\biegst\\biegt\\biegen\\biegt\\biegen} &
\parbox[t][][t]{2cm}{\normalfont bog\\bogst\\bog\\bogen\\bogt\\bogen}\\
\end{tabular}
\begin{tabular}{l}
\parbox[t][][t]{8cm}{}\\
\parbox[t][][t]{8cm}{\normalfont \footnotesize 
etw biegen - 
to bend sth - 
(jdm) etw biegen - 
to bend (or flex) sth (to sb) - 
biegen - 
to inflect - 
auf Biegen oder (o. und) Brechen fam - 
by hook or by crook - 
es geht auf Biegen oder Brechen fam - 
it's all or nothing (or do or die) 
}\\
\end{tabular}
}
%===bieten===
\card{\normalfont \Huge bieten}{
\begin{tabular}{lll}
\parbox[t][][t]{2.0 cm}{\normalfont \raggedleft ich\\du\\er/sie/es\\wir\\ihr\\sie} &    
\parbox[t][][t]{2cm}{\normalfont biete\\bietest\\bietet\\bieten\\bietet\\bieten} &
\parbox[t][][t]{2cm}{\normalfont bot\\botest\\bot\\boten\\botet\\boten}\\
\end{tabular}
\begin{tabular}{l}
\parbox[t][][t]{8cm}{}\\
\parbox[t][][t]{8cm}{\normalfont \footnotesize 
etw (für etw akk) bieten - 
to offer sth (for sth) - 
etw (für etw akk) bieten - 
(bei Auktionen) - 
to bid sth (for sth) - 
er hat 2.000 Euro auf das Gemälde geboten - 
he bid 2,000 euros for the painting - 
wer bietet mehr? - 
any more bids? - 
mehr/weniger bieten als jd 
}\\
\end{tabular}
}
%===binden===
\card{\normalfont \Huge binden}{
\begin{tabular}{lll}
\parbox[t][][t]{2.0 cm}{\normalfont \raggedleft ich\\du\\er/sie/es\\wir\\ihr\\sie} &    
\parbox[t][][t]{2cm}{\normalfont binde\\bindest\\bindet\\binden\\bindet\\binden} &
\parbox[t][][t]{2cm}{\normalfont band\\bandest\\band\\banden\\bandet\\banden}\\
\end{tabular}
\begin{tabular}{l}
\parbox[t][][t]{8cm}{}\\
\parbox[t][][t]{8cm}{\normalfont \footnotesize 
etw (zu etw dat) binden - 
to bind (or tie) sth (to sth) - 
Fichtenzweige wurden zu Kränzen gebunden - 
pine twigs were tied (or bound)  (together) into wreaths - 
binden Sie mir bitte einen Strauß roter Rosen! - 
make up a bunch of red roses for me, please - 
bindest du mir bitte die Krawatte? - 
can you do (up) my tie (for me), please? - 
kannst du mir bitte die Schürze hinten binden? - 
can you tie my apron at the back for me, please? 
}\\
\end{tabular}
}
%===bitten===
\card{\normalfont \Huge bitten}{
\begin{tabular}{lll}
\parbox[t][][t]{2.0 cm}{\normalfont \raggedleft ich\\du\\er/sie/es\\wir\\ihr\\sie} &    
\parbox[t][][t]{2cm}{\normalfont bitte\\bittest\\bittet\\bitten\\bittet\\bitten} &
\parbox[t][][t]{2cm}{\normalfont bat\\batest\\bat\\baten\\batet\\baten}\\
\end{tabular}
\begin{tabular}{l}
\parbox[t][][t]{8cm}{}\\
\parbox[t][][t]{8cm}{\normalfont \footnotesize 
bite - 
Biss m - 
Add to my favourites Preselect for export to vocabulary trainer View selected vocabulary - 
bite of an insect - 
Stich m - 
Add to my favourites Preselect for export to vocabulary trainer View selected vocabulary - 
bite mark - 
Bisswunde f - 
Add to my favourites Preselect for export to vocabulary trainer View selected vocabulary - 
snake/dog bite 
}\\
\end{tabular}
}
%===blasen===
\card{\normalfont \Huge blasen}{
\begin{tabular}{lll}
\parbox[t][][t]{2.0 cm}{\normalfont \raggedleft ich\\du\\er/sie/es\\wir\\ihr\\sie} &    
\parbox[t][][t]{2cm}{\normalfont blase\\bläst\\bläst\\blasen\\blast\\blasen} &
\parbox[t][][t]{2cm}{\normalfont blies\\bliesest\\blies\\bliesen\\bliest\\bliesen}\\
\end{tabular}
\begin{tabular}{l}
\parbox[t][][t]{8cm}{}\\
\parbox[t][][t]{8cm}{\normalfont \footnotesize 
blasen - 
to blow - 
auf etw akk - 
blasen - 
to blow on sth - 
auf eine Brandwunde blasen - 
to blow on a burn - 
blasen - 
to play - 
auf etw akk/in etw dat 
}\\
\end{tabular}
}
%===bleiben===
\card{\normalfont \Huge bleiben}{
\begin{tabular}{lll}
\parbox[t][][t]{2.0 cm}{\normalfont \raggedleft ich\\du\\er/sie/es\\wir\\ihr\\sie} &    
\parbox[t][][t]{2cm}{\normalfont bleibe\\bleibst\\bleibt\\bleiben\\bleibt\\bleiben} &
\parbox[t][][t]{2cm}{\normalfont blieb\\bliebst\\blieb\\blieben\\bliebt\\blieben}\\
\end{tabular}
\begin{tabular}{l}
\parbox[t][][t]{8cm}{}\\
\parbox[t][][t]{8cm}{\normalfont \footnotesize 
bleiben - 
to stay - 
bleiben Sie doch noch! - 
do stay! - 
ich bleibe noch zwei Jahre in der Schule - 
I'll be staying at school another two years - 
ich bleibe heute etwas länger im Büro - 
I'll be working late today - 
bleiben Sie am Apparat! - 
hold the line! 
}\\
\end{tabular}
}
%===bleichen===
\card{\normalfont \Huge bleichen}{
\begin{tabular}{lll}
\parbox[t][][t]{2.0 cm}{\normalfont \raggedleft ich\\du\\er/sie/es\\wir\\ihr\\sie} &    
\parbox[t][][t]{2cm}{\normalfont bleiche\\bleichst\\bleicht\\bleichen\\bleicht\\bleichen} &
\parbox[t][][t]{2cm}{\normalfont bleichte\\bleichtest\\bleichte\\bleichten\\bleichtet\\bleichten}\\
\end{tabular}
\begin{tabular}{l}
\parbox[t][][t]{8cm}{}\\
\parbox[t][][t]{8cm}{\normalfont \footnotesize 
etw bleichen - 
to bleach sth - 
bleichen - 
to become faded - 
bleich - 
pale - 
bleich (vor etw dat) werden - 
to go (or turn) pale (with sth) - 
er wurde bleich vor Entsetzen/Schreck - 
he paled (or went pale) 
}\\
\end{tabular}
}
%===blühen===
\card{\normalfont \Huge blühen}{
\begin{tabular}{lll}
\parbox[t][][t]{2.0 cm}{\normalfont \raggedleft ich\\du\\er/sie/es\\wir\\ihr\\sie} &    
\parbox[t][][t]{2cm}{\normalfont blühe\\blühst\\blüht\\blühen\\blüht\\blühen} &
\parbox[t][][t]{2cm}{\normalfont blühte\\blühtest\\blühte\\blühten\\blühtet\\blühten}\\
\end{tabular}
\begin{tabular}{l}
\parbox[t][][t]{8cm}{}\\
\parbox[t][][t]{8cm}{\normalfont \footnotesize 
blühen - 
to bloom - 
blühen - 
to flower - 
zum Blühen kommen (zu blühen beginnen) - 
to (come into) blossom - 
blühen (und gedeihen) - 
to flourish - 
blühen (und gedeihen) - 
to thrive 
}\\
\end{tabular}
}
%===braten===
\card{\normalfont \Huge braten}{
\begin{tabular}{lll}
\parbox[t][][t]{2.0 cm}{\normalfont \raggedleft ich\\du\\er/sie/es\\wir\\ihr\\sie} &    
\parbox[t][][t]{2cm}{\normalfont brate\\brätst\\brät\\braten\\bratet\\braten} &
\parbox[t][][t]{2cm}{\normalfont briet\\brietest \\briet\\brieten\\brietet\\brieten}\\
\end{tabular}
\begin{tabular}{l}
\parbox[t][][t]{8cm}{}\\
\parbox[t][][t]{8cm}{\normalfont \footnotesize 
etw braten - 
(in der Pfanne garen) - 
to fry sth - 
etw braten - 
(am Spieß garen) - 
to roast sth (on a spit) - 
(sich dat) etw braten - 
to fry (oneself) sth - 
etw knusprig (o. kross) - 
braten 
}\\
\end{tabular}
}
%===brauchen===
\card{\normalfont \Huge brauchen}{
\begin{tabular}{lll}
\parbox[t][][t]{2.0 cm}{\normalfont \raggedleft ich\\du\\er/sie/es\\wir\\ihr\\sie} &    
\parbox[t][][t]{2cm}{\normalfont brauche\\brauchst\\braucht\\brauchen\\braucht\\brauchen} &
\parbox[t][][t]{2cm}{\normalfont brauchte\\brauchtest\\brauchte\\brauchten\\brauchtet\\brauchten}\\
\end{tabular}
\begin{tabular}{l}
\parbox[t][][t]{8cm}{}\\
\parbox[t][][t]{8cm}{\normalfont \footnotesize 
jdn/etw brauchen - 
to need sb/sth - 
ich habe alles, was ich brauche - 
I have everything I need - 
wozu brauchst du das? - 
what do you need that for? - 
brauchst du noch etwas? - 
do you require anything else? - 
brauchst du das Messer gerade, oder kann ich es mir mal kurz ausleihen? - 
are you using this knife or can I borrow it for a minute? 
}\\
\end{tabular}
}
%===brechen===
\card{\normalfont \Huge brechen}{
\begin{tabular}{lll}
\parbox[t][][t]{2.0 cm}{\normalfont \raggedleft ich\\du\\er/sie/es\\wir\\ihr\\sie} &    
\parbox[t][][t]{2cm}{\normalfont breche\\brichst\\bricht\\brechen\\brecht\\brechen} &
\parbox[t][][t]{2cm}{\normalfont brach\\brachst\\brach\\brachen\\bracht\\brachen}\\
\end{tabular}
\begin{tabular}{l}
\parbox[t][][t]{8cm}{}\\
\parbox[t][][t]{8cm}{\normalfont \footnotesize 
etw brechen - 
to break sth - 
etw von etw dat - 
brechen - 
to break sth off sth - 
Zweige von den Bäumen brechen - 
to break twigs off trees - 
Schiefer/Stein/Marmor brechen - 
to cut slate/stone/marble - 
Schiefer/Stein/Marmor brechen 
}\\
\end{tabular}
}
%===brennen===
\card{\normalfont \Huge brennen}{
\begin{tabular}{lll}
\parbox[t][][t]{2.0 cm}{\normalfont \raggedleft ich\\du\\er/sie/es\\wir\\ihr\\sie} &    
\parbox[t][][t]{2cm}{\normalfont brenne\\brennst\\brennt\\brennen\\brennt\\brennen} &
\parbox[t][][t]{2cm}{\normalfont brannte\\branntest\\brannte\\brannten\\branntet\\brannten}\\
\end{tabular}
\begin{tabular}{l}
\parbox[t][][t]{8cm}{}\\
\parbox[t][][t]{8cm}{\normalfont \footnotesize 
brennen - 
to be on fire - 
lichterloh brennen - 
to be ablaze - 
zu brennen anfangen - 
to start burning - 
zu brennen anfangen - 
to catch fire - 
brennend - 
burning 
}\\
\end{tabular}
}
%===bringen===
\card{\normalfont \Huge bringen}{
\begin{tabular}{lll}
\parbox[t][][t]{2.0 cm}{\normalfont \raggedleft ich\\du\\er/sie/es\\wir\\ihr\\sie} &    
\parbox[t][][t]{2cm}{\normalfont bringe\\bringst\\bringt\\bringen\\bringt\\bringen} &
\parbox[t][][t]{2cm}{\normalfont brachte\\brachtest\\brachte\\brachten\\brachtet\\brachten}\\
\end{tabular}
\begin{tabular}{l}
\parbox[t][][t]{8cm}{}\\
\parbox[t][][t]{8cm}{\normalfont \footnotesize 
etw bringen - 
to bring sth - 
etw bringen - 
(hinbringen a.) - 
to take sth - 
den Müll nach draußen bringen - 
to take/bring out sep the rubbish Brit (or garbage) - 
etw in Stellung bringen - 
to position sth - 
(jdm) etw bringen 
}\\
\end{tabular}
}
%===buchen===
\card{\normalfont \Huge buchen}{
\begin{tabular}{lll}
\parbox[t][][t]{2.0 cm}{\normalfont \raggedleft ich\\du\\er/sie/es\\wir\\ihr\\sie} &    
\parbox[t][][t]{2cm}{\normalfont buche\\buchst\\bucht\\buchen\\bucht\\buchen} &
\parbox[t][][t]{2cm}{\normalfont buchte\\buchtest\\buchte\\buchten\\buchtet\\buchten}\\
\end{tabular}
\begin{tabular}{l}
\parbox[t][][t]{8cm}{}\\
\parbox[t][][t]{8cm}{\normalfont \footnotesize 
etw (bei einem Reisebüro) buchen - 
to book (or reserve) sth (at a travel agent) - 
etw (als etw akk) buchen - 
to enter sth (as sth) - 
etw buchen - 
to register sth - 
etw als Erfolg/Sieg buchen - 
to mark (or - 
fam chalk) up a success/victory (for oneself) - 
etw unter etw akk buchen 
}\\
\end{tabular}
}
%===danken===
\card{\normalfont \Huge danken}{
\begin{tabular}{lll}
\parbox[t][][t]{2.0 cm}{\normalfont \raggedleft ich\\du\\er/sie/es\\wir\\ihr\\sie} &    
\parbox[t][][t]{2cm}{\normalfont danke\\dankst\\dankt\\danken\\dankt\\danken} &
\parbox[t][][t]{2cm}{\normalfont dankte\\danktest\\dankte\\dankten\\danktet\\dankten}\\
\end{tabular}
\begin{tabular}{l}
\parbox[t][][t]{8cm}{}\\
\parbox[t][][t]{8cm}{\normalfont \footnotesize 
danken - 
to express one's thanks - 
danken (danke sagen) - 
to say thanks - 
sie dankte und legte auf - 
she said thanks and put the phone down - 
lasset uns danken - 
REL - 
let us thank the Lord - 
(ich) danke 
}\\
\end{tabular}
}
%===dauern===
\card{\normalfont \Huge dauern}{
\begin{tabular}{lll}
\parbox[t][][t]{2.0 cm}{\normalfont \raggedleft ich\\du\\er/sie/es\\wir\\ihr\\sie} &    
\parbox[t][][t]{2cm}{\normalfont dauere\\dauerst\\dauert\\dauern\\dauert\\dauern} &
\parbox[t][][t]{2cm}{\normalfont dauerte\\dauertest\\dauerte\\dauerten\\dauertet\\dauerten}\\
\end{tabular}
\begin{tabular}{l}
\parbox[t][][t]{8cm}{}\\
\parbox[t][][t]{8cm}{\normalfont \footnotesize 
dauern - 
to last - 
eine Stunde/einen Tag/lang/länger dauern - 
to last an hour/a day/a long time/longer - 
dieser Krach dauert jetzt schon den ganzen Tag - 
this racket has been going on all (or the whole) day now - 
der Film dauert 3 Stunden - 
the film is 3 hours long - 
dauern - 
to take 
}\\
\end{tabular}
}
%===denken===
\card{\normalfont \Huge denken}{
\begin{tabular}{lll}
\parbox[t][][t]{2.0 cm}{\normalfont \raggedleft ich\\du\\er/sie/es\\wir\\ihr\\sie} &    
\parbox[t][][t]{2cm}{\normalfont denke\\denkst\\denkt\\denken\\denkt\\denken} &
\parbox[t][][t]{2cm}{\normalfont dachte\\dachtest\\dachte\\dachten\\dachtet\\dachten}\\
\end{tabular}
\begin{tabular}{l}
\parbox[t][][t]{8cm}{}\\
\parbox[t][][t]{8cm}{\normalfont \footnotesize 
denken - 
to think - 
ich denke, also bin ich - 
I think, therefore I am - 
jdm zu denken geben - 
to give sb food for thought (or something to think about) - 
das gab mir zu denken - 
that made me think - 
langsam/schnell denken - 
to be a slow/quick thinker 
}\\
\end{tabular}
}
%===dienen===
\card{\normalfont \Huge dienen}{
\begin{tabular}{lll}
\parbox[t][][t]{2.0 cm}{\normalfont \raggedleft ich\\du\\er/sie/es\\wir\\ihr\\sie} &    
\parbox[t][][t]{2cm}{\normalfont diene\\dienst\\dient\\dienen\\dient\\dienen} &
\parbox[t][][t]{2cm}{\normalfont diente\\dientest\\diente\\dienten\\dientet\\dienten}\\
\end{tabular}
\begin{tabular}{l}
\parbox[t][][t]{8cm}{}\\
\parbox[t][][t]{8cm}{\normalfont \footnotesize 
etw dat - 
dienen - 
to be (important) for sth - 
jds Interessen dienen - 
to serve sb's interests - 
jds Sicherheit dienen - 
for sb's safety - 
zum Verständnis einer S. gen - 
dienen - 
to help in understanding sth 
}\\
\end{tabular}
}
%===diskutieren===
\card{\normalfont \Huge diskutieren}{
\begin{tabular}{lll}
\parbox[t][][t]{2.0 cm}{\normalfont \raggedleft ich\\du\\er/sie/es\\wir\\ihr\\sie} &    
\parbox[t][][t]{2cm}{\normalfont diskutiere\\diskutierst\\diskutiert\\diskutieren\\diskutiert\\diskutieren} &
\parbox[t][][t]{2cm}{\normalfont diskutierte\\diskutiertest\\diskutierte\\diskutierten\\diskutiertet\\diskutierten}\\
\end{tabular}
\begin{tabular}{l}
\parbox[t][][t]{8cm}{}\\
\parbox[t][][t]{8cm}{\normalfont \footnotesize 
etw diskutieren - 
to discuss sth - 
etw abschließend diskutieren - 
to discuss sth conclusively - 
etw ausgiebig diskutieren - 
to discuss sth at length - 
etw erschöpfend diskutieren - 
to have exhaustive discussions about sth - 
etw zu Ende diskutieren - 
to finish discussing sth 
}\\
\end{tabular}
}
%===dreschen===
\card{\normalfont \Huge dreschen}{
\begin{tabular}{lll}
\parbox[t][][t]{2.0 cm}{\normalfont \raggedleft ich\\du\\er/sie/es\\wir\\ihr\\sie} &    
\parbox[t][][t]{2cm}{\normalfont dresche\\drischst\\drischt\\dreschen\\drescht\\dreschen} &
\parbox[t][][t]{2cm}{\normalfont drosch\\droschst\\drosch\\droschen\\droscht\\droschen}\\
\end{tabular}
\begin{tabular}{l}
\parbox[t][][t]{8cm}{}\\
\parbox[t][][t]{8cm}{\normalfont \footnotesize 
etw dreschen - 
to thresh sth - 
jdn dreschen - 
to thrash sb - 
jd grün und blau dreschen - 
to beat sb black and blue - 
jdm eine dreschen - 
fam - 
to land sb one Brit - 
sich akk 
}\\
\end{tabular}
}
%===dringen===
\card{\normalfont \Huge dringen}{
\begin{tabular}{lll}
\parbox[t][][t]{2.0 cm}{\normalfont \raggedleft ich\\du\\er/sie/es\\wir\\ihr\\sie} &    
\parbox[t][][t]{2cm}{\normalfont dringe\\dringst\\dringt\\dringen\\dringt\\dringen} &
\parbox[t][][t]{2cm}{\normalfont drang\\drangst\\drang\\drangen\\drangt\\drangen}\\
\end{tabular}
\begin{tabular}{l}
\parbox[t][][t]{8cm}{}\\
\parbox[t][][t]{8cm}{\normalfont \footnotesize 
durch etw akk/in etw akk - 
dringen - 
to penetrate sth - 
durch die Bewölkung/den Nebel/in den Nachthimmel dringen - 
to pierce the clouds/fog/the night sky - 
durch etw akk - 
dringen - 
to force one's/it's way through sth - 
an etw akk/zu jdm dringen - 
to get through to (or reach) sth/sb 
}\\
\end{tabular}
}
%===drucken===
\card{\normalfont \Huge drucken}{
\begin{tabular}{lll}
\parbox[t][][t]{2.0 cm}{\normalfont \raggedleft ich\\du\\er/sie/es\\wir\\ihr\\sie} &    
\parbox[t][][t]{2cm}{\normalfont drucke\\druckst\\druckt\\drucken\\druckt\\drucken} &
\parbox[t][][t]{2cm}{\normalfont druckte\\drucktest\\druckte\\druckten\\drucktet\\druckten}\\
\end{tabular}
\begin{tabular}{l}
\parbox[t][][t]{8cm}{}\\
\parbox[t][][t]{8cm}{\normalfont \footnotesize 
(jdm) etw drucken - 
to print sth (for sb) - 
etw auf etw dat - 
drucken - 
to print sth on sth - 
drucken - 
to print - 
etw drücken - 
to press sth - 
einen Knopf drücken 
}\\
\end{tabular}
}
%===drücken===
\card{\normalfont \Huge drücken}{
\begin{tabular}{lll}
\parbox[t][][t]{2.0 cm}{\normalfont \raggedleft ich\\du\\er/sie/es\\wir\\ihr\\sie} &    
\parbox[t][][t]{2cm}{\normalfont drücke\\drückst\\drückt\\drücken\\drückt\\drücken} &
\parbox[t][][t]{2cm}{\normalfont drückte\\drücktest\\drückte\\drückten\\drücktet\\drückten}\\
\end{tabular}
\begin{tabular}{l}
\parbox[t][][t]{8cm}{}\\
\parbox[t][][t]{8cm}{\normalfont \footnotesize 
etw drücken - 
to press sth - 
einen Knopf drücken - 
to push (or - 
Brit a. press)  a button - 
bei Alarm Knopf drücken - 
please press the button in case of alarm - 
etw aus etw dat - 
drücken - 
to squeeze sth out of sth 
}\\
\end{tabular}
}
%===dürfen===
\card{\normalfont \Huge dürfen}{
\begin{tabular}{lll}
\parbox[t][][t]{2.0 cm}{\normalfont \raggedleft ich\\du\\er/sie/es\\wir\\ihr\\sie} &    
\parbox[t][][t]{2cm}{\normalfont darf\\darfst\\darf\\dürfen\\dürft\\dürfen} &
\parbox[t][][t]{2cm}{\normalfont durfte\\durftest\\durfte\\durften\\durftet\\durften}\\
\end{tabular}
\begin{tabular}{l}
\parbox[t][][t]{8cm}{}\\
\parbox[t][][t]{8cm}{\normalfont \footnotesize 
etw (nicht) tun dürfen - 
to (not) be allowed to do sth - 
darf man hier parken? - 
are you allowed (or is it permitted) to park here? - 
hier darf man nicht rauchen - 
smoking is not allowed (or permitted) here - 
darf ich heute Abend ins Kino gehen? - 
can I go to the cinema tonight? - 
jd/etw darf etw nicht tun - 
sb/sth mustn't do sth (or 
}\\
\end{tabular}
}
%===ehren===
\card{\normalfont \Huge ehren}{
\begin{tabular}{lll}
\parbox[t][][t]{2.0 cm}{\normalfont \raggedleft ich\\du\\er/sie/es\\wir\\ihr\\sie} &    
\parbox[t][][t]{2cm}{\normalfont ehre\\ehrst\\ehrt\\ehren\\ehrt\\ehren} &
\parbox[t][][t]{2cm}{\normalfont ehrte\\ehrtest\\ehrte\\ehrten\\ehrtet\\ehrten}\\
\end{tabular}
\begin{tabular}{l}
\parbox[t][][t]{8cm}{}\\
\parbox[t][][t]{8cm}{\normalfont \footnotesize 
jdn (durch etw akk - 
(o. mit dat - 
)) ehren - 
to honour (or - 
Am -or) sb (with sth) - 
jdn ehren - 
to make sb feel honoured (or - 
Am -ored) - 
dieser Besuch ehrt uns sehr - 
we are very much honoured by this visit 
}\\
\end{tabular}
}
%===einfallen===
\card{\normalfont \Huge einfallen}{
\begin{tabular}{lll}
\parbox[t][][t]{2.0 cm}{\normalfont \raggedleft ich\\du\\er/sie/es\\wir\\ihr\\sie} &    
\parbox[t][][t]{2cm}{\normalfont falle ein\\fällst ein\\fällt ein\\fallen ein\\fallt ein\\fallen ein} &
\parbox[t][][t]{2cm}{\normalfont fiel ein\\fielst ein\\fiel ein\\fielen ein\\fielt ein\\fielen ein}\\
\end{tabular}
\begin{tabular}{l}
\parbox[t][][t]{8cm}{}\\
\parbox[t][][t]{8cm}{\normalfont \footnotesize 
etw fällt jdm ein - 
sb thinks of sth - 
sich dat etwas einfallen lassen - 
to think of sth - 
was fällt Ihnen ein! - 
what do you think you're doing! - 
etw fällt jdm ein - 
sb remembers sth - 
der Name will mir einfach nicht einfallen! - 
the name just won't come to me! 
}\\
\end{tabular}
}
%===einkaufen===
\card{\normalfont \Huge einkaufen}{
\begin{tabular}{lll}
\parbox[t][][t]{2.0 cm}{\normalfont \raggedleft ich\\du\\er/sie/es\\wir\\ihr\\sie} &    
\parbox[t][][t]{2cm}{\normalfont kaufe ein\\kaufst ein\\kauft ein\\kaufen ein\\kauft ein\\kaufen ein} &
\parbox[t][][t]{2cm}{\normalfont kaufte ein\\kauftest ein\\kaufte ein\\kauften ein\\kauftet ein\\kauften ein}\\
\end{tabular}
\begin{tabular}{l}
\parbox[t][][t]{8cm}{}\\
\parbox[t][][t]{8cm}{\normalfont \footnotesize 
etw einkaufen - 
to buy sth - 
etw billig/günstiger/teuer einkaufen - 
to buy sth cheaply/at a more favourable price/at an expensive price (or to pay little/less/a lot for sth) - 
(bei jdm/in etw dat) einkaufen - 
to shop (at sb's/sth) - 
einkaufen gehen - 
to go shopping - 
sich akk in etw akk - 
einkaufen 
}\\
\end{tabular}
}
%===einladen===
\card{\normalfont \Huge einladen}{
\begin{tabular}{lll}
\parbox[t][][t]{2.0 cm}{\normalfont \raggedleft ich\\du\\er/sie/es\\wir\\ihr\\sie} &    
\parbox[t][][t]{2cm}{\normalfont lade ein\\lädst ein\\lädt ein\\laden ein\\ladet ein\\laden ein} &
\parbox[t][][t]{2cm}{\normalfont lud ein\\ludst ein\\lud ein\\luden ein\\ludet ein\\luden ein}\\
\end{tabular}
\begin{tabular}{l}
\parbox[t][][t]{8cm}{}\\
\parbox[t][][t]{8cm}{\normalfont \footnotesize 
jdn (zu etw dat/in etw akk) einladen - 
to invite sb (to sth) - 
ich bin zu meinem Cousin in die USA eingeladen - 
my cousin (who lives) in the USA has invited me to stay with him - 
wir sind morgen eingeladen - 
we've been invited out tomorrow - 
jdn zu etw dat/in etw akk - 
einladen - 
to invite sb for/(out) to sth - 
jdn zum Essen einladen 
}\\
\end{tabular}
}
%===einrichten===
\card{\normalfont \Huge einrichten}{
\begin{tabular}{lll}
\parbox[t][][t]{2.0 cm}{\normalfont \raggedleft ich\\du\\er/sie/es\\wir\\ihr\\sie} &    
\parbox[t][][t]{2cm}{\normalfont richte ein\\richtest ein\\richtet ein\\richten ein\\richtet ein\\richten ein} &
\parbox[t][][t]{2cm}{\normalfont richtete ein\\richtetest ein\\richtete ein\\richteten ein\\richtetet ein\\richteten ein}\\
\end{tabular}
\begin{tabular}{l}
\parbox[t][][t]{8cm}{}\\
\parbox[t][][t]{8cm}{\normalfont \footnotesize 
(jdm) etw (irgendwie) einrichten - 
to furnish sth (somehow) (for sb) - 
die Wohnung war schon fertig eingerichtet - 
the flat was already furnished - 
etw anders einrichten - 
to furnish sth differently - 
etw neu einrichten - 
to refurnish (or refit) sth - 
eine Apotheke/eine Praxis/ein Labor einrichten - 
to fit out sep 
}\\
\end{tabular}
}
%===einschalten===
\card{\normalfont \Huge einschalten}{
\begin{tabular}{lll}
\parbox[t][][t]{2.0 cm}{\normalfont \raggedleft ich\\du\\er/sie/es\\wir\\ihr\\sie} &    
\parbox[t][][t]{2cm}{\normalfont schalte ein\\schaltest ein\\schaltet ein\\schalten ein\\schaltet ein\\schalten ein} &
\parbox[t][][t]{2cm}{\normalfont schaltete ein\\schaltetest ein\\schaltete ein\\schalteten ein\\schaltetet ein\\schalteten ein}\\
\end{tabular}
\begin{tabular}{l}
\parbox[t][][t]{8cm}{}\\
\parbox[t][][t]{8cm}{\normalfont \footnotesize 
etw einschalten - 
to switch (or turn) on sep sth - 
den Computer einschalten - 
to turn the computer on - 
den Fernseher einschalten - 
to put (or switch) - 
(or turn) on sep the TV - 
den ersten Gang einschalten - 
to engage first gear form - 
den ersten Gang einschalten 
}\\
\end{tabular}
}
%===einsetzen===
\card{\normalfont \Huge einsetzen}{
\begin{tabular}{lll}
\parbox[t][][t]{2.0 cm}{\normalfont \raggedleft ich\\du\\er/sie/es\\wir\\ihr\\sie} &    
\parbox[t][][t]{2cm}{\normalfont setze ein\\setzt ein\\setzt ein\\setzen ein\\setzt ein\\setzen ein} &
\parbox[t][][t]{2cm}{\normalfont setzte ein\\setztest ein\\setzte ein\\setzten ein\\setztet ein\\setzten ein}\\
\end{tabular}
\begin{tabular}{l}
\parbox[t][][t]{8cm}{}\\
\parbox[t][][t]{8cm}{\normalfont \footnotesize 
etw (in etw akk) einsetzen - 
to insert sth (in sth) - 
etw (in etw akk) einsetzen - 
to put sth in (sth) - 
Blumen in ein Beet einsetzen - 
to plant flowers in a flowerbed - 
Edelsteine in ein Collier/einen Ring einsetzen - 
to inset (or mount) jewels in a necklace/ring - 
Maschinenteile einsetzen - 
to mount component parts 
}\\
\end{tabular}
}
%===einsteigen===
\card{\normalfont \Huge einsteigen}{
\begin{tabular}{lll}
\parbox[t][][t]{2.0 cm}{\normalfont \raggedleft ich\\du\\er/sie/es\\wir\\ihr\\sie} &    
\parbox[t][][t]{2cm}{\normalfont steige ein\\steigst ein\\steigt ein\\steigen ein\\steigt ein\\steigen ein} &
\parbox[t][][t]{2cm}{\normalfont stieg ein\\stiegst ein\\stieg ein\\stiegen ein\\stiegt ein\\stiegen ein}\\
\end{tabular}
\begin{tabular}{l}
\parbox[t][][t]{8cm}{}\\
\parbox[t][][t]{8cm}{\normalfont \footnotesize 
(in etw akk) einsteigen - 
to get on (sth) - 
in ein Auto/Taxi einsteigen - 
to get in(to) a car/taxi - 
in einen Zug einsteigen - 
to get on (or - 
form board)  a train - 
einsteigen! - 
all aboard! - 
(in etw akk) einsteigen 
}\\
\end{tabular}
}
%===einstellen===
\card{\normalfont \Huge einstellen}{
\begin{tabular}{lll}
\parbox[t][][t]{2.0 cm}{\normalfont \raggedleft ich\\du\\er/sie/es\\wir\\ihr\\sie} &    
\parbox[t][][t]{2cm}{\normalfont stelle ein\\stellst ein\\stellt ein\\stellen ein\\stellt ein\\stellen ein} &
\parbox[t][][t]{2cm}{\normalfont stellte ein\\stelltest ein\\stellte ein\\stellten ein\\stelltet ein\\stellten ein}\\
\end{tabular}
\begin{tabular}{l}
\parbox[t][][t]{8cm}{}\\
\parbox[t][][t]{8cm}{\normalfont \footnotesize 
jdn (als etw) einstellen - 
to employ (or take on) sb (as sth) - 
Arbeitskräfte einstellen - 
to take on employees - 
sie wurde als Redaktionsassistentin eingestellt - 
she was given a job as (an) editorial assistant - 
etw einstellen - 
to stop (or break off) sth - 
eine Suche einstellen - 
to call off (or abandon)  a search 
}\\
\end{tabular}
}
%===einziehen===
\card{\normalfont \Huge einziehen}{
\begin{tabular}{lll}
\parbox[t][][t]{2.0 cm}{\normalfont \raggedleft ich\\du\\er/sie/es\\wir\\ihr\\sie} &    
\parbox[t][][t]{2cm}{\normalfont ziehe ein\\ziehst ein\\zieht ein\\ziehen ein\\zieht ein\\ziehen ein} &
\parbox[t][][t]{2cm}{\normalfont zog ein\\zogst ein\\zog ein\\zogen ein\\zogt ein\\zogen ein}\\
\end{tabular}
\begin{tabular}{l}
\parbox[t][][t]{8cm}{}\\
\parbox[t][][t]{8cm}{\normalfont \footnotesize 
etw einziehen - 
to draw in sth sep - 
zieh den Bauch ein! - 
keep your tummy in! - 
der Hund zog den Schwanz ein - 
the dog put its tail between its legs - 
mit eingezogenem Schwanz a. fig - 
with his/her/its tail between his/her/its legs - 
die Fühler/Krallen einziehen - 
to retract (or draw in) its feelers/claws 
}\\
\end{tabular}
}
%===empfangen===
\card{\normalfont \Huge empfangen}{
\begin{tabular}{lll}
\parbox[t][][t]{2.0 cm}{\normalfont \raggedleft ich\\du\\er/sie/es\\wir\\ihr\\sie} &    
\parbox[t][][t]{2cm}{\normalfont empfange\\empfängst\\empfängt\\empfangen\\empfangt\\empfangen} &
\parbox[t][][t]{2cm}{\normalfont empfing\\empfingst\\empfing\\empfingen\\empfingt\\empfingen}\\
\end{tabular}
\begin{tabular}{l}
\parbox[t][][t]{8cm}{}\\
\parbox[t][][t]{8cm}{\normalfont \footnotesize 
etw empfangen - 
to receive sth - 
etw lässt sich empfangen - 
sth can be received - 
das 4. Programm lässt sich nicht gut empfangen - 
Channel 4 is difficult to receive - 
eine E-Mail empfangen - 
to receive an email - 
jdn empfangen - 
to welcome (or greet) 
}\\
\end{tabular}
}
%===empfehlen===
\card{\normalfont \Huge empfehlen}{
\begin{tabular}{lll}
\parbox[t][][t]{2.0 cm}{\normalfont \raggedleft ich\\du\\er/sie/es\\wir\\ihr\\sie} &    
\parbox[t][][t]{2cm}{\normalfont empfehle\\empfiehlst\\empfiehlt\\empfehlen\\empfehlt\\empfehlen} &
\parbox[t][][t]{2cm}{\normalfont empfahl\\empfahlst\\empfahl\\empfahlen\\empfahlt\\empfahlen}\\
\end{tabular}
\begin{tabular}{l}
\parbox[t][][t]{8cm}{}\\
\parbox[t][][t]{8cm}{\normalfont \footnotesize 
(jdm) etw empfehlen - 
to recommend sth to sb - 
zu empfehlen sein - 
to be recommended - 
dieses Hotel ist zu empfehlen - 
this hotel is (to be) recommended - 
jdm jdn (als etw) empfehlen - 
to recommend sb to sb (as sth) - 
ich empfehle Ihnen diese junge Dame (als neue Mitarbeiterin) - 
I recommend this young lady to you (as a colleague) 
}\\
\end{tabular}
}
%===enthalten===
\card{\normalfont \Huge enthalten}{
\begin{tabular}{lll}
\parbox[t][][t]{2.0 cm}{\normalfont \raggedleft ich\\du\\er/sie/es\\wir\\ihr\\sie} &    
\parbox[t][][t]{2cm}{\normalfont enthalte\\enthältst\\enthält\\enthalten\\enthaltet\\enthalten} &
\parbox[t][][t]{2cm}{\normalfont enthielt\\enthieltest\\enthielt\\enthielten\\enthieltet\\enthielten}\\
\end{tabular}
\begin{tabular}{l}
\parbox[t][][t]{8cm}{}\\
\parbox[t][][t]{8cm}{\normalfont \footnotesize 
etw enthalten - 
to contain sth - 
etw enthalten - 
to include sth - 
in etw dat (mit) enthalten sein - 
to be included in (with) sth - 
sich akk - 
enthalten - 
to abstain - 
sich akk einer S. gen 
}\\
\end{tabular}
}
%===entlassen===
\card{\normalfont \Huge entlassen}{
\begin{tabular}{lll}
\parbox[t][][t]{2.0 cm}{\normalfont \raggedleft ich\\du\\er/sie/es\\wir\\ihr\\sie} &    
\parbox[t][][t]{2cm}{\normalfont entlasse\\entlässt\\entlässt\\entlassen\\entlasst\\entlassen} &
\parbox[t][][t]{2cm}{\normalfont entließ\\entließt\\entließ\\entließen\\entließt\\entließen}\\
\end{tabular}
\begin{tabular}{l}
\parbox[t][][t]{8cm}{}\\
\parbox[t][][t]{8cm}{\normalfont \footnotesize 
jdn entlassen - 
(Stellen abbauen) - 
to make sb redundant - 
jdn entlassen - 
(gehen lassen) - 
to dismiss - 
jdn entlassen - 
to dismiss sb - 
jdn entlassen - 
MED, MILIT 
}\\
\end{tabular}
}
%===entscheiden===
\card{\normalfont \Huge entscheiden}{
\begin{tabular}{lll}
\parbox[t][][t]{2.0 cm}{\normalfont \raggedleft ich\\du\\er/sie/es\\wir\\ihr\\sie} &    
\parbox[t][][t]{2cm}{\normalfont entscheide\\entscheidest\\entscheidet\\entscheiden\\entscheidet\\entscheiden} &
\parbox[t][][t]{2cm}{\normalfont entschied\\entschiedest \\entschied\\entschieden\\entschiedet\\entschieden}\\
\end{tabular}
\begin{tabular}{l}
\parbox[t][][t]{8cm}{}\\
\parbox[t][][t]{8cm}{\normalfont \footnotesize 
entscheiden, dass/ob/was/wann/wie ... - 
to decide that/whether/what/when/how ... - 
entscheiden, dass/ob/was/wann/wie ... (gerichtlich) - 
to rule that/whether/what/when/how ... - 
etw entscheiden - 
to settle sth - 
etw (für jdn (o. zugunsten einer Person)) entscheiden - 
to settle sth (in sb's favour (or - 
Am -or) ) - 
entschieden sein 
}\\
\end{tabular}
}
%===entschuldigen===
\card{\normalfont \Huge entschuldigen}{
\begin{tabular}{lll}
\parbox[t][][t]{2.0 cm}{\normalfont \raggedleft ich\\du\\er/sie/es\\wir\\ihr\\sie} &    
\parbox[t][][t]{2cm}{\normalfont entschuldige\\entschuldigst\\entschuldigt\\entschuldigen\\entschuldigt\\entschuldigen} &
\parbox[t][][t]{2cm}{\normalfont entschuldigte\\entschuldigtest\\entschuldigte\\entschuldigten\\entschuldigtet\\entschuldigten}\\
\end{tabular}
\begin{tabular}{l}
\parbox[t][][t]{8cm}{}\\
\parbox[t][][t]{8cm}{\normalfont \footnotesize 
entschuldigen Sie, können Sie mir sagen, wie ich zum Bahnhof komme? - 
excuse me, could you tell me how to get to the station? - 
entschuldigen Sie bitte, was sagten Sie da gerade? - 
sorry, what were you just saying there? - 
sich akk (bei jdm) (für etw akk/wegen einer S. gen) entschuldigen - 
to apologize (to sb) (for sth) - 
sich akk (bei jdm) (für etw akk/wegen einer S. gen) entschuldigen - 
to say sorry (to sb) (for sth) - 
ich muss mich bei Ihnen wegen meines Zuspätkommens entschuldigen - 
I'm terribly sorry I'm so late 
}\\
\end{tabular}
}
%===entsprechen===
\card{\normalfont \Huge entsprechen}{
\begin{tabular}{lll}
\parbox[t][][t]{2.0 cm}{\normalfont \raggedleft ich\\du\\er/sie/es\\wir\\ihr\\sie} &    
\parbox[t][][t]{2cm}{\normalfont entspreche\\entsprichst\\entspricht\\entsprechen\\entsprecht\\entsprechen} &
\parbox[t][][t]{2cm}{\normalfont entsprach\\entsprachst\\entsprach\\entsprachen\\entspracht\\entsprachen}\\
\end{tabular}
\begin{tabular}{l}
\parbox[t][][t]{8cm}{}\\
\parbox[t][][t]{8cm}{\normalfont \footnotesize 
etw dat - 
entsprechen - 
to correspond to (or tally with) sth - 
der Artikel in der Zeitung entsprach nicht ganz den Tatsachen - 
the article in the newspaper wasn't quite in accordance with the facts - 
etw dat - 
entsprechen - 
to fulfil (or - 
Am usu -ll) - 
(or meet) 
}\\
\end{tabular}
}
%===entstehen===
\card{\normalfont \Huge entstehen}{
\begin{tabular}{lll}
\parbox[t][][t]{2.0 cm}{\normalfont \raggedleft ich\\du\\er/sie/es\\wir\\ihr\\sie} &    
\parbox[t][][t]{2cm}{\normalfont entstehe\\entstehst\\entsteht\\entstehen\\entsteht\\entstehen} &
\parbox[t][][t]{2cm}{\normalfont entstand\\entstandest\\entstand\\entstanden\\entstandet\\entstanden}\\
\end{tabular}
\begin{tabular}{l}
\parbox[t][][t]{8cm}{}\\
\parbox[t][][t]{8cm}{\normalfont \footnotesize 
(aus etw dat/durch etw akk) entstehen - 
to come into being (from sth) - 
(aus etw dat/durch etw akk) entstehen - 
to be created (from sth) - 
aus diesem kleinen Pflänzchen wird ein großer Baum entstehen - 
a great tree will grow from this sapling - 
das Haus war in nur 8 Monaten entstanden - 
the house was built in only eight months - 
im Entstehen begriffen sein geh - 
to be in the process of development (or emerging) 
}\\
\end{tabular}
}
%===enttäuschen===
\card{\normalfont \Huge enttäuschen}{
\begin{tabular}{lll}
\parbox[t][][t]{2.0 cm}{\normalfont \raggedleft ich\\du\\er/sie/es\\wir\\ihr\\sie} &    
\parbox[t][][t]{2cm}{\normalfont enttäusche\\enttäuschst\\enttäuscht\\enttäuschen\\enttäuscht\\enttäuschen} &
\parbox[t][][t]{2cm}{\normalfont enttäuschte\\enttäuschtest\\enttäuschte\\enttäuschten\\enttäuschtet\\enttäuschten}\\
\end{tabular}
\begin{tabular}{l}
\parbox[t][][t]{8cm}{}\\
\parbox[t][][t]{8cm}{\normalfont \footnotesize 
jdn enttäuschen - 
to disappoint sb - 
jds Hoffnungen enttäuschen - 
to dash sb's hopes - 
jds Vertrauen enttäuschen - 
to betray sb's trust - 
die Mannschaft hat sehr enttäuscht - 
the team was very disappointing 
}\\
\end{tabular}
}
%===entwickeln===
\card{\normalfont \Huge entwickeln}{
\begin{tabular}{lll}
\parbox[t][][t]{2.0 cm}{\normalfont \raggedleft ich\\du\\er/sie/es\\wir\\ihr\\sie} &    
\parbox[t][][t]{2cm}{\normalfont entwickle \\entwickelst\\entwickelt\\entwickeln\\entwickelt\\entwickeln} &
\parbox[t][][t]{2cm}{\normalfont entwickelte\\entwickeltest\\entwickelte\\entwickelten\\entwickeltet\\entwickelten}\\
\end{tabular}
\begin{tabular}{l}
\parbox[t][][t]{8cm}{}\\
\parbox[t][][t]{8cm}{\normalfont \footnotesize 
etw entwickeln - 
to develop sth - 
etw entwickeln - 
to develop sth - 
einen Plan entwickeln - 
to develop (or devise)  a plan - 
einen Film entwickeln - 
to develop a film - 
etw entwickeln - 
to produce sth 
}\\
\end{tabular}
}
%===erfahren===
\card{\normalfont \Huge erfahren}{
\begin{tabular}{lll}
\parbox[t][][t]{2.0 cm}{\normalfont \raggedleft ich\\du\\er/sie/es\\wir\\ihr\\sie} &    
\parbox[t][][t]{2cm}{\normalfont erfahre\\erfährst\\erfährt\\erfahren\\erfahrt\\erfahren} &
\parbox[t][][t]{2cm}{\normalfont erfuhr\\erfuhrst\\erfuhr\\erfuhren\\erfuhrt\\erfuhren}\\
\end{tabular}
\begin{tabular}{l}
\parbox[t][][t]{8cm}{}\\
\parbox[t][][t]{8cm}{\normalfont \footnotesize 
etw (von jdm) (über jdn/etw) erfahren - 
Nachricht, Neuigkeit etc. - 
to hear (or find out) sth (from sb) (about sb/sth) - 
etw erfahren - 
to learn of sth - 
darf man Ihre Absichten erfahren? - 
might we enquire as to your intentions? - 
etw erfahren - 
to experience sth - 
in seinem Leben hat er viel Liebe erfahren 
}\\
\end{tabular}
}
%===erfinden===
\card{\normalfont \Huge erfinden}{
\begin{tabular}{lll}
\parbox[t][][t]{2.0 cm}{\normalfont \raggedleft ich\\du\\er/sie/es\\wir\\ihr\\sie} &    
\parbox[t][][t]{2cm}{\normalfont erfinde\\erfindest\\erfindet\\erfinden\\erfindet\\erfinden} &
\parbox[t][][t]{2cm}{\normalfont erfand\\erfandest \\erfand\\erfanden\\erfandet\\erfanden}\\
\end{tabular}
\begin{tabular}{l}
\parbox[t][][t]{8cm}{}\\
\parbox[t][][t]{8cm}{\normalfont \footnotesize 
etw erfinden - 
to invent sth - 
etw erfinden - 
to invent (or - 
sep make up) sth - 
frei erfunden sein - 
to be completely fictitious 
}\\
\end{tabular}
}
%===erfüllen===
\card{\normalfont \Huge erfüllen}{
\begin{tabular}{lll}
\parbox[t][][t]{2.0 cm}{\normalfont \raggedleft ich\\du\\er/sie/es\\wir\\ihr\\sie} &    
\parbox[t][][t]{2cm}{\normalfont erfülle\\erfüllst\\erfüllt\\erfüllen\\erfüllt\\erfüllen} &
\parbox[t][][t]{2cm}{\normalfont erfüllte\\erfülltest\\erfüllte\\erfüllten\\erfülltet\\erfüllten}\\
\end{tabular}
\begin{tabular}{l}
\parbox[t][][t]{8cm}{}\\
\parbox[t][][t]{8cm}{\normalfont \footnotesize 
etw erfüllen - 
to fulfil (or - 
Am usu -ll) - 
(or carry out) sth - 
welche Funktion erfüllt sie im Betrieb? - 
what is her function in the company? - 
mein altes Auto erfüllt seinen Zweck - 
my old car serves its purpose - 
jdn erfüllen - 
to come over sb 
}\\
\end{tabular}
}
%===erhalten===
\card{\normalfont \Huge erhalten}{
\begin{tabular}{lll}
\parbox[t][][t]{2.0 cm}{\normalfont \raggedleft ich\\du\\er/sie/es\\wir\\ihr\\sie} &    
\parbox[t][][t]{2cm}{\normalfont erhalte\\erhältst\\erhält\\erhalten\\erhaltet\\erhalten} &
\parbox[t][][t]{2cm}{\normalfont erhielt\\erhieltest\\erhielt\\erhielten\\erhieltet\\erhielten}\\
\end{tabular}
\begin{tabular}{l}
\parbox[t][][t]{8cm}{}\\
\parbox[t][][t]{8cm}{\normalfont \footnotesize 
etw (von jdm) erhalten - 
to receive sth (from sb) - 
etw (von jdm) erhalten - 
Antwort, Brief, Geschenk - 
receive - 
etw (von jdm) erhalten - 
Befehl - 
to be issued with (or receive) - 
den Auftrag erhalten, etw zu tun - 
to be given (or assigned) the task of doing sth 
}\\
\end{tabular}
}
%===erhöhen===
\card{\normalfont \Huge erhöhen}{
\begin{tabular}{lll}
\parbox[t][][t]{2.0 cm}{\normalfont \raggedleft ich\\du\\er/sie/es\\wir\\ihr\\sie} &    
\parbox[t][][t]{2cm}{\normalfont erhöhe\\erhöhst\\erhöht\\erhöhen\\erhöht\\erhöhen} &
\parbox[t][][t]{2cm}{\normalfont erhöhte\\erhöhtest\\erhöhte\\erhöhten\\erhöhtet\\erhöhten}\\
\end{tabular}
\begin{tabular}{l}
\parbox[t][][t]{8cm}{}\\
\parbox[t][][t]{8cm}{\normalfont \footnotesize 
etw (um etw akk) erhöhen - 
to raise sth (by sth) - 
die Mauern wurden um zwei Meter erhöht - 
the walls were raised by two metres - 
etw (auf etw akk/um etw akk) erhöhen - 
to increase sth (to sth/by sth) - 
etw erhöhen - 
to heighten sth - 
etw erhöhen - 
to sharpen sth 
}\\
\end{tabular}
}
%===erinnern===
\card{\normalfont \Huge erinnern}{
\begin{tabular}{lll}
\parbox[t][][t]{2.0 cm}{\normalfont \raggedleft ich\\du\\er/sie/es\\wir\\ihr\\sie} &    
\parbox[t][][t]{2cm}{\normalfont erinnere\\erinnerst\\erinnert\\erinnern\\erinnert\\erinnern} &
\parbox[t][][t]{2cm}{\normalfont erinnerte\\erinnertest\\erinnerte\\erinnerten\\erinnertet\\erinnerten}\\
\end{tabular}
\begin{tabular}{l}
\parbox[t][][t]{8cm}{}\\
\parbox[t][][t]{8cm}{\normalfont \footnotesize 
jdn an etw akk - 
erinnern - 
to remind sb about sth - 
jdn daran erinnern, etw zu tun - 
to remind sb to do sth - 
jdn an jdn/etw erinnern - 
to remind sb of sb/sth - 
sich akk an jdn/etw erinnern - 
to remember sb/sth - 
wenn ich mich recht erinnere, ... 
}\\
\end{tabular}
}
%===erkälten===
\card{\normalfont \Huge erkälten}{
\begin{tabular}{lll}
\parbox[t][][t]{2.0 cm}{\normalfont \raggedleft ich\\du\\er/sie/es\\wir\\ihr\\sie} &    
\parbox[t][][t]{2cm}{\normalfont erkälte\\erkältest\\erkältet\\erkälten\\erkältet\\erkälten} &
\parbox[t][][t]{2cm}{\normalfont erkältete\\erkältetest\\erkältete\\erkälteten\\erkältetet\\erkälteten}\\
\end{tabular}
\begin{tabular}{l}
\parbox[t][][t]{8cm}{}\\
\parbox[t][][t]{8cm}{\normalfont \footnotesize 
sich akk - 
erkälten - 
to catch a cold - 
sich dat etw erkälten - 
to catch a chill in one's sth - 
erkalten - 
to become cold - 
erkalten - 
to cool (down) - 
erkalten 
}\\
\end{tabular}
}
%===erkennen===
\card{\normalfont \Huge erkennen}{
\begin{tabular}{lll}
\parbox[t][][t]{2.0 cm}{\normalfont \raggedleft ich\\du\\er/sie/es\\wir\\ihr\\sie} &    
\parbox[t][][t]{2cm}{\normalfont erkenne\\erkennst\\erkennt\\erkennen\\erkennt\\erkennen} &
\parbox[t][][t]{2cm}{\normalfont erkannte\\erkanntest\\erkannte\\erkannten\\erkanntet\\erkannten}\\
\end{tabular}
\begin{tabular}{l}
\parbox[t][][t]{8cm}{}\\
\parbox[t][][t]{8cm}{\normalfont \footnotesize 
jdn/etw erkennen - 
to discern sb/sth - 
er ist der Täter, ich habe ihn gleich erkannt! - 
he's the culprit, I recognized him straight away - 
etw erkennen lassen - 
to show sth - 
jdm zu erkennen geben, dass ... - 
to make it clear to sb that ... - 
jdn/etw (an etw dat) erkennen - 
to recognize sb/sth (by sth) 
}\\
\end{tabular}
}
%===erklären===
\card{\normalfont \Huge erklären}{
\begin{tabular}{lll}
\parbox[t][][t]{2.0 cm}{\normalfont \raggedleft ich\\du\\er/sie/es\\wir\\ihr\\sie} &    
\parbox[t][][t]{2cm}{\normalfont erkläre\\erklärst\\erklärt\\erklären\\erklärt\\erklären} &
\parbox[t][][t]{2cm}{\normalfont erklärte\\erklärtest\\erklärte\\erklärten\\erklärtet\\erklärten}\\
\end{tabular}
\begin{tabular}{l}
\parbox[t][][t]{8cm}{}\\
\parbox[t][][t]{8cm}{\normalfont \footnotesize 
(jdm) etw (an etw dat) erklären - 
to explain sth (to sb) (using sth) - 
jdm erklären, dass/wieso ... - 
to explain to sb that/why ... - 
(jdm) etw erklären - 
to interpret sth (to sb) - 
etw erklären - 
to explain sth - 
etw erklären - 
to announce sth 
}\\
\end{tabular}
}
%===erlauben===
\card{\normalfont \Huge erlauben}{
\begin{tabular}{lll}
\parbox[t][][t]{2.0 cm}{\normalfont \raggedleft ich\\du\\er/sie/es\\wir\\ihr\\sie} &    
\parbox[t][][t]{2cm}{\normalfont erlaube\\erlaubst\\erlaubt\\erlauben\\erlaubt\\erlauben} &
\parbox[t][][t]{2cm}{\normalfont erlaubte\\erlaubtest\\erlaubte\\erlaubten\\erlaubtet\\erlaubten}\\
\end{tabular}
\begin{tabular}{l}
\parbox[t][][t]{8cm}{}\\
\parbox[t][][t]{8cm}{\normalfont \footnotesize 
jdm etw erlauben - 
to allow (or permit) sb to do sth - 
du erlaubst deinem Kind zu viel - 
you let your child get away with too much - 
jdm erlauben, etw zu tun - 
to allow (or permit) sb to do sth - 
etw ist (nicht) erlaubt - 
sth is (not) allowed (or permitted) - 
es ist (nicht) erlaubt, etw irgendwo zu tun - 
it is (not) permissible to do sth somewhere 
}\\
\end{tabular}
}
%===erleben===
\card{\normalfont \Huge erleben}{
\begin{tabular}{lll}
\parbox[t][][t]{2.0 cm}{\normalfont \raggedleft ich\\du\\er/sie/es\\wir\\ihr\\sie} &    
\parbox[t][][t]{2cm}{\normalfont erlebe\\erlebst\\erlebt\\erleben\\erlebt\\erleben} &
\parbox[t][][t]{2cm}{\normalfont erlebte\\erlebtest\\erlebte\\erlebten\\erlebtet\\erlebten}\\
\end{tabular}
\begin{tabular}{l}
\parbox[t][][t]{8cm}{}\\
\parbox[t][][t]{8cm}{\normalfont \footnotesize 
etw erleben - 
to live to see sth - 
dass ich das (noch) erleben muss! - 
couldn't I have been spared that?! - 
etw erleben - 
to experience sth - 
wunderschöne Tage/einen wunderschönen Urlaub irgendwo erleben - 
to have a wonderful time/holiday somewhere - 
was hast du denn alles in Dänemark erlebt? - 
what did you do/see in Denmark? 
}\\
\end{tabular}
}
%===erledigen===
\card{\normalfont \Huge erledigen}{
\begin{tabular}{lll}
\parbox[t][][t]{2.0 cm}{\normalfont \raggedleft ich\\du\\er/sie/es\\wir\\ihr\\sie} &    
\parbox[t][][t]{2cm}{\normalfont erledige\\erledigst\\erledigt\\erledigen\\erledigt\\erledigen} &
\parbox[t][][t]{2cm}{\normalfont erledigte\\erledigtest\\erledigte\\erledigten\\erledigtet\\erledigten}\\
\end{tabular}
\begin{tabular}{l}
\parbox[t][][t]{8cm}{}\\
\parbox[t][][t]{8cm}{\normalfont \footnotesize 
etw erledigen - 
to carry out sth sep - 
Besorgungen erledigen - 
to do some (or the) shopping - 
Formalitäten erledigen - 
to complete formalities - 
wird erledigt! fam - 
I'll/we'll etc. get on (or - 
Brit on to) it (right away)! - 
erledigt 
}\\
\end{tabular}
}
%===erlöschen===
\card{\normalfont \Huge erlöschen}{
\begin{tabular}{lll}
\parbox[t][][t]{2.0 cm}{\normalfont \raggedleft ich\\du\\er/sie/es\\wir\\ihr\\sie} &    
\parbox[t][][t]{2cm}{\normalfont erlösche\\erlischst\\erlischt\\erlöschen\\erlöscht\\erlöschen} &
\parbox[t][][t]{2cm}{\normalfont erlosch\\erloschest\\erlosch\\erloschen\\erloscht\\erloschen}\\
\end{tabular}
\begin{tabular}{l}
\parbox[t][][t]{8cm}{}\\
\parbox[t][][t]{8cm}{\normalfont \footnotesize 
erlöschen - 
to stop burning - 
erlöschen - 
to go out - 
dieser Vulkan ist vor 100 Jahren erloschen - 
the volcano became dormant 100 years ago - 
erlöschen - 
to fizzle out - 
erlöschen - 
to expire 
}\\
\end{tabular}
}
%===eröffnen===
\card{\normalfont \Huge eröffnen}{
\begin{tabular}{lll}
\parbox[t][][t]{2.0 cm}{\normalfont \raggedleft ich\\du\\er/sie/es\\wir\\ihr\\sie} &    
\parbox[t][][t]{2cm}{\normalfont eröffne\\eröffnest\\eröffnet\\eröffnen\\eröffnet\\eröffnen} &
\parbox[t][][t]{2cm}{\normalfont eröffnete\\eröffnetest\\eröffnete\\eröffneten\\eröffnetet\\eröffneten}\\
\end{tabular}
\begin{tabular}{l}
\parbox[t][][t]{8cm}{}\\
\parbox[t][][t]{8cm}{\normalfont \footnotesize 
etw eröffnen - 
to open sth - 
etw eröffnen - 
to open sth - 
etw eröffnen - 
to institute sth - 
etw eröffnen - 
to open sth - 
etw für eröffnet erklären geh - 
to declare sth open form 
}\\
\end{tabular}
}
%===erreichen===
\card{\normalfont \Huge erreichen}{
\begin{tabular}{lll}
\parbox[t][][t]{2.0 cm}{\normalfont \raggedleft ich\\du\\er/sie/es\\wir\\ihr\\sie} &    
\parbox[t][][t]{2cm}{\normalfont erreiche\\erreichst\\erreicht\\erreichen\\erreicht\\erreichen} &
\parbox[t][][t]{2cm}{\normalfont erreichte\\erreichtest\\erreichte\\erreichten\\erreichtet\\erreichten}\\
\end{tabular}
\begin{tabular}{l}
\parbox[t][][t]{8cm}{}\\
\parbox[t][][t]{8cm}{\normalfont \footnotesize 
etw erreichen - 
to catch sth - 
etw erreichen - 
to get to sth - 
jdn erreichen - 
to reach sb - 
jdn erreichen - 
to contact sb - 
jdn erreichen - 
to get hold of sb fam 
}\\
\end{tabular}
}
%===erscheinen===
\card{\normalfont \Huge erscheinen}{
\begin{tabular}{lll}
\parbox[t][][t]{2.0 cm}{\normalfont \raggedleft ich\\du\\er/sie/es\\wir\\ihr\\sie} &    
\parbox[t][][t]{2cm}{\normalfont erscheine\\erscheinst\\erscheint\\erscheinen\\erscheint\\erscheinen} &
\parbox[t][][t]{2cm}{\normalfont erschien\\erschienst\\erschien\\erschienen\\erschient\\erschienen}\\
\end{tabular}
\begin{tabular}{l}
\parbox[t][][t]{8cm}{}\\
\parbox[t][][t]{8cm}{\normalfont \footnotesize 
erscheinen - 
to appear - 
du sollst sofort beim Chef erscheinen! - 
the boss wants to see you straight away! - 
sie war des Öfteren unpünktlich erschienen - 
she had often arrived late - 
erscheinen - 
to be able to be seen - 
am sechsten Tag erschien endlich Land am Horizont - 
on the sixth day we/they etc. finally sighted land 
}\\
\end{tabular}
}
%===erschrecken===
\card{\normalfont \Huge erschrecken}{
\begin{tabular}{lll}
\parbox[t][][t]{2.0 cm}{\normalfont \raggedleft ich\\du\\er/sie/es\\wir\\ihr\\sie} &    
\parbox[t][][t]{2cm}{\normalfont erschrecke\\erschrickst\\erschrickt\\erschrecken\\erschreckt\\erschrecken} &
\parbox[t][][t]{2cm}{\normalfont erschrak\\erschrakst\\erschrak\\erschraken\\erschrakt\\erschraken}\\
\end{tabular}
\begin{tabular}{l}
\parbox[t][][t]{8cm}{}\\
\parbox[t][][t]{8cm}{\normalfont \footnotesize 
jdn erschrecken - 
to give sb a fright - 
jdn erschrecken - 
to alarm sb - 
jdn erschrecken - 
to shock sb - 
(vor jdm/etw) erschrecken - 
to get a fright (from sb/sth) - 
erschrecken Sie nicht, ich bin's nur! - 
don't get a fright, it's only me! 
}\\
\end{tabular}
}
%===erwähnen===
\card{\normalfont \Huge erwähnen}{
\begin{tabular}{lll}
\parbox[t][][t]{2.0 cm}{\normalfont \raggedleft ich\\du\\er/sie/es\\wir\\ihr\\sie} &    
\parbox[t][][t]{2cm}{\normalfont erwähne\\erwähnst\\erwähnt\\erwähnen\\erwähnt\\erwähnen} &
\parbox[t][][t]{2cm}{\normalfont erwähnte\\erwähntest\\erwähnte\\erwähnten\\erwähntet\\erwähnten}\\
\end{tabular}
\begin{tabular}{l}
\parbox[t][][t]{8cm}{}\\
\parbox[t][][t]{8cm}{\normalfont \footnotesize 
jdn/etw erwähnen - 
to mention sb/sth - 
(jdm gegenüber) erwähnen, dass ... - 
to mention (to sb) that ... 
}\\
\end{tabular}
}
%===erwarten===
\card{\normalfont \Huge erwarten}{
\begin{tabular}{lll}
\parbox[t][][t]{2.0 cm}{\normalfont \raggedleft ich\\du\\er/sie/es\\wir\\ihr\\sie} &    
\parbox[t][][t]{2cm}{\normalfont erwarte\\erwartest\\erwartet\\erwarten\\erwartet\\erwarten} &
\parbox[t][][t]{2cm}{\normalfont erwartete\\erwartetest\\erwartete\\erwarteten\\erwartetet\\erwarteten}\\
\end{tabular}
\begin{tabular}{l}
\parbox[t][][t]{8cm}{}\\
\parbox[t][][t]{8cm}{\normalfont \footnotesize 
jdn/etw erwarten - 
to expect sb/sth - 
etw erwarten - 
to wait for (or - 
form to await) sth - 
etw von jdm erwarten - 
to expect sth from sb - 
von jdm erwarten, dass ... - 
to expect sb to do sth - 
von jdm zu erwarten sein 
}\\
\end{tabular}
}
%===erzählen===
\card{\normalfont \Huge erzählen}{
\begin{tabular}{lll}
\parbox[t][][t]{2.0 cm}{\normalfont \raggedleft ich\\du\\er/sie/es\\wir\\ihr\\sie} &    
\parbox[t][][t]{2cm}{\normalfont erzähle\\erzählst\\erzählt\\erzählen\\erzählt\\erzählen} &
\parbox[t][][t]{2cm}{\normalfont erzählte\\erzähltest\\erzählte\\erzählten\\erzähltet\\erzählten}\\
\end{tabular}
\begin{tabular}{l}
\parbox[t][][t]{8cm}{}\\
\parbox[t][][t]{8cm}{\normalfont \footnotesize 
erzählen - 
explain - 
erzählen - 
tell - 
erzählen - 
relate - 
(jdm) etw erzählen - 
to tell (sb sth) - 
(jdm) seine Erlebnisse erzählen - 
to tell (sb) about one's experiences 
}\\
\end{tabular}
}
%===essen===
\card{\normalfont \Huge essen}{
\begin{tabular}{lll}
\parbox[t][][t]{2.0 cm}{\normalfont \raggedleft ich\\du\\er/sie/es\\wir\\ihr\\sie} &    
\parbox[t][][t]{2cm}{\normalfont esse\\isst\\isst\\essen\\esst\\essen} &
\parbox[t][][t]{2cm}{\normalfont aß\\aßt\\aß\\aßen\\aßt\\aßen}\\
\end{tabular}
\begin{tabular}{l}
\parbox[t][][t]{8cm}{}\\
\parbox[t][][t]{8cm}{\normalfont \footnotesize 
etw essen - 
to eat sth - 
essen Sie gern Äpfel? - 
do you like apples? - 
ich esse kein Fleisch - 
I don't eat meat - 
ich esse am liebsten Schokoladeneis - 
I like chocolate ice cream most (or best) of all - 
etw zum Nachtisch essen - 
to have sth for dessert 
}\\
\end{tabular}
}
%===existieren===
\card{\normalfont \Huge existieren}{
\begin{tabular}{lll}
\parbox[t][][t]{2.0 cm}{\normalfont \raggedleft ich\\du\\er/sie/es\\wir\\ihr\\sie} &    
\parbox[t][][t]{2cm}{\normalfont existiere\\existierst\\existiert\\existieren\\existiert\\existieren} &
\parbox[t][][t]{2cm}{\normalfont existierte\\existiertest\\existierte\\existierten\\existiertet\\existierten}\\
\end{tabular}
\begin{tabular}{l}
\parbox[t][][t]{8cm}{}\\
\parbox[t][][t]{8cm}{\normalfont \footnotesize 
existieren - 
to exist - 
existieren - 
to be in existence - 
(von etw dat) existieren - 
to live (on sth) - 
(von etw dat) existieren - 
to keep alive (on sth) iron 
}\\
\end{tabular}
}
%===fahren===
\card{\normalfont \Huge fahren}{
\begin{tabular}{lll}
\parbox[t][][t]{2.0 cm}{\normalfont \raggedleft ich\\du\\er/sie/es\\wir\\ihr\\sie} &    
\parbox[t][][t]{2cm}{\normalfont fahre\\fährst\\fährt\\fahren\\fahrt\\fahren} &
\parbox[t][][t]{2cm}{\normalfont fuhr\\fuhrst\\fuhr\\fuhren\\fuhrt\\fuhren}\\
\end{tabular}
\begin{tabular}{l}
\parbox[t][][t]{8cm}{}\\
\parbox[t][][t]{8cm}{\normalfont \footnotesize 
fahren - 
to go - 
mit dem Bus/der Straßenbahn/dem Taxi/dem Zug fahren - 
to go by bus/tram/taxi/train - 
erster/zweiter Klasse fahren - 
to travel (or go) first/second class - 
wie fährt man von hier am besten zum Bahnhof? - 
what's the best way to the station from here? - 
fahren Sie nach Heidelberg/zum Flughafen? - 
are you going to Heidelberg/to the airport? 
}\\
\end{tabular}
}
%===fallen===
\card{\normalfont \Huge fallen}{
\begin{tabular}{lll}
\parbox[t][][t]{2.0 cm}{\normalfont \raggedleft ich\\du\\er/sie/es\\wir\\ihr\\sie} &    
\parbox[t][][t]{2cm}{\normalfont falle\\fällst\\fällt\\fallen\\fallt\\fallen} &
\parbox[t][][t]{2cm}{\normalfont fiel\\fielst\\fiel\\fielen\\fielt\\fielen}\\
\end{tabular}
\begin{tabular}{l}
\parbox[t][][t]{8cm}{}\\
\parbox[t][][t]{8cm}{\normalfont \footnotesize 
fallen apple - 
abgefallener Apfel - 
Add to my favourites Preselect for export to vocabulary trainer View selected vocabulary - 
fallen arches MED - 
Senkfüße pl - 
Add to my favourites Preselect for export to vocabulary trainer View selected vocabulary - 
fallen leaf - 
heruntergefallenes Blatt - 
Add to my favourites Preselect for export to vocabulary trainer View selected vocabulary - 
fallen leaves 
}\\
\end{tabular}
}
%===fangen===
\card{\normalfont \Huge fangen}{
\begin{tabular}{lll}
\parbox[t][][t]{2.0 cm}{\normalfont \raggedleft ich\\du\\er/sie/es\\wir\\ihr\\sie} &    
\parbox[t][][t]{2cm}{\normalfont fange\\fängst\\fängt\\fangen\\fangt\\fangen} &
\parbox[t][][t]{2cm}{\normalfont fing\\fingst\\fing\\fingen\\fingt\\fingen}\\
\end{tabular}
\begin{tabular}{l}
\parbox[t][][t]{8cm}{}\\
\parbox[t][][t]{8cm}{\normalfont \footnotesize 
jdn fangen - 
to catch (or apprehend) sb - 
einen Dieb fangen - 
to catch a thief - 
etw fangen - 
to catch sth - 
etw fangen - 
to catch sth - 
fangen - 
to catch 
}\\
\end{tabular}
}
%===fassen===
\card{\normalfont \Huge fassen}{
\begin{tabular}{lll}
\parbox[t][][t]{2.0 cm}{\normalfont \raggedleft ich\\du\\er/sie/es\\wir\\ihr\\sie} &    
\parbox[t][][t]{2cm}{\normalfont fasse\\fasst\\fasst\\fassen\\fasst\\fassen} &
\parbox[t][][t]{2cm}{\normalfont fasste\\fasstest\\fasste\\fassten\\fasstet\\fassten}\\
\end{tabular}
\begin{tabular}{l}
\parbox[t][][t]{8cm}{}\\
\parbox[t][][t]{8cm}{\normalfont \footnotesize 
etw fassen - 
to grasp sth - 
jds Hand fassen - 
to take sb's hand - 
jdn an/bei etw dat - 
fassen - 
to seize sb by sth - 
jdn am Arm fassen - 
to seize sb's arm (or sb by the arm) - 
jdn bei der Hand fassen 
}\\
\end{tabular}
}
%===fehlen===
\card{\normalfont \Huge fehlen}{
\begin{tabular}{lll}
\parbox[t][][t]{2.0 cm}{\normalfont \raggedleft ich\\du\\er/sie/es\\wir\\ihr\\sie} &    
\parbox[t][][t]{2cm}{\normalfont fehle\\fehlst\\fehlt\\fehlen\\fehlt\\fehlen} &
\parbox[t][][t]{2cm}{\normalfont fehlte\\fehltest\\fehlte\\fehlten\\fehltet\\fehlten}\\
\end{tabular}
\begin{tabular}{l}
\parbox[t][][t]{8cm}{}\\
\parbox[t][][t]{8cm}{\normalfont \footnotesize 
fehlen - 
to be missing - 
besondere Kennzeichen fehlen - 
there are no distinguishing marks - 
du kannst ihm den fehlenden Vater nicht ersetzen - 
you can't replace his father - 
und wieder ein Zitat deiner Mutter, das durfte ja nicht fehlen! iron - 
you couldn't leave that out, you had to quote your mother! - 
etw fehlt jdm - 
sb is lacking sth 
}\\
\end{tabular}
}
%===feiern===
\card{\normalfont \Huge feiern}{
\begin{tabular}{lll}
\parbox[t][][t]{2.0 cm}{\normalfont \raggedleft ich\\du\\er/sie/es\\wir\\ihr\\sie} &    
\parbox[t][][t]{2cm}{\normalfont feiere\\feierst\\feiert\\feiern\\feiert\\feiern} &
\parbox[t][][t]{2cm}{\normalfont feierte\\feiertest\\feierte\\feierten\\feiertet\\feierten}\\
\end{tabular}
\begin{tabular}{l}
\parbox[t][][t]{8cm}{}\\
\parbox[t][][t]{8cm}{\normalfont \footnotesize 
etw feiern - 
to celebrate sth - 
seinen Geburtstag feiern - 
to celebrate one's birthday - 
eine Party feiern - 
to have a party - 
jdn feiern - 
to acclaim sb - 
feiern - 
to celebrate 
}\\
\end{tabular}
}
%===feixen===
\card{\normalfont \Huge feixen}{
\begin{tabular}{lll}
\parbox[t][][t]{2.0 cm}{\normalfont \raggedleft ich\\du\\er/sie/es\\wir\\ihr\\sie} &    
\parbox[t][][t]{2cm}{\normalfont feixe\\feixt\\feixt\\feixen\\feixt\\feixen} &
\parbox[t][][t]{2cm}{\normalfont feixte\\feixtest\\feixte\\feixten\\feixtet\\feixten}\\
\end{tabular}
\begin{tabular}{l}
\parbox[t][][t]{8cm}{}\\
\parbox[t][][t]{8cm}{\normalfont \footnotesize 
feixen - 
to smirk 
}\\
\end{tabular}
}
%===festhalten===
\card{\normalfont \Huge festhalten}{
\begin{tabular}{lll}
\parbox[t][][t]{2.0 cm}{\normalfont \raggedleft ich\\du\\er/sie/es\\wir\\ihr\\sie} &    
\parbox[t][][t]{2cm}{\normalfont halte fest\\hältst fest\\hält fest\\halten fest\\haltet fest\\halten fest} &
\parbox[t][][t]{2cm}{\normalfont hielt fest\\hieltest fest\\hielt fest\\hielten fest\\hieltet fest\\hielten fest}\\
\end{tabular}
\begin{tabular}{l}
\parbox[t][][t]{8cm}{}\\
\parbox[t][][t]{8cm}{\normalfont \footnotesize 
jdn (an etw dat) festhalten - 
to grab (or seize) sb (by sth) - 
er hielt sie am Ärmel fest - 
he grabbed her by the sleeve - 
jdn/etw festhalten - 
to hold sb/sth tight (or tightly) - 
jdn festhalten - 
to detain (or hold) sb - 
festhalten, dass ... - 
to record the fact that ... 
}\\
\end{tabular}
}
%===finden===
\card{\normalfont \Huge finden}{
\begin{tabular}{lll}
\parbox[t][][t]{2.0 cm}{\normalfont \raggedleft ich\\du\\er/sie/es\\wir\\ihr\\sie} &    
\parbox[t][][t]{2cm}{\normalfont finde\\findest\\findet\\finden\\findet\\finden} &
\parbox[t][][t]{2cm}{\normalfont fand\\fandest\\fand\\fanden\\fandet\\fanden}\\
\end{tabular}
\begin{tabular}{l}
\parbox[t][][t]{8cm}{}\\
\parbox[t][][t]{8cm}{\normalfont \footnotesize 
jdn/etw finden - 
to find sb/sth - 
es muss doch (irgendwo) zu finden sein! - 
it has to be (found) somewhere! - 
ich finde das (richtige) Wort nicht - 
I can't find (or think of) the (right) word - 
die Polizei fand eine heiße Spur - 
the police discovered a firm lead - 
jdn/etw finden - 
to find sb/sth 
}\\
\end{tabular}
}
%===flechten===
\card{\normalfont \Huge flechten}{
\begin{tabular}{lll}
\parbox[t][][t]{2.0 cm}{\normalfont \raggedleft ich\\du\\er/sie/es\\wir\\ihr\\sie} &    
\parbox[t][][t]{2cm}{\normalfont flechte\\flichtest\\flichtet\\flechten\\flechtet\\flechten} &
\parbox[t][][t]{2cm}{\normalfont flocht\\flochtest\\flocht\\flochten\\flochtet\\flochten}\\
\end{tabular}
\begin{tabular}{l}
\parbox[t][][t]{8cm}{}\\
\parbox[t][][t]{8cm}{\normalfont \footnotesize 
etw flechten - 
to plait (or - 
esp Am braid) sth - 
sich/jdm die Haare (zu Zöpfen (o. in Zöpfe)) flechten - 
to plait (or - 
esp Am braid) one's/sb's hair - 
einen Korb/Kranz/eine Matte flechten - 
to weave (or make)  a basket/wreath/mat - 
etw zu etw dat - 
flechten 
}\\
\end{tabular}
}
%===fliegen===
\card{\normalfont \Huge fliegen}{
\begin{tabular}{lll}
\parbox[t][][t]{2.0 cm}{\normalfont \raggedleft ich\\du\\er/sie/es\\wir\\ihr\\sie} &    
\parbox[t][][t]{2cm}{\normalfont fliege\\fliegst\\fliegt\\fliegen\\fliegt\\fliegen} &
\parbox[t][][t]{2cm}{\normalfont flog\\flogst\\flog\\flogen\\flogt\\flogen}\\
\end{tabular}
\begin{tabular}{l}
\parbox[t][][t]{8cm}{}\\
\parbox[t][][t]{8cm}{\normalfont \footnotesize 
fliegen - 
to fly - 
(irgendwohin) fliegen - 
to fly (somewhere) - 
wann fliegt die nächste Maschine (nach Paris)? - 
when is the next flight (to Paris)? - 
(aus etw dat) fliegen - 
to get kicked (or - 
fam chucked) out (of sth) - 
aus einer Firma fliegen 
}\\
\end{tabular}
}
%===fliehen===
\card{\normalfont \Huge fliehen}{
\begin{tabular}{lll}
\parbox[t][][t]{2.0 cm}{\normalfont \raggedleft ich\\du\\er/sie/es\\wir\\ihr\\sie} &    
\parbox[t][][t]{2cm}{\normalfont fliehe\\fliehst\\flieht\\fliehen\\flieht\\fliehen} &
\parbox[t][][t]{2cm}{\normalfont floh\\flohst\\floh\\flohen\\floht\\flohen}\\
\end{tabular}
\begin{tabular}{l}
\parbox[t][][t]{8cm}{}\\
\parbox[t][][t]{8cm}{\normalfont \footnotesize 
fliehen - 
to escape - 
fliehen - 
to flee - 
aus dem Gefängnis fliehen - 
to escape from prison - 
fliehen - 
to flee - 
vor der Polizei/einem Sturm fliehen - 
to flee from the police/before a storm 
}\\
\end{tabular}
}
%===fließen===
\card{\normalfont \Huge fließen}{
\begin{tabular}{lll}
\parbox[t][][t]{2.0 cm}{\normalfont \raggedleft ich\\du\\er/sie/es\\wir\\ihr\\sie} &    
\parbox[t][][t]{2cm}{\normalfont fließe\\fließt\\fließt\\fließen\\fließt\\fließen} &
\parbox[t][][t]{2cm}{\normalfont floss\\flossest\\floss\\flossen\\flosst\\flossen}\\
\end{tabular}
\begin{tabular}{l}
\parbox[t][][t]{8cm}{}\\
\parbox[t][][t]{8cm}{\normalfont \footnotesize 
fließen - 
to flow - 
fließen (sich dahinbewegen) - 
to flow - 
fließen (sich dahinbewegen) - 
to move - 
fließen METEO (einströmen) - 
to move - 
es fließt kein Wasser aus dem Hahn - 
there's no water coming from the tap 
}\\
\end{tabular}
}
%===fluchen===
\card{\normalfont \Huge fluchen}{
\begin{tabular}{lll}
\parbox[t][][t]{2.0 cm}{\normalfont \raggedleft ich\\du\\er/sie/es\\wir\\ihr\\sie} &    
\parbox[t][][t]{2cm}{\normalfont fluche\\fluchst\\flucht\\fluchen\\flucht\\fluchen} &
\parbox[t][][t]{2cm}{\normalfont fluchte\\fluchtest\\fluchte\\fluchten\\fluchtet\\fluchten}\\
\end{tabular}
\begin{tabular}{l}
\parbox[t][][t]{8cm}{}\\
\parbox[t][][t]{8cm}{\normalfont \footnotesize 
(auf/über jdn/etw) fluchen - 
to curse (or swear) at sb/sth - 
jdn/etw fluchen - 
to curse sb/sth 
}\\
\end{tabular}
}
%===folgen===
\card{\normalfont \Huge folgen}{
\begin{tabular}{lll}
\parbox[t][][t]{2.0 cm}{\normalfont \raggedleft ich\\du\\er/sie/es\\wir\\ihr\\sie} &    
\parbox[t][][t]{2cm}{\normalfont folge\\folgst\\folgt\\folgen\\folgt\\folgen} &
\parbox[t][][t]{2cm}{\normalfont folgte\\folgtest\\folgte\\folgten\\folgtet\\folgten}\\
\end{tabular}
\begin{tabular}{l}
\parbox[t][][t]{8cm}{}\\
\parbox[t][][t]{8cm}{\normalfont \footnotesize 
jdm/etw folgen - 
to follow sb/sth - 
folgen Sie mir unauffällig! - 
follow me quietly - 
(auf etw/jdn) folgen - 
to follow (sth/sb) - 
es folgt die Ziehung der Lottozahlen - 
the lottery draw will follow - 
auf etw akk - 
folgen 
}\\
\end{tabular}
}
%===fordern===
\card{\normalfont \Huge fordern}{
\begin{tabular}{lll}
\parbox[t][][t]{2.0 cm}{\normalfont \raggedleft ich\\du\\er/sie/es\\wir\\ihr\\sie} &    
\parbox[t][][t]{2cm}{\normalfont fordere\\forderst\\fordert\\fordern\\fordert\\fordern} &
\parbox[t][][t]{2cm}{\normalfont forderte\\fordertest\\forderte\\forderten\\fordertet\\forderten}\\
\end{tabular}
\begin{tabular}{l}
\parbox[t][][t]{8cm}{}\\
\parbox[t][][t]{8cm}{\normalfont \footnotesize 
etw (von jdm) fordern - 
to demand sth (from sb) - 
etw (von jdm) fordern - 
to require sth (of (or from) sb) - 
etw fordern - 
to claim sth - 
der Flugzeugabsturz forderte 123 Menschenleben - 
the (aero)plane crash claimed 123 lives - 
jdn/ein Tier fordern - 
to make demands on sb/an animal 
}\\
\end{tabular}
}
%===fressen===
\card{\normalfont \Huge fressen}{
\begin{tabular}{lll}
\parbox[t][][t]{2.0 cm}{\normalfont \raggedleft ich\\du\\er/sie/es\\wir\\ihr\\sie} &    
\parbox[t][][t]{2cm}{\normalfont fresse\\frisst\\frisst\\fressen\\fresst\\fressen} &
\parbox[t][][t]{2cm}{\normalfont fraß\\fraßt\\fraß\\fraßen\\fraßt\\fraßen}\\
\end{tabular}
\begin{tabular}{l}
\parbox[t][][t]{8cm}{}\\
\parbox[t][][t]{8cm}{\normalfont \footnotesize 
(aus etw dat/von etw dat) fressen - 
Tiere - 
to eat (or feed)  (out of/from sth) - 
(gierig) fressen - 
to guzzle fam - 
(gierig) fressen - 
to scoff fam - 
für drei fressen - 
to eat enough for a whole army fam - 
(an etw dat) fressen 
}\\
\end{tabular}
}
%===freuen===
\card{\normalfont \Huge freuen}{
\begin{tabular}{lll}
\parbox[t][][t]{2.0 cm}{\normalfont \raggedleft ich\\du\\er/sie/es\\wir\\ihr\\sie} &    
\parbox[t][][t]{2cm}{\normalfont freue\\freust\\freut\\freuen\\freut\\freuen} &
\parbox[t][][t]{2cm}{\normalfont freute\\freutest\\freute\\freuten\\freutet\\freuten}\\
\end{tabular}
\begin{tabular}{l}
\parbox[t][][t]{8cm}{}\\
\parbox[t][][t]{8cm}{\normalfont \footnotesize 
sich akk (über etw akk) freuen - 
to be glad (or pleased)  (about sth) - 
sich akk über ein Geschenk freuen - 
to be pleased with a present - 
sich akk sehr (über etw akk) freuen - 
to be delighted (with sth) - 
sich akk für jdn freuen - 
to be pleased (or glad) for sb('s sake) - 
sich akk mit jdm freuen - 
to share sb's happiness 
}\\
\end{tabular}
}
%===frieren===
\card{\normalfont \Huge frieren}{
\begin{tabular}{lll}
\parbox[t][][t]{2.0 cm}{\normalfont \raggedleft ich\\du\\er/sie/es\\wir\\ihr\\sie} &    
\parbox[t][][t]{2cm}{\normalfont friere\\frierst\\friert\\frieren\\friert\\frieren} &
\parbox[t][][t]{2cm}{\normalfont fror\\frorst\\fror\\froren\\frort\\froren}\\
\end{tabular}
\begin{tabular}{l}
\parbox[t][][t]{8cm}{}\\
\parbox[t][][t]{8cm}{\normalfont \footnotesize 
jd friert - 
sb is cold - 
ich friere an den Füßen - 
my feet are (freezing) cold - 
frieren Flüssigkeit, Boden - 
to freeze - 
der Boden ist (hart) gefroren - 
the ground is frozen (hard) - 
es friert - 
it's freezing 
}\\
\end{tabular}
}
%===frühstücken===
\card{\normalfont \Huge frühstücken}{
\begin{tabular}{lll}
\parbox[t][][t]{2.0 cm}{\normalfont \raggedleft ich\\du\\er/sie/es\\wir\\ihr\\sie} &    
\parbox[t][][t]{2cm}{\normalfont frühstücke\\frühstückst\\frühstückt\\frühstücken\\frühstückt\\frühstücken} &
\parbox[t][][t]{2cm}{\normalfont frühstückte\\frühstücktest\\frühstückte\\frühstückten\\frühstücktet\\frühstückten}\\
\end{tabular}
\begin{tabular}{l}
\parbox[t][][t]{8cm}{}\\
\parbox[t][][t]{8cm}{\normalfont \footnotesize 
frühstücken - 
to have (one's) breakfast - 
frühstücken - 
to breakfast form - 
sie frühstücken immer um 8 Uhr - 
they always have breakfast at 8 o'clock - 
etw frühstücken - 
to have sth for breakfast - 
etw frühstücken - 
to breakfast on sth form 
}\\
\end{tabular}
}
%===fühlen===
\card{\normalfont \Huge fühlen}{
\begin{tabular}{lll}
\parbox[t][][t]{2.0 cm}{\normalfont \raggedleft ich\\du\\er/sie/es\\wir\\ihr\\sie} &    
\parbox[t][][t]{2cm}{\normalfont fühle\\fühlst\\fühlt\\fühlen\\fühlt\\fühlen} &
\parbox[t][][t]{2cm}{\normalfont fühlte\\fühltest\\fühlte\\fühlten\\fühltet\\fühlten}\\
\end{tabular}
\begin{tabular}{l}
\parbox[t][][t]{8cm}{}\\
\parbox[t][][t]{8cm}{\normalfont \footnotesize 
etw fühlen - 
to feel sth - 
fühlen - 
to feel sth - 
Achtung/Verachtung für jdn fühlen - 
to feel respect/contempt for sb - 
Erbarmen/Mitleid mit jdm fühlen - 
to feel pity/sympathy for sb - 
fühlen, dass ... - 
to feel (that) ... 
}\\
\end{tabular}
}
%===führen===
\card{\normalfont \Huge führen}{
\begin{tabular}{lll}
\parbox[t][][t]{2.0 cm}{\normalfont \raggedleft ich\\du\\er/sie/es\\wir\\ihr\\sie} &    
\parbox[t][][t]{2cm}{\normalfont führe\\führst\\führt\\führen\\führt\\führen} &
\parbox[t][][t]{2cm}{\normalfont führte\\führtest\\führte\\führten\\führtet\\führten}\\
\end{tabular}
\begin{tabular}{l}
\parbox[t][][t]{8cm}{}\\
\parbox[t][][t]{8cm}{\normalfont \footnotesize 
jdn aus etw dat/in etw akk - 
führen - 
to lead sb into/out of sth - 
jdn in einen Raum führen - 
to lead (or usher) sb into a room - 
jdn durch/über etw akk - 
führen - 
to lead sb through/across (or over) sth - 
eine alte Dame über die Straße führen - 
to help an old lady across (or over) the road 
}\\
\end{tabular}
}
%===füllen===
\card{\normalfont \Huge füllen}{
\begin{tabular}{lll}
\parbox[t][][t]{2.0 cm}{\normalfont \raggedleft ich\\du\\er/sie/es\\wir\\ihr\\sie} &    
\parbox[t][][t]{2cm}{\normalfont fülle\\füllst\\füllt\\füllen\\füllt\\füllen} &
\parbox[t][][t]{2cm}{\normalfont füllte\\fülltest\\füllte\\füllten\\fülltet\\füllten}\\
\end{tabular}
\begin{tabular}{l}
\parbox[t][][t]{8cm}{}\\
\parbox[t][][t]{8cm}{\normalfont \footnotesize 
etw (mit etw dat) füllen - 
to fill sth (with sth) - 
halb gefüllt - 
half-full - 
etw (mit etw dat) füllen - 
to stuff sth (with sth) - 
etw in etw akk - 
füllen - 
to put sth into sth - 
etw in etw akk 
}\\
\end{tabular}
}
%===funktionieren===
\card{\normalfont \Huge funktionieren}{
\begin{tabular}{lll}
\parbox[t][][t]{2.0 cm}{\normalfont \raggedleft ich\\du\\er/sie/es\\wir\\ihr\\sie} &    
\parbox[t][][t]{2cm}{\normalfont funktioniere\\funktionierst\\funktioniert\\funktionieren\\funktioniert\\funktionieren} &
\parbox[t][][t]{2cm}{\normalfont funktionierte\\funktioniertest\\funktionierte\\funktionierten\\funktioniertet\\funktionierten}\\
\end{tabular}
\begin{tabular}{l}
\parbox[t][][t]{8cm}{}\\
\parbox[t][][t]{8cm}{\normalfont \footnotesize 
funktionieren - 
to work - 
funktionieren Maschine a. - 
to operate - 
funktionieren Maschine a. - 
to function - 
wie funktioniert dieses Gerät? - 
how does this device work? - 
funktionieren - 
to work out 
}\\
\end{tabular}
}
%===fürchten===
\card{\normalfont \Huge fürchten}{
\begin{tabular}{lll}
\parbox[t][][t]{2.0 cm}{\normalfont \raggedleft ich\\du\\er/sie/es\\wir\\ihr\\sie} &    
\parbox[t][][t]{2cm}{\normalfont fürchte\\fürchtest\\fürchtet\\fürchten\\fürchtet\\fürchten} &
\parbox[t][][t]{2cm}{\normalfont fürchtete\\fürchtetest\\fürchtete\\fürchteten\\fürchtetet\\fürchteten}\\
\end{tabular}
\begin{tabular}{l}
\parbox[t][][t]{8cm}{}\\
\parbox[t][][t]{8cm}{\normalfont \footnotesize 
jdn/etw fürchten - 
to fear (or be afraid of) sb/sth - 
gefürchtet sein - 
to be feared - 
jdn das Fürchten lehren - 
to teach sb the meaning of fear - 
etw fürchten - 
to fear sth - 
fürchten, dass ... - 
to fear that ... 
}\\
\end{tabular}
}
%===gären===
\card{\normalfont \Huge gären}{
\begin{tabular}{lll}
\parbox[t][][t]{2.0 cm}{\normalfont \raggedleft ich\\du\\er/sie/es\\wir\\ihr\\sie} &    
\parbox[t][][t]{2cm}{\normalfont gäre\\gärst\\gärt\\gären\\gärt\\gären} &
\parbox[t][][t]{2cm}{\normalfont gor\\gorst\\gor \\goren\\gort \\goren}\\
\end{tabular}
\begin{tabular}{l}
\parbox[t][][t]{8cm}{}\\
\parbox[t][][t]{8cm}{\normalfont \footnotesize 
gären - 
to ferment - 
gären - 
to seethe - 
etw gärt in jdm - 
sth is making sb seethe - 
die Wut hatte schon lange in ihm gegärt - 
he had been seething with fury a long time - 
etw garen - 
to cook sth 
}\\
\end{tabular}
}
%===geben===
\card{\normalfont \Huge geben}{
\begin{tabular}{lll}
\parbox[t][][t]{2.0 cm}{\normalfont \raggedleft ich\\du\\er/sie/es\\wir\\ihr\\sie} &    
\parbox[t][][t]{2cm}{\normalfont gebe\\gibst\\gibt\\geben\\gebt\\geben} &
\parbox[t][][t]{2cm}{\normalfont gab\\gabst\\gab\\gaben\\gabt\\gaben}\\
\end{tabular}
\begin{tabular}{l}
\parbox[t][][t]{8cm}{}\\
\parbox[t][][t]{8cm}{\normalfont \footnotesize 
jdm etw geben - 
to give sb sth - 
jdm etw geben - 
to give sth to sb - 
gibst du mir bitte mal das Brot? - 
could you give (or hand) me the bread, please? (or pass) - 
ich würde alles darum geben, ihn noch einmal zu sehen - 
I would give anything to see him again - 
geben - 
to deal 
}\\
\end{tabular}
}
%===gebrauchen===
\card{\normalfont \Huge gebrauchen}{
\begin{tabular}{lll}
\parbox[t][][t]{2.0 cm}{\normalfont \raggedleft ich\\du\\er/sie/es\\wir\\ihr\\sie} &    
\parbox[t][][t]{2cm}{\normalfont gebrauche\\gebrauchst\\gebraucht\\gebrauchen\\gebraucht\\gebrauchen} &
\parbox[t][][t]{2cm}{\normalfont gebrauchte\\gebrauchtest\\gebrauchte\\gebrauchten\\gebrauchtet\\gebrauchten}\\
\end{tabular}
\begin{tabular}{l}
\parbox[t][][t]{8cm}{}\\
\parbox[t][][t]{8cm}{\normalfont \footnotesize 
etw gebrauchen - 
to use sth - 
ein gebrauchtes Auto - 
a used (or second-hand) car - 
nicht mehr zu gebrauchen sein - 
to be no longer (of) any use - 
nicht mehr zu gebrauchen sein - 
to be useless - 
das kann ich gut gebrauchen - 
I can really use that 
}\\
\end{tabular}
}
%===gedeihen===
\card{\normalfont \Huge gedeihen}{
\begin{tabular}{lll}
\parbox[t][][t]{2.0 cm}{\normalfont \raggedleft ich\\du\\er/sie/es\\wir\\ihr\\sie} &    
\parbox[t][][t]{2cm}{\normalfont gedeihe\\gedeihst\\gedeiht\\gedeihen\\gedeiht\\gedeihen} &
\parbox[t][][t]{2cm}{\normalfont gedieh\\gediehst\\gedieh\\gediehen\\gedieht\\gediehen}\\
\end{tabular}
\begin{tabular}{l}
\parbox[t][][t]{8cm}{}\\
\parbox[t][][t]{8cm}{\normalfont \footnotesize 
gedeihen - 
to flourish - 
gedeihen - 
to thrive - 
gedeihen - 
to make headway (or progress) - 
Gedeihen - 
flourishing - 
Gedeihen - 
thriving 
}\\
\end{tabular}
}
%===fallen===
\card{\normalfont \Huge fallen}{
\begin{tabular}{lll}
\parbox[t][][t]{2.0 cm}{\normalfont \raggedleft ich\\du\\er/sie/es\\wir\\ihr\\sie} &    
\parbox[t][][t]{2cm}{\normalfont falle\\fällst\\fällt\\fallen\\fallt\\fallen} &
\parbox[t][][t]{2cm}{\normalfont fiel\\fielst\\fiel\\fielen\\fielt\\fielen}\\
\end{tabular}
\begin{tabular}{l}
\parbox[t][][t]{8cm}{}\\
\parbox[t][][t]{8cm}{\normalfont \footnotesize 
fallen apple - 
abgefallener Apfel - 
Add to my favourites Preselect for export to vocabulary trainer View selected vocabulary - 
fallen arches MED - 
Senkfüße pl - 
Add to my favourites Preselect for export to vocabulary trainer View selected vocabulary - 
fallen leaf - 
heruntergefallenes Blatt - 
Add to my favourites Preselect for export to vocabulary trainer View selected vocabulary - 
fallen leaves 
}\\
\end{tabular}
}
%===gehen===
\card{\normalfont \Huge gehen}{
\begin{tabular}{lll}
\parbox[t][][t]{2.0 cm}{\normalfont \raggedleft ich\\du\\er/sie/es\\wir\\ihr\\sie} &    
\parbox[t][][t]{2cm}{\normalfont gehe\\gehst\\geht\\gehen\\geht\\gehen} &
\parbox[t][][t]{2cm}{\normalfont ging\\gingst\\ging\\gingen\\gingt\\gingen}\\
\end{tabular}
\begin{tabular}{l}
\parbox[t][][t]{8cm}{}\\
\parbox[t][][t]{8cm}{\normalfont \footnotesize 
gehen - 
to walk (somewhere) - 
gehen wir oder fahren wir mit dem Auto? - 
shall we walk or drive? - 
die kleine Sarah lernt gehen - 
little Sarah is learning to walk - 
(im Zimmer) auf und ab gehen - 
to walk up and down (or to pace)  (the room) - 
über die Brücke/Straße gehen - 
to cross (over) the bridge/street 
}\\
\end{tabular}
}
%===gehören===
\card{\normalfont \Huge gehören}{
\begin{tabular}{lll}
\parbox[t][][t]{2.0 cm}{\normalfont \raggedleft ich\\du\\er/sie/es\\wir\\ihr\\sie} &    
\parbox[t][][t]{2cm}{\normalfont gehöre\\gehörst\\gehört\\gehören\\gehört\\gehören} &
\parbox[t][][t]{2cm}{\normalfont gehörte\\gehörtest\\gehörte\\gehörten\\gehörtet\\gehörten}\\
\end{tabular}
\begin{tabular}{l}
\parbox[t][][t]{8cm}{}\\
\parbox[t][][t]{8cm}{\normalfont \footnotesize 
jdm gehören - 
to belong to sb - 
jdm gehören - 
to be sb's - 
ihm gehören mehrere Häuser - 
he owns several houses - 
ihre ganze Liebe gehört ihrem Sohn - 
she gives all her love to her son - 
mein Herz gehört einem anderen - 
my heart belongs to another poet 
}\\
\end{tabular}
}
%===gelten===
\card{\normalfont \Huge gelten}{
\begin{tabular}{lll}
\parbox[t][][t]{2.0 cm}{\normalfont \raggedleft ich\\du\\er/sie/es\\wir\\ihr\\sie} &    
\parbox[t][][t]{2cm}{\normalfont gelte\\giltst\\gilt\\gelten\\geltet\\gelten} &
\parbox[t][][t]{2cm}{\normalfont galt\\galtest\\galt\\galten\\galtet\\galten}\\
\end{tabular}
\begin{tabular}{l}
\parbox[t][][t]{8cm}{}\\
\parbox[t][][t]{8cm}{\normalfont \footnotesize 
(für jdn) gelten - 
Regelung - 
to be valid (for sb) - 
(für jdn) gelten - 
Bestimmungen - 
to apply (to sb) - 
(für jdn) gelten - 
Gesetz - 
to be in force - 
(für jdn) gelten 
}\\
\end{tabular}
}
%===genesen===
\card{\normalfont \Huge genesen}{
\begin{tabular}{lll}
\parbox[t][][t]{2.0 cm}{\normalfont \raggedleft ich\\du\\er/sie/es\\wir\\ihr\\sie} &    
\parbox[t][][t]{2cm}{\normalfont genese\\genest\\genest\\genesen\\genest\\genesen} &
\parbox[t][][t]{2cm}{\normalfont genas\\genasest\\genas\\genasen\\genast\\genasen}\\
\end{tabular}
\begin{tabular}{l}
\parbox[t][][t]{8cm}{}\\
\parbox[t][][t]{8cm}{\normalfont \footnotesize 
(nach etw dat/von etw dat) genesen - 
to recover (after/from sth) - 
(nach etw dat/von etw dat) genesen - 
to convalesce - 
Genese - 
genesis no pl form 
}\\
\end{tabular}
}
%===genießen===
\card{\normalfont \Huge genießen}{
\begin{tabular}{lll}
\parbox[t][][t]{2.0 cm}{\normalfont \raggedleft ich\\du\\er/sie/es\\wir\\ihr\\sie} &    
\parbox[t][][t]{2cm}{\normalfont genieße\\genießt\\genießt\\genießen\\genießt\\genießen} &
\parbox[t][][t]{2cm}{\normalfont genoss\\genossest\\genoss\\genossen\\genosst\\genossen}\\
\end{tabular}
\begin{tabular}{l}
\parbox[t][][t]{8cm}{}\\
\parbox[t][][t]{8cm}{\normalfont \footnotesize 
etw genießen - 
to enjoy (or relish) sth - 
etw genießen - 
(bewusst kosten) - 
to savour (or - 
Am -or) sth - 
etw genießen  (essen) - 
to eat sth - 
etw genießen  (trinken) - 
to drink sth 
}\\
\end{tabular}
}
%===geraten===
\card{\normalfont \Huge geraten}{
\begin{tabular}{lll}
\parbox[t][][t]{2.0 cm}{\normalfont \raggedleft ich\\du\\er/sie/es\\wir\\ihr\\sie} &    
\parbox[t][][t]{2cm}{\normalfont gerate\\gerätst\\gerät\\geraten\\geratet\\geraten} &
\parbox[t][][t]{2cm}{\normalfont geriet\\gerietest\\geriet\\gerieten\\gerietet\\gerieten}\\
\end{tabular}
\begin{tabular}{l}
\parbox[t][][t]{8cm}{}\\
\parbox[t][][t]{8cm}{\normalfont \footnotesize 
irgendwohin geraten - 
to get to somewhere - 
in schlechte Gesellschaft/eine Schlägerei/einen Stau geraten - 
to get into bad company/a fight/a traffic jam - 
an einen Ort geraten - 
to come to a place - 
(mit etw dat) an/in/unter etw akk - 
geraten - 
to get (sth) caught in/under sth - 
unter einen Lastwagen geraten 
}\\
\end{tabular}
}
%===geschehen===
\card{\normalfont \Huge geschehen}{
\begin{tabular}{lll}
\parbox[t][][t]{2.0 cm}{\normalfont \raggedleft ich\\du\\er/sie/es\\wir\\ihr\\sie} &    
\parbox[t][][t]{2cm}{\normalfont geschehe\\geschiehst\\geschieht\\geschehen\\gescheht\\geschehen} &
\parbox[t][][t]{2cm}{\normalfont geschah\\geschahst\\geschah\\geschahen\\geschaht\\geschahen}\\
\end{tabular}
\begin{tabular}{l}
\parbox[t][][t]{8cm}{}\\
\parbox[t][][t]{8cm}{\normalfont \footnotesize 
geschehen - 
to happen - 
geschehen - 
to occur - 
es muss etwas geschehen - 
something's got to be done - 
geschehen - 
to be carried out (or done) - 
ein Mord geschieht - 
a murder is committed 
}\\
\end{tabular}
}
%===gewinnen===
\card{\normalfont \Huge gewinnen}{
\begin{tabular}{lll}
\parbox[t][][t]{2.0 cm}{\normalfont \raggedleft ich\\du\\er/sie/es\\wir\\ihr\\sie} &    
\parbox[t][][t]{2cm}{\normalfont gewinne\\gewinnst\\gewinnt\\gewinnen\\gewinnt\\gewinnen} &
\parbox[t][][t]{2cm}{\normalfont gewann\\gewannst\\gewann\\gewannen\\gewannt\\gewannen}\\
\end{tabular}
\begin{tabular}{l}
\parbox[t][][t]{8cm}{}\\
\parbox[t][][t]{8cm}{\normalfont \footnotesize 
etw gewinnen - 
to win sth - 
etw gewinnen - 
to win sth - 
ein Spiel gegen jdn gewinnen - 
to beat sb in a game - 
jdn (für etw akk) gewinnen - 
to win sb over (to sth) - 
jdn als Freund gewinnen - 
to win (or gain) sb as a friend 
}\\
\end{tabular}
}
%===gewöhnen===
\card{\normalfont \Huge gewöhnen}{
\begin{tabular}{lll}
\parbox[t][][t]{2.0 cm}{\normalfont \raggedleft ich\\du\\er/sie/es\\wir\\ihr\\sie} &    
\parbox[t][][t]{2cm}{\normalfont gewöhne\\gewöhnst\\gewöhnt\\gewöhnen\\gewöhnt\\gewöhnen} &
\parbox[t][][t]{2cm}{\normalfont gewöhnte\\gewöhntest\\gewöhnte\\gewöhnten\\gewöhntet\\gewöhnten}\\
\end{tabular}
\begin{tabular}{l}
\parbox[t][][t]{8cm}{}\\
\parbox[t][][t]{8cm}{\normalfont \footnotesize 
jdn an etw akk - 
gewöhnen - 
to make sb used (or accustomed) to (or accustom sb to) sth - 
ein Tier an sich/etw akk - 
gewöhnen - 
to make an animal get used to one/sth - 
ein Haustier an Sauberkeit gewöhnen - 
to house-train a pet - 
an jdn/etw gewöhnt sein jdn/etw gewöhnt sein fam - 
to be used (or accustomed) to sb/sth 
}\\
\end{tabular}
}
%===gießen===
\card{\normalfont \Huge gießen}{
\begin{tabular}{lll}
\parbox[t][][t]{2.0 cm}{\normalfont \raggedleft ich\\du\\er/sie/es\\wir\\ihr\\sie} &    
\parbox[t][][t]{2cm}{\normalfont gieße\\gießt\\gießt\\gießen\\gießt\\gießen} &
\parbox[t][][t]{2cm}{\normalfont goss\\gossest\\goss\\gossen\\gosst\\gossen}\\
\end{tabular}
\begin{tabular}{l}
\parbox[t][][t]{8cm}{}\\
\parbox[t][][t]{8cm}{\normalfont \footnotesize 
etw gießen - 
to water sth - 
etw gießen - 
to pour sth - 
ein Glas randvoll gießen - 
to fill (up sep ) a glass to the brim - 
etw in etw akk - 
gießen - 
to pour sth in(to) sth - 
etw auf etw akk/über etw akk 
}\\
\end{tabular}
}
%===glauben===
\card{\normalfont \Huge glauben}{
\begin{tabular}{lll}
\parbox[t][][t]{2.0 cm}{\normalfont \raggedleft ich\\du\\er/sie/es\\wir\\ihr\\sie} &    
\parbox[t][][t]{2cm}{\normalfont glaube\\glaubst\\glaubt\\glauben\\glaubt\\glauben} &
\parbox[t][][t]{2cm}{\normalfont glaubte\\glaubtest\\glaubte\\glaubten\\glaubtet\\glaubten}\\
\end{tabular}
\begin{tabular}{l}
\parbox[t][][t]{8cm}{}\\
\parbox[t][][t]{8cm}{\normalfont \footnotesize 
(jdm) etw glauben - 
to believe sth (of sb's) - 
das glaubst du doch selbst nicht! - 
you don't really believe that, do you! (or can't be serious!) - 
ob du es glaubst oder nicht, aber... - 
believe it or not, but... - 
jdm jedes Wort glauben - 
to believe every word sb says - 
kaum (o. nicht) zu glauben - 
unbelievable 
}\\
\end{tabular}
}
%===gleichen===
\card{\normalfont \Huge gleichen}{
\begin{tabular}{lll}
\parbox[t][][t]{2.0 cm}{\normalfont \raggedleft ich\\du\\er/sie/es\\wir\\ihr\\sie} &    
\parbox[t][][t]{2cm}{\normalfont gleiche\\gleichst\\gleicht\\gleichen\\gleicht\\gleichen} &
\parbox[t][][t]{2cm}{\normalfont glich\\glichst\\glich\\glichen\\glicht\\glichen}\\
\end{tabular}
\begin{tabular}{l}
\parbox[t][][t]{8cm}{}\\
\parbox[t][][t]{8cm}{\normalfont \footnotesize 
jdm/etw gleichen - 
to be (just) like sb/sth - 
sich dat - 
gleichen - 
to be alike (or similar) - 
Fusion unter Gleichen - 
merger among equals - 
gleich (übereinstimmend) - 
same attr - 
gleich (gleichwertig) 
}\\
\end{tabular}
}
%===gleiten===
\card{\normalfont \Huge gleiten}{
\begin{tabular}{lll}
\parbox[t][][t]{2.0 cm}{\normalfont \raggedleft ich\\du\\er/sie/es\\wir\\ihr\\sie} &    
\parbox[t][][t]{2cm}{\normalfont gleite\\gleitest\\gleitet\\gleiten\\gleitet\\gleiten} &
\parbox[t][][t]{2cm}{\normalfont glitt\\glittest\\glitt\\glitten\\glittet\\glitten}\\
\end{tabular}
\begin{tabular}{l}
\parbox[t][][t]{8cm}{}\\
\parbox[t][][t]{8cm}{\normalfont \footnotesize 
(durch etw akk/über etw akk o dat) gleiten - 
to glide (through/over sth) - 
(durch etw akk/über etw akk o dat) gleiten - 
Wolke - 
to sail (through/over sth) - 
(durch etw akk/in etw akk/über etw akk) gleiten - 
to glide (through/into/over sth) - 
(durch etw akk/in etw akk/über etw akk) gleiten - 
Schlange a. - 
to slide (or slip)  (through/into/over sth) 
}\\
\end{tabular}
}
%===glimmen===
\card{\normalfont \Huge glimmen}{
\begin{tabular}{lll}
\parbox[t][][t]{2.0 cm}{\normalfont \raggedleft ich\\du\\er/sie/es\\wir\\ihr\\sie} &    
\parbox[t][][t]{2cm}{\normalfont glimme\\glimmst\\glimmt\\glimmen\\glimmt\\glimmen} &
\parbox[t][][t]{2cm}{\normalfont glimmte \\glimst\\glimmte \\glimmten\\glimmtet\\glimmten }\\
\end{tabular}
\begin{tabular}{l}
\parbox[t][][t]{8cm}{}\\
\parbox[t][][t]{8cm}{\normalfont \footnotesize 
glimmen - 
to glow - 
glimmen Feuer, Asche a. - 
to smoulder - 
glimmen Feuer, Asche a. - 
Am usu to smolder - 
glimmende Asche - 
embers - 
glimmende Asche - 
hot ashes 
}\\
\end{tabular}
}
%===graben===
\card{\normalfont \Huge graben}{
\begin{tabular}{lll}
\parbox[t][][t]{2.0 cm}{\normalfont \raggedleft ich\\du\\er/sie/es\\wir\\ihr\\sie} &    
\parbox[t][][t]{2cm}{\normalfont grabe\\gräbst\\gräbt\\graben\\grabt\\graben} &
\parbox[t][][t]{2cm}{\normalfont grub\\grubst\\grub\\gruben\\grubt\\gruben}\\
\end{tabular}
\begin{tabular}{l}
\parbox[t][][t]{8cm}{}\\
\parbox[t][][t]{8cm}{\normalfont \footnotesize 
graben - 
Graben m - 
Add to my favourites Preselect for export to vocabulary trainer View selected vocabulary - 
graben - 
to dig - 
nach etw dat - 
graben - 
to dig for sth - 
etw graben - 
Loch 
}\\
\end{tabular}
}
%===greifen===
\card{\normalfont \Huge greifen}{
\begin{tabular}{lll}
\parbox[t][][t]{2.0 cm}{\normalfont \raggedleft ich\\du\\er/sie/es\\wir\\ihr\\sie} &    
\parbox[t][][t]{2cm}{\normalfont greife\\greifst\\greift\\greifen\\greift\\greifen} &
\parbox[t][][t]{2cm}{\normalfont griff\\griffst\\griff\\griffen\\grifft\\griffen}\\
\end{tabular}
\begin{tabular}{l}
\parbox[t][][t]{8cm}{}\\
\parbox[t][][t]{8cm}{\normalfont \footnotesize 
(sich dat) etw greifen - 
to take sth - 
(sich dat) etw greifen - 
Essen - 
to help oneself to sth - 
aus dem Leben gegriffen sein - 
to be taken from real life - 
(sich dat) etw (mit etw dat) greifen - 
to take hold of sth (with sth) - 
(sich dat) etw (mit etw dat) greifen 
}\\
\end{tabular}
}
%===gründen===
\card{\normalfont \Huge gründen}{
\begin{tabular}{lll}
\parbox[t][][t]{2.0 cm}{\normalfont \raggedleft ich\\du\\er/sie/es\\wir\\ihr\\sie} &    
\parbox[t][][t]{2cm}{\normalfont gründe\\gründest\\gründet\\gründen\\gründet\\gründen} &
\parbox[t][][t]{2cm}{\normalfont gründete\\gründetest\\gründete\\gründeten\\gründetet\\gründeten}\\
\end{tabular}
\begin{tabular}{l}
\parbox[t][][t]{8cm}{}\\
\parbox[t][][t]{8cm}{\normalfont \footnotesize 
etw gründen - 
to found sth - 
einen Betrieb/eine Firma gründen - 
to establish (or set up)  a business/firm - 
eine Partei gründen - 
to form (or establish)  a party - 
eine Universität gründen - 
to found (or establish)  a university - 
etw auf etw akk - 
gründen 
}\\
\end{tabular}
}
%===grüßen===
\card{\normalfont \Huge grüßen}{
\begin{tabular}{lll}
\parbox[t][][t]{2.0 cm}{\normalfont \raggedleft ich\\du\\er/sie/es\\wir\\ihr\\sie} &    
\parbox[t][][t]{2cm}{\normalfont grüße\\grüßt\\grüßt\\grüßen\\grüßt\\grüßen} &
\parbox[t][][t]{2cm}{\normalfont grüßte\\grüßtest\\grüßte\\grüßten\\grüßtet\\grüßten}\\
\end{tabular}
\begin{tabular}{l}
\parbox[t][][t]{8cm}{}\\
\parbox[t][][t]{8cm}{\normalfont \footnotesize 
jdn grüßen - 
to greet sb - 
jdn grüßen - 
MILIT - 
to salute sb - 
sei (mir) gegrüßt! geh - 
greetings! form - 
grüß dich! fam - 
hello there! fam - 
jdn von jdm grüßen 
}\\
\end{tabular}
}
%===gucken===
\card{\normalfont \Huge gucken}{
\begin{tabular}{lll}
\parbox[t][][t]{2.0 cm}{\normalfont \raggedleft ich\\du\\er/sie/es\\wir\\ihr\\sie} &    
\parbox[t][][t]{2cm}{\normalfont gucke\\guckst\\guckt\\gucken\\guckt\\gucken} &
\parbox[t][][t]{2cm}{\normalfont guckte\\gucktest\\guckte\\guckten\\gucktet\\guckten}\\
\end{tabular}
\begin{tabular}{l}
\parbox[t][][t]{8cm}{}\\
\parbox[t][][t]{8cm}{\normalfont \footnotesize 
(in etw akk/durch etw akk/aus etw dat) gucken - 
to look (in/through/out of sth) - 
was guckst du so dumm! - 
take that silly look off your face! - 
ich habe schon Weihnachtsgeschenke gekauft, aber nicht gucken! - 
I've already bought the Christmas presents, so no peeping! - 
aus etw dat - 
gucken - 
to stick out of sth - 
was guckt denn da aus der Tasche? 
}\\
\end{tabular}
}
%===haben===
\card{\normalfont \Huge haben}{
\begin{tabular}{lll}
\parbox[t][][t]{2.0 cm}{\normalfont \raggedleft ich\\du\\er/sie/es\\wir\\ihr\\sie} &    
\parbox[t][][t]{2cm}{\normalfont habe\\hast\\hat\\haben\\habt\\haben} &
\parbox[t][][t]{2cm}{\normalfont hatte\\hattest\\hatte\\hatten\\hattet\\hatten}\\
\end{tabular}
\begin{tabular}{l}
\parbox[t][][t]{8cm}{}\\
\parbox[t][][t]{8cm}{\normalfont \footnotesize 
jdn/etw haben - 
to have (got) sb/sth - 
wir haben zwei Autos - 
we have (or we've)  (got) two cars - 
sie hatte gestern Geburtstag - 
it was her birthday yesterday - 
sie hat ihn zum Mann - 
he is her husband - 
die/wir habens (ja) hum - 
they/we can afford it 
}\\
\end{tabular}
}
%===halten===
\card{\normalfont \Huge halten}{
\begin{tabular}{lll}
\parbox[t][][t]{2.0 cm}{\normalfont \raggedleft ich\\du\\er/sie/es\\wir\\ihr\\sie} &    
\parbox[t][][t]{2cm}{\normalfont halte\\hältst\\hält\\halten\\haltet\\halten} &
\parbox[t][][t]{2cm}{\normalfont hielt\\hieltest\\hielt\\hielten\\hieltet\\hielten}\\
\end{tabular}
\begin{tabular}{l}
\parbox[t][][t]{8cm}{}\\
\parbox[t][][t]{8cm}{\normalfont \footnotesize 
(jdm) jdn/etw halten - 
to hold sb/sth (for sb) - 
du musst das Seil ganz fest halten - 
you must keep a tight grip on the rope - 
hältst du bitte kurz meine Tasche? - 
would you please hold my bag for a moment? - 
kannst du (das) mal einen Moment halten? - 
can you hold that for a second? - 
jdn/etw im Arm halten - 
to hold sb/sth in one's arms 
}\\
\end{tabular}
}
%===handeln===
\card{\normalfont \Huge handeln}{
\begin{tabular}{lll}
\parbox[t][][t]{2.0 cm}{\normalfont \raggedleft ich\\du\\er/sie/es\\wir\\ihr\\sie} &    
\parbox[t][][t]{2cm}{\normalfont handle\\handelst\\handelt\\handeln\\handelt\\handeln} &
\parbox[t][][t]{2cm}{\normalfont handelte\\handeltest\\handelte\\handelten\\handeltet\\handelten}\\
\end{tabular}
\begin{tabular}{l}
\parbox[t][][t]{8cm}{}\\
\parbox[t][][t]{8cm}{\normalfont \footnotesize 
mit etw dat (o. ÖKON in etw dat) - 
handeln - 
to trade (or deal) in sth - 
in Dollar gehandelt werden - 
to be traded in dollars - 
zu drei Dollar gehandelt werden - 
to be traded at (or for) three dollars - 
er handelt mit (o. in) Gebrauchtwagen - 
he trades (or deals) in second-hand cars - 
er handelt mit (o. in) Gebrauchtwagen 
}\\
\end{tabular}
}
%===hängen===
\card{\normalfont \Huge hängen}{
\begin{tabular}{lll}
\parbox[t][][t]{2.0 cm}{\normalfont \raggedleft ich\\du\\er/sie/es\\wir\\ihr\\sie} &    
\parbox[t][][t]{2cm}{\normalfont hänge\\hängst\\hängt\\hängen\\hängt\\hängen} &
\parbox[t][][t]{2cm}{\normalfont hing\\hingst\\hing\\hingen\\hingt\\hingen}\\
\end{tabular}
\begin{tabular}{l}
\parbox[t][][t]{8cm}{}\\
\parbox[t][][t]{8cm}{\normalfont \footnotesize 
hängen - 
to hang - 
das Bild hängt nicht gerade - 
the picture's not hanging straight - 
(herabhängen) an etw dat/über etw akk/von etw dat - 
hängen - 
to hang on sth/over sth/from sth - 
hängt die Wäsche noch an der Leine? - 
is the washing still hanging on the line? - 
die Spinne hing an einem Faden von der Decke 
}\\
\end{tabular}
}
%===hassen===
\card{\normalfont \Huge hassen}{
\begin{tabular}{lll}
\parbox[t][][t]{2.0 cm}{\normalfont \raggedleft ich\\du\\er/sie/es\\wir\\ihr\\sie} &    
\parbox[t][][t]{2cm}{\normalfont hasse\\hasst\\hasst\\hassen\\hasst\\hassen} &
\parbox[t][][t]{2cm}{\normalfont hasste\\hasstest\\hasste\\hassten\\hasstet\\hassten}\\
\end{tabular}
\begin{tabular}{l}
\parbox[t][][t]{8cm}{}\\
\parbox[t][][t]{8cm}{\normalfont \footnotesize 
jdn hassen - 
to hate sb - 
etw hassen - 
to hate (or loathe) - 
(or detest) sth - 
es hassen, etw zu tun - 
to hate doing sth - 
Pest - 
Pest - 
die Pest 
}\\
\end{tabular}
}
%===heben===
\card{\normalfont \Huge heben}{
\begin{tabular}{lll}
\parbox[t][][t]{2.0 cm}{\normalfont \raggedleft ich\\du\\er/sie/es\\wir\\ihr\\sie} &    
\parbox[t][][t]{2cm}{\normalfont hebe\\hebst\\hebt\\heben\\hebt\\heben} &
\parbox[t][][t]{2cm}{\normalfont hob\\hobst\\hob\\hoben\\hobt\\hoben}\\
\end{tabular}
\begin{tabular}{l}
\parbox[t][][t]{8cm}{}\\
\parbox[t][][t]{8cm}{\normalfont \footnotesize 
etw heben - 
to lift (or raise) sth - 
etw heben - 
Hebezeug a. - 
to hoist sth - 
etw heben - 
(vom Boden) - 
to pick up sth sep - 
hebt eure Füße! - 
pick your feet up! 
}\\
\end{tabular}
}
%===heiraten===
\card{\normalfont \Huge heiraten}{
\begin{tabular}{lll}
\parbox[t][][t]{2.0 cm}{\normalfont \raggedleft ich\\du\\er/sie/es\\wir\\ihr\\sie} &    
\parbox[t][][t]{2cm}{\normalfont heirate\\heiratest\\heiratet\\heiraten\\heiratet\\heiraten} &
\parbox[t][][t]{2cm}{\normalfont heiratete\\heiratetest\\heiratete\\heirateten\\heiratetet\\heirateten}\\
\end{tabular}
\begin{tabular}{l}
\parbox[t][][t]{8cm}{}\\
\parbox[t][][t]{8cm}{\normalfont \footnotesize 
jdn heiraten - 
to marry sb - 
heiraten - 
to get married - 
wir wollen nächsten Monat heiraten - 
we want to get married next month - 
sie hat reich geheiratet - 
she married into money - 
„wir heiraten“ - 
“we are getting married” 
}\\
\end{tabular}
}
%===heißen===
\card{\normalfont \Huge heißen}{
\begin{tabular}{lll}
\parbox[t][][t]{2.0 cm}{\normalfont \raggedleft ich\\du\\er/sie/es\\wir\\ihr\\sie} &    
\parbox[t][][t]{2cm}{\normalfont heiße\\heißt\\heißt\\heißen\\heißt\\heißen} &
\parbox[t][][t]{2cm}{\normalfont hieß\\hießest\\hieß\\hießen\\hießt\\hießen}\\
\end{tabular}
\begin{tabular}{l}
\parbox[t][][t]{8cm}{}\\
\parbox[t][][t]{8cm}{\normalfont \footnotesize 
heißen - 
to be called - 
wie heißen Sie? - 
what's your name? - 
ich heiße Schmitz - 
my name is Schmitz - 
wie soll das Baby denn heißen? - 
what shall we call (or will we name) the baby? - 
er heißt jetzt anders - 
he has changed his name 
}\\
\end{tabular}
}
%===heizen===
\card{\normalfont \Huge heizen}{
\begin{tabular}{lll}
\parbox[t][][t]{2.0 cm}{\normalfont \raggedleft ich\\du\\er/sie/es\\wir\\ihr\\sie} &    
\parbox[t][][t]{2cm}{\normalfont heize\\heizt\\heizt\\heizen\\heizt\\heizen} &
\parbox[t][][t]{2cm}{\normalfont heizte\\heiztest\\heizte\\heizten\\heiztet\\heizten}\\
\end{tabular}
\begin{tabular}{l}
\parbox[t][][t]{8cm}{}\\
\parbox[t][][t]{8cm}{\normalfont \footnotesize 
(mit etw dat) heizen - 
„womit heizt ihr zu Hause?“ — „wir heizen mit Gas“ - 
“how is your house heated?” — “it's gas-heated” - 
heizen - 
to give off heat - 
etw heizen - 
to heat sth - 
etw heizen - 
to stoke sth 
}\\
\end{tabular}
}
%===helfen===
\card{\normalfont \Huge helfen}{
\begin{tabular}{lll}
\parbox[t][][t]{2.0 cm}{\normalfont \raggedleft ich\\du\\er/sie/es\\wir\\ihr\\sie} &    
\parbox[t][][t]{2cm}{\normalfont helfe\\hilfst\\hilft\\helfen\\helft\\helfen} &
\parbox[t][][t]{2cm}{\normalfont half\\halfst\\half\\halfen\\halft\\halfen}\\
\end{tabular}
\begin{tabular}{l}
\parbox[t][][t]{8cm}{}\\
\parbox[t][][t]{8cm}{\normalfont \footnotesize 
jdm (bei etw dat/in etw dat) helfen - 
to help sb (with/in sth) - 
warte mal, ich helfe dir - 
wait, I'll help you - 
können/könnten Sie mir mal/bitte helfen? - 
could/would you help me please/a minute? - 
jdm aus etw dat/in etw akk - 
helfen - 
to help sb out of/into sth - 
darf ich Ihnen in den Mantel helfen? 
}\\
\end{tabular}
}
%===herkommen===
\card{\normalfont \Huge herkommen}{
\begin{tabular}{lll}
\parbox[t][][t]{2.0 cm}{\normalfont \raggedleft ich\\du\\er/sie/es\\wir\\ihr\\sie} &    
\parbox[t][][t]{2cm}{\normalfont komme her\\kommst her\\kommt her\\kommen her\\kommt her\\kommen her} &
\parbox[t][][t]{2cm}{\normalfont kam her\\kamst her\\kam her\\kamen her\\kamt her\\kamen her}\\
\end{tabular}
\begin{tabular}{l}
\parbox[t][][t]{8cm}{}\\
\parbox[t][][t]{8cm}{\normalfont \footnotesize 
herkommen - 
to come here - 
kannst du mal herkommen? - 
can you come here a minute? - 
von wo kommst du denn so spät noch her? - 
where have you come from at (or been until) this late hour? - 
von irgendwo herkommen - 
to come from somewhere - 
irgendwo herkommen - 
to come from somewhere 
}\\
\end{tabular}
}
%===herrschen===
\card{\normalfont \Huge herrschen}{
\begin{tabular}{lll}
\parbox[t][][t]{2.0 cm}{\normalfont \raggedleft ich\\du\\er/sie/es\\wir\\ihr\\sie} &    
\parbox[t][][t]{2cm}{\normalfont herrsche\\herrschst\\herrscht\\herrschen\\herrscht\\herrschen} &
\parbox[t][][t]{2cm}{\normalfont herrschte\\herrschtest\\herrschte\\herrschten\\herrschtet\\herrschten}\\
\end{tabular}
\begin{tabular}{l}
\parbox[t][][t]{8cm}{}\\
\parbox[t][][t]{8cm}{\normalfont \footnotesize 
(über jdn/etw) herrschen - 
to rule (or govern)  ((over) sb/sth) - 
diese Partei herrscht seit 1918 - 
this party has been in power since 1918 - 
herrschen - 
to hold sway - 
herrschen - 
to prevail - 
herrschen - 
to be prevalent 
}\\
\end{tabular}
}
%===herstellen===
\card{\normalfont \Huge herstellen}{
\begin{tabular}{lll}
\parbox[t][][t]{2.0 cm}{\normalfont \raggedleft ich\\du\\er/sie/es\\wir\\ihr\\sie} &    
\parbox[t][][t]{2cm}{\normalfont stelle her\\stellst her\\stellt her\\stellen her\\stellt her\\stellen her} &
\parbox[t][][t]{2cm}{\normalfont stellte her\\stelltest her\\stellte her\\stellten her\\stelltet her\\stellten her}\\
\end{tabular}
\begin{tabular}{l}
\parbox[t][][t]{8cm}{}\\
\parbox[t][][t]{8cm}{\normalfont \footnotesize 
etw herstellen - 
to produce (or manufacture) sth - 
die Schnitzereien sind alle von Hand hergestellt - 
the carvings are all made (or produced) by hand - 
herstellende Industrie - 
producing industry - 
etw herstellen - 
to establish (or make) sth - 
etw (zu jdm/etw) herstellen - 
to put sth here (next to sb/sth) 
}\\
\end{tabular}
}
%===hinterlassen===
\card{\normalfont \Huge hinterlassen}{
\begin{tabular}{lll}
\parbox[t][][t]{2.0 cm}{\normalfont \raggedleft ich\\du\\er/sie/es\\wir\\ihr\\sie} &    
\parbox[t][][t]{2cm}{\normalfont hinterlasse\\hinterlässt\\hinterlässt\\hinterlassen\\hinterlasst\\hinterlassen} &
\parbox[t][][t]{2cm}{\normalfont hinterließ\\hinterließt\\hinterließ\\hinterließen\\hinterließt\\hinterließen}\\
\end{tabular}
\begin{tabular}{l}
\parbox[t][][t]{8cm}{}\\
\parbox[t][][t]{8cm}{\normalfont \footnotesize 
jdm etw hinterlassen - 
to leave (or - 
form bequeath) - 
(or will) sb sth - 
jdn hinterlassen - 
to leave sb - 
er hinterlässt eine Frau und drei Kinder - 
he leaves a wife and three children - 
er hinterlässt eine Frau und drei Kinder - 
he is survived by a wife and three children 
}\\
\end{tabular}
}
%===hoffen===
\card{\normalfont \Huge hoffen}{
\begin{tabular}{lll}
\parbox[t][][t]{2.0 cm}{\normalfont \raggedleft ich\\du\\er/sie/es\\wir\\ihr\\sie} &    
\parbox[t][][t]{2cm}{\normalfont hoffe\\hoffst\\hofft\\hoffen\\hofft\\hoffen} &
\parbox[t][][t]{2cm}{\normalfont hoffte\\hofftest\\hoffte\\hofften\\hofftet\\hofften}\\
\end{tabular}
\begin{tabular}{l}
\parbox[t][][t]{8cm}{}\\
\parbox[t][][t]{8cm}{\normalfont \footnotesize 
hoffen - 
to hope - 
hoffen, dass ... - 
to hope (that) ... - 
auf etw akk - 
hoffen - 
to hope for sth - 
auf jdn hoffen - 
to put one's trust in sb - 
auf Gott hoffen 
}\\
\end{tabular}
}
%===holen===
\card{\normalfont \Huge holen}{
\begin{tabular}{lll}
\parbox[t][][t]{2.0 cm}{\normalfont \raggedleft ich\\du\\er/sie/es\\wir\\ihr\\sie} &    
\parbox[t][][t]{2cm}{\normalfont hole\\holst\\holt\\holen\\holt\\holen} &
\parbox[t][][t]{2cm}{\normalfont holte\\holtest\\holte\\holten\\holtet\\holten}\\
\end{tabular}
\begin{tabular}{l}
\parbox[t][][t]{8cm}{}\\
\parbox[t][][t]{8cm}{\normalfont \footnotesize 
etw holen - 
to get (or fetch) - 
(or go for) sth - 
jdm etw holen - 
to get sb sth (or sth for sb) - 
könntest du mir bitte meine Brille holen? - 
could you please (go and) get me my glasses (or my glasses for me) ? - 
etw (aus etw dat/von etw dat) holen - 
to get sth (out of/from sth) - 
jdn holen 
}\\
\end{tabular}
}
%===hören===
\card{\normalfont \Huge hören}{
\begin{tabular}{lll}
\parbox[t][][t]{2.0 cm}{\normalfont \raggedleft ich\\du\\er/sie/es\\wir\\ihr\\sie} &    
\parbox[t][][t]{2cm}{\normalfont höre\\hörst\\hört\\hören\\hört\\hören} &
\parbox[t][][t]{2cm}{\normalfont hörte\\hörtest\\hörte\\hörten\\hörtet\\hörten}\\
\end{tabular}
\begin{tabular}{l}
\parbox[t][][t]{8cm}{}\\
\parbox[t][][t]{8cm}{\normalfont \footnotesize 
jdn/etw hören - 
to hear sb/sth - 
können Sie mich (noch) hören?, - 
hören Sie mich (noch)? - 
are you (still) able to hear me? - 
ich höre Sie nicht (gut) - 
I can't understand (or hear) you (very well) - 
jdn etw tun hören - 
to hear sb doing sth - 
ich habe dich ja gar nicht kommen hören! 
}\\
\end{tabular}
}
%===informieren===
\card{\normalfont \Huge informieren}{
\begin{tabular}{lll}
\parbox[t][][t]{2.0 cm}{\normalfont \raggedleft ich\\du\\er/sie/es\\wir\\ihr\\sie} &    
\parbox[t][][t]{2cm}{\normalfont informiere\\informierst\\informiert\\informieren\\informiert\\informieren} &
\parbox[t][][t]{2cm}{\normalfont informierte\\informiertest\\informierte\\informierten\\informiertet\\informierten}\\
\end{tabular}
\begin{tabular}{l}
\parbox[t][][t]{8cm}{}\\
\parbox[t][][t]{8cm}{\normalfont \footnotesize 
jdn (über etw akk) informieren - 
to inform sb (about/of sth) - 
jd ist gut informiert - 
sb is well-informed - 
sich akk (über etw akk) informieren - 
to find out (about sth) - 
sich akk (über etw akk) informieren - 
to inform oneself 
}\\
\end{tabular}
}
%===interessieren===
\card{\normalfont \Huge interessieren}{
\begin{tabular}{lll}
\parbox[t][][t]{2.0 cm}{\normalfont \raggedleft ich\\du\\er/sie/es\\wir\\ihr\\sie} &    
\parbox[t][][t]{2cm}{\normalfont interessiere\\interessierst\\interessiert\\interessieren\\interessiert\\interessieren} &
\parbox[t][][t]{2cm}{\normalfont interessierte\\interessiertest\\interessierte\\interessierten\\interessiertet\\interessierten}\\
\end{tabular}
\begin{tabular}{l}
\parbox[t][][t]{8cm}{}\\
\parbox[t][][t]{8cm}{\normalfont \footnotesize 
jdn interessieren - 
to interest sb - 
dein Vorschlag interessiert mich sehr - 
your suggestion interests me greatly - 
das hat Sie nicht zu interessieren! - 
that's no concern of yours! - 
jdn für etw akk - 
interessieren - 
to interest sb in sth - 
sich akk für jdn/etw interessieren 
}\\
\end{tabular}
}
%===kämpfen===
\card{\normalfont \Huge kämpfen}{
\begin{tabular}{lll}
\parbox[t][][t]{2.0 cm}{\normalfont \raggedleft ich\\du\\er/sie/es\\wir\\ihr\\sie} &    
\parbox[t][][t]{2cm}{\normalfont kämpfe\\kämpfst\\kämpft\\kämpfen\\kämpft\\kämpfen} &
\parbox[t][][t]{2cm}{\normalfont kämpfte\\kämpftest\\kämpfte\\kämpften\\kämpftet\\kämpften}\\
\end{tabular}
\begin{tabular}{l}
\parbox[t][][t]{8cm}{}\\
\parbox[t][][t]{8cm}{\normalfont \footnotesize 
(für/gegen jdn/etw) kämpfen - 
to fight (for/against sb/sth) - 
bis auf den letzten Mann kämpfen - 
to fight to the last man - 
(gegen jdn) kämpfen - 
to fight (against sb) - 
(gegen jdn) kämpfen - 
to contend (with sb) - 
um etw akk - 
kämpfen 
}\\
\end{tabular}
}
%===kaufen===
\card{\normalfont \Huge kaufen}{
\begin{tabular}{lll}
\parbox[t][][t]{2.0 cm}{\normalfont \raggedleft ich\\du\\er/sie/es\\wir\\ihr\\sie} &    
\parbox[t][][t]{2cm}{\normalfont kaufe\\kaufst\\kauft\\kaufen\\kauft\\kaufen} &
\parbox[t][][t]{2cm}{\normalfont kaufte\\kauftest\\kaufte\\kauften\\kauftet\\kauften}\\
\end{tabular}
\begin{tabular}{l}
\parbox[t][][t]{8cm}{}\\
\parbox[t][][t]{8cm}{\normalfont \footnotesize 
(jdm/sich) etw kaufen - 
to buy (sb/oneself) sth - 
(jdm/sich) etw kaufen - 
to buy (or - 
form purchase) sth (for sb/oneself) - 
er hat sich ein neues Auto gekauft - 
he('s) bought (himself) a new car - 
ich fange mit dem Kaufen der Geschenke immer viel zu spät an - 
I always start buying the presents much too late - 
etw auf eigene/fremde Rechnung kaufen 
}\\
\end{tabular}
}
%===kehren===
\card{\normalfont \Huge kehren}{
\begin{tabular}{lll}
\parbox[t][][t]{2.0 cm}{\normalfont \raggedleft ich\\du\\er/sie/es\\wir\\ihr\\sie} &    
\parbox[t][][t]{2cm}{\normalfont kehre\\kehrst\\kehrt\\kehren\\kehrt\\kehren} &
\parbox[t][][t]{2cm}{\normalfont kehrte\\kehrtest\\kehrte\\kehrten\\kehrtet\\kehrten}\\
\end{tabular}
\begin{tabular}{l}
\parbox[t][][t]{8cm}{}\\
\parbox[t][][t]{8cm}{\normalfont \footnotesize 
etw irgendwohin kehren - 
to turn sth somewhere - 
kehre die Innenseite nach außen - 
turn it inside out - 
in sich akk gekehrt - 
pensive - 
in sich akk gekehrt - 
lost in thought - 
er ist ein stiller, in sich gekehrter Mensch - 
he is a quiet, introverted person 
}\\
\end{tabular}
}
%===keimen===
\card{\normalfont \Huge keimen}{
\begin{tabular}{lll}
\parbox[t][][t]{2.0 cm}{\normalfont \raggedleft ich\\du\\er/sie/es\\wir\\ihr\\sie} &    
\parbox[t][][t]{2cm}{\normalfont keime\\keimst\\keimt\\keimen\\keimt\\keimen} &
\parbox[t][][t]{2cm}{\normalfont keimte\\keimtest\\keimte\\keimten\\keimtet\\keimten}\\
\end{tabular}
\begin{tabular}{l}
\parbox[t][][t]{8cm}{}\\
\parbox[t][][t]{8cm}{\normalfont \footnotesize 
keimen - 
to germinate - 
die alten Kartoffeln/Zwiebeln fangen an zu keimen - 
the old potatoes/onions are beginning to sprout/ put out shoots - 
diese chemische Behandlung soll die Kartoffeln am Keimen hindern - 
this chemical treatment is supposed to prevent the potatoes (from) sprouting - 
keimen - 
to stir - 
diese Bemerkung ließ bei ihr einen ersten, leisen Verdacht keimen - 
this comment aroused a first sneaking (or slight) suspicion in her 
}\\
\end{tabular}
}
%===kennen===
\card{\normalfont \Huge kennen}{
\begin{tabular}{lll}
\parbox[t][][t]{2.0 cm}{\normalfont \raggedleft ich\\du\\er/sie/es\\wir\\ihr\\sie} &    
\parbox[t][][t]{2cm}{\normalfont kenne\\kennst\\kennt\\kennen\\kennt\\kennen} &
\parbox[t][][t]{2cm}{\normalfont kannte\\kanntest\\kannte\\kannten\\kanntet\\kannten}\\
\end{tabular}
\begin{tabular}{l}
\parbox[t][][t]{8cm}{}\\
\parbox[t][][t]{8cm}{\normalfont \footnotesize 
jdn/etw kennen - 
to know sb/sth - 
ich kenne ihn noch von unserer gemeinsamen Studienzeit - 
I know him from our time at college together - 
kennst du das Buch/diesen Film? - 
have you read this book/seen this film? - 
ich kenne das Gefühl - 
I know the feeling - 
jdn als jdn kennen - 
to know sb as sb 
}\\
\end{tabular}
}
%===klagen===
\card{\normalfont \Huge klagen}{
\begin{tabular}{lll}
\parbox[t][][t]{2.0 cm}{\normalfont \raggedleft ich\\du\\er/sie/es\\wir\\ihr\\sie} &    
\parbox[t][][t]{2cm}{\normalfont klage\\klagst\\klagt\\klagen\\klagt\\klagen} &
\parbox[t][][t]{2cm}{\normalfont klagte\\klagtest\\klagte\\klagten\\klagtet\\klagten}\\
\end{tabular}
\begin{tabular}{l}
\parbox[t][][t]{8cm}{}\\
\parbox[t][][t]{8cm}{\normalfont \footnotesize 
(über etw akk) klagen - 
to moan (or grumble) - 
(or complain)  (about sth) - 
sie klagt regelmäßig über Kopfschmerzen - 
she regularly complains of having headaches - 
um jdn/etw klagen - 
to mourn (for (or over) ) sb/for (or over) sth - 
über etw akk - 
klagen - 
to mourn sth 
}\\
\end{tabular}
}
%===klappen===
\card{\normalfont \Huge klappen}{
\begin{tabular}{lll}
\parbox[t][][t]{2.0 cm}{\normalfont \raggedleft ich\\du\\er/sie/es\\wir\\ihr\\sie} &    
\parbox[t][][t]{2cm}{\normalfont klappe\\klappst\\klappt\\klappen\\klappt\\klappen} &
\parbox[t][][t]{2cm}{\normalfont klappte\\klapptest\\klappte\\klappten\\klapptet\\klappten}\\
\end{tabular}
\begin{tabular}{l}
\parbox[t][][t]{8cm}{}\\
\parbox[t][][t]{8cm}{\normalfont \footnotesize 
etw irgendwohin klappen - 
to fold sth somewhere - 
einen Deckel/eine Klappe nach oben/unten klappen - 
to lift up (or raise) /lower a lid/flap - 
(irgendwie) klappen - 
to work out (somehow) - 
alles hat geklappt - 
everything went as planned (or (off) all right) - 
(ein bisschen Glück und) es könnte klappen - 
it might work (with a bit of luck) 
}\\
\end{tabular}
}
%===kleben===
\card{\normalfont \Huge kleben}{
\begin{tabular}{lll}
\parbox[t][][t]{2.0 cm}{\normalfont \raggedleft ich\\du\\er/sie/es\\wir\\ihr\\sie} &    
\parbox[t][][t]{2cm}{\normalfont klebe\\klebst\\klebt\\kleben\\klebt\\kleben} &
\parbox[t][][t]{2cm}{\normalfont klebte\\klebtest\\klebte\\klebten\\klebtet\\klebten}\\
\end{tabular}
\begin{tabular}{l}
\parbox[t][][t]{8cm}{}\\
\parbox[t][][t]{8cm}{\normalfont \footnotesize 
kleben - 
to be sticky - 
(an etw dat) kleben - 
to stick (to sth) - 
an der Tür kleben - 
to stick on the door - 
(an jdm/etw/in etw dat) kleben bleiben - 
to stick (to sb/sth/in sth) - 
an etw dat - 
kleben 
}\\
\end{tabular}
}
%===klettern===
\card{\normalfont \Huge klettern}{
\begin{tabular}{lll}
\parbox[t][][t]{2.0 cm}{\normalfont \raggedleft ich\\du\\er/sie/es\\wir\\ihr\\sie} &    
\parbox[t][][t]{2cm}{\normalfont klettere\\kletterst\\klettert\\klettern\\klettert\\klettern} &
\parbox[t][][t]{2cm}{\normalfont kletterte\\klettertest\\kletterte\\kletterten\\klettertet\\kletterten}\\
\end{tabular}
\begin{tabular}{l}
\parbox[t][][t]{8cm}{}\\
\parbox[t][][t]{8cm}{\normalfont \footnotesize 
(auf etw akk o dat) klettern - 
to climb ((on) sth) - 
(auf etw akk o dat) klettern - 
(mühsam) - 
to clamber (up (or on) sth) - 
auf einen Baum klettern - 
to climb a tree - 
aufs Dach klettern - 
to climb onto the roof - 
klettern 
}\\
\end{tabular}
}
%===klingen===
\card{\normalfont \Huge klingen}{
\begin{tabular}{lll}
\parbox[t][][t]{2.0 cm}{\normalfont \raggedleft ich\\du\\er/sie/es\\wir\\ihr\\sie} &    
\parbox[t][][t]{2cm}{\normalfont klinge\\klingst\\klingt\\klingen\\klingt\\klingen} &
\parbox[t][][t]{2cm}{\normalfont klang\\klangst\\klang\\klangen\\klangt\\klangen}\\
\end{tabular}
\begin{tabular}{l}
\parbox[t][][t]{8cm}{}\\
\parbox[t][][t]{8cm}{\normalfont \footnotesize 
klingen Glas - 
to clink - 
die Gläser klingen lassen - 
to clink glasses (in a toast) - 
die Gläser klingen lassen Glocke - 
to ring - 
dumpf/hell klingen - 
to have a dull/clear ring - 
klingen - 
to sound 
}\\
\end{tabular}
}
%===klopfen===
\card{\normalfont \Huge klopfen}{
\begin{tabular}{lll}
\parbox[t][][t]{2.0 cm}{\normalfont \raggedleft ich\\du\\er/sie/es\\wir\\ihr\\sie} &    
\parbox[t][][t]{2cm}{\normalfont klopfe\\klopfst\\klopft\\klopfen\\klopft\\klopfen} &
\parbox[t][][t]{2cm}{\normalfont klopfte\\klopftest\\klopfte\\klopften\\klopftet\\klopften}\\
\end{tabular}
\begin{tabular}{l}
\parbox[t][][t]{8cm}{}\\
\parbox[t][][t]{8cm}{\normalfont \footnotesize 
(an etw akk/auf etw akk/gegen etw akk) klopfen - 
to knock (at/on/against sth) (with sth) - 
(gegen etw akk) klopfen - 
Specht - 
to hammer (against sth) - 
jdm auf etw akk - 
klopfen - 
to pat sb on sth - 
jdm auf etw akk - 
klopfen 
}\\
\end{tabular}
}
%===knien===
\card{\normalfont \Huge knien}{
\begin{tabular}{lll}
\parbox[t][][t]{2.0 cm}{\normalfont \raggedleft ich\\du\\er/sie/es\\wir\\ihr\\sie} &    
\parbox[t][][t]{2cm}{\normalfont knie\\kniest\\kniet\\knien\\kniet\\knien} &
\parbox[t][][t]{2cm}{\normalfont kniete\\knietest\\kniete\\knieten\\knietet\\knieten}\\
\end{tabular}
\begin{tabular}{l}
\parbox[t][][t]{8cm}{}\\
\parbox[t][][t]{8cm}{\normalfont \footnotesize 
(auf etw akk/vor jdm/etw) knien - 
to kneel (on sth/in front of (or - 
form before) sb/sth) - 
im Knien - 
on one's knees - 
im Knien - 
kneeling (down) - 
sich akk auf etw akk - 
knien - 
to kneel (down) on sth 
}\\
\end{tabular}
}
%===kochen===
\card{\normalfont \Huge kochen}{
\begin{tabular}{lll}
\parbox[t][][t]{2.0 cm}{\normalfont \raggedleft ich\\du\\er/sie/es\\wir\\ihr\\sie} &    
\parbox[t][][t]{2cm}{\normalfont koche\\kochst\\kocht\\kochen\\kocht\\kochen} &
\parbox[t][][t]{2cm}{\normalfont kochte\\kochtest\\kochte\\kochten\\kochtet\\kochten}\\
\end{tabular}
\begin{tabular}{l}
\parbox[t][][t]{8cm}{}\\
\parbox[t][][t]{8cm}{\normalfont \footnotesize 
kochen - 
to cook - 
dort kocht man sehr scharf/pikant - 
the food there is very hot/spicy - 
kochen - 
to boil - 
etw zum Kochen bringen - 
to bring sth to the boil - 
kochend heiß - 
boiling hot 
}\\
\end{tabular}
}
%===kommen===
\card{\normalfont \Huge kommen}{
\begin{tabular}{lll}
\parbox[t][][t]{2.0 cm}{\normalfont \raggedleft ich\\du\\er/sie/es\\wir\\ihr\\sie} &    
\parbox[t][][t]{2cm}{\normalfont komme\\kommst\\kommt\\kommen\\kommt\\kommen} &
\parbox[t][][t]{2cm}{\normalfont kam\\kamst\\kam\\kamen\\kamt\\kamen}\\
\end{tabular}
\begin{tabular}{l}
\parbox[t][][t]{8cm}{}\\
\parbox[t][][t]{8cm}{\normalfont \footnotesize 
kommen - 
to come - 
kommen - 
to arrive - 
ich bin gerade gekommen - 
I just arrived (or got here) - 
ich komme schon! - 
I'm coming! - 
sie kommen morgen aus Berlin - 
they're arriving (or coming) from Berlin tomorrow 
}\\
\end{tabular}
}
%===können===
\card{\normalfont \Huge können}{
\begin{tabular}{lll}
\parbox[t][][t]{2.0 cm}{\normalfont \raggedleft ich\\du\\er/sie/es\\wir\\ihr\\sie} &    
\parbox[t][][t]{2cm}{\normalfont kann\\kannst\\kann\\können\\könnt\\können} &
\parbox[t][][t]{2cm}{\normalfont konnte\\konntest\\konnte\\konnten\\konntet\\konnten}\\
\end{tabular}
\begin{tabular}{l}
\parbox[t][][t]{8cm}{}\\
\parbox[t][][t]{8cm}{\normalfont \footnotesize 
etw tun können - 
to be able to do sth - 
sie tut, was sie kann - 
she does her best - 
ich werde sehen, was ich tun kann - 
I'll see what I can do - 
was kann man da tun? - 
what can be done in such a case? - 
ich kann da nichts dazu tun - 
I can't do anything to help (or about it) 
}\\
\end{tabular}
}
%===konzentrieren===
\card{\normalfont \Huge konzentrieren}{
\begin{tabular}{lll}
\parbox[t][][t]{2.0 cm}{\normalfont \raggedleft ich\\du\\er/sie/es\\wir\\ihr\\sie} &    
\parbox[t][][t]{2cm}{\normalfont konzentriere\\konzentrierst\\konzentriert\\konzentrieren\\konzentriert\\konzentrieren} &
\parbox[t][][t]{2cm}{\normalfont konzentrierte\\konzentriertest\\konzentrierte\\konzentrierten\\konzentriertet\\konzentrierten}\\
\end{tabular}
\begin{tabular}{l}
\parbox[t][][t]{8cm}{}\\
\parbox[t][][t]{8cm}{\normalfont \footnotesize 
sich akk (auf etw akk) konzentrieren - 
to concentrate (on sth) - 
etw (auf etw akk) konzentrieren - 
to concentrate sth (on sth) - 
etw konzentrieren - 
to concentrate sth - 
konzentrieren - 
concentrate - 
konzentrieren - 
focus 
}\\
\end{tabular}
}
%===korrigieren===
\card{\normalfont \Huge korrigieren}{
\begin{tabular}{lll}
\parbox[t][][t]{2.0 cm}{\normalfont \raggedleft ich\\du\\er/sie/es\\wir\\ihr\\sie} &    
\parbox[t][][t]{2cm}{\normalfont korrigiere\\korrigierst\\korrigiert\\korrigieren\\korrigiert\\korrigieren} &
\parbox[t][][t]{2cm}{\normalfont korrigierte\\korrigiertest\\korrigierte\\korrigierten\\korrigiertet\\korrigierten}\\
\end{tabular}
\begin{tabular}{l}
\parbox[t][][t]{8cm}{}\\
\parbox[t][][t]{8cm}{\normalfont \footnotesize 
etw korrigieren - 
to correct sth - 
eine Klassenarbeit/einen Aufsatz korrigieren - 
to mark a test/an essay - 
ein Manuskript korrigieren - 
to proofread a manuscript - 
korrigiert - 
corrected - 
korrigiert Aufsatz, Arbeit - 
marked 
}\\
\end{tabular}
}
%===kosten===
\card{\normalfont \Huge kosten}{
\begin{tabular}{lll}
\parbox[t][][t]{2.0 cm}{\normalfont \raggedleft ich\\du\\er/sie/es\\wir\\ihr\\sie} &    
\parbox[t][][t]{2cm}{\normalfont koste\\kostest\\kostet\\kosten\\kostet\\kosten} &
\parbox[t][][t]{2cm}{\normalfont kostete\\kostetest\\kostete\\kosteten\\kostetet\\kosteten}\\
\end{tabular}
\begin{tabular}{l}
\parbox[t][][t]{8cm}{}\\
\parbox[t][][t]{8cm}{\normalfont \footnotesize 
etw kosten - 
to cost sth - 
jdn etw kosten - 
to cost sb sth - 
der Computer hat mich 1.000 Euro gekostet - 
the computer cost me 1,000 euros - 
sich dat etw etwas kosten lassen fam - 
to be prepared to spend a lot on sth fam - 
jdn etw kosten - 
to take (up) sb's sth 
}\\
\end{tabular}
}
%===kreischen===
\card{\normalfont \Huge kreischen}{
\begin{tabular}{lll}
\parbox[t][][t]{2.0 cm}{\normalfont \raggedleft ich\\du\\er/sie/es\\wir\\ihr\\sie} &    
\parbox[t][][t]{2cm}{\normalfont kreische\\kreischst\\kreischt\\kreischen\\kreischt\\kreischen} &
\parbox[t][][t]{2cm}{\normalfont kreischte\\kreischtest\\kreischte\\kreischten\\kreischtet\\kreischten}\\
\end{tabular}
\begin{tabular}{l}
\parbox[t][][t]{8cm}{}\\
\parbox[t][][t]{8cm}{\normalfont \footnotesize 
kreischen - 
to squawk - 
kreischen - 
to squeal - 
kreischen - 
to shriek - 
kreischen - 
to screech 
}\\
\end{tabular}
}
%===kriechen===
\card{\normalfont \Huge kriechen}{
\begin{tabular}{lll}
\parbox[t][][t]{2.0 cm}{\normalfont \raggedleft ich\\du\\er/sie/es\\wir\\ihr\\sie} &    
\parbox[t][][t]{2cm}{\normalfont krieche\\kriechst\\kriecht\\kriechen\\kriecht\\kriechen} &
\parbox[t][][t]{2cm}{\normalfont kroch\\krochst\\kroch\\krochen\\krocht\\krochen}\\
\end{tabular}
\begin{tabular}{l}
\parbox[t][][t]{8cm}{}\\
\parbox[t][][t]{8cm}{\normalfont \footnotesize 
(irgendwohin) kriechen - 
to crawl (somewhere) - 
nicht mehr kriechen können - 
to be on one's last legs - 
kriechen - 
to creep by - 
kriechen - 
to creep (or crawl)  (along) - 
(vor jdm) kriechen - 
to grovel (before sb) 
}\\
\end{tabular}
}
%===kriegen===
\card{\normalfont \Huge kriegen}{
\begin{tabular}{lll}
\parbox[t][][t]{2.0 cm}{\normalfont \raggedleft ich\\du\\er/sie/es\\wir\\ihr\\sie} &    
\parbox[t][][t]{2cm}{\normalfont kriege\\kriegst\\kriegt\\kriegen\\kriegt\\kriegen} &
\parbox[t][][t]{2cm}{\normalfont kriegte\\kriegtest\\kriegte\\kriegten\\kriegtet\\kriegten}\\
\end{tabular}
\begin{tabular}{l}
\parbox[t][][t]{8cm}{}\\
\parbox[t][][t]{8cm}{\normalfont \footnotesize 
etw (von jdm) kriegen - 
to get sth (from sb) - 
ich nehme diesen Ring, was kriegen Sie dafür (von mir)? - 
I'll take this ring, what do you want for it (or what do I owe you for it) ? - 
ich kriege noch 20 Euro von dir - 
you still owe me 20 euros - 
das Buch ist nirgends zu kriegen - 
you can't get that book anywhere - 
etw getan kriegen - 
to get sth done 
}\\
\end{tabular}
}
%===kümmern===
\card{\normalfont \Huge kümmern}{
\begin{tabular}{lll}
\parbox[t][][t]{2.0 cm}{\normalfont \raggedleft ich\\du\\er/sie/es\\wir\\ihr\\sie} &    
\parbox[t][][t]{2cm}{\normalfont kümmere\\kümmerst\\kümmert\\kümmern\\kümmert\\kümmern} &
\parbox[t][][t]{2cm}{\normalfont kümmerte\\kümmertest\\kümmerte\\kümmerten\\kümmertet\\kümmerten}\\
\end{tabular}
\begin{tabular}{l}
\parbox[t][][t]{8cm}{}\\
\parbox[t][][t]{8cm}{\normalfont \footnotesize 
etw/jd kümmert jdn - 
sth/sb concerns sb - 
was kümmert mich das? - 
what concern is that of mine? - 
es hat ihn noch nie gekümmert, was andere von ihm dachten - 
it never worried him what other people thought of him - 
das traurige Kind kümmert mich - 
I feel sorry for the sad child - 
kümmern - 
to become stunted 
}\\
\end{tabular}
}
%===kündigen===
\card{\normalfont \Huge kündigen}{
\begin{tabular}{lll}
\parbox[t][][t]{2.0 cm}{\normalfont \raggedleft ich\\du\\er/sie/es\\wir\\ihr\\sie} &    
\parbox[t][][t]{2cm}{\normalfont kündige\\kündigst\\kündigt\\kündigen\\kündigt\\kündigen} &
\parbox[t][][t]{2cm}{\normalfont kündigte\\kündigtest\\kündigte\\kündigten\\kündigtet\\kündigten}\\
\end{tabular}
\begin{tabular}{l}
\parbox[t][][t]{8cm}{}\\
\parbox[t][][t]{8cm}{\normalfont \footnotesize 
etw kündigen - 
to hand in one's notice - 
etw kündigen - 
to quit - 
seine Arbeit/seinen Job/seine Stelle kündigen - 
to hand in one's notice - 
kündigen - 
to cancel - 
kündigen - 
to terminate 
}\\
\end{tabular}
}
%===kürzen===
\card{\normalfont \Huge kürzen}{
\begin{tabular}{lll}
\parbox[t][][t]{2.0 cm}{\normalfont \raggedleft ich\\du\\er/sie/es\\wir\\ihr\\sie} &    
\parbox[t][][t]{2cm}{\normalfont kürze\\kürzt\\kürzt\\kürzen\\kürzt\\kürzen} &
\parbox[t][][t]{2cm}{\normalfont kürzte\\kürztest\\kürzte\\kürzten\\kürztet\\kürzten}\\
\end{tabular}
\begin{tabular}{l}
\parbox[t][][t]{8cm}{}\\
\parbox[t][][t]{8cm}{\normalfont \footnotesize 
etw (um etw akk) kürzen - 
to shorten sth (by sth) - 
können Sie mir die Hose um einen Zentimeter kürzen? - 
can you shorten these trousers for me by a centimetre? - 
etw kürzen - 
to shorten sth - 
ich habe meinen Artikel um die Hälfte gekürzt - 
I've shortened my article by fifty percent - 
das Buch wurde vom Verlag auf lediglich 150 Seiten gekürzt - 
the publishers shortened the book to a mere 150 pages 
}\\
\end{tabular}
}
%===lächeln===
\card{\normalfont \Huge lächeln}{
\begin{tabular}{lll}
\parbox[t][][t]{2.0 cm}{\normalfont \raggedleft ich\\du\\er/sie/es\\wir\\ihr\\sie} &    
\parbox[t][][t]{2cm}{\normalfont lächle\\lächelst\\lächelt\\lächeln\\lächelt\\lächeln} &
\parbox[t][][t]{2cm}{\normalfont lächelte\\lächeltest\\lächelte\\lächelten\\lächeltet\\lächelten}\\
\end{tabular}
\begin{tabular}{l}
\parbox[t][][t]{8cm}{}\\
\parbox[t][][t]{8cm}{\normalfont \footnotesize 
lächeln - 
to smile - 
(über jdn/etw) lächeln - 
to grin (or smirk)  (at sb/sth) - 
Lächeln - 
smile - 
ein müdes Lächeln - 
a weary smile 
}\\
\end{tabular}
}
%===lachen===
\card{\normalfont \Huge lachen}{
\begin{tabular}{lll}
\parbox[t][][t]{2.0 cm}{\normalfont \raggedleft ich\\du\\er/sie/es\\wir\\ihr\\sie} &    
\parbox[t][][t]{2cm}{\normalfont lache\\lachst\\lacht\\lachen\\lacht\\lachen} &
\parbox[t][][t]{2cm}{\normalfont lachte\\lachtest\\lachte\\lachten\\lachtet\\lachten}\\
\end{tabular}
\begin{tabular}{l}
\parbox[t][][t]{8cm}{}\\
\parbox[t][][t]{8cm}{\normalfont \footnotesize 
lachen - 
to laugh - 
lach du nur! fam - 
you can laugh! fam - 
das wäre doch gelacht fam - 
it would be ridiculous - 
über etw akk - 
lachen - 
to laugh at sth - 
breit lachen 
}\\
\end{tabular}
}
%===laden===
\card{\normalfont \Huge laden}{
\begin{tabular}{lll}
\parbox[t][][t]{2.0 cm}{\normalfont \raggedleft ich\\du\\er/sie/es\\wir\\ihr\\sie} &    
\parbox[t][][t]{2cm}{\normalfont lade\\lädst\\lädt\\laden\\ladet\\laden} &
\parbox[t][][t]{2cm}{\normalfont lud\\ludest\\lud\\luden\\ludet\\luden}\\
\end{tabular}
\begin{tabular}{l}
\parbox[t][][t]{8cm}{}\\
\parbox[t][][t]{8cm}{\normalfont \footnotesize 
laden - 
beladen - 
Add to my favourites Preselect for export to vocabulary trainer View selected vocabulary - 
to be laden with sth - 
mit etw dat beladen sein - 
Add to my favourites Preselect for export to vocabulary trainer View selected vocabulary - 
the table was laden with food - 
der Tisch war überreichlich gedeckt - 
Add to my favourites Preselect for export to vocabulary trainer View selected vocabulary - 
laden with presents for everyone 
}\\
\end{tabular}
}
%===landen===
\card{\normalfont \Huge landen}{
\begin{tabular}{lll}
\parbox[t][][t]{2.0 cm}{\normalfont \raggedleft ich\\du\\er/sie/es\\wir\\ihr\\sie} &    
\parbox[t][][t]{2cm}{\normalfont lande\\landest\\landet\\landen\\landet\\landen} &
\parbox[t][][t]{2cm}{\normalfont landete\\landetest\\landete\\landeten\\landetet\\landeten}\\
\end{tabular}
\begin{tabular}{l}
\parbox[t][][t]{8cm}{}\\
\parbox[t][][t]{8cm}{\normalfont \footnotesize 
landen Flugzeug, Raumschiff, Vogel - 
to land - 
(auf etw dat/in einer Stadt) landen - 
to land (on sth/in a city) - 
auf dem Mond landen - 
to land on the moon - 
irgendwo landen - 
to land somewhere - 
das Schiff ist auf einer Sandbank gelandet - 
the ship ran aground on a sandbank 
}\\
\end{tabular}
}
%===lassen===
\card{\normalfont \Huge lassen}{
\begin{tabular}{lll}
\parbox[t][][t]{2.0 cm}{\normalfont \raggedleft ich\\du\\er/sie/es\\wir\\ihr\\sie} &    
\parbox[t][][t]{2cm}{\normalfont lasse\\lässt\\lässt\\lassen\\lasst\\lassen} &
\parbox[t][][t]{2cm}{\normalfont ließ\\ließt \\ließ\\ließen\\ließt\\ließen}\\
\end{tabular}
\begin{tabular}{l}
\parbox[t][][t]{8cm}{}\\
\parbox[t][][t]{8cm}{\normalfont \footnotesize 
etw lassen - 
to stop sth - 
etw lassen - 
(verzichten) - 
to refrain from doing sth - 
etw lassen - 
(nicht tun) - 
to not do sth - 
etw lassen - 
(sich nicht bemühen) 
}\\
\end{tabular}
}
%===laufen===
\card{\normalfont \Huge laufen}{
\begin{tabular}{lll}
\parbox[t][][t]{2.0 cm}{\normalfont \raggedleft ich\\du\\er/sie/es\\wir\\ihr\\sie} &    
\parbox[t][][t]{2cm}{\normalfont laufe\\läufst\\läuft\\laufen\\lauft\\laufen} &
\parbox[t][][t]{2cm}{\normalfont lief\\liefst\\lief\\liefen\\lieft\\liefen}\\
\end{tabular}
\begin{tabular}{l}
\parbox[t][][t]{8cm}{}\\
\parbox[t][][t]{8cm}{\normalfont \footnotesize 
laufen - 
to run - 
sie lief, um die Straßenbahn noch zu erwischen - 
she ran to catch the tram - 
sie lief, was sie nur konnte - 
she ran as fast as she could - 
so lauf doch! - 
come on, hurry up! - 
aus dem Haus laufen - 
to run out of the house 
}\\
\end{tabular}
}
%===leben===
\card{\normalfont \Huge leben}{
\begin{tabular}{lll}
\parbox[t][][t]{2.0 cm}{\normalfont \raggedleft ich\\du\\er/sie/es\\wir\\ihr\\sie} &    
\parbox[t][][t]{2cm}{\normalfont lebe\\lebst\\lebt\\leben\\lebt\\leben} &
\parbox[t][][t]{2cm}{\normalfont lebte\\lebtest\\lebte\\lebten\\lebtet\\lebten}\\
\end{tabular}
\begin{tabular}{l}
\parbox[t][][t]{8cm}{}\\
\parbox[t][][t]{8cm}{\normalfont \footnotesize 
leben - 
to live - 
Gott sei Dank, er lebt (noch) - 
Thank God, he's (still) alive - 
lebst du noch? hum - 
are you still in the land of the living? hum - 
man lebt nur einmal - 
you only have one life to live - 
daraufhin wollte er nicht mehr leben - 
after that he didn't want to go on living 
}\\
\end{tabular}
}
%===legen===
\card{\normalfont \Huge legen}{
\begin{tabular}{lll}
\parbox[t][][t]{2.0 cm}{\normalfont \raggedleft ich\\du\\er/sie/es\\wir\\ihr\\sie} &    
\parbox[t][][t]{2cm}{\normalfont lege\\legst\\legt\\legen\\legt\\legen} &
\parbox[t][][t]{2cm}{\normalfont legte\\legtest\\legte\\legten\\legtet\\legten}\\
\end{tabular}
\begin{tabular}{l}
\parbox[t][][t]{8cm}{}\\
\parbox[t][][t]{8cm}{\normalfont \footnotesize 
etw legen - 
to put (or place) sth on its side - 
etw irgendwohin legen - 
to put sth somewhere - 
die Betonung auf ein Wort legen - 
to stress a word - 
er legte den Kopf an ihre Schulter - 
he leaned his head on her shoulder - 
jdm eine Binde/die Hände vor die Augen legen - 
to blindfold sb/to put one's hands over sb's eyes 
}\\
\end{tabular}
}
%===lehnen===
\card{\normalfont \Huge lehnen}{
\begin{tabular}{lll}
\parbox[t][][t]{2.0 cm}{\normalfont \raggedleft ich\\du\\er/sie/es\\wir\\ihr\\sie} &    
\parbox[t][][t]{2cm}{\normalfont lehne\\lehnst\\lehnt\\lehnen\\lehnt\\lehnen} &
\parbox[t][][t]{2cm}{\normalfont lehnte\\lehntest\\lehnte\\lehnten\\lehntet\\lehnten}\\
\end{tabular}
\begin{tabular}{l}
\parbox[t][][t]{8cm}{}\\
\parbox[t][][t]{8cm}{\normalfont \footnotesize 
etw an etw akk/gegen etw akk - 
lehnen - 
to lean sth against sth - 
an etw dat - 
lehnen - 
to lean against sth - 
sich akk an jdn/etw lehnen - 
to lean on sb/sth - 
sich akk über etw akk - 
lehnen 
}\\
\end{tabular}
}
%===lehren===
\card{\normalfont \Huge lehren}{
\begin{tabular}{lll}
\parbox[t][][t]{2.0 cm}{\normalfont \raggedleft ich\\du\\er/sie/es\\wir\\ihr\\sie} &    
\parbox[t][][t]{2cm}{\normalfont lehre\\lehrst\\lehrt\\lehren\\lehrt\\lehren} &
\parbox[t][][t]{2cm}{\normalfont lehrte\\lehrtest\\lehrte\\lehrten\\lehrtet\\lehrten}\\
\end{tabular}
\begin{tabular}{l}
\parbox[t][][t]{8cm}{}\\
\parbox[t][][t]{8cm}{\normalfont \footnotesize 
etw lehren - 
to teach sth - 
etw lehren - 
(an der Uni) - 
to lecture in sth - 
jdn (etw akk) lehren - 
to teach sb (sth) - 
wer hat dich zeichnen gelehrt? - 
who taught you to draw? - 
das lehrte ihn das Fürchten 
}\\
\end{tabular}
}
%===leiden===
\card{\normalfont \Huge leiden}{
\begin{tabular}{lll}
\parbox[t][][t]{2.0 cm}{\normalfont \raggedleft ich\\du\\er/sie/es\\wir\\ihr\\sie} &    
\parbox[t][][t]{2cm}{\normalfont leide\\leidest\\leidet\\leiden\\leidet\\leiden} &
\parbox[t][][t]{2cm}{\normalfont litt\\littest\\litt\\litten\\littet\\litten}\\
\end{tabular}
\begin{tabular}{l}
\parbox[t][][t]{8cm}{}\\
\parbox[t][][t]{8cm}{\normalfont \footnotesize 
leiden - 
to suffer - 
an etw dat - 
leiden - 
to suffer from sth - 
leiden - 
to suffer - 
unter jdm leiden - 
to suffer because of sb - 
unter etw dat 
}\\
\end{tabular}
}
%===leihen===
\card{\normalfont \Huge leihen}{
\begin{tabular}{lll}
\parbox[t][][t]{2.0 cm}{\normalfont \raggedleft ich\\du\\er/sie/es\\wir\\ihr\\sie} &    
\parbox[t][][t]{2cm}{\normalfont leihe\\leihst\\leiht\\leihen\\leiht\\leihen} &
\parbox[t][][t]{2cm}{\normalfont lieh\\liehst\\lieh\\liehen\\lieht\\liehen}\\
\end{tabular}
\begin{tabular}{l}
\parbox[t][][t]{8cm}{}\\
\parbox[t][][t]{8cm}{\normalfont \footnotesize 
jdm etw leihen - 
to lend sb sth - 
geliehen - 
borrowed - 
sich dat etw (von jdm) leihen - 
to borrow sth (from sb) - 
leihen (entleihen) - 
borrow - 
leihen (verleihen) - 
lend 
}\\
\end{tabular}
}
%===leisten===
\card{\normalfont \Huge leisten}{
\begin{tabular}{lll}
\parbox[t][][t]{2.0 cm}{\normalfont \raggedleft ich\\du\\er/sie/es\\wir\\ihr\\sie} &    
\parbox[t][][t]{2cm}{\normalfont leiste\\leistest\\leistet\\leisten\\leistet\\leisten} &
\parbox[t][][t]{2cm}{\normalfont leistete\\leistetest\\leistete\\leisteten\\leistetet\\leisteten}\\
\end{tabular}
\begin{tabular}{l}
\parbox[t][][t]{8cm}{}\\
\parbox[t][][t]{8cm}{\normalfont \footnotesize 
etw leisten - 
to achieve sth - 
etw leisten - 
(an Arbeitsleistung erbringen) - 
to do sth - 
für heute haben wir genug geleistet - 
we've done enough for today - 
ich hatte gehofft, sie würde mehr leisten - 
I had hoped she would do a better job - 
(etwas) Anerkennenswertes/Besonderes/Erstaunliches leisten 
}\\
\end{tabular}
}
%===leiten===
\card{\normalfont \Huge leiten}{
\begin{tabular}{lll}
\parbox[t][][t]{2.0 cm}{\normalfont \raggedleft ich\\du\\er/sie/es\\wir\\ihr\\sie} &    
\parbox[t][][t]{2cm}{\normalfont leite\\leitest\\leitet\\leiten\\leitet\\leiten} &
\parbox[t][][t]{2cm}{\normalfont leitete\\leitetest\\leitete\\leiteten\\leitetet\\leiteten}\\
\end{tabular}
\begin{tabular}{l}
\parbox[t][][t]{8cm}{}\\
\parbox[t][][t]{8cm}{\normalfont \footnotesize 
etw leiten - 
to run (or be in charge of) sth - 
eine Abteilung leiten - 
to be head of (or run)  a department - 
eine Firma leiten - 
to run (or manage)  a company - 
ein Labor/eine Redaktion leiten - 
to be head (or in charge) of a laboratory/an editorial office - 
eine Schule leiten - 
to be head (or headmaster) 
}\\
\end{tabular}
}
%===lesen===
\card{\normalfont \Huge lesen}{
\begin{tabular}{lll}
\parbox[t][][t]{2.0 cm}{\normalfont \raggedleft ich\\du\\er/sie/es\\wir\\ihr\\sie} &    
\parbox[t][][t]{2cm}{\normalfont lese\\liest\\liest\\lesen\\lest\\lesen} &
\parbox[t][][t]{2cm}{\normalfont las\\lasest\\las\\lasen\\last\\lasen}\\
\end{tabular}
\begin{tabular}{l}
\parbox[t][][t]{8cm}{}\\
\parbox[t][][t]{8cm}{\normalfont \footnotesize 
etw lesen - 
to read sth - 
etw lesen - 
to proofread (or read through (and correct)) sth - 
einfach/kaum/nicht/schwer zu lesen sein - 
to be easy/almost impossible/impossible/difficult to read - 
etw (in etw akk) lesen - 
to read sth (into sth) - 
etw aus etw dat - 
lesen 
}\\
\end{tabular}
}
%===lieben===
\card{\normalfont \Huge lieben}{
\begin{tabular}{lll}
\parbox[t][][t]{2.0 cm}{\normalfont \raggedleft ich\\du\\er/sie/es\\wir\\ihr\\sie} &    
\parbox[t][][t]{2cm}{\normalfont liebe\\liebst\\liebt\\lieben\\liebt\\lieben} &
\parbox[t][][t]{2cm}{\normalfont liebte\\liebtest\\liebte\\liebten\\liebtet\\liebten}\\
\end{tabular}
\begin{tabular}{l}
\parbox[t][][t]{8cm}{}\\
\parbox[t][][t]{8cm}{\normalfont \footnotesize 
jdn lieben - 
to love sb - 
sich akk - 
lieben - 
to love each other (or one another) - 
jdn/etw lieben lernen - 
to come (or learn) to love sb/sth - 
sich akk - 
lieben lernen - 
to come (or learn) to love each other (or one another) 
}\\
\end{tabular}
}
%===liefern===
\card{\normalfont \Huge liefern}{
\begin{tabular}{lll}
\parbox[t][][t]{2.0 cm}{\normalfont \raggedleft ich\\du\\er/sie/es\\wir\\ihr\\sie} &    
\parbox[t][][t]{2cm}{\normalfont liefere\\lieferst\\liefert\\liefern\\liefert\\liefern} &
\parbox[t][][t]{2cm}{\normalfont lieferte\\liefertest\\lieferte\\lieferten\\liefertet\\lieferten}\\
\end{tabular}
\begin{tabular}{l}
\parbox[t][][t]{8cm}{}\\
\parbox[t][][t]{8cm}{\normalfont \footnotesize 
(jdm) etw liefern - 
to deliver sth (to sb) - 
(jdm) etw liefern - 
to supply (sb with) sth - 
etw an jdn/etw liefern - 
to deliver sth to sb/sth - 
(jdm) etw liefern - 
to provide sth (for sb) - 
etw liefern - 
to yield sth 
}\\
\end{tabular}
}
%===liegen===
\card{\normalfont \Huge liegen}{
\begin{tabular}{lll}
\parbox[t][][t]{2.0 cm}{\normalfont \raggedleft ich\\du\\er/sie/es\\wir\\ihr\\sie} &    
\parbox[t][][t]{2cm}{\normalfont liege\\liegst\\liegt\\liegen\\liegt\\liegen} &
\parbox[t][][t]{2cm}{\normalfont lag\\lagst\\lag\\lagen\\lagt\\lagen}\\
\end{tabular}
\begin{tabular}{l}
\parbox[t][][t]{8cm}{}\\
\parbox[t][][t]{8cm}{\normalfont \footnotesize 
liegen - 
to lie - 
ich liege noch im Bett - 
I'm still (lying) in bed - 
während der Krankheit musste sie liegen - 
while she was ill she had to lie down all the time - 
Weinflaschen müssen liegen - 
wine bottles should lie flat - 
hast du irgendwo meine Schlüssel liegen gesehen? - 
have you seen my keys lying (around) anywhere? 
}\\
\end{tabular}
}
%===loben===
\card{\normalfont \Huge loben}{
\begin{tabular}{lll}
\parbox[t][][t]{2.0 cm}{\normalfont \raggedleft ich\\du\\er/sie/es\\wir\\ihr\\sie} &    
\parbox[t][][t]{2cm}{\normalfont lobe\\lobst\\lobt\\loben\\lobt\\loben} &
\parbox[t][][t]{2cm}{\normalfont lobte\\lobtest\\lobte\\lobten\\lobtet\\lobten}\\
\end{tabular}
\begin{tabular}{l}
\parbox[t][][t]{8cm}{}\\
\parbox[t][][t]{8cm}{\normalfont \footnotesize 
jdn/etw loben - 
to praise sb/sth - 
sich akk - 
loben - 
to praise oneself - 
zu loben sein - 
to be praiseworthy (or worthy of praise) - 
sich dat etw loben - 
to prefer sth - 
da lobe ich mir die guten alten Zeiten 
}\\
\end{tabular}
}
%===lohnen===
\card{\normalfont \Huge lohnen}{
\begin{tabular}{lll}
\parbox[t][][t]{2.0 cm}{\normalfont \raggedleft ich\\du\\er/sie/es\\wir\\ihr\\sie} &    
\parbox[t][][t]{2cm}{\normalfont lohne\\lohnst\\lohnt\\lohnen\\lohnt\\lohnen} &
\parbox[t][][t]{2cm}{\normalfont lohnte\\lohntest\\lohnte\\lohnten\\lohntet\\lohnten}\\
\end{tabular}
\begin{tabular}{l}
\parbox[t][][t]{8cm}{}\\
\parbox[t][][t]{8cm}{\normalfont \footnotesize 
sich akk (für jdn) lohnen - 
to be worthwhile (or worth it)  (for sb) - 
unsere Mühe hat sich gelohnt - 
it was worth the effort (or trouble) - 
unsere Mühe hat sich gelohnt - 
our efforts were worth it (or worthwhile) - 
sich akk - 
lohnen - 
to be worth seeing (or going to see) - 
sich akk 
}\\
\end{tabular}
}
%===löschen===
\card{\normalfont \Huge löschen}{
\begin{tabular}{lll}
\parbox[t][][t]{2.0 cm}{\normalfont \raggedleft ich\\du\\er/sie/es\\wir\\ihr\\sie} &    
\parbox[t][][t]{2cm}{\normalfont lösche\\löschst\\löscht\\löschen\\löscht\\löschen} &
\parbox[t][][t]{2cm}{\normalfont löschte\\löschtest\\löschte\\löschten\\löschtet\\löschten}\\
\end{tabular}
\begin{tabular}{l}
\parbox[t][][t]{8cm}{}\\
\parbox[t][][t]{8cm}{\normalfont \footnotesize 
etw löschen - 
Feuer, Flammen - 
to extinguish (or - 
sep put out) sth (with sth) - 
das Licht löschen - 
to switch (or turn) off (or out) the light(s) sep - 
das Licht löschen - 
to put out the light(s) sep - 
etw löschen - 
to delete (or remove) sth 
}\\
\end{tabular}
}
%===lösen===
\card{\normalfont \Huge lösen}{
\begin{tabular}{lll}
\parbox[t][][t]{2.0 cm}{\normalfont \raggedleft ich\\du\\er/sie/es\\wir\\ihr\\sie} &    
\parbox[t][][t]{2cm}{\normalfont löse\\löst\\löst\\lösen\\löst\\lösen} &
\parbox[t][][t]{2cm}{\normalfont löste\\löstest\\löste\\lösten\\löstet\\lösten}\\
\end{tabular}
\begin{tabular}{l}
\parbox[t][][t]{8cm}{}\\
\parbox[t][][t]{8cm}{\normalfont \footnotesize 
etw (von etw dat) lösen - 
to remove sth (from sth) - 
eine Briefmarke von einem Umschlag lösen - 
to get a stamp off from an envelope - 
das Fleisch vom Knochen lösen - 
to take the meat off the bone - 
den Schmutz lösen - 
to remove the dirt - 
etw aus dem Zusammenhang lösen - 
fig 
}\\
\end{tabular}
}
%===lügen===
\card{\normalfont \Huge lügen}{
\begin{tabular}{lll}
\parbox[t][][t]{2.0 cm}{\normalfont \raggedleft ich\\du\\er/sie/es\\wir\\ihr\\sie} &    
\parbox[t][][t]{2cm}{\normalfont lüge\\lügst\\lügt\\lügen\\lügt\\lügen} &
\parbox[t][][t]{2cm}{\normalfont log\\logst\\log\\logen\\logt\\logen}\\
\end{tabular}
\begin{tabular}{l}
\parbox[t][][t]{8cm}{}\\
\parbox[t][][t]{8cm}{\normalfont \footnotesize 
etw lügen - 
to make up sth sep - 
das Blaue vom Himmel herunterlügen - 
to charm the birds out of the trees - 
lügen - 
to lie - 
etw ist gelogen - 
sth is a lie - 
das ist alles gelogen - 
that's a total lie 
}\\
\end{tabular}
}
%===machen===
\card{\normalfont \Huge machen}{
\begin{tabular}{lll}
\parbox[t][][t]{2.0 cm}{\normalfont \raggedleft ich\\du\\er/sie/es\\wir\\ihr\\sie} &    
\parbox[t][][t]{2cm}{\normalfont mache\\machst\\macht\\machen\\macht\\machen} &
\parbox[t][][t]{2cm}{\normalfont machte\\machtest\\machte\\machten\\machtet\\machten}\\
\end{tabular}
\begin{tabular}{l}
\parbox[t][][t]{8cm}{}\\
\parbox[t][][t]{8cm}{\normalfont \footnotesize 
etw machen - 
to do sth - 
hast du die Kartoffeln/Türen/das Badezimmer gemacht? - 
have you done the potatoes/doors/bathroom? - 
etw machen - 
to make sth - 
Fotos (von jdm/etw) machen - 
to take photos (of sb/sth) - 
Gedichte machen - 
to write poems 
}\\
\end{tabular}
}
%===mahlen===
\card{\normalfont \Huge mahlen}{
\begin{tabular}{lll}
\parbox[t][][t]{2.0 cm}{\normalfont \raggedleft ich\\du\\er/sie/es\\wir\\ihr\\sie} &    
\parbox[t][][t]{2cm}{\normalfont mahle\\mahlst\\mahlt\\mahlen\\mahlt\\mahlen} &
\parbox[t][][t]{2cm}{\normalfont mahlte\\mahltest\\mahlte\\mahlten\\mahltet\\mahlten}\\
\end{tabular}
\begin{tabular}{l}
\parbox[t][][t]{8cm}{}\\
\parbox[t][][t]{8cm}{\normalfont \footnotesize 
etw (zu etw dat) mahlen - 
to grind sth (into sth) - 
Getreide mahlen - 
to grind grain - 
Mehl mahlen - 
to grind flour - 
gemahlen - 
ground - 
gemahlener Kaffee - 
ground coffee 
}\\
\end{tabular}
}
%===malen===
\card{\normalfont \Huge malen}{
\begin{tabular}{lll}
\parbox[t][][t]{2.0 cm}{\normalfont \raggedleft ich\\du\\er/sie/es\\wir\\ihr\\sie} &    
\parbox[t][][t]{2cm}{\normalfont male\\malst\\malt\\malen\\malt\\malen} &
\parbox[t][][t]{2cm}{\normalfont malte\\maltest\\malte\\malten\\maltet\\malten}\\
\end{tabular}
\begin{tabular}{l}
\parbox[t][][t]{8cm}{}\\
\parbox[t][][t]{8cm}{\normalfont \footnotesize 
malen - 
to paint - 
malen (künstlerisch darstellen) - 
to paint - 
ein Bild/Porträt malen - 
to paint a picture/portrait - 
Schilder malen - 
to paint signs - 
einen Hintergrund malen - 
to paint a background 
}\\
\end{tabular}
}
%===meiden===
\card{\normalfont \Huge meiden}{
\begin{tabular}{lll}
\parbox[t][][t]{2.0 cm}{\normalfont \raggedleft ich\\du\\er/sie/es\\wir\\ihr\\sie} &    
\parbox[t][][t]{2cm}{\normalfont meide\\meidest\\meidet\\meiden\\meidet\\meiden} &
\parbox[t][][t]{2cm}{\normalfont mied\\miedest\\mied\\mieden\\miedet\\mieden}\\
\end{tabular}
\begin{tabular}{l}
\parbox[t][][t]{8cm}{}\\
\parbox[t][][t]{8cm}{\normalfont \footnotesize 
jdn meiden - 
to avoid (or steer clear of) sb - 
etw meiden - 
to avoid sth - 
Alkohol meiden - 
to avoid (or abstain from) - 
(or - 
form eschew) alcohol - 
etw meiden (scheuen, ausweichen) - 
to shun sth 
}\\
\end{tabular}
}
%===meinen===
\card{\normalfont \Huge meinen}{
\begin{tabular}{lll}
\parbox[t][][t]{2.0 cm}{\normalfont \raggedleft ich\\du\\er/sie/es\\wir\\ihr\\sie} &    
\parbox[t][][t]{2cm}{\normalfont meine\\meinst\\meint\\meinen\\meint\\meinen} &
\parbox[t][][t]{2cm}{\normalfont meinte\\meintest\\meinte\\meinten\\meintet\\meinten}\\
\end{tabular}
\begin{tabular}{l}
\parbox[t][][t]{8cm}{}\\
\parbox[t][][t]{8cm}{\normalfont \footnotesize 
meinen(, dass) - 
to think (or - 
fam reckon)  (that) - 
ich würde/man möchte meinen, ... - 
I/one (or you) would think ... - 
meinen Sie? - 
(do) you think so? (or - 
fam reckon (so)) - 
meinen - 
to say 
}\\
\end{tabular}
}
%===melden===
\card{\normalfont \Huge melden}{
\begin{tabular}{lll}
\parbox[t][][t]{2.0 cm}{\normalfont \raggedleft ich\\du\\er/sie/es\\wir\\ihr\\sie} &    
\parbox[t][][t]{2cm}{\normalfont melde\\meldest\\meldet\\melden\\meldet\\melden} &
\parbox[t][][t]{2cm}{\normalfont meldete\\meldetest\\meldete\\meldeten\\meldetet\\meldeten}\\
\end{tabular}
\begin{tabular}{l}
\parbox[t][][t]{8cm}{}\\
\parbox[t][][t]{8cm}{\normalfont \footnotesize 
(jdm) etw melden - 
to report sth (to sb) - 
jdn (bei jdm) melden - 
to report sb (to sb) - 
der Behörde eine Adressänderung melden - 
to notify the authorities of a change of address - 
eine Geburt/einen Todesfall melden - 
to register a birth/a death - 
etw im Personalbüro melden - 
to report sth to the personnel office 
}\\
\end{tabular}
}
%===merken===
\card{\normalfont \Huge merken}{
\begin{tabular}{lll}
\parbox[t][][t]{2.0 cm}{\normalfont \raggedleft ich\\du\\er/sie/es\\wir\\ihr\\sie} &    
\parbox[t][][t]{2cm}{\normalfont merke\\merkst\\merkt\\merken\\merkt\\merken} &
\parbox[t][][t]{2cm}{\normalfont merkte\\merktest\\merkte\\merkten\\merktet\\merkten}\\
\end{tabular}
\begin{tabular}{l}
\parbox[t][][t]{8cm}{}\\
\parbox[t][][t]{8cm}{\normalfont \footnotesize 
etw merken - 
to feel sth - 
es war kaum zu merken - 
it was scarcely noticeable - 
etw (von etw dat) merken - 
to notice sth (of sth) - 
ich habe nichts davon gemerkt - 
I didn't notice a thing (or anything) - 
das merkt jeder/keiner! - 
everyone/no one will notice! 
}\\
\end{tabular}
}
%===messen===
\card{\normalfont \Huge messen}{
\begin{tabular}{lll}
\parbox[t][][t]{2.0 cm}{\normalfont \raggedleft ich\\du\\er/sie/es\\wir\\ihr\\sie} &    
\parbox[t][][t]{2cm}{\normalfont messe\\misst\\misst\\messen\\messt\\messen} &
\parbox[t][][t]{2cm}{\normalfont maß\\maßest\\maß\\maßen\\maßt\\maßen}\\
\end{tabular}
\begin{tabular}{l}
\parbox[t][][t]{8cm}{}\\
\parbox[t][][t]{8cm}{\normalfont \footnotesize 
etw messen - 
to measure sth - 
jds Blutdruck/Temperatur messen - 
to take sb's blood pressure/temperature - 
etw messen - 
to measure sth - 
etw an etw dat - 
messen - 
to judge sth by sth - 
gemessen an etw dat 
}\\
\end{tabular}
}
%===mieten===
\card{\normalfont \Huge mieten}{
\begin{tabular}{lll}
\parbox[t][][t]{2.0 cm}{\normalfont \raggedleft ich\\du\\er/sie/es\\wir\\ihr\\sie} &    
\parbox[t][][t]{2cm}{\normalfont miete\\mietest\\mietet\\mieten\\mietet\\mieten} &
\parbox[t][][t]{2cm}{\normalfont mietete\\mietetest\\mietete\\mieteten\\mietetet\\mieteten}\\
\end{tabular}
\begin{tabular}{l}
\parbox[t][][t]{8cm}{}\\
\parbox[t][][t]{8cm}{\normalfont \footnotesize 
etw mieten - 
to rent sth - 
etw mieten - 
(Boot, Wagen a.) - 
to rent sth (or - 
Brit a. hire) - 
etw mieten - 
(Haus, Wohnung, Büro a.) - 
to lease sth - 
Miete 
}\\
\end{tabular}
}
%===misslingen===
\card{\normalfont \Huge misslingen}{
\begin{tabular}{lll}
\parbox[t][][t]{2.0 cm}{\normalfont \raggedleft ich\\du\\er/sie/es\\wir\\ihr\\sie} &    
\parbox[t][][t]{2cm}{\normalfont misslinge\\misslingst\\misslingt\\misslingen\\misslingt\\misslingen} &
\parbox[t][][t]{2cm}{\normalfont mislang\\mislangst\\mislang\\mislangen\\mislangt\\mislangen}\\
\end{tabular}
\begin{tabular}{l}
\parbox[t][][t]{8cm}{}\\
\parbox[t][][t]{8cm}{\normalfont \footnotesize 
misslingen - 
to fail - 
misslingen - 
to be a failure - 
misslingen - 
to be unsuccessful - 
es misslingt jdm, etw zu tun - 
sb fails (in their (or an) attempt) to do sth - 
eine misslungene Ehe - 
a failed (or an unsuccessful) marriage 
}\\
\end{tabular}
}
%===mitteilen===
\card{\normalfont \Huge mitteilen}{
\begin{tabular}{lll}
\parbox[t][][t]{2.0 cm}{\normalfont \raggedleft ich\\du\\er/sie/es\\wir\\ihr\\sie} &    
\parbox[t][][t]{2cm}{\normalfont teile mit\\teilst mit\\teilt mit\\teilen mit\\teilt mit\\teilen mit} &
\parbox[t][][t]{2cm}{\normalfont teilte mit\\teiltest mit\\teilte mit\\teilten mit\\teiltet mit\\teilten mit}\\
\end{tabular}
\begin{tabular}{l}
\parbox[t][][t]{8cm}{}\\
\parbox[t][][t]{8cm}{\normalfont \footnotesize 
jdm etw mitteilen - 
to tell sb (or - 
form inform sb of) sth - 
jdm mitteilen, dass - 
to tell (or - 
form inform) sb that - 
sich akk (jdm) mitteilen - 
to communicate (with sb) - 
sich akk jdm mitteilen - 
to communicate itself to sb 
}\\
\end{tabular}
}
%===mögen===
\card{\normalfont \Huge mögen}{
\begin{tabular}{lll}
\parbox[t][][t]{2.0 cm}{\normalfont \raggedleft ich\\du\\er/sie/es\\wir\\ihr\\sie} &    
\parbox[t][][t]{2cm}{\normalfont mag\\magst\\mag\\mögen\\mögt\\mögen} &
\parbox[t][][t]{2cm}{\normalfont mochte\\mochtest\\mochte\\mochten\\mochtet\\mochten}\\
\end{tabular}
\begin{tabular}{l}
\parbox[t][][t]{8cm}{}\\
\parbox[t][][t]{8cm}{\normalfont \footnotesize 
etw tun mögen - 
to want to do sth - 
ich mag dich nicht mehr sehen! - 
I don't want to see you any more! - 
ich mag dich nicht gerne allein lassen - 
I don't like to leave you alone (or leaving you alone) - 
Stefan hat noch nie Fisch essen mögen - 
Stefan has never liked fish - 
das hätte ich sehen mögen! - 
I would have liked to see that! 
}\\
\end{tabular}
}
%===nehmen===
\card{\normalfont \Huge nehmen}{
\begin{tabular}{lll}
\parbox[t][][t]{2.0 cm}{\normalfont \raggedleft ich\\du\\er/sie/es\\wir\\ihr\\sie} &    
\parbox[t][][t]{2cm}{\normalfont nehme\\nimmst\\nimmt\\nehmen\\nehmt\\nehmen} &
\parbox[t][][t]{2cm}{\normalfont nahm\\nahmst\\nahm\\nahmen\\nahmt\\nahmen}\\
\end{tabular}
\begin{tabular}{l}
\parbox[t][][t]{8cm}{}\\
\parbox[t][][t]{8cm}{\normalfont \footnotesize 
(sich dat) etw nehmen - 
to take sth - 
(sich dat) etw nehmen - 
(sich bedienen) - 
to help oneself to sth - 
jdn am Arm/an der Hand nehmen - 
to take sb's arm/hand (or sb by the arm/hand) - 
etw in die Hand nehmen - 
to take sth in one's hand - 
(sich dat) etw nehmen 
}\\
\end{tabular}
}
%===nennen===
\card{\normalfont \Huge nennen}{
\begin{tabular}{lll}
\parbox[t][][t]{2.0 cm}{\normalfont \raggedleft ich\\du\\er/sie/es\\wir\\ihr\\sie} &    
\parbox[t][][t]{2cm}{\normalfont nenne\\nennst\\nennt\\nennen\\nennt\\nennen} &
\parbox[t][][t]{2cm}{\normalfont nannte\\nanntest\\nannte\\nannten\\nanntet\\nannten}\\
\end{tabular}
\begin{tabular}{l}
\parbox[t][][t]{8cm}{}\\
\parbox[t][][t]{8cm}{\normalfont \footnotesize 
jdn/etw nennen - 
to name (or call) sb/sth - 
genannt - 
known as - 
nennen - 
to call - 
Freunde dürfen mich Johnny nennen - 
friends may call me Johnny - 
etw nennen - 
to call sth 
}\\
\end{tabular}
}
%===nutzen===
\card{\normalfont \Huge nutzen}{
\begin{tabular}{lll}
\parbox[t][][t]{2.0 cm}{\normalfont \raggedleft ich\\du\\er/sie/es\\wir\\ihr\\sie} &    
\parbox[t][][t]{2cm}{\normalfont nutze\\nutzt\\nutzt\\nutzen\\nutzt\\nutzen} &
\parbox[t][][t]{2cm}{\normalfont nutzte\\nutztest\\nutzte\\nutzten\\nutztet\\nutzten}\\
\end{tabular}
\begin{tabular}{l}
\parbox[t][][t]{8cm}{}\\
\parbox[t][][t]{8cm}{\normalfont \footnotesize 
(jdm) (etwas) nutzen (o. nützen) - 
to be of use (to sb) - 
und was soll das nutzen, wenn ich mich ein drittes Mal darum bemühe? - 
and what's the use in me giving it a third go? - 
drohe ihm, das nützt immer! - 
threaten him, that always helps! - 
schön, dass meine Ermahnungen doch etwas genutzt/genützt haben - 
good, my warnings weren't a complete waste of time - 
(jdm) nichts nutzen (o. nützen) - 
to not do (sb) any good 
}\\
\end{tabular}
}
%===öffnen===
\card{\normalfont \Huge öffnen}{
\begin{tabular}{lll}
\parbox[t][][t]{2.0 cm}{\normalfont \raggedleft ich\\du\\er/sie/es\\wir\\ihr\\sie} &    
\parbox[t][][t]{2cm}{\normalfont öffne\\öffnest\\öffnet\\öffnen\\öffnet\\öffnen} &
\parbox[t][][t]{2cm}{\normalfont öffnete\\öffnetest\\öffnete\\öffneten\\öffnetet\\öffneten}\\
\end{tabular}
\begin{tabular}{l}
\parbox[t][][t]{8cm}{}\\
\parbox[t][][t]{8cm}{\normalfont \footnotesize 
etw öffnen - 
to open sth - 
„hier öffnen“ - 
“open here (or this end) ” - 
die Tür quietscht immer beim Öffnen - 
the door always squeaks when you open it - 
(jdm) öffnen - 
to open the door (for sb) - 
sich akk - 
öffnen 
}\\
\end{tabular}
}
%===ordnen===
\card{\normalfont \Huge ordnen}{
\begin{tabular}{lll}
\parbox[t][][t]{2.0 cm}{\normalfont \raggedleft ich\\du\\er/sie/es\\wir\\ihr\\sie} &    
\parbox[t][][t]{2cm}{\normalfont ordne\\ordnest\\ordnet\\ordnen\\ordnet\\ordnen} &
\parbox[t][][t]{2cm}{\normalfont ordnete\\ordnetest\\ordnete\\ordneten\\ordnetet\\ordneten}\\
\end{tabular}
\begin{tabular}{l}
\parbox[t][][t]{8cm}{}\\
\parbox[t][][t]{8cm}{\normalfont \footnotesize 
etw ordnen - 
to arrange (or order) sth - 
etw neu ordnen - 
to rearrange (or reorganize) sth - 
etw ordnen - 
to put sth in order - 
etw ordnen - 
to sort (or straighten) sth out - 
sich akk - 
ordnen 
}\\
\end{tabular}
}
%===packen===
\card{\normalfont \Huge packen}{
\begin{tabular}{lll}
\parbox[t][][t]{2.0 cm}{\normalfont \raggedleft ich\\du\\er/sie/es\\wir\\ihr\\sie} &    
\parbox[t][][t]{2cm}{\normalfont packe\\packst\\packt\\packen\\packt\\packen} &
\parbox[t][][t]{2cm}{\normalfont packte\\packtest\\packte\\packten\\packtet\\packten}\\
\end{tabular}
\begin{tabular}{l}
\parbox[t][][t]{8cm}{}\\
\parbox[t][][t]{8cm}{\normalfont \footnotesize 
jdn/etw packen - 
to grab (hold of) sb/sth - 
jdn/etw packen - 
to seize sb/sth - 
wenn ich dich packe/zu packen kriege ... - 
when I get hold of you ... - 
jdn/etw bei etw dat/an etw dat - 
packen - 
to grab (or seize) sb/sth by sth - 
jdn an (o. bei) dem Kragen packen 
}\\
\end{tabular}
}
%===passieren===
\card{\normalfont \Huge passieren}{
\begin{tabular}{lll}
\parbox[t][][t]{2.0 cm}{\normalfont \raggedleft ich\\du\\er/sie/es\\wir\\ihr\\sie} &    
\parbox[t][][t]{2cm}{\normalfont passiere\\passierst\\passiert\\passieren\\passiert\\passieren} &
\parbox[t][][t]{2cm}{\normalfont passierte\\passiertest\\passierte\\passierten\\passiertet\\passierten}\\
\end{tabular}
\begin{tabular}{l}
\parbox[t][][t]{8cm}{}\\
\parbox[t][][t]{8cm}{\normalfont \footnotesize 
passieren - 
to happen - 
ist was passiert? - 
has something happened? - 
wie konnte das nur passieren? - 
how could that happen? - 
... sonst passiert was! fam - 
... or there'll be trouble! fam - 
so etwas passiert eben - 
things like that do happen sometimes 
}\\
\end{tabular}
}
%===pfeifen===
\card{\normalfont \Huge pfeifen}{
\begin{tabular}{lll}
\parbox[t][][t]{2.0 cm}{\normalfont \raggedleft ich\\du\\er/sie/es\\wir\\ihr\\sie} &    
\parbox[t][][t]{2cm}{\normalfont pfeife\\pfeifst\\pfeift\\pfeifen\\pfeift\\pfeifen} &
\parbox[t][][t]{2cm}{\normalfont pfiff\\pfiffst\\pfiff\\pfiffen\\pfifft\\pfiffen}\\
\end{tabular}
\begin{tabular}{l}
\parbox[t][][t]{8cm}{}\\
\parbox[t][][t]{8cm}{\normalfont \footnotesize 
pfeifen - 
to whistle - 
auf etw akk - 
pfeifen - 
not to give a damn about sth - 
ich pfeife auf euer Mitleid! - 
I don't need your sympathy! - 
Pfeifen im Walde fam - 
to be whistling in the wind fam - 
pfeifen 
}\\
\end{tabular}
}
%===pflegen===
\card{\normalfont \Huge pflegen}{
\begin{tabular}{lll}
\parbox[t][][t]{2.0 cm}{\normalfont \raggedleft ich\\du\\er/sie/es\\wir\\ihr\\sie} &    
\parbox[t][][t]{2cm}{\normalfont pflege\\pflegst\\pflegt\\pflegen\\pflegt\\pflegen} &
\parbox[t][][t]{2cm}{\normalfont pflegte\\pflegtest\\pflegte\\pflegten\\pflegtet\\pflegten}\\
\end{tabular}
\begin{tabular}{l}
\parbox[t][][t]{8cm}{}\\
\parbox[t][][t]{8cm}{\normalfont \footnotesize 
jdn pflegen - 
to care for (or look after) - 
(or nurse) sb - 
etw pflegen - 
to tend sth - 
etw (mit etw dat) pflegen - 
to look after sth (with sth) - 
etw (mit etw dat) pflegen - 
to treat sth (with sth) - 
etw zu tun pflegen 
}\\
\end{tabular}
}
%===planen===
\card{\normalfont \Huge planen}{
\begin{tabular}{lll}
\parbox[t][][t]{2.0 cm}{\normalfont \raggedleft ich\\du\\er/sie/es\\wir\\ihr\\sie} &    
\parbox[t][][t]{2cm}{\normalfont plane\\planst\\plant\\planen\\plant\\planen} &
\parbox[t][][t]{2cm}{\normalfont plante\\plantest\\plante\\planten\\plantet\\planten}\\
\end{tabular}
\begin{tabular}{l}
\parbox[t][][t]{8cm}{}\\
\parbox[t][][t]{8cm}{\normalfont \footnotesize 
etw planen - 
to plan sth - 
für heute Abend habe ich bisher noch nichts geplant - 
I haven't got anything planned yet for tonight - 
planen, etw zu tun - 
to be planning to do sth - 
Planum - 
planed subgrade - 
Plane - 
tarpaulin 
}\\
\end{tabular}
}
%===preisen===
\card{\normalfont \Huge preisen}{
\begin{tabular}{lll}
\parbox[t][][t]{2.0 cm}{\normalfont \raggedleft ich\\du\\er/sie/es\\wir\\ihr\\sie} &    
\parbox[t][][t]{2cm}{\normalfont preise\\preist\\preist\\preisen\\preist\\preisen} &
\parbox[t][][t]{2cm}{\normalfont pries\\priesest\\pries\\priesen\\priest\\priesen}\\
\end{tabular}
\begin{tabular}{l}
\parbox[t][][t]{8cm}{}\\
\parbox[t][][t]{8cm}{\normalfont \footnotesize 
jdn/etw preisen - 
to praise (or extol) - 
(or - 
form laud) sb/sth - 
sich akk glücklich preisen (können) - 
to (be able to) count (or consider) oneself lucky - 
Preis für +akk - 
price of - 
Preise werden übertroffen - 
prices are being topped 
}\\
\end{tabular}
}
%===probieren===
\card{\normalfont \Huge probieren}{
\begin{tabular}{lll}
\parbox[t][][t]{2.0 cm}{\normalfont \raggedleft ich\\du\\er/sie/es\\wir\\ihr\\sie} &    
\parbox[t][][t]{2cm}{\normalfont probiere\\probierst\\probiert\\probieren\\probiert\\probieren} &
\parbox[t][][t]{2cm}{\normalfont probierte\\probiertest\\probierte\\probierten\\probiertet\\probierten}\\
\end{tabular}
\begin{tabular}{l}
\parbox[t][][t]{8cm}{}\\
\parbox[t][][t]{8cm}{\normalfont \footnotesize 
etw probieren - 
to try (or taste) - 
(or sample) sth - 
es (mit etw dat) probieren - 
to try (or to have a go (or try) at) it (with sth) - 
ich habe es schon mit vielen Diäten probiert - 
I have already tried many diets - 
probieren, etw zu tun - 
to try to do sth - 
ich werde probieren, sie zu überreden 
}\\
\end{tabular}
}
%===protestieren===
\card{\normalfont \Huge protestieren}{
\begin{tabular}{lll}
\parbox[t][][t]{2.0 cm}{\normalfont \raggedleft ich\\du\\er/sie/es\\wir\\ihr\\sie} &    
\parbox[t][][t]{2cm}{\normalfont protestiere\\protestierst\\protestiert\\protestieren\\protestiert\\protestieren} &
\parbox[t][][t]{2cm}{\normalfont protestierte\\protestiertest\\protestierte\\protestierten\\protestiertet\\protestierten}\\
\end{tabular}
\begin{tabular}{l}
\parbox[t][][t]{8cm}{}\\
\parbox[t][][t]{8cm}{\normalfont \footnotesize 
gegen jdn/etw protestieren - 
JUR - 
to protest sb/sth - 
(gegen etw akk) protestieren - 
to protest (or make a protest)  (against (or about) sth) - 
dagegen protestieren, dass jd/etw etw tut - 
to protest against (or about) sb('s)/sth('s) doing sth - 
er protestierte lautstark gegen seine Verurteilung - 
he protested loudly against his conviction 
}\\
\end{tabular}
}
%===prüfen===
\card{\normalfont \Huge prüfen}{
\begin{tabular}{lll}
\parbox[t][][t]{2.0 cm}{\normalfont \raggedleft ich\\du\\er/sie/es\\wir\\ihr\\sie} &    
\parbox[t][][t]{2cm}{\normalfont prüfe\\prüfst\\prüft\\prüfen\\prüft\\prüfen} &
\parbox[t][][t]{2cm}{\normalfont prüfte\\prüftest\\prüfte\\prüften\\prüftet\\prüften}\\
\end{tabular}
\begin{tabular}{l}
\parbox[t][][t]{8cm}{}\\
\parbox[t][][t]{8cm}{\normalfont \footnotesize 
jdn (in etw dat) prüfen - 
to examine sb (in (or on) sth) - 
ich möchte darin geprüft werden, was ich auch tatsächlich gelernt habe - 
I want to be examined on what I really studied - 
Deutsch/Latein prüfen - 
to be the examiner for German/Latin - 
jdn im Hauptfach/Nebenfach prüfen - 
to examine sb in his/her main/minor subject - 
jds Kenntnisse prüfen - 
to test sb's knowledge 
}\\
\end{tabular}
}
%===rächen===
\card{\normalfont \Huge rächen}{
\begin{tabular}{lll}
\parbox[t][][t]{2.0 cm}{\normalfont \raggedleft ich\\du\\er/sie/es\\wir\\ihr\\sie} &    
\parbox[t][][t]{2cm}{\normalfont räche\\rächst\\rächt\\rächen\\rächt\\rächen} &
\parbox[t][][t]{2cm}{\normalfont rächte\\rächtest\\rächte\\rächten\\rächtet\\rächten}\\
\end{tabular}
\begin{tabular}{l}
\parbox[t][][t]{8cm}{}\\
\parbox[t][][t]{8cm}{\normalfont \footnotesize 
etw (an jdm) rächen - 
to take revenge (on sb) for sth - 
jdn rächen - 
to avenge (or take (or exact) revenge for) sb - 
sich akk (an jdm) (für etw akk) rächen - 
to take (or exact) one's revenge (or avenge oneself)  (on sb) (for sth) - 
sich akk (an jdm) (durch etw akk) rächen - 
to come back and haunt sb (as a result of sth) - 
früher oder später rächt sich das viele Rauchen - 
sooner or later (the) heavy smoking will take its toll 
}\\
\end{tabular}
}
%===raten===
\card{\normalfont \Huge raten}{
\begin{tabular}{lll}
\parbox[t][][t]{2.0 cm}{\normalfont \raggedleft ich\\du\\er/sie/es\\wir\\ihr\\sie} &    
\parbox[t][][t]{2cm}{\normalfont rate\\rätst\\rät\\raten\\ratet\\raten} &
\parbox[t][][t]{2cm}{\normalfont riet\\rietest\\riet\\rieten\\rietet\\rieten}\\
\end{tabular}
\begin{tabular}{l}
\parbox[t][][t]{8cm}{}\\
\parbox[t][][t]{8cm}{\normalfont \footnotesize 
(jdm) zu etw dat - 
raten - 
to advise (sb to do) sth - 
(jdm) zu etw dat - 
raten - 
to recommend sth (to sb) - 
wenn Sie mich fragen, ich würde (Ihnen) zu einem Kompromiss raten - 
if you ask me, I'd advise (you to) compromise - 
jdm raten, etw zu tun - 
to advise sb to do sth 
}\\
\end{tabular}
}
%===reagieren===
\card{\normalfont \Huge reagieren}{
\begin{tabular}{lll}
\parbox[t][][t]{2.0 cm}{\normalfont \raggedleft ich\\du\\er/sie/es\\wir\\ihr\\sie} &    
\parbox[t][][t]{2cm}{\normalfont reagiere\\reagierst\\reagiert\\reagieren\\reagiert\\reagieren} &
\parbox[t][][t]{2cm}{\normalfont reagierte\\reagiertest\\reagierte\\reagierten\\reagiertet\\reagierten}\\
\end{tabular}
\begin{tabular}{l}
\parbox[t][][t]{8cm}{}\\
\parbox[t][][t]{8cm}{\normalfont \footnotesize 
(auf etw akk) reagieren - 
to react (to sth) - 
ich habe ihn um eine Antwort gebeten, aber er hat noch nicht reagiert - 
I have asked him for an answer but he hasn't come back to me yet - 
empfindlich/sauer (auf etw akk) reagieren - 
to be sensitive (to sth)/peeved (at sth) - 
(mit etw dat) reagieren - 
to react (with sth) - 
heftig reagieren - 
to react brisk 
}\\
\end{tabular}
}
%===rechnen===
\card{\normalfont \Huge rechnen}{
\begin{tabular}{lll}
\parbox[t][][t]{2.0 cm}{\normalfont \raggedleft ich\\du\\er/sie/es\\wir\\ihr\\sie} &    
\parbox[t][][t]{2cm}{\normalfont rechne\\rechnest\\rechnet\\rechnen\\rechnet\\rechnen} &
\parbox[t][][t]{2cm}{\normalfont rechnete\\rechnetest\\rechnete\\rechneten\\rechnetet\\rechneten}\\
\end{tabular}
\begin{tabular}{l}
\parbox[t][][t]{8cm}{}\\
\parbox[t][][t]{8cm}{\normalfont \footnotesize 
etw rechnen - 
to calculate sth - 
etw im Kopf rechnen - 
to do a sum in one's head - 
was für einen Blödsinn hast du da gerechnet? fam - 
how did you get that absurd result? - 
etw rechnen - 
to work out sth sep - 
etw rechnen - 
to calculate sth 
}\\
\end{tabular}
}
%===reden===
\card{\normalfont \Huge reden}{
\begin{tabular}{lll}
\parbox[t][][t]{2.0 cm}{\normalfont \raggedleft ich\\du\\er/sie/es\\wir\\ihr\\sie} &    
\parbox[t][][t]{2cm}{\normalfont rede\\redest\\redet\\reden\\redet\\reden} &
\parbox[t][][t]{2cm}{\normalfont redete\\redetest\\redete\\redeten\\redetet\\redeten}\\
\end{tabular}
\begin{tabular}{l}
\parbox[t][][t]{8cm}{}\\
\parbox[t][][t]{8cm}{\normalfont \footnotesize 
reden - 
to talk - 
reden - 
to speak - 
mit jdm (über jdn/etw) reden - 
to talk to sb (about sb/sth) - 
über manche Themen wurde zu Hause nie geredet - 
some topics were never discussed at home - 
wie redest du denn mit deinem Vater! - 
that's no way to talk to (or speak with) your father 
}\\
\end{tabular}
}
%===regeln===
\card{\normalfont \Huge regeln}{
\begin{tabular}{lll}
\parbox[t][][t]{2.0 cm}{\normalfont \raggedleft ich\\du\\er/sie/es\\wir\\ihr\\sie} &    
\parbox[t][][t]{2cm}{\normalfont regle \\regelst\\regelt\\regeln\\regelt\\regeln} &
\parbox[t][][t]{2cm}{\normalfont regelte\\regeltest\\regelte\\regelten\\regeltet\\regelten}\\
\end{tabular}
\begin{tabular}{l}
\parbox[t][][t]{8cm}{}\\
\parbox[t][][t]{8cm}{\normalfont \footnotesize 
etw regeln - 
to settle (or see to) sth - 
etw regeln - 
to sort sth out - 
ein Problem regeln - 
to resolve a problem - 
sich akk - 
regeln lassen - 
to be able to be settled - 
mit etwas gutem Willen lässt sich alles regeln 
}\\
\end{tabular}
}
%===regieren===
\card{\normalfont \Huge regieren}{
\begin{tabular}{lll}
\parbox[t][][t]{2.0 cm}{\normalfont \raggedleft ich\\du\\er/sie/es\\wir\\ihr\\sie} &    
\parbox[t][][t]{2cm}{\normalfont regiere\\regierst\\regiert\\regieren\\regiert\\regieren} &
\parbox[t][][t]{2cm}{\normalfont regierte\\regiertest\\regierte\\regierten\\regiertet\\regierten}\\
\end{tabular}
\begin{tabular}{l}
\parbox[t][][t]{8cm}{}\\
\parbox[t][][t]{8cm}{\normalfont \footnotesize 
regieren - 
to rule - 
regieren - 
to reign - 
über jdn/etw regieren - 
to rule (or reign) over sb/sth - 
ein Land regieren - 
to rule (or govern)  a country - 
ein Land regieren - 
Monarch a. 
}\\
\end{tabular}
}
%===reiben===
\card{\normalfont \Huge reiben}{
\begin{tabular}{lll}
\parbox[t][][t]{2.0 cm}{\normalfont \raggedleft ich\\du\\er/sie/es\\wir\\ihr\\sie} &    
\parbox[t][][t]{2cm}{\normalfont reibe\\reibst\\reibt\\reiben\\reibt\\reiben} &
\parbox[t][][t]{2cm}{\normalfont rieb\\riebst\\rieb\\rieben\\riebt\\rieben}\\
\end{tabular}
\begin{tabular}{l}
\parbox[t][][t]{8cm}{}\\
\parbox[t][][t]{8cm}{\normalfont \footnotesize 
etw reiben - 
to rub sth - 
etw auf etw akk/in etw akk reiben - 
to rub sth onto/into sth - 
etw aus etw dat/von etw dat - 
reiben - 
to rub sth out of/off sth - 
etw reiben - 
to grate sth - 
sich akk (an etw dat) reiben 
}\\
\end{tabular}
}
%===reichen===
\card{\normalfont \Huge reichen}{
\begin{tabular}{lll}
\parbox[t][][t]{2.0 cm}{\normalfont \raggedleft ich\\du\\er/sie/es\\wir\\ihr\\sie} &    
\parbox[t][][t]{2cm}{\normalfont reiche\\reichst\\reicht\\reichen\\reicht\\reichen} &
\parbox[t][][t]{2cm}{\normalfont reichte\\reichtest\\reichte\\reichten\\reichtet\\reichten}\\
\end{tabular}
\begin{tabular}{l}
\parbox[t][][t]{8cm}{}\\
\parbox[t][][t]{8cm}{\normalfont \footnotesize 
reichen - 
to be enough (or sufficient) - 
die Vorräte reichen noch Monate - 
the stores will last for months still - 
der Zucker muss noch bis Montag reichen - 
the sugar must last till Monday - 
reicht das Licht zum Lesen? - 
is there enough light to read by? - 
dazu reicht meine Geduld nicht - 
I haven't got enough patience 
}\\
\end{tabular}
}
%===reisen===
\card{\normalfont \Huge reisen}{
\begin{tabular}{lll}
\parbox[t][][t]{2.0 cm}{\normalfont \raggedleft ich\\du\\er/sie/es\\wir\\ihr\\sie} &    
\parbox[t][][t]{2cm}{\normalfont reise\\reist\\reist\\reisen\\reist\\reisen} &
\parbox[t][][t]{2cm}{\normalfont reiste\\reistest\\reiste\\reisten\\reistet\\reisten}\\
\end{tabular}
\begin{tabular}{l}
\parbox[t][][t]{8cm}{}\\
\parbox[t][][t]{8cm}{\normalfont \footnotesize 
reisen - 
to travel - 
wohin werdet ihr in eurem Urlaub reisen? - 
where are you going (to) on holiday? - 
reisen - 
to leave - 
reisen - 
to travel as a rep - 
im Mai wird unser Vertreter wieder reisen - 
our representative will be on the road again in May 
}\\
\end{tabular}
}
%===reißen===
\card{\normalfont \Huge reißen}{
\begin{tabular}{lll}
\parbox[t][][t]{2.0 cm}{\normalfont \raggedleft ich\\du\\er/sie/es\\wir\\ihr\\sie} &    
\parbox[t][][t]{2cm}{\normalfont reiße\\reißt\\reißt\\reißen\\reißt\\reißen} &
\parbox[t][][t]{2cm}{\normalfont riss\\rissest\\riss\\rissen\\risst\\rissen}\\
\end{tabular}
\begin{tabular}{l}
\parbox[t][][t]{8cm}{}\\
\parbox[t][][t]{8cm}{\normalfont \footnotesize 
(an etw dat) reißen - 
Seil, Faden, Band - 
to break (or snap)  (at sth) - 
(an etw dat) reißen - 
Papier, Stoff - 
to tear (or rip)  (at sth) - 
das Seil riss unter dem Gewicht - 
the rope broke (or snapped) under the weight - 
jdm reißt etw - 
sb's sth breaks (or snaps) /tears 
}\\
\end{tabular}
}
%===reiten===
\card{\normalfont \Huge reiten}{
\begin{tabular}{lll}
\parbox[t][][t]{2.0 cm}{\normalfont \raggedleft ich\\du\\er/sie/es\\wir\\ihr\\sie} &    
\parbox[t][][t]{2cm}{\normalfont reite\\reitest\\reitet\\reiten\\reitet\\reiten} &
\parbox[t][][t]{2cm}{\normalfont ritt\\rittest \\ritt\\ritten\\rittet\\ritten}\\
\end{tabular}
\begin{tabular}{l}
\parbox[t][][t]{8cm}{}\\
\parbox[t][][t]{8cm}{\normalfont \footnotesize 
reiten - 
to ride - 
bist du schon mal geritten? - 
have you ever been riding? - 
wissen Sie, wo man hier reiten lernen kann? - 
do you know where it's possible to take riding lessons round here? - 
auf etw dat - 
reiten - 
to ride (on) sth - 
bist du schon mal auf einem Pony geritten? 
}\\
\end{tabular}
}
%===rennen===
\card{\normalfont \Huge rennen}{
\begin{tabular}{lll}
\parbox[t][][t]{2.0 cm}{\normalfont \raggedleft ich\\du\\er/sie/es\\wir\\ihr\\sie} &    
\parbox[t][][t]{2cm}{\normalfont renne\\rennst\\rennt\\rennen\\rennt\\rennen} &
\parbox[t][][t]{2cm}{\normalfont rannte\\ranntest\\rannte\\rannten\\ranntet\\rannten}\\
\end{tabular}
\begin{tabular}{l}
\parbox[t][][t]{8cm}{}\\
\parbox[t][][t]{8cm}{\normalfont \footnotesize 
rennen - 
to run - 
zu jdm rennen - 
to run (off) to sb - 
dann renn' doch zu deiner Mama - 
why don't you run off to your mummy - 
sie rennt bei jeder Kleinigkeit zur Geschäftsleitung - 
she's always going up to management with every little triviality - 
die arme Frau rennt dauernd zur Polizei - 
that poor woman's always running to the police 
}\\
\end{tabular}
}
%===reservieren===
\card{\normalfont \Huge reservieren}{
\begin{tabular}{lll}
\parbox[t][][t]{2.0 cm}{\normalfont \raggedleft ich\\du\\er/sie/es\\wir\\ihr\\sie} &    
\parbox[t][][t]{2cm}{\normalfont reserviere\\reservierst\\reserviert\\reservieren\\reserviert\\reservieren} &
\parbox[t][][t]{2cm}{\normalfont reservierte\\reserviertest\\reservierte\\reservierten\\reserviertet\\reservierten}\\
\end{tabular}
\begin{tabular}{l}
\parbox[t][][t]{8cm}{}\\
\parbox[t][][t]{8cm}{\normalfont \footnotesize 
(jdm (o. für jdn)) etw reservieren - 
to reserve sth (for sb) - 
einen Sitzplatz reservieren - 
to reserve (or book)  a seat - 
ich möchte drei Plätze reservieren - 
I'd like to book (or reserve) three seats 
}\\
\end{tabular}
}
%===retten===
\card{\normalfont \Huge retten}{
\begin{tabular}{lll}
\parbox[t][][t]{2.0 cm}{\normalfont \raggedleft ich\\du\\er/sie/es\\wir\\ihr\\sie} &    
\parbox[t][][t]{2cm}{\normalfont rette\\rettest\\rettet\\retten\\rettet\\retten} &
\parbox[t][][t]{2cm}{\normalfont rettete\\rettetest\\rettete\\retteten\\rettetet\\retteten}\\
\end{tabular}
\begin{tabular}{l}
\parbox[t][][t]{8cm}{}\\
\parbox[t][][t]{8cm}{\normalfont \footnotesize 
jdn/etw (vor jdm/etw) retten - 
to save sb/sth (from sb/sth) - 
ein geschickter Restaurator wird das Gemälde noch retten können - 
a skilled restorer will still be able to save the painting - 
sie konnte ihren Schmuck durch die Flucht hindurch retten - 
she was able to save her jewellery while fleeing - 
rettend - 
which saved the day - 
das ist der rettende Einfall! - 
that's the idea that will save the day! 
}\\
\end{tabular}
}
%===riechen===
\card{\normalfont \Huge riechen}{
\begin{tabular}{lll}
\parbox[t][][t]{2.0 cm}{\normalfont \raggedleft ich\\du\\er/sie/es\\wir\\ihr\\sie} &    
\parbox[t][][t]{2cm}{\normalfont rieche\\riechst\\riecht\\riechen\\riecht\\riechen} &
\parbox[t][][t]{2cm}{\normalfont roch\\rochst\\roch\\rochen\\rocht\\rochen}\\
\end{tabular}
\begin{tabular}{l}
\parbox[t][][t]{8cm}{}\\
\parbox[t][][t]{8cm}{\normalfont \footnotesize 
riechen (duften) - 
to smell - 
riechen (stinken a.) - 
to stink pej - 
riechen (stinken a.) - 
to reek pej - 
das riecht hier ja so angebrannt - 
there's a real smell of burning here - 
nach etw dat - 
riechen 
}\\
\end{tabular}
}
%===ringen===
\card{\normalfont \Huge ringen}{
\begin{tabular}{lll}
\parbox[t][][t]{2.0 cm}{\normalfont \raggedleft ich\\du\\er/sie/es\\wir\\ihr\\sie} &    
\parbox[t][][t]{2cm}{\normalfont ringe\\ringst\\ringt\\ringen\\ringt\\ringen} &
\parbox[t][][t]{2cm}{\normalfont rang\\rangst\\rang\\rangen\\rangt\\rangen}\\
\end{tabular}
\begin{tabular}{l}
\parbox[t][][t]{8cm}{}\\
\parbox[t][][t]{8cm}{\normalfont \footnotesize 
(mit jdm) ringen - 
to wrestle (with sb) - 
mit sich dat - 
ringen - 
to wrestle with oneself - 
nach Atem (o. Luft) - 
ringen - 
to struggle for breath - 
um etw akk - 
ringen 
}\\
\end{tabular}
}
%===rufen===
\card{\normalfont \Huge rufen}{
\begin{tabular}{lll}
\parbox[t][][t]{2.0 cm}{\normalfont \raggedleft ich\\du\\er/sie/es\\wir\\ihr\\sie} &    
\parbox[t][][t]{2cm}{\normalfont rufe\\rufst\\ruft\\rufen\\ruft\\rufen} &
\parbox[t][][t]{2cm}{\normalfont rief\\riefst\\rief\\riefen\\rieft\\riefen}\\
\end{tabular}
\begin{tabular}{l}
\parbox[t][][t]{8cm}{}\\
\parbox[t][][t]{8cm}{\normalfont \footnotesize 
rufen - 
to cry out - 
(nach jdm) rufen - 
to call (for sb) - 
rufen - 
to call - 
die Pflicht ruft - 
duty calls - 
(zu etw dat) rufen - 
to call (to sth) 
}\\
\end{tabular}
}
%===ruhen===
\card{\normalfont \Huge ruhen}{
\begin{tabular}{lll}
\parbox[t][][t]{2.0 cm}{\normalfont \raggedleft ich\\du\\er/sie/es\\wir\\ihr\\sie} &    
\parbox[t][][t]{2cm}{\normalfont ruhe\\ruhst\\ruht\\ruhen\\ruht\\ruhen} &
\parbox[t][][t]{2cm}{\normalfont ruhte\\ruhtest\\ruhte\\ruhten\\ruhtet\\ruhten}\\
\end{tabular}
\begin{tabular}{l}
\parbox[t][][t]{8cm}{}\\
\parbox[t][][t]{8cm}{\normalfont \footnotesize 
ruhen - 
to (have a) rest - 
(ich) wünsche, wohl geruht zu haben! geh - 
I trust you had a good night's sleep? - 
nicht eher ruhen werden, bis ..., nicht ruhen und rasten, bis ... - 
to not rest until ... - 
auf etw dat - 
ruhen - 
to rest on sth - 
auf jdm/etw ruhen 
}\\
\end{tabular}
}
%===rühren===
\card{\normalfont \Huge rühren}{
\begin{tabular}{lll}
\parbox[t][][t]{2.0 cm}{\normalfont \raggedleft ich\\du\\er/sie/es\\wir\\ihr\\sie} &    
\parbox[t][][t]{2cm}{\normalfont rühre\\rührst\\rührt\\rühren\\rührt\\rühren} &
\parbox[t][][t]{2cm}{\normalfont rührte\\rührtest\\rührte\\rührten\\rührtet\\rührten}\\
\end{tabular}
\begin{tabular}{l}
\parbox[t][][t]{8cm}{}\\
\parbox[t][][t]{8cm}{\normalfont \footnotesize 
etw rühren - 
to stir sth - 
jdn/etw rühren - 
to move sb/to touch sth - 
jds Gemüt/Herz rühren - 
to touch sb/sb's heart - 
das kann mich nicht rühren - 
that doesn't bother me - 
gerührt - 
moved pred 
}\\
\end{tabular}
}
%===sagen===
\card{\normalfont \Huge sagen}{
\begin{tabular}{lll}
\parbox[t][][t]{2.0 cm}{\normalfont \raggedleft ich\\du\\er/sie/es\\wir\\ihr\\sie} &    
\parbox[t][][t]{2cm}{\normalfont sage\\sagst\\sagt\\sagen\\sagt\\sagen} &
\parbox[t][][t]{2cm}{\normalfont sagte\\sagtest\\sagte\\sagten\\sagtet\\sagten}\\
\end{tabular}
\begin{tabular}{l}
\parbox[t][][t]{8cm}{}\\
\parbox[t][][t]{8cm}{\normalfont \footnotesize 
etw (zu jdm) sagen - 
to say sth (to sb) - 
sagen, dass/ob ... - 
to say (that)/whether ... - 
sagen, wann/wie/warum ... - 
to say when/how/why ... - 
man sagt von ihr, dass ... - 
it is said of her that ... - 
warum haben Sie das nicht gleich gesagt? - 
why didn't you say that (or so) before? 
}\\
\end{tabular}
}
%===sammeln===
\card{\normalfont \Huge sammeln}{
\begin{tabular}{lll}
\parbox[t][][t]{2.0 cm}{\normalfont \raggedleft ich\\du\\er/sie/es\\wir\\ihr\\sie} &    
\parbox[t][][t]{2cm}{\normalfont sammle \\sammelst\\sammelt\\sammeln\\sammelt\\sammeln} &
\parbox[t][][t]{2cm}{\normalfont sammelte\\sammeltest\\sammelte\\sammelten\\sammeltet\\sammelten}\\
\end{tabular}
\begin{tabular}{l}
\parbox[t][][t]{8cm}{}\\
\parbox[t][][t]{8cm}{\normalfont \footnotesize 
etw sammeln - 
to pick (or gather) sth - 
etw sammeln - 
to gather sth - 
etw von der Erde sammeln - 
to pick up sth sep  (off the ground) - 
etw sammeln - 
to collect sth - 
etw sammeln - 
to collect sth (in) 
}\\
\end{tabular}
}
%===saufen===
\card{\normalfont \Huge saufen}{
\begin{tabular}{lll}
\parbox[t][][t]{2.0 cm}{\normalfont \raggedleft ich\\du\\er/sie/es\\wir\\ihr\\sie} &    
\parbox[t][][t]{2cm}{\normalfont saufe\\säufst\\säuft\\saufen\\sauft\\saufen} &
\parbox[t][][t]{2cm}{\normalfont soff\\soffst\\soff\\soffen\\sofft\\soffen}\\
\end{tabular}
\begin{tabular}{l}
\parbox[t][][t]{8cm}{}\\
\parbox[t][][t]{8cm}{\normalfont \footnotesize 
etw saufen - 
to drink sth - 
etw saufen - 
to knock back sth sep fam - 
etw saufen - 
to drink sth - 
saufen - 
to drink - 
saufen - 
to (be/go on the) booze fam 
}\\
\end{tabular}
}
%===saugen===
\card{\normalfont \Huge saugen}{
\begin{tabular}{lll}
\parbox[t][][t]{2.0 cm}{\normalfont \raggedleft ich\\du\\er/sie/es\\wir\\ihr\\sie} &    
\parbox[t][][t]{2cm}{\normalfont sauge\\saugst\\saugt\\saugen\\saugt\\saugen} &
\parbox[t][][t]{2cm}{\normalfont sog \\sogst \\sog\\sogen \\sogt \\sogen }\\
\end{tabular}
\begin{tabular}{l}
\parbox[t][][t]{8cm}{}\\
\parbox[t][][t]{8cm}{\normalfont \footnotesize 
(an etw dat) saugen - 
to suck ((on) sth) - 
saugen - 
to vacuum - 
saugen - 
to hoover Brit - 
etw (aus etw dat) saugen - 
to suck sth (from sth) - 
etw saugen - 
to vacuum (or 
}\\
\end{tabular}
}
%===schaden===
\card{\normalfont \Huge schaden}{
\begin{tabular}{lll}
\parbox[t][][t]{2.0 cm}{\normalfont \raggedleft ich\\du\\er/sie/es\\wir\\ihr\\sie} &    
\parbox[t][][t]{2cm}{\normalfont schade\\schadest\\schadet\\schaden\\schadet\\schaden} &
\parbox[t][][t]{2cm}{\normalfont schadete\\schadetest\\schadete\\schadeten\\schadetet\\schadeten}\\
\end{tabular}
\begin{tabular}{l}
\parbox[t][][t]{8cm}{}\\
\parbox[t][][t]{8cm}{\normalfont \footnotesize 
jdm/sich schaden - 
to do harm to sb/oneself - 
etw dat/etw dat sehr schaden - 
to damage/to do great damage to sth - 
Arbeit hat noch keinem geschadet fam - 
work never did (or has never done) anybody any harm - 
es kann nichts schaden, wenn jd etw tut - 
it would do no harm if sb does sth (or for sb to do sth) - 
schadet das was? fam - 
so what? 
}\\
\end{tabular}
}
%===schaffen===
\card{\normalfont \Huge schaffen}{
\begin{tabular}{lll}
\parbox[t][][t]{2.0 cm}{\normalfont \raggedleft ich\\du\\er/sie/es\\wir\\ihr\\sie} &    
\parbox[t][][t]{2cm}{\normalfont schaffe\\schaffst\\schafft\\schaffen\\schafft\\schaffen} &
\parbox[t][][t]{2cm}{\normalfont schaffte\\schafftest\\schaffte\\schafften\\schafftet\\schafften}\\
\end{tabular}
\begin{tabular}{l}
\parbox[t][][t]{8cm}{}\\
\parbox[t][][t]{8cm}{\normalfont \footnotesize 
etw schaffen - 
to manage (to do) sth - 
wie schaffst du das nur? - 
how do you (manage to) do it (all)? - 
schaffst du es noch? - 
can you manage? - 
ich schaffe es nicht mehr - 
I can't manage (or cope) any more - 
ich schaffe es nicht mehr - 
I can't go on 
}\\
\end{tabular}
}
%===schalten===
\card{\normalfont \Huge schalten}{
\begin{tabular}{lll}
\parbox[t][][t]{2.0 cm}{\normalfont \raggedleft ich\\du\\er/sie/es\\wir\\ihr\\sie} &    
\parbox[t][][t]{2cm}{\normalfont schalte\\schaltest\\schaltet\\schalten\\schaltet\\schalten} &
\parbox[t][][t]{2cm}{\normalfont schaltete\\schaltetest\\schaltete\\schalteten\\schaltetet\\schalteten}\\
\end{tabular}
\begin{tabular}{l}
\parbox[t][][t]{8cm}{}\\
\parbox[t][][t]{8cm}{\normalfont \footnotesize 
schalten - 
to change gear - 
schalten - 
to get it fam - 
schalten - 
to catch on fam - 
schalten - 
to switch to - 
schalten und walten - 
to manage things as one pleases 
}\\
\end{tabular}
}
%===schätzen===
\card{\normalfont \Huge schätzen}{
\begin{tabular}{lll}
\parbox[t][][t]{2.0 cm}{\normalfont \raggedleft ich\\du\\er/sie/es\\wir\\ihr\\sie} &    
\parbox[t][][t]{2cm}{\normalfont schätze\\schätzt\\schätzt\\schätzen\\schätzt\\schätzen} &
\parbox[t][][t]{2cm}{\normalfont schätzte\\schätztest\\schätzte\\schätzten\\schätztet\\schätzten}\\
\end{tabular}
\begin{tabular}{l}
\parbox[t][][t]{8cm}{}\\
\parbox[t][][t]{8cm}{\normalfont \footnotesize 
jdn/etw (auf etw akk)schätzen - 
to guess (or reckon) that sb/sth is sth - 
jdn/etw auf ein bestimmtes Alter schätzen - 
to guess sb's/sth's age - 
meistens werde ich jünger geschätzt - 
people usually think I'm younger - 
jdn auf eine bestimmte Größe/etw auf eine bestimmte Höhe schätzen - 
to guess the height of sb/sth - 
ich schätze sein Gewicht auf ca. 100 kg - 
I reckon he weighs about 100 kilos 
}\\
\end{tabular}
}
%===schauen===
\card{\normalfont \Huge schauen}{
\begin{tabular}{lll}
\parbox[t][][t]{2.0 cm}{\normalfont \raggedleft ich\\du\\er/sie/es\\wir\\ihr\\sie} &    
\parbox[t][][t]{2cm}{\normalfont schaue\\schaust\\schaut\\schauen\\schaut\\schauen} &
\parbox[t][][t]{2cm}{\normalfont schaute\\schautest\\schaute\\schauten\\schautet\\schauten}\\
\end{tabular}
\begin{tabular}{l}
\parbox[t][][t]{8cm}{}\\
\parbox[t][][t]{8cm}{\normalfont \footnotesize 
schauen - 
to look - 
auf die Uhr schauen - 
to look at the clock - 
auf jdn/etw schauen - 
to look at sb/sth - 
um sich akk - 
schauen - 
to look around - 
um sich akk 
}\\
\end{tabular}
}
%===scheiden===
\card{\normalfont \Huge scheiden}{
\begin{tabular}{lll}
\parbox[t][][t]{2.0 cm}{\normalfont \raggedleft ich\\du\\er/sie/es\\wir\\ihr\\sie} &    
\parbox[t][][t]{2cm}{\normalfont scheide\\scheidest\\scheidet\\scheiden\\scheidet\\scheiden} &
\parbox[t][][t]{2cm}{\normalfont schied\\schiedest\\schied\\schieden\\schiedet\\schieden}\\
\end{tabular}
\begin{tabular}{l}
\parbox[t][][t]{8cm}{}\\
\parbox[t][][t]{8cm}{\normalfont \footnotesize 
jdn scheiden - 
to divorce sb - 
sich akk (von jdm) scheiden lassen - 
to get divorced (from sb) - 
geschieden - 
divorced - 
wir sind geschiedene Leute fig - 
it's all over between us - 
etw scheiden - 
to dissolve sth 
}\\
\end{tabular}
}
%===scheinen===
\card{\normalfont \Huge scheinen}{
\begin{tabular}{lll}
\parbox[t][][t]{2.0 cm}{\normalfont \raggedleft ich\\du\\er/sie/es\\wir\\ihr\\sie} &    
\parbox[t][][t]{2cm}{\normalfont scheine\\scheinst\\scheint\\scheinen\\scheint\\scheinen} &
\parbox[t][][t]{2cm}{\normalfont schien\\schienst\\schien\\schienen\\schient\\schienen}\\
\end{tabular}
\begin{tabular}{l}
\parbox[t][][t]{8cm}{}\\
\parbox[t][][t]{8cm}{\normalfont \footnotesize 
scheinen - 
to shine - 
scheinen - 
to shine - 
etw zu sein scheinen - 
to appear (or seem) to be sth - 
es scheint, dass/als (ob) ... - 
it appears (or seems) that/as if ... - 
wie es scheint, hast du recht - 
it appears (or seems)  (that) you are right 
}\\
\end{tabular}
}
%===scheitern===
\card{\normalfont \Huge scheitern}{
\begin{tabular}{lll}
\parbox[t][][t]{2.0 cm}{\normalfont \raggedleft ich\\du\\er/sie/es\\wir\\ihr\\sie} &    
\parbox[t][][t]{2cm}{\normalfont scheitere\\scheiterst\\scheitert\\scheitern\\scheitert\\scheitern} &
\parbox[t][][t]{2cm}{\normalfont scheiterte\\scheitertest\\scheiterte\\scheiterten\\scheitertet\\scheiterten}\\
\end{tabular}
\begin{tabular}{l}
\parbox[t][][t]{8cm}{}\\
\parbox[t][][t]{8cm}{\normalfont \footnotesize 
(an jdm/etw) scheitern - 
to fail (or be unsuccessful)  (because of sb/sth) - 
etw scheitert an etw dat - 
sth flounders (or runs aground) on sth fig - 
kläglich scheitern - 
to fail miserably - 
Scheitern - 
failure - 
das Scheitern der Verhandlungen - 
the breakdown of the talks (or negotiations) 
}\\
\end{tabular}
}
%===schelten===
\card{\normalfont \Huge schelten}{
\begin{tabular}{lll}
\parbox[t][][t]{2.0 cm}{\normalfont \raggedleft ich\\du\\er/sie/es\\wir\\ihr\\sie} &    
\parbox[t][][t]{2cm}{\normalfont schelte\\schiltst\\schilt\\schelten\\scheltet\\schelten} &
\parbox[t][][t]{2cm}{\normalfont schalt\\schaltest \\schalt\\schalten\\schaltet\\schalten}\\
\end{tabular}
\begin{tabular}{l}
\parbox[t][][t]{8cm}{}\\
\parbox[t][][t]{8cm}{\normalfont \footnotesize 
jdn (für (o. wegen) etw) schelten - 
to scold sb dated - 
(or - 
form reprimand sb)  (for sth/doing sth) - 
jdn (für (o. wegen) etw) schelten - 
to tell sb off (for sth/doing sth) - 
jdn (für (o. wegen) etw) schelten - 
to give sb a dressing-down fam - 
jdn (für (o. wegen) etw) schelten - 
(ewig schimpfen) 
}\\
\end{tabular}
}
%===schenken===
\card{\normalfont \Huge schenken}{
\begin{tabular}{lll}
\parbox[t][][t]{2.0 cm}{\normalfont \raggedleft ich\\du\\er/sie/es\\wir\\ihr\\sie} &    
\parbox[t][][t]{2cm}{\normalfont schenke\\schenkst\\schenkt\\schenken\\schenkt\\schenken} &
\parbox[t][][t]{2cm}{\normalfont schenkte\\schenktest\\schenkte\\schenkten\\schenktet\\schenkten}\\
\end{tabular}
\begin{tabular}{l}
\parbox[t][][t]{8cm}{}\\
\parbox[t][][t]{8cm}{\normalfont \footnotesize 
jdm etw (zu etw dat) schenken - 
to give sb sth as a present (or gift)  (for sth) - 
jdm etw (zu etw dat) schenken - 
(zu einem Anlass) - 
to present sb with sth (on the occasion of sth) form - 
den Rest von dem Geld schenke ich dir - 
you can keep the rest of the money - 
ich möchte nichts geschenkt haben! - 
I don't want any presents! - 
ich nehme nichts geschenkt! 
}\\
\end{tabular}
}
%===schicken===
\card{\normalfont \Huge schicken}{
\begin{tabular}{lll}
\parbox[t][][t]{2.0 cm}{\normalfont \raggedleft ich\\du\\er/sie/es\\wir\\ihr\\sie} &    
\parbox[t][][t]{2cm}{\normalfont schicke\\schickst\\schickt\\schicken\\schickt\\schicken} &
\parbox[t][][t]{2cm}{\normalfont schickte\\schicktest\\schickte\\schickten\\schicktet\\schickten}\\
\end{tabular}
\begin{tabular}{l}
\parbox[t][][t]{8cm}{}\\
\parbox[t][][t]{8cm}{\normalfont \footnotesize 
(jdm) etw schicken - 
to send (sb) sth - 
(jdm) etw schicken - 
ÖKON - 
to dispatch (or despatch) sth (to sb) - 
etw mit der Post schicken - 
to send sth by post (or - 
Am mail) - 
etw mit der Post schicken - 
to post (or 
}\\
\end{tabular}
}
%===schieben===
\card{\normalfont \Huge schieben}{
\begin{tabular}{lll}
\parbox[t][][t]{2.0 cm}{\normalfont \raggedleft ich\\du\\er/sie/es\\wir\\ihr\\sie} &    
\parbox[t][][t]{2cm}{\normalfont schiebe\\schiebst\\schiebt\\schieben\\schiebt\\schieben} &
\parbox[t][][t]{2cm}{\normalfont schob\\schobst\\schob\\schoben\\schobt\\schoben}\\
\end{tabular}
\begin{tabular}{l}
\parbox[t][][t]{8cm}{}\\
\parbox[t][][t]{8cm}{\normalfont \footnotesize 
etw (irgendwohin) schieben - 
to push sth (somewhere) - 
er schob den Einkaufswagen durch den Supermarkt - 
he wheeled the shopping trolley through the supermarket - 
jdn/etw schieben - 
to push (or - 
fam shove) sb/sth - 
lass uns den Schrank in die Ecke schieben - 
let's shift the cupboard into the corner - 
jdn schieben 
}\\
\end{tabular}
}
%===schießen===
\card{\normalfont \Huge schießen}{
\begin{tabular}{lll}
\parbox[t][][t]{2.0 cm}{\normalfont \raggedleft ich\\du\\er/sie/es\\wir\\ihr\\sie} &    
\parbox[t][][t]{2cm}{\normalfont schieße\\schießt\\schießt\\schießen\\schießt\\schießen} &
\parbox[t][][t]{2cm}{\normalfont schoss\\schossest\\schoss\\schossen\\schosst\\schossen}\\
\end{tabular}
\begin{tabular}{l}
\parbox[t][][t]{8cm}{}\\
\parbox[t][][t]{8cm}{\normalfont \footnotesize 
(mit etw dat) schießen - 
to shoot (with sth) - 
auf jdn/etw schießen - 
to shoot at sb/sth - 
schießen - 
(o. zum Schießen - 
) gehen - 
to go shooting - 
(an etw akk/auf etw akk/in etw akk) schießen - 
to shoot (at/into sth) 
}\\
\end{tabular}
}
%===schimpfen===
\card{\normalfont \Huge schimpfen}{
\begin{tabular}{lll}
\parbox[t][][t]{2.0 cm}{\normalfont \raggedleft ich\\du\\er/sie/es\\wir\\ihr\\sie} &    
\parbox[t][][t]{2cm}{\normalfont schimpfe\\schimpfst\\schimpft\\schimpfen\\schimpft\\schimpfen} &
\parbox[t][][t]{2cm}{\normalfont schimpfte\\schimpftest\\schimpfte\\schimpften\\schimpftet\\schimpften}\\
\end{tabular}
\begin{tabular}{l}
\parbox[t][][t]{8cm}{}\\
\parbox[t][][t]{8cm}{\normalfont \footnotesize 
(auf (o. über) jdn/etw) schimpfen - 
to grumble (about sb/sth) - 
schimpfen - 
to (curse and) swear - 
wie ein Rohrspatz schimpfen - 
to curse like a washerwoman (or sailor) - 
schimpfen - 
to grumble - 
mit jdm schimpfen - 
to scold sb 
}\\
\end{tabular}
}
%===schinden===
\card{\normalfont \Huge schinden}{
\begin{tabular}{lll}
\parbox[t][][t]{2.0 cm}{\normalfont \raggedleft ich\\du\\er/sie/es\\wir\\ihr\\sie} &    
\parbox[t][][t]{2cm}{\normalfont schinde mich\\schindest dich\\schindet sich\\schinden uns\\schindet euch\\schinden sich} &
\parbox[t][][t]{2cm}{\normalfont schindete \\schindetest\\schindete \\schindeten \\schindetet\\schindeten}\\
\end{tabular}
\begin{tabular}{l}
\parbox[t][][t]{8cm}{}\\
\parbox[t][][t]{8cm}{\normalfont \footnotesize 
sich akk (mit etw dat) schinden - 
to work oneself to death (at/over sth) - 
sich akk (mit etw dat) schinden - 
to slave (away) (at sth) - 
sich akk (mit etw dat) schinden - 
to work like a Trojan (at sth) Brit - 
jdn schinden - 
to work (or treat) sb like a slave - 
jdn schinden - 
to work sb into the ground 
}\\
\end{tabular}
}
%===schlafen===
\card{\normalfont \Huge schlafen}{
\begin{tabular}{lll}
\parbox[t][][t]{2.0 cm}{\normalfont \raggedleft ich\\du\\er/sie/es\\wir\\ihr\\sie} &    
\parbox[t][][t]{2cm}{\normalfont schlafe\\schläfst\\schläft\\schlafen\\schlaft\\schlafen} &
\parbox[t][][t]{2cm}{\normalfont schlief\\schliefst\\schlief\\schliefen\\schlieft\\schliefen}\\
\end{tabular}
\begin{tabular}{l}
\parbox[t][][t]{8cm}{}\\
\parbox[t][][t]{8cm}{\normalfont \footnotesize 
schlafen - 
to sleep - 
schlafen - 
to be asleep - 
bei dem Lärm kann doch kein Mensch schlafen! - 
nobody can sleep with that noise (going on)! - 
darüber muss ich erst schlafen - 
I'll have to sleep over that - 
schlaf gut (o. - 
geh schlafen Sie wohl) 
}\\
\end{tabular}
}
%===schlagen===
\card{\normalfont \Huge schlagen}{
\begin{tabular}{lll}
\parbox[t][][t]{2.0 cm}{\normalfont \raggedleft ich\\du\\er/sie/es\\wir\\ihr\\sie} &    
\parbox[t][][t]{2cm}{\normalfont schlage\\schlägst\\schlägt\\schlagen\\schlagt\\schlagen} &
\parbox[t][][t]{2cm}{\normalfont schlug\\schlugst\\schlug\\schlugen\\schlugt\\schlugen}\\
\end{tabular}
\begin{tabular}{l}
\parbox[t][][t]{8cm}{}\\
\parbox[t][][t]{8cm}{\normalfont \footnotesize 
jdn schlagen - 
to hit (or - 
form strike) sb - 
jdn schlagen - 
(mit der Faust) - 
to punch sb - 
jdn schlagen - 
(mit der flachen Hand) - 
to slap sb - 
sie schlug ihm das Heft um die Ohren 
}\\
\end{tabular}
}
%===schleichen===
\card{\normalfont \Huge schleichen}{
\begin{tabular}{lll}
\parbox[t][][t]{2.0 cm}{\normalfont \raggedleft ich\\du\\er/sie/es\\wir\\ihr\\sie} &    
\parbox[t][][t]{2cm}{\normalfont schleiche\\schleichst\\schleicht\\schleichen\\schleicht\\schleichen} &
\parbox[t][][t]{2cm}{\normalfont schlich\\schlichst\\schlich\\schlichen\\schlicht\\schlichen}\\
\end{tabular}
\begin{tabular}{l}
\parbox[t][][t]{8cm}{}\\
\parbox[t][][t]{8cm}{\normalfont \footnotesize 
(irgendwohin) schleichen - 
to creep (or - 
liter steal) - 
(or - 
pej sneak)  (somewhere) - 
schleichen - 
to prowl - 
schleichen - 
to crawl along - 
sich akk irgendwohin schleichen 
}\\
\end{tabular}
}
%===schleifen===
\card{\normalfont \Huge schleifen}{
\begin{tabular}{lll}
\parbox[t][][t]{2.0 cm}{\normalfont \raggedleft ich\\du\\er/sie/es\\wir\\ihr\\sie} &    
\parbox[t][][t]{2cm}{\normalfont schleife\\schleifst\\schleift\\schleifen\\schleift\\schleifen} &
\parbox[t][][t]{2cm}{\normalfont schleifte\\schleiftest\\schleifte\\schleiften\\schleiftet\\schleiften}\\
\end{tabular}
\begin{tabular}{l}
\parbox[t][][t]{8cm}{}\\
\parbox[t][][t]{8cm}{\normalfont \footnotesize 
etw/jdn schleifen - 
to drag sth/sb - 
jdn schleifen - 
to drag sb - 
etw schleifen - 
to raze sth to the ground - 
etw schleifen - 
to tear sth down - 
schleifen - 
to rub (or scrape) 
}\\
\end{tabular}
}
%===schlingen===
\card{\normalfont \Huge schlingen}{
\begin{tabular}{lll}
\parbox[t][][t]{2.0 cm}{\normalfont \raggedleft ich\\du\\er/sie/es\\wir\\ihr\\sie} &    
\parbox[t][][t]{2cm}{\normalfont schlinge\\schlingst\\schlingt\\schlingen\\schlingt\\schlingen} &
\parbox[t][][t]{2cm}{\normalfont schlang\\schlangst\\schlang\\schlangen\\schlangt\\schlangen}\\
\end{tabular}
\begin{tabular}{l}
\parbox[t][][t]{8cm}{}\\
\parbox[t][][t]{8cm}{\normalfont \footnotesize 
etw (um etw akk) schlingen - 
to wind sth (about sth) - 
etw zu einem Knoten schlingen - 
to tie (or knot) sth - 
die Arme um jdn schlingen - 
to wrap one's arms around sb - 
sich akk (um etw akk) schlingen - 
to wind (or coil) itself (around sth) - 
sich akk (um etw akk) schlingen - 
to creep (around sth) 
}\\
\end{tabular}
}
%===schmecken===
\card{\normalfont \Huge schmecken}{
\begin{tabular}{lll}
\parbox[t][][t]{2.0 cm}{\normalfont \raggedleft ich\\du\\er/sie/es\\wir\\ihr\\sie} &    
\parbox[t][][t]{2cm}{\normalfont schmecke\\schmeckst\\schmeckt\\schmecken\\schmeckt\\schmecken} &
\parbox[t][][t]{2cm}{\normalfont schmeckte\\schmecktest\\schmeckte\\schmeckten\\schmecktet\\schmeckten}\\
\end{tabular}
\begin{tabular}{l}
\parbox[t][][t]{8cm}{}\\
\parbox[t][][t]{8cm}{\normalfont \footnotesize 
hat es geschmeckt? - 
did you enjoy it? - 
hat es geschmeckt? - 
was it OK? - 
hat es geschmeckt? - 
was everything to your satisfaction? form - 
so, ich hoffe, es schmeckt! - 
so, I hope you enjoy it! - 
na, schmeckt's? — klar, und wie! - 
well, is it OK? — you bet! 
}\\
\end{tabular}
}
%===schmeißen===
\card{\normalfont \Huge schmeißen}{
\begin{tabular}{lll}
\parbox[t][][t]{2.0 cm}{\normalfont \raggedleft ich\\du\\er/sie/es\\wir\\ihr\\sie} &    
\parbox[t][][t]{2cm}{\normalfont schmeiße\\schmeißt\\schmeißt\\schmeißen\\schmeißt\\schmeißen} &
\parbox[t][][t]{2cm}{\normalfont schmiss\\schmissest\\schmiss\\schmissen\\schmisst\\schmissen}\\
\end{tabular}
\begin{tabular}{l}
\parbox[t][][t]{8cm}{}\\
\parbox[t][][t]{8cm}{\normalfont \footnotesize 
etw (irgendwohin/nach jdm) schmeißen - 
to throw (or - 
fam chuck) sth (somewhere/at sb) - 
etw (irgendwohin/nach jdm) schmeißen - 
(mit Kraft) - 
to hurl (or fling) sth (somewhere/at sb) - 
etw (für jdn) schmeißen - 
to stand sth (for sb) - 
eine Party schmeißen - 
to throw a party 
}\\
\end{tabular}
}
%===schmelzen===
\card{\normalfont \Huge schmelzen}{
\begin{tabular}{lll}
\parbox[t][][t]{2.0 cm}{\normalfont \raggedleft ich\\du\\er/sie/es\\wir\\ihr\\sie} &    
\parbox[t][][t]{2cm}{\normalfont schmelze\\schmilzt\\schmilzt\\schmelzen\\schmelzt\\schmelzen} &
\parbox[t][][t]{2cm}{\normalfont schmolz\\schmolzest\\schmolz\\schmolzen\\schmolzt\\schmolzen}\\
\end{tabular}
\begin{tabular}{l}
\parbox[t][][t]{8cm}{}\\
\parbox[t][][t]{8cm}{\normalfont \footnotesize 
schmelzen Eis, Schnee, Metalle, usw. - 
to melt - 
schmelzen - 
to melt - 
jds Herz zum Schmelzen bringen - 
to melt sb's heart - 
schmelzen - 
to melt away - 
schmelzen Vermögen - 
to dwindle 
}\\
\end{tabular}
}
%===schnauben===
\card{\normalfont \Huge schnauben}{
\begin{tabular}{lll}
\parbox[t][][t]{2.0 cm}{\normalfont \raggedleft ich\\du\\er/sie/es\\wir\\ihr\\sie} &    
\parbox[t][][t]{2cm}{\normalfont schnaube\\schnaubst\\schnaubt\\schnauben\\schnaubt\\schnauben} &
\parbox[t][][t]{2cm}{\normalfont schnaubte \\schnaubtest \\schnaubte\\schnaubten\\schnaubtet\\schnaubten }\\
\end{tabular}
\begin{tabular}{l}
\parbox[t][][t]{8cm}{}\\
\parbox[t][][t]{8cm}{\normalfont \footnotesize 
(vor etw dat) schnauben - 
to snort (with sth) - 
vor Wut schnauben - 
to snort with rage - 
schnauben - 
to snort - 
wütend schnaubend - 
snorting with rage - 
schnauben - 
to blow one's nose 
}\\
\end{tabular}
}
%===schneiden===
\card{\normalfont \Huge schneiden}{
\begin{tabular}{lll}
\parbox[t][][t]{2.0 cm}{\normalfont \raggedleft ich\\du\\er/sie/es\\wir\\ihr\\sie} &    
\parbox[t][][t]{2cm}{\normalfont schneide\\schneidest\\schneidet\\schneiden\\schneidet\\schneiden} &
\parbox[t][][t]{2cm}{\normalfont schnitt\\schnittest\\schnitt\\schnitten\\schnittet\\schnitten}\\
\end{tabular}
\begin{tabular}{l}
\parbox[t][][t]{8cm}{}\\
\parbox[t][][t]{8cm}{\normalfont \footnotesize 
etw schneiden - 
to cut sth - 
Wurst in die Suppe schneiden - 
to slice sausage into the soup - 
etw schneiden - 
to cut (or trim) sth - 
einen Baum schneiden - 
to prune a tree - 
das Gras schneiden - 
to cut (or mow) the grass 
}\\
\end{tabular}
}
%===schrauben===
\card{\normalfont \Huge schrauben}{
\begin{tabular}{lll}
\parbox[t][][t]{2.0 cm}{\normalfont \raggedleft ich\\du\\er/sie/es\\wir\\ihr\\sie} &    
\parbox[t][][t]{2cm}{\normalfont schraube\\schraubst\\schraubt\\schrauben\\schraubt\\schrauben} &
\parbox[t][][t]{2cm}{\normalfont schraubte\\schraubtest\\schraubte\\schraubten\\schraubtet\\schraubten}\\
\end{tabular}
\begin{tabular}{l}
\parbox[t][][t]{8cm}{}\\
\parbox[t][][t]{8cm}{\normalfont \footnotesize 
etw (an (o. auf) etw akk) schrauben - 
to screw sth (into/onto sth) - 
etw (auf etw akk) schrauben - 
to push sth up (to sth) - 
etw höher/niedriger schrauben - 
to raise/lower sth - 
etw fester/loser schrauben - 
to tighten/loosen sth - 
eine Glühbirne aus der Fassung schrauben - 
to unscrew a light bulb 
}\\
\end{tabular}
}
%===schrecken===
\card{\normalfont \Huge schrecken}{
\begin{tabular}{lll}
\parbox[t][][t]{2.0 cm}{\normalfont \raggedleft ich\\du\\er/sie/es\\wir\\ihr\\sie} &    
\parbox[t][][t]{2cm}{\normalfont schrecke\\schreckst\\schreckt\\schrecken\\schreckt\\schrecken} &
\parbox[t][][t]{2cm}{\normalfont schreckte\\schrecktest\\schreckte\\schreckten\\schrecktet\\schreckten}\\
\end{tabular}
\begin{tabular}{l}
\parbox[t][][t]{8cm}{}\\
\parbox[t][][t]{8cm}{\normalfont \footnotesize 
etw schreckt jdn - 
sth frightens (or scares) sb - 
(aus etw dat) schrecken - 
to be startled (out of sth) - 
Schrecken - 
fright - 
Schrecken - 
horror - 
Schrecken (stärker) - 
terror 
}\\
\end{tabular}
}
%===schreiben===
\card{\normalfont \Huge schreiben}{
\begin{tabular}{lll}
\parbox[t][][t]{2.0 cm}{\normalfont \raggedleft ich\\du\\er/sie/es\\wir\\ihr\\sie} &    
\parbox[t][][t]{2cm}{\normalfont schreibe\\schreibst\\schreibt\\schreiben\\schreibt\\schreiben} &
\parbox[t][][t]{2cm}{\normalfont schrieb\\schriebst\\schrieb\\schrieben\\schriebt\\schrieben}\\
\end{tabular}
\begin{tabular}{l}
\parbox[t][][t]{8cm}{}\\
\parbox[t][][t]{8cm}{\normalfont \footnotesize 
etw schreiben - 
to write sth - 
etw schreiben - 
LIT, MUS a. - 
to compose sth - 
eine Arbeit schreiben - 
SCH - 
to do a paper - 
jdm einen Brief schreiben - 
to write a letter to sb (or sb a letter) 
}\\
\end{tabular}
}
%===schreien===
\card{\normalfont \Huge schreien}{
\begin{tabular}{lll}
\parbox[t][][t]{2.0 cm}{\normalfont \raggedleft ich\\du\\er/sie/es\\wir\\ihr\\sie} &    
\parbox[t][][t]{2cm}{\normalfont schreie\\schreist\\schreit\\schreien\\schreit\\schreien} &
\parbox[t][][t]{2cm}{\normalfont schrie\\schriest\\schrie\\schrien\\schriet\\schrien}\\
\end{tabular}
\begin{tabular}{l}
\parbox[t][][t]{8cm}{}\\
\parbox[t][][t]{8cm}{\normalfont \footnotesize 
schreien - 
to yell - 
(mit jdm) schreien - 
to shout (at sb) - 
schreien - 
to cry - 
schreien (Eule) - 
to screech - 
(nach jdm) schreien - 
to shout (for sb) 
}\\
\end{tabular}
}
%===schreiten===
\card{\normalfont \Huge schreiten}{
\begin{tabular}{lll}
\parbox[t][][t]{2.0 cm}{\normalfont \raggedleft ich\\du\\er/sie/es\\wir\\ihr\\sie} &    
\parbox[t][][t]{2cm}{\normalfont schreite\\schreitest\\schreitet\\schreiten\\schreitet\\schreiten} &
\parbox[t][][t]{2cm}{\normalfont schritt\\schrittest\\schritt\\schritten\\schrittet\\schritten}\\
\end{tabular}
\begin{tabular}{l}
\parbox[t][][t]{8cm}{}\\
\parbox[t][][t]{8cm}{\normalfont \footnotesize 
(irgendwohin) schreiten - 
to stride (somewhere) - 
(zu etw dat) schreiten - 
to proceed (with sth) - 
zur Tat schreiten - 
to get down to action (or work) - 
zur Abstimmung schreiten - 
to go to a vote - 
Wahlurne - 
ballot box 
}\\
\end{tabular}
}
%===schützen===
\card{\normalfont \Huge schützen}{
\begin{tabular}{lll}
\parbox[t][][t]{2.0 cm}{\normalfont \raggedleft ich\\du\\er/sie/es\\wir\\ihr\\sie} &    
\parbox[t][][t]{2cm}{\normalfont schütze\\schützt\\schützt\\schützen\\schützt\\schützen} &
\parbox[t][][t]{2cm}{\normalfont schützte\\schütztest\\schützte\\schützten\\schütztet\\schützten}\\
\end{tabular}
\begin{tabular}{l}
\parbox[t][][t]{8cm}{}\\
\parbox[t][][t]{8cm}{\normalfont \footnotesize 
jdn (vor jdm/etw) schützen - 
to protect sb (against (or from) sb/sth) - 
sich akk (vor etw dat - 
(o. gegen etw akk - 
)) schützen - 
to protect oneself (against sth) - 
Gott schütze dich! - 
may the Lord protect you! - 
etw (vor etw dat) schützen - 
to keep sth away from sth 
}\\
\end{tabular}
}
%===schweben===
\card{\normalfont \Huge schweben}{
\begin{tabular}{lll}
\parbox[t][][t]{2.0 cm}{\normalfont \raggedleft ich\\du\\er/sie/es\\wir\\ihr\\sie} &    
\parbox[t][][t]{2cm}{\normalfont schwebe\\schwebst\\schwebt\\schweben\\schwebt\\schweben} &
\parbox[t][][t]{2cm}{\normalfont schwebte\\schwebtest\\schwebte\\schwebten\\schwebtet\\schwebten}\\
\end{tabular}
\begin{tabular}{l}
\parbox[t][][t]{8cm}{}\\
\parbox[t][][t]{8cm}{\normalfont \footnotesize 
(irgendwo) schweben - 
to float (somewhere) - 
(irgendwo) schweben - 
Drachenflieger, Vogel - 
to hover (somewhere) - 
in Lebensgefahr schweben - 
to be in danger of one's life - 
in Lebensgefahr schweben - 
(Patient) - 
to be in a critical condition 
}\\
\end{tabular}
}
%===schweigen===
\card{\normalfont \Huge schweigen}{
\begin{tabular}{lll}
\parbox[t][][t]{2.0 cm}{\normalfont \raggedleft ich\\du\\er/sie/es\\wir\\ihr\\sie} &    
\parbox[t][][t]{2cm}{\normalfont schweige\\schweigst\\schweigt\\schweigen\\schweigt\\schweigen} &
\parbox[t][][t]{2cm}{\normalfont schwieg\\schwiegst\\schwieg\\schwiegen\\schwiegt\\schwiegen}\\
\end{tabular}
\begin{tabular}{l}
\parbox[t][][t]{8cm}{}\\
\parbox[t][][t]{8cm}{\normalfont \footnotesize 
schweigen - 
to remain silent (or keep quiet) - 
schweig, ich will kein Wort mehr hören - 
(that's) enough, I don't want to hear another word (from you) - 
er schwieg betroffen, als er das hörte - 
he was so shocked he couldn't say anything - 
in schweigender Anklage - 
in silent reproach - 
zu etw akk schweigen - 
to say nothing in (or make no) reply to sth 
}\\
\end{tabular}
}
%===schwellen===
\card{\normalfont \Huge schwellen}{
\begin{tabular}{lll}
\parbox[t][][t]{2.0 cm}{\normalfont \raggedleft ich\\du\\er/sie/es\\wir\\ihr\\sie} &    
\parbox[t][][t]{2cm}{\normalfont schwelle\\schwillst\\schwillt\\schwellen\\schwellt\\schwellen} &
\parbox[t][][t]{2cm}{\normalfont schwoll\\schwollst\\schwoll\\schwollen\\schwollt\\schwollen}\\
\end{tabular}
\begin{tabular}{l}
\parbox[t][][t]{8cm}{}\\
\parbox[t][][t]{8cm}{\normalfont \footnotesize 
schwellen - 
to swell (up) - 
der Knöchel ist ja ganz geschwollen - 
the ankle is very swollen - 
schwellen - 
to grow - 
jdm schwillt der Kamm fam - 
sb gets too big for their boots fam - 
jdm schwillt der Kamm fam - 
sb gets cocky 
}\\
\end{tabular}
}
%===schwimmen===
\card{\normalfont \Huge schwimmen}{
\begin{tabular}{lll}
\parbox[t][][t]{2.0 cm}{\normalfont \raggedleft ich\\du\\er/sie/es\\wir\\ihr\\sie} &    
\parbox[t][][t]{2cm}{\normalfont schwimme\\schwimmst\\schwimmt\\schwimmen\\schwimmt\\schwimmen} &
\parbox[t][][t]{2cm}{\normalfont schwamm\\schwammst\\schwamm\\schwammen\\schwammt\\schwammen}\\
\end{tabular}
\begin{tabular}{l}
\parbox[t][][t]{8cm}{}\\
\parbox[t][][t]{8cm}{\normalfont \footnotesize 
schwimmen - 
to swim - 
ich kann nicht schwimmen - 
I can't swim - 
schwimmen gehen - 
to go swimming - 
auf etw dat - 
(o. in etw dat - 
) - 
schwimmen 
}\\
\end{tabular}
}
%===schwinden===
\card{\normalfont \Huge schwinden}{
\begin{tabular}{lll}
\parbox[t][][t]{2.0 cm}{\normalfont \raggedleft ich\\du\\er/sie/es\\wir\\ihr\\sie} &    
\parbox[t][][t]{2cm}{\normalfont schwinde\\schwindest\\schwindet\\schwinden\\schwindet\\schwinden} &
\parbox[t][][t]{2cm}{\normalfont schwand\\schwandest \\schwand\\schwanden\\schwandet\\schwanden}\\
\end{tabular}
\begin{tabular}{l}
\parbox[t][][t]{8cm}{}\\
\parbox[t][][t]{8cm}{\normalfont \footnotesize 
schwinden - 
to run out - 
schwinden - 
to dwindle - 
im Schwinden begriffen sein - 
to be running out (or dwindling) - 
etw schwindet - 
sth is fading away - 
etw schwindet Effekt, (schmerzstillende) Wirkung - 
to be wearing off 
}\\
\end{tabular}
}
%===schwingen===
\card{\normalfont \Huge schwingen}{
\begin{tabular}{lll}
\parbox[t][][t]{2.0 cm}{\normalfont \raggedleft ich\\du\\er/sie/es\\wir\\ihr\\sie} &    
\parbox[t][][t]{2cm}{\normalfont schwinge\\schwingst\\schwingt\\schwingen\\schwingt\\schwingen} &
\parbox[t][][t]{2cm}{\normalfont schwang\\schwangst\\schwang\\schwangen\\schwangt\\schwangen}\\
\end{tabular}
\begin{tabular}{l}
\parbox[t][][t]{8cm}{}\\
\parbox[t][][t]{8cm}{\normalfont \footnotesize 
etw schwingen - 
to wave sth - 
etw schwingen - 
to brandish sth - 
er schwang die Axt - 
he brandished the axe - 
jdn/etw schwingen - 
to swing sb/sth - 
der Dirigent schwingt seinen Taktstock - 
the conductor flourishes his baton 
}\\
\end{tabular}
}
%===schwitzen===
\card{\normalfont \Huge schwitzen}{
\begin{tabular}{lll}
\parbox[t][][t]{2.0 cm}{\normalfont \raggedleft ich\\du\\er/sie/es\\wir\\ihr\\sie} &    
\parbox[t][][t]{2cm}{\normalfont schwitze\\schwitzt\\schwitzt\\schwitzen\\schwitzt\\schwitzen} &
\parbox[t][][t]{2cm}{\normalfont schwitzte\\schwitztest\\schwitzte\\schwitzten\\schwitztet\\schwitzten}\\
\end{tabular}
\begin{tabular}{l}
\parbox[t][][t]{8cm}{}\\
\parbox[t][][t]{8cm}{\normalfont \footnotesize 
schwitzen - 
to sweat (or - 
form perspire) - 
schwitzen - 
to steam (or become steamed) up - 
schwitzen - 
to sweat over sth - 
er schwitzt noch immer über der schwierigen Rechenaufgabe - 
he's still sweating over the difficult sums - 
sich akk nass schwitzen 
}\\
\end{tabular}
}
%===schwören===
\card{\normalfont \Huge schwören}{
\begin{tabular}{lll}
\parbox[t][][t]{2.0 cm}{\normalfont \raggedleft ich\\du\\er/sie/es\\wir\\ihr\\sie} &    
\parbox[t][][t]{2cm}{\normalfont schwöre\\schwörst\\schwört\\schwören\\schwört\\schwören} &
\parbox[t][][t]{2cm}{\normalfont schwor\\schworst\\schwor\\schworen\\schwort\\schworen}\\
\end{tabular}
\begin{tabular}{l}
\parbox[t][][t]{8cm}{}\\
\parbox[t][][t]{8cm}{\normalfont \footnotesize 
schwören - 
to swear - 
auf die Verfassung schwören - 
to swear on the constitution - 
(auf jdn/etw) schwören - 
to swear (by sb/on (or by) sth) - 
er schwört auf Vitamin C - 
he swears by vitamin C - 
etw schwören - 
to swear sth 
}\\
\end{tabular}
}
%===sehen===
\card{\normalfont \Huge sehen}{
\begin{tabular}{lll}
\parbox[t][][t]{2.0 cm}{\normalfont \raggedleft ich\\du\\er/sie/es\\wir\\ihr\\sie} &    
\parbox[t][][t]{2cm}{\normalfont sehe\\siehst\\sieht\\sehen\\seht\\sehen} &
\parbox[t][][t]{2cm}{\normalfont sah\\sahst\\sah\\sahen\\saht\\sahen}\\
\end{tabular}
\begin{tabular}{l}
\parbox[t][][t]{8cm}{}\\
\parbox[t][][t]{8cm}{\normalfont \footnotesize 
jdn/etw sehen - 
to see sb/sth - 
siehst du irgendwo meine Schlüssel? - 
can you see my keys anywhere? - 
man darf dich bei mir nicht sehen - 
you can't be seen with me - 
ich habe ihn vom Fenster aus gesehen - 
I saw him from the window - 
das sieht man, das kann man sehen - 
you can see that 
}\\
\end{tabular}
}
%===sein===
\card{\normalfont \Huge sein}{
\begin{tabular}{lll}
\parbox[t][][t]{2.0 cm}{\normalfont \raggedleft ich\\du\\er/sie/es\\wir\\ihr\\sie} &    
\parbox[t][][t]{2cm}{\normalfont bin\\bist\\ist\\sind\\seid\\sind} &
\parbox[t][][t]{2cm}{\normalfont war\\warst\\war\\waren\\wart\\waren}\\
\end{tabular}
\begin{tabular}{l}
\parbox[t][][t]{8cm}{}\\
\parbox[t][][t]{8cm}{\normalfont \footnotesize 
wie ist das Wetter? - 
what's the weather like? - 
wie ist der Wein? - 
how's the wine? - 
wie wäre es mit einem Kaffee? - 
how about a coffee? - 
wie war das noch mit morgen/dem Klempner? - 
what was that about tomorrow/a plumber? - 
also, wie ist es? macht ihr mit? - 
so, what about it? do you want to join in? 
}\\
\end{tabular}
}
%===senden===
\card{\normalfont \Huge senden}{
\begin{tabular}{lll}
\parbox[t][][t]{2.0 cm}{\normalfont \raggedleft ich\\du\\er/sie/es\\wir\\ihr\\sie} &    
\parbox[t][][t]{2cm}{\normalfont sende\\sendest\\sendet\\senden\\sendet\\senden} &
\parbox[t][][t]{2cm}{\normalfont sandte \\sandtest\\sandte\\sandten \\sandtet \\sandten }\\
\end{tabular}
\begin{tabular}{l}
\parbox[t][][t]{8cm}{}\\
\parbox[t][][t]{8cm}{\normalfont \footnotesize 
senden - 
to broadcast - 
ein Fernsehspiel senden - 
to broadcast a television play - 
ein Signal/eine Botschaft senden - 
to transmit a signal/message - 
senden - 
to be on the air - 
jdn/etw senden - 
to send sb/sth 
}\\
\end{tabular}
}
%===setzen===
\card{\normalfont \Huge setzen}{
\begin{tabular}{lll}
\parbox[t][][t]{2.0 cm}{\normalfont \raggedleft ich\\du\\er/sie/es\\wir\\ihr\\sie} &    
\parbox[t][][t]{2cm}{\normalfont setze\\setzt\\setzt\\setzen\\setzt\\setzen} &
\parbox[t][][t]{2cm}{\normalfont setzte\\setztest\\setzte\\setzten\\setztet\\setzten}\\
\end{tabular}
\begin{tabular}{l}
\parbox[t][][t]{8cm}{}\\
\parbox[t][][t]{8cm}{\normalfont \footnotesize 
jdn/etw irgendwohin setzen - 
to put (or place) sb/sth somewhere - 
ein Kind jdm auf den Schoß setzen - 
to put (or sit)  a child on sb's lap - 
den Topf auf den Herd setzen - 
to place the pot on the stove - 
das Glas an den Mund setzen - 
to put the glass to one's lips - 
einen Hund auf eine Fährte setzen - 
to put a dog on a trail 
}\\
\end{tabular}
}
%===sichern===
\card{\normalfont \Huge sichern}{
\begin{tabular}{lll}
\parbox[t][][t]{2.0 cm}{\normalfont \raggedleft ich\\du\\er/sie/es\\wir\\ihr\\sie} &    
\parbox[t][][t]{2cm}{\normalfont sichere\\sicherst\\sichert\\sichern\\sichert\\sichern} &
\parbox[t][][t]{2cm}{\normalfont sicherte\\sichertest\\sicherte\\sicherten\\sichertet\\sicherten}\\
\end{tabular}
\begin{tabular}{l}
\parbox[t][][t]{8cm}{}\\
\parbox[t][][t]{8cm}{\normalfont \footnotesize 
etw (durch etw akk - 
(o. mit etw dat - 
)) (gegen etw akk) sichern - 
to safeguard sth (with sth) (from sth) - 
die Grenzen/den Staat sichern - 
to safeguard the borders/state - 
eine Schusswaffe sichern - 
to put on a safety catch on a firearm - 
die Tür/Fenster sichern - 
to secure the door/windows 
}\\
\end{tabular}
}
%===sieden===
\card{\normalfont \Huge sieden}{
\begin{tabular}{lll}
\parbox[t][][t]{2.0 cm}{\normalfont \raggedleft ich\\du\\er/sie/es\\wir\\ihr\\sie} &    
\parbox[t][][t]{2cm}{\normalfont siede\\siedest\\siedet\\sieden\\siedet\\sieden} &
\parbox[t][][t]{2cm}{\normalfont siedete \\siedetest\\siedete \\siedeten \\siedetet\\siedeten}\\
\end{tabular}
\begin{tabular}{l}
\parbox[t][][t]{8cm}{}\\
\parbox[t][][t]{8cm}{\normalfont \footnotesize 
sieden - 
to boil - 
etw zum Sieden bringen - 
to bring sth to the boil - 
jdn (mit etw dat) zum Sieden bringen - 
to drive sb mad (with sth) - 
siedend heiß fam - 
boiling (or scalding) hot - 
es ist mir siedend heiß eingefallen, dass ... fig fam - 
I suddenly remembered that ... 
}\\
\end{tabular}
}
%===singen===
\card{\normalfont \Huge singen}{
\begin{tabular}{lll}
\parbox[t][][t]{2.0 cm}{\normalfont \raggedleft ich\\du\\er/sie/es\\wir\\ihr\\sie} &    
\parbox[t][][t]{2cm}{\normalfont singe\\singst\\singt\\singen\\singt\\singen} &
\parbox[t][][t]{2cm}{\normalfont sang\\sangst\\sang\\sangen\\sangt\\sangen}\\
\end{tabular}
\begin{tabular}{l}
\parbox[t][][t]{8cm}{}\\
\parbox[t][][t]{8cm}{\normalfont \footnotesize 
singen - 
to sing - 
singen (Vögel a.) - 
to carol liter - 
zu etw dat - 
singen - 
to sing to sth - 
singen - 
to squeal sl - 
singen 
}\\
\end{tabular}
}
%===sinken===
\card{\normalfont \Huge sinken}{
\begin{tabular}{lll}
\parbox[t][][t]{2.0 cm}{\normalfont \raggedleft ich\\du\\er/sie/es\\wir\\ihr\\sie} &    
\parbox[t][][t]{2cm}{\normalfont sinke\\sinkst\\sinkt\\sinken\\sinkt\\sinken} &
\parbox[t][][t]{2cm}{\normalfont sank\\sankst\\sank\\sanken\\sankt\\sanken}\\
\end{tabular}
\begin{tabular}{l}
\parbox[t][][t]{8cm}{}\\
\parbox[t][][t]{8cm}{\normalfont \footnotesize 
sinken - 
to sink - 
sinken Schiff - 
to go down - 
sinken Schiff - 
to founder - 
auf den Grund sinken - 
to sink to the bottom - 
sich akk - 
sinken lassen 
}\\
\end{tabular}
}
%===sinnen===
\card{\normalfont \Huge sinnen}{
\begin{tabular}{lll}
\parbox[t][][t]{2.0 cm}{\normalfont \raggedleft ich\\du\\er/sie/es\\wir\\ihr\\sie} &    
\parbox[t][][t]{2cm}{\normalfont sinne\\sinnst\\sinnt\\sinnen\\sinnt\\sinnen} &
\parbox[t][][t]{2cm}{\normalfont sann\\sannst\\sann\\sannen\\sannt\\sannen}\\
\end{tabular}
\begin{tabular}{l}
\parbox[t][][t]{8cm}{}\\
\parbox[t][][t]{8cm}{\normalfont \footnotesize 
(über etw akk) sinnen - 
to brood (or muse)  (over sth) - 
(über etw akk) sinnen - 
to ponder ((on) sth) - 
(über etw akk) sinnen - 
to reflect (on sth) - 
sinnend - 
brooding/broodingly - 
sinnend - 
musing/musingly 
}\\
\end{tabular}
}
%===sitzen===
\card{\normalfont \Huge sitzen}{
\begin{tabular}{lll}
\parbox[t][][t]{2.0 cm}{\normalfont \raggedleft ich\\du\\er/sie/es\\wir\\ihr\\sie} &    
\parbox[t][][t]{2cm}{\normalfont sitze\\sitzt\\sitzt\\sitzen\\sitzt\\sitzen} &
\parbox[t][][t]{2cm}{\normalfont saß\\saßest\\saß\\saßen\\saßt\\saßen}\\
\end{tabular}
\begin{tabular}{l}
\parbox[t][][t]{8cm}{}\\
\parbox[t][][t]{8cm}{\normalfont \footnotesize 
sitzen - 
to sit - 
sitzen (auf einer Kante, Vogel a.) - 
to perch - 
wir saßen auf Barhockern und tranken ein Bier - 
we perched on bar stools and had a beer - 
sitz! (Befehl an Hund) - 
sit! - 
(bitte) bleib/bleiben Sie sitzen! - 
(please) don't get up 
}\\
\end{tabular}
}
%===sollen===
\card{\normalfont \Huge sollen}{
\begin{tabular}{lll}
\parbox[t][][t]{2.0 cm}{\normalfont \raggedleft ich\\du\\er/sie/es\\wir\\ihr\\sie} &    
\parbox[t][][t]{2cm}{\normalfont soll\\sollst\\soll\\sollen\\sollt\\sollen} &
\parbox[t][][t]{2cm}{\normalfont sollte\\solltest\\sollte\\sollten\\solltet\\sollten}\\
\end{tabular}
\begin{tabular}{l}
\parbox[t][][t]{8cm}{}\\
\parbox[t][][t]{8cm}{\normalfont \footnotesize 
jd soll etw tun - 
sb is to do sth - 
sie soll am Tag zwei Tabletten einnehmen - 
she's to (or she must) take two pills a day - 
er soll sofort kommen - 
he is to come immediately - 
ich soll Ihnen sagen, dass ... - 
I am (or I've been asked) to tell you that ... - 
ich soll dir schöne Grüße von Richard bestellen - 
Richard asked me to give you his best wishes 
}\\
\end{tabular}
}
%===sorgen===
\card{\normalfont \Huge sorgen}{
\begin{tabular}{lll}
\parbox[t][][t]{2.0 cm}{\normalfont \raggedleft ich\\du\\er/sie/es\\wir\\ihr\\sie} &    
\parbox[t][][t]{2cm}{\normalfont sorge\\sorgst\\sorgt\\sorgen\\sorgt\\sorgen} &
\parbox[t][][t]{2cm}{\normalfont sorgte\\sorgtest\\sorgte\\sorgten\\sorgtet\\sorgten}\\
\end{tabular}
\begin{tabular}{l}
\parbox[t][][t]{8cm}{}\\
\parbox[t][][t]{8cm}{\normalfont \footnotesize 
für jdn sorgen - 
to provide for sb - 
für jdn sorgen - 
to look after sb - 
für etw akk - 
sorgen - 
to get sth - 
ich sorge für die Getränke - 
I'll get (or take care of) the drinks - 
für gute Stimmung/die Musik sorgen 
}\\
\end{tabular}
}
%===spalten===
\card{\normalfont \Huge spalten}{
\begin{tabular}{lll}
\parbox[t][][t]{2.0 cm}{\normalfont \raggedleft ich\\du\\er/sie/es\\wir\\ihr\\sie} &    
\parbox[t][][t]{2cm}{\normalfont spalte\\spaltest\\spaltet\\spalten\\spaltet\\spalten} &
\parbox[t][][t]{2cm}{\normalfont spaltete\\spaltetest\\spaltete\\spalteten\\spaltetet\\spalteten}\\
\end{tabular}
\begin{tabular}{l}
\parbox[t][][t]{8cm}{}\\
\parbox[t][][t]{8cm}{\normalfont \footnotesize 
etw spalten - 
to split (or - 
liter cleave) sth - 
Holz spalten - 
in +akk - 
to chop wood into - 
etw spalten - 
to rend (or divide) sth - 
die Partei spalten - 
to split (or divide) the party 
}\\
\end{tabular}
}
%===sparen===
\card{\normalfont \Huge sparen}{
\begin{tabular}{lll}
\parbox[t][][t]{2.0 cm}{\normalfont \raggedleft ich\\du\\er/sie/es\\wir\\ihr\\sie} &    
\parbox[t][][t]{2cm}{\normalfont spare\\sparst\\spart\\sparen\\spart\\sparen} &
\parbox[t][][t]{2cm}{\normalfont sparte\\spartest\\sparte\\sparten\\spartet\\sparten}\\
\end{tabular}
\begin{tabular}{l}
\parbox[t][][t]{8cm}{}\\
\parbox[t][][t]{8cm}{\normalfont \footnotesize 
etw sparen - 
to save sth - 
etw sparen - 
to save sth - 
Arbeit/Energie/Strom/Zeit sparen - 
to save work/energy/electricity/time - 
jdm/sich etw sparen - 
to spare sb/oneself sth - 
jdm/sich die Mühe/den Ärger sparen - 
to spare sb/oneself the effort/trouble 
}\\
\end{tabular}
}
%===spazieren===
\card{\normalfont \Huge spazieren}{
\begin{tabular}{lll}
\parbox[t][][t]{2.0 cm}{\normalfont \raggedleft ich\\du\\er/sie/es\\wir\\ihr\\sie} &    
\parbox[t][][t]{2cm}{\normalfont spaziere\\spazierst\\spaziert\\spazieren\\spaziert\\spazieren} &
\parbox[t][][t]{2cm}{\normalfont spazierte\\spaziertest\\spazierte\\spazierten\\spaziertet\\spazierten}\\
\end{tabular}
\begin{tabular}{l}
\parbox[t][][t]{8cm}{}\\
\parbox[t][][t]{8cm}{\normalfont \footnotesize 
spazieren - 
to stroll (or walk) - 
(auf und ab) spazieren - 
to stroll (up and down) - 
jdn/etw spazieren führen - 
to take sb/sth for a walk - 
spazieren fahren - 
to go for a drive - 
jdn spazieren fahren - 
to take sb out for a drive 
}\\
\end{tabular}
}
%===speien===
\card{\normalfont \Huge speien}{
\begin{tabular}{lll}
\parbox[t][][t]{2.0 cm}{\normalfont \raggedleft ich\\du\\er/sie/es\\wir\\ihr\\sie} &    
\parbox[t][][t]{2cm}{\normalfont speie\\speist\\speit\\speien\\speit\\speien} &
\parbox[t][][t]{2cm}{\normalfont spie\\spiest\\spie\\spien\\spiet\\spien}\\
\end{tabular}
\begin{tabular}{l}
\parbox[t][][t]{8cm}{}\\
\parbox[t][][t]{8cm}{\normalfont \footnotesize 
etw (auf etw akk) speien - 
to spew sth (onto sth) - 
etw (irgendwohin) speien - 
to spit sth (somewhere) - 
Gift - 
poison - 
Gift - 
toxin spec - 
Gift (Schlangengift) - 
venom 
}\\
\end{tabular}
}
%===spielen===
\card{\normalfont \Huge spielen}{
\begin{tabular}{lll}
\parbox[t][][t]{2.0 cm}{\normalfont \raggedleft ich\\du\\er/sie/es\\wir\\ihr\\sie} &    
\parbox[t][][t]{2cm}{\normalfont spiele\\spielst\\spielt\\spielen\\spielt\\spielen} &
\parbox[t][][t]{2cm}{\normalfont spielte\\spieltest\\spielte\\spielten\\spieltet\\spielten}\\
\end{tabular}
\begin{tabular}{l}
\parbox[t][][t]{8cm}{}\\
\parbox[t][][t]{8cm}{\normalfont \footnotesize 
etw spielen - 
to play sth - 
Lotto spielen - 
to play the lottery - 
Basketball/Schach/Tennis spielen - 
to play basketball/chess/tennis - 
Gitarre/Klavier spielen - 
to play the guitar/piano - 
etw spielen - 
to play sth 
}\\
\end{tabular}
}
%===spinnen===
\card{\normalfont \Huge spinnen}{
\begin{tabular}{lll}
\parbox[t][][t]{2.0 cm}{\normalfont \raggedleft ich\\du\\er/sie/es\\wir\\ihr\\sie} &    
\parbox[t][][t]{2cm}{\normalfont spinne\\spinnst\\spinnt\\spinnen\\spinnt\\spinnen} &
\parbox[t][][t]{2cm}{\normalfont spann\\spannst\\spann\\spannen\\spannt\\spannen}\\
\end{tabular}
\begin{tabular}{l}
\parbox[t][][t]{8cm}{}\\
\parbox[t][][t]{8cm}{\normalfont \footnotesize 
spinnen - 
to spin - 
Wolle spinnen - 
to spin wool - 
spinnen - 
to invent (or concoct) (or spin) - 
eine Geschichte/Lüge spinnen - 
to spin (or invent)  a story/lie - 
spinnen - 
to spin 
}\\
\end{tabular}
}
%===sprechen===
\card{\normalfont \Huge sprechen}{
\begin{tabular}{lll}
\parbox[t][][t]{2.0 cm}{\normalfont \raggedleft ich\\du\\er/sie/es\\wir\\ihr\\sie} &    
\parbox[t][][t]{2cm}{\normalfont spreche\\sprichst\\spricht\\sprechen\\sprecht\\sprechen} &
\parbox[t][][t]{2cm}{\normalfont sprach\\sprachst\\sprach\\sprachen\\spracht\\sprachen}\\
\end{tabular}
\begin{tabular}{l}
\parbox[t][][t]{8cm}{}\\
\parbox[t][][t]{8cm}{\normalfont \footnotesize 
sprechen - 
to speak - 
sprechen - 
to talk - 
kann das Kind schon sprechen? - 
can the baby talk yet? - 
ich konnte vor Aufregung kaum sprechen - 
I could hardly speak for excitement - 
sprich!/sprechen Sie! - 
speak! 
}\\
\end{tabular}
}
%===sprießen===
\card{\normalfont \Huge sprießen}{
\begin{tabular}{lll}
\parbox[t][][t]{2.0 cm}{\normalfont \raggedleft ich\\du\\er/sie/es\\wir\\ihr\\sie} &    
\parbox[t][][t]{2cm}{\normalfont sprieße\\sprießt\\sprießt\\sprießen\\sprießt\\sprießen} &
\parbox[t][][t]{2cm}{\normalfont sprosste\\sprosstest\\sprosste\\sprossten\\sprosstet\\sprossten}\\
\end{tabular}
\begin{tabular}{l}
\parbox[t][][t]{8cm}{}\\
\parbox[t][][t]{8cm}{\normalfont \footnotesize 
sprießen - 
to spring up (or shoot) - 
sprießen Bart, Brüste, Haare - 
to grow 
}\\
\end{tabular}
}
%===springen===
\card{\normalfont \Huge springen}{
\begin{tabular}{lll}
\parbox[t][][t]{2.0 cm}{\normalfont \raggedleft ich\\du\\er/sie/es\\wir\\ihr\\sie} &    
\parbox[t][][t]{2cm}{\normalfont springe\\springst\\springt\\springen\\springt\\springen} &
\parbox[t][][t]{2cm}{\normalfont sprang\\sprangst\\sprang\\sprangen\\sprangt\\sprangen}\\
\end{tabular}
\begin{tabular}{l}
\parbox[t][][t]{8cm}{}\\
\parbox[t][][t]{8cm}{\normalfont \footnotesize 
springen - 
to jump (or leap) - 
die Kinder sprangen hin und her - 
the children leapt (or jumped) about - 
der Hase sprang über die Wiese - 
the rabbit leapt (or bounded) across the meadow - 
springen - 
to jump - 
springen - 
to jump 
}\\
\end{tabular}
}
%===spülen===
\card{\normalfont \Huge spülen}{
\begin{tabular}{lll}
\parbox[t][][t]{2.0 cm}{\normalfont \raggedleft ich\\du\\er/sie/es\\wir\\ihr\\sie} &    
\parbox[t][][t]{2cm}{\normalfont spüle\\spülst\\spült\\spülen\\spült\\spülen} &
\parbox[t][][t]{2cm}{\normalfont spülte\\spültest\\spülte\\spülten\\spültet\\spülten}\\
\end{tabular}
\begin{tabular}{l}
\parbox[t][][t]{8cm}{}\\
\parbox[t][][t]{8cm}{\normalfont \footnotesize 
spülen - 
to wash up - 
spülen - 
to flush - 
etw spülen - 
to wash up sth sep - 
etw irgendwohin spülen - 
to wash sth somewhere - 
das Meer spülte die Leiche an Land - 
the sea washed the body ashore 
}\\
\end{tabular}
}
%===starten===
\card{\normalfont \Huge starten}{
\begin{tabular}{lll}
\parbox[t][][t]{2.0 cm}{\normalfont \raggedleft ich\\du\\er/sie/es\\wir\\ihr\\sie} &    
\parbox[t][][t]{2cm}{\normalfont starte\\startest\\startet\\starten\\startet\\starten} &
\parbox[t][][t]{2cm}{\normalfont startete\\startetest\\startete\\starteten\\startetet\\starteten}\\
\end{tabular}
\begin{tabular}{l}
\parbox[t][][t]{8cm}{}\\
\parbox[t][][t]{8cm}{\normalfont \footnotesize 
starten LUFT - 
to take off - 
starten RAUM - 
to lift (or blast) off - 
starten RAUM - 
to be launched - 
(zu etw dat) starten - 
to start ((on) sth) - 
die Läufer sind gestartet! - 
the runners have started (or are off) ! 
}\\
\end{tabular}
}
%===stattfinden===
\card{\normalfont \Huge stattfinden}{
\begin{tabular}{lll}
\parbox[t][][t]{2.0 cm}{\normalfont \raggedleft ich\\du\\er/sie/es\\wir\\ihr\\sie} &    
\parbox[t][][t]{2cm}{\normalfont finde statt\\findest statt\\findet statt\\finden statt\\findet statt\\finden statt} &
\parbox[t][][t]{2cm}{\normalfont fand statt\\fandest statt\\fand statt\\fanden statt\\fandet statt\\fanden statt}\\
\end{tabular}
\begin{tabular}{l}
\parbox[t][][t]{8cm}{}\\
\parbox[t][][t]{8cm}{\normalfont \footnotesize 
stattfinden - 
to take place - 
stattfinden Veranstaltung a. - 
to be held - 
stattfinden - 
to take place - 
stattfinden - 
to happen 
}\\
\end{tabular}
}
%===stechen===
\card{\normalfont \Huge stechen}{
\begin{tabular}{lll}
\parbox[t][][t]{2.0 cm}{\normalfont \raggedleft ich\\du\\er/sie/es\\wir\\ihr\\sie} &    
\parbox[t][][t]{2cm}{\normalfont steche\\stichst\\sticht\\stechen\\stecht\\stechen} &
\parbox[t][][t]{2cm}{\normalfont stach\\stachst\\stach\\stachen\\stacht\\stachen}\\
\end{tabular}
\begin{tabular}{l}
\parbox[t][][t]{8cm}{}\\
\parbox[t][][t]{8cm}{\normalfont \footnotesize 
stechen - 
to prick - 
stechen Werkzeug - 
to be sharp - 
stechen - 
to sting - 
stechen Mücken, Moskitos - 
to bite - 
(mit etw dat) durch etw akk/in etw akk - 
stechen 
}\\
\end{tabular}
}
%===stecken===
\card{\normalfont \Huge stecken}{
\begin{tabular}{lll}
\parbox[t][][t]{2.0 cm}{\normalfont \raggedleft ich\\du\\er/sie/es\\wir\\ihr\\sie} &    
\parbox[t][][t]{2cm}{\normalfont stecke\\steckst\\steckt\\stecken\\steckt\\stecken} &
\parbox[t][][t]{2cm}{\normalfont steckte \\stecktest\\steckte\\steckten\\stecktet \\steckten }\\
\end{tabular}
\begin{tabular}{l}
\parbox[t][][t]{8cm}{}\\
\parbox[t][][t]{8cm}{\normalfont \footnotesize 
in/hinter/zwischen etw dat - 
stecken - 
to be stuck in/behind/between sth - 
in/hinter/zwischen etw dat - 
stecken - 
Dorn, Gräte, Kugel - 
to be (sticking (or lodged) ) in/behind/between sth - 
(in etw dat) stecken bleiben - 
to get stuck (in sth) - 
das Auto blieb im Schlamm stecken 
}\\
\end{tabular}
}
%===stehen===
\card{\normalfont \Huge stehen}{
\begin{tabular}{lll}
\parbox[t][][t]{2.0 cm}{\normalfont \raggedleft ich\\du\\er/sie/es\\wir\\ihr\\sie} &    
\parbox[t][][t]{2cm}{\normalfont stehe\\stehst\\steht\\stehen\\steht\\stehen} &
\parbox[t][][t]{2cm}{\normalfont stand\\standest\\stand\\standen\\standet\\standen}\\
\end{tabular}
\begin{tabular}{l}
\parbox[t][][t]{8cm}{}\\
\parbox[t][][t]{8cm}{\normalfont \footnotesize 
stehen Mensch - 
to stand - 
stehen längliche Gegenstände a. - 
to be (placed) upright - 
ich kann nicht mehr stehen - 
I can't stand up any longer - 
unsere alte Schule steht noch - 
our old school is still standing (or is still there) - 
nach dem Erdbeben waren nur ein paar Häuser stehen geblieben - 
after the earthquake, only a few houses were left standing 
}\\
\end{tabular}
}
%===stehlen===
\card{\normalfont \Huge stehlen}{
\begin{tabular}{lll}
\parbox[t][][t]{2.0 cm}{\normalfont \raggedleft ich\\du\\er/sie/es\\wir\\ihr\\sie} &    
\parbox[t][][t]{2cm}{\normalfont stehle\\stiehlst\\stiehlt\\stehlen\\stehlt\\stehlen} &
\parbox[t][][t]{2cm}{\normalfont stahl\\stahlst\\stahl\\stahlen\\stahlt\\stahlen}\\
\end{tabular}
\begin{tabular}{l}
\parbox[t][][t]{8cm}{}\\
\parbox[t][][t]{8cm}{\normalfont \footnotesize 
(jdm) etw stehlen - 
to steal (or - 
hum purloin)  (sb's) sth - 
das/er/sie usw. kann mir gestohlen bleiben! fam - 
to hell with it/him/her etc.! fam - 
das/er/sie usw. kann mir gestohlen bleiben! fam - 
he/she etc. can go take a running jump! fam - 
das/er/sie usw. kann mir gestohlen bleiben! fam - 
he/she etc. can go hang fam - 
dem lieben Gott die Zeit stehlen 
}\\
\end{tabular}
}
%===steigen===
\card{\normalfont \Huge steigen}{
\begin{tabular}{lll}
\parbox[t][][t]{2.0 cm}{\normalfont \raggedleft ich\\du\\er/sie/es\\wir\\ihr\\sie} &    
\parbox[t][][t]{2cm}{\normalfont steige\\steigst\\steigt\\steigen\\steigt\\steigen} &
\parbox[t][][t]{2cm}{\normalfont stieg\\stiegst\\stieg\\stiegen\\stiegt\\stiegen}\\
\end{tabular}
\begin{tabular}{l}
\parbox[t][][t]{8cm}{}\\
\parbox[t][][t]{8cm}{\normalfont \footnotesize 
steigen - 
to climb - 
auf etw akk - 
steigen - 
to climb (up) sth - 
durchs Fenster steigen - 
to climb through the window - 
auf etw akk - 
steigen - 
to get on(to) sth 
}\\
\end{tabular}
}
%===stellen===
\card{\normalfont \Huge stellen}{
\begin{tabular}{lll}
\parbox[t][][t]{2.0 cm}{\normalfont \raggedleft ich\\du\\er/sie/es\\wir\\ihr\\sie} &    
\parbox[t][][t]{2cm}{\normalfont stelle\\stellst\\stellt\\stellen\\stellt\\stellen} &
\parbox[t][][t]{2cm}{\normalfont stellte\\stelltest\\stellte\\stellten\\stelltet\\stellten}\\
\end{tabular}
\begin{tabular}{l}
\parbox[t][][t]{8cm}{}\\
\parbox[t][][t]{8cm}{\normalfont \footnotesize 
sich akk irgendwohin stellen - 
to go and stand somewhere - 
sich akk irgendwohin stellen - 
(herkommen) - 
to come and stand somewhere - 
sich akk irgendwohin stellen - 
(Stellung beziehen) - 
to take up position somewhere - 
sich akk ans Ende der Schlange stellen - 
to go/come to the back (or end) of the queue (or 
}\\
\end{tabular}
}
%===sterben===
\card{\normalfont \Huge sterben}{
\begin{tabular}{lll}
\parbox[t][][t]{2.0 cm}{\normalfont \raggedleft ich\\du\\er/sie/es\\wir\\ihr\\sie} &    
\parbox[t][][t]{2cm}{\normalfont sterbe\\stirbst\\stirbt\\sterben\\sterbt\\sterben} &
\parbox[t][][t]{2cm}{\normalfont starb\\starbst\\starb\\starben\\starbt\\starben}\\
\end{tabular}
\begin{tabular}{l}
\parbox[t][][t]{8cm}{}\\
\parbox[t][][t]{8cm}{\normalfont \footnotesize 
(an etw dat) sterben - 
to die (of sth) - 
mein Großonkel ist schon lange gestorben - 
my great uncle died a long time ago (or has been dead for years) - 
daran wirst du (schon) nicht sterben! hum fam - 
it won't kill you! fam - 
als Held sterben - 
to die a hero('s death) - 
(fast) vor etw dat - 
sterben 
}\\
\end{tabular}
}
%===stimmen===
\card{\normalfont \Huge stimmen}{
\begin{tabular}{lll}
\parbox[t][][t]{2.0 cm}{\normalfont \raggedleft ich\\du\\er/sie/es\\wir\\ihr\\sie} &    
\parbox[t][][t]{2cm}{\normalfont stimme\\stimmst\\stimmt\\stimmen\\stimmt\\stimmen} &
\parbox[t][][t]{2cm}{\normalfont stimmte\\stimmtest\\stimmte\\stimmten\\stimmtet\\stimmten}\\
\end{tabular}
\begin{tabular}{l}
\parbox[t][][t]{8cm}{}\\
\parbox[t][][t]{8cm}{\normalfont \footnotesize 
stimmen - 
to be right (or correct) - 
es stimmt, dass jd etw ist/tut - 
it is true that sb is/does sth - 
stimmt! fam - 
right! - 
habe ich nicht völlig Recht? — stimmt! - 
don't you think I'm right? — yes, I do! - 
stimmen - 
to be correct 
}\\
\end{tabular}
}
%===stinken===
\card{\normalfont \Huge stinken}{
\begin{tabular}{lll}
\parbox[t][][t]{2.0 cm}{\normalfont \raggedleft ich\\du\\er/sie/es\\wir\\ihr\\sie} &    
\parbox[t][][t]{2cm}{\normalfont stinke\\stinkst\\stinkt\\stinken\\stinkt\\stinken} &
\parbox[t][][t]{2cm}{\normalfont stank\\stankst\\stank\\stanken\\stankt\\stanken}\\
\end{tabular}
\begin{tabular}{l}
\parbox[t][][t]{8cm}{}\\
\parbox[t][][t]{8cm}{\normalfont \footnotesize 
(nach etw dat) stinken - 
to stink (or reek)  (of sth) - 
stinken - 
to stink - 
die Sache stinkt - 
the whole business stinks (or is (very) fishy) - 
jdm stinkt etw - 
sb is fed up (to the back teeth)  (or is sick to death) with sth fam - 
etw (an jdm/etw) stinkt jdm - 
sth (about sb/sth) sickens sb 
}\\
\end{tabular}
}
%===stoppen===
\card{\normalfont \Huge stoppen}{
\begin{tabular}{lll}
\parbox[t][][t]{2.0 cm}{\normalfont \raggedleft ich\\du\\er/sie/es\\wir\\ihr\\sie} &    
\parbox[t][][t]{2cm}{\normalfont stoppe\\stoppst\\stoppt\\stoppen\\stoppt\\stoppen} &
\parbox[t][][t]{2cm}{\normalfont stoppte\\stopptest\\stoppte\\stoppten\\stopptet\\stoppten}\\
\end{tabular}
\begin{tabular}{l}
\parbox[t][][t]{8cm}{}\\
\parbox[t][][t]{8cm}{\normalfont \footnotesize 
jdn/etw stoppen - 
to stop sb/sth - 
etw stoppen - 
to stop (or put a stop to) sth - 
etw stoppen - 
to bring sth to a halt (or stop) (or standstill) - 
die Verhandlungsgespräche sind gestoppt worden - 
the negotiations have broken down - 
die Ausführung stoppen - 
INFORM, TECH 
}\\
\end{tabular}
}
%===stören===
\card{\normalfont \Huge stören}{
\begin{tabular}{lll}
\parbox[t][][t]{2.0 cm}{\normalfont \raggedleft ich\\du\\er/sie/es\\wir\\ihr\\sie} &    
\parbox[t][][t]{2cm}{\normalfont störe\\störst\\stört\\stören\\stört\\stören} &
\parbox[t][][t]{2cm}{\normalfont störte\\störtest\\störte\\störten\\störtet\\störten}\\
\end{tabular}
\begin{tabular}{l}
\parbox[t][][t]{8cm}{}\\
\parbox[t][][t]{8cm}{\normalfont \footnotesize 
jdn (bei etw dat) stören - 
to disturb (or bother) sb (when he/she is doing sth) - 
bitte lassen Sie sich nicht stören! - 
please don't let me disturb you! - 
entschuldigen Sie, wenn ich Sie störe - 
I'm sorry to bother you (or if I'm disturbing you) - 
tut mir leid, wenn ich dich störe, aber könntest du mir sagen, wie spät es ist? - 
sorry to trouble (or bother) you, but could you tell me the time? - 
störe mich jetzt nicht! - 
don't bother (or disturb) me now! 
}\\
\end{tabular}
}
%===stoßen===
\card{\normalfont \Huge stoßen}{
\begin{tabular}{lll}
\parbox[t][][t]{2.0 cm}{\normalfont \raggedleft ich\\du\\er/sie/es\\wir\\ihr\\sie} &    
\parbox[t][][t]{2cm}{\normalfont stoße\\stößt\\stößt\\stoßen\\stoßt\\stoßen} &
\parbox[t][][t]{2cm}{\normalfont stieß\\stieß(e)st\\stieß\\stießen\\stießt\\stießen}\\
\end{tabular}
\begin{tabular}{l}
\parbox[t][][t]{8cm}{}\\
\parbox[t][][t]{8cm}{\normalfont \footnotesize 
jdn stoßen - 
to push sb - 
jdn stoßen - 
(stark) - 
to shove sb - 
jdn stoßen - 
(leicht) - 
to poke sb - 
er hat sie die Treppe hinuntergestoßen - 
he shoved her down the stairs 
}\\
\end{tabular}
}
%===strahlen===
\card{\normalfont \Huge strahlen}{
\begin{tabular}{lll}
\parbox[t][][t]{2.0 cm}{\normalfont \raggedleft ich\\du\\er/sie/es\\wir\\ihr\\sie} &    
\parbox[t][][t]{2cm}{\normalfont strahle\\strahlst\\strahlt\\strahlen\\strahlt\\strahlen} &
\parbox[t][][t]{2cm}{\normalfont strahlte\\strahltest\\strahlte\\strahlten\\strahltet\\strahlten}\\
\end{tabular}
\begin{tabular}{l}
\parbox[t][][t]{8cm}{}\\
\parbox[t][][t]{8cm}{\normalfont \footnotesize 
irgendwohin strahlen - 
to shine somewhere - 
auf jdn strahlen - 
to shine on sb - 
jdm ins Gesicht/auf jds Gesicht strahlen - 
to shine (straight) into sb's eyes - 
strahlen - 
to be radioactive - 
(vor etw dat) strahlen - 
to beam (or be radiant)  (with sth) 
}\\
\end{tabular}
}
%===streben===
\card{\normalfont \Huge streben}{
\begin{tabular}{lll}
\parbox[t][][t]{2.0 cm}{\normalfont \raggedleft ich\\du\\er/sie/es\\wir\\ihr\\sie} &    
\parbox[t][][t]{2cm}{\normalfont strebe\\strebst\\strebt\\streben\\strebt\\streben} &
\parbox[t][][t]{2cm}{\normalfont strebte\\strebtest\\strebte\\strebten\\strebtet\\strebten}\\
\end{tabular}
\begin{tabular}{l}
\parbox[t][][t]{8cm}{}\\
\parbox[t][][t]{8cm}{\normalfont \footnotesize 
nach etw dat - 
streben - 
to strive (or try hard) for sth - 
danach streben, etw zu tun - 
to strive (or try hard) to do sth - 
streben - 
to make one's way purposefully - 
zum Ausgang/zur Tür/an den Strand streben - 
to make (or head) for the exit/door/beach - 
Streben 
}\\
\end{tabular}
}
%===streichen===
\card{\normalfont \Huge streichen}{
\begin{tabular}{lll}
\parbox[t][][t]{2.0 cm}{\normalfont \raggedleft ich\\du\\er/sie/es\\wir\\ihr\\sie} &    
\parbox[t][][t]{2cm}{\normalfont streiche\\streichst\\streicht\\streichen\\streicht\\streichen} &
\parbox[t][][t]{2cm}{\normalfont strich\\strichst\\strich\\strichen\\stricht\\strichen}\\
\end{tabular}
\begin{tabular}{l}
\parbox[t][][t]{8cm}{}\\
\parbox[t][][t]{8cm}{\normalfont \footnotesize 
etw (mit etw dat) streichen - 
to paint sth (with sth) - 
etw (auf etw akk) streichen - 
to spread sth (on sth) - 
(sich dat) Butter aufs Brot streichen - 
to put butter on one's bread - 
(sich dat) Butter aufs Brot streichen - 
to butter one's bread - 
etw streichen - 
to delete sth 
}\\
\end{tabular}
}
%===streiten===
\card{\normalfont \Huge streiten}{
\begin{tabular}{lll}
\parbox[t][][t]{2.0 cm}{\normalfont \raggedleft ich\\du\\er/sie/es\\wir\\ihr\\sie} &    
\parbox[t][][t]{2cm}{\normalfont streite\\streitest\\streitet\\streiten\\streitet\\streiten} &
\parbox[t][][t]{2cm}{\normalfont stritt\\strittest\\stritt\\stritten\\strittet\\stritten}\\
\end{tabular}
\begin{tabular}{l}
\parbox[t][][t]{8cm}{}\\
\parbox[t][][t]{8cm}{\normalfont \footnotesize 
(mit jdm) streiten - 
to argue (or quarrel)  (with sb) - 
mit jdm über etw akk - 
streiten - 
to argue with sb about sth - 
darüber lässt sich streiten - 
that's open to argument (or debatable) - 
sich akk (miteinander) streiten - 
to quarrel (or argue)  (with each other) - 
habt ihr euch wieder gestritten? 
}\\
\end{tabular}
}
%===studieren===
\card{\normalfont \Huge studieren}{
\begin{tabular}{lll}
\parbox[t][][t]{2.0 cm}{\normalfont \raggedleft ich\\du\\er/sie/es\\wir\\ihr\\sie} &    
\parbox[t][][t]{2cm}{\normalfont studiere\\studierst\\studiert\\studieren\\studiert\\studieren} &
\parbox[t][][t]{2cm}{\normalfont studierte\\studiertest\\studierte\\studierten\\studiertet\\studierten}\\
\end{tabular}
\begin{tabular}{l}
\parbox[t][][t]{8cm}{}\\
\parbox[t][][t]{8cm}{\normalfont \footnotesize 
studieren - 
to study - 
sie studiert noch - 
she is still a student - 
studieren wollen - 
to want to go to ( Am  a) university/college - 
ich studiere derzeit im fünften/sechsten Semester - 
I'm in my third year (at university/college) - 
etw studieren - 
to study (or 
}\\
\end{tabular}
}
%===stürzen===
\card{\normalfont \Huge stürzen}{
\begin{tabular}{lll}
\parbox[t][][t]{2.0 cm}{\normalfont \raggedleft ich\\du\\er/sie/es\\wir\\ihr\\sie} &    
\parbox[t][][t]{2cm}{\normalfont stürze\\stürzt\\stürzt\\stürzen\\stürzt\\stürzen} &
\parbox[t][][t]{2cm}{\normalfont stürzte\\stürztest\\stürzte\\stürzten\\stürztet\\stürzten}\\
\end{tabular}
\begin{tabular}{l}
\parbox[t][][t]{8cm}{}\\
\parbox[t][][t]{8cm}{\normalfont \footnotesize 
stürzen - 
to fall - 
ich wäre fast gestürzt - 
I nearly fell (down (or over) ) - 
schwer stürzen - 
to fall heavily - 
(aus (o. von) etw) stürzen - 
to fall (out of (or from) - 
(or off) sth) - 
vom Dach/Tisch/Fahrrad/Pferd stürzen 
}\\
\end{tabular}
}
%===stützen===
\card{\normalfont \Huge stützen}{
\begin{tabular}{lll}
\parbox[t][][t]{2.0 cm}{\normalfont \raggedleft ich\\du\\er/sie/es\\wir\\ihr\\sie} &    
\parbox[t][][t]{2cm}{\normalfont stütze\\stützt\\stützt\\stützen\\stützt\\stützen} &
\parbox[t][][t]{2cm}{\normalfont stützte\\stütztest\\stützte\\stützten\\stütztet\\stützten}\\
\end{tabular}
\begin{tabular}{l}
\parbox[t][][t]{8cm}{}\\
\parbox[t][][t]{8cm}{\normalfont \footnotesize 
jdn/etw stützen - 
to support sb/sth - 
etw stützen - 
to support sth - 
etw stützen - 
to prop sth up - 
etw (auf etw akk) stützen - 
to rest sth (on sth) - 
die Ellbogen auf den Tisch stützen - 
to rest (or prop) one's elbows on the table 
}\\
\end{tabular}
}
%===suchen===
\card{\normalfont \Huge suchen}{
\begin{tabular}{lll}
\parbox[t][][t]{2.0 cm}{\normalfont \raggedleft ich\\du\\er/sie/es\\wir\\ihr\\sie} &    
\parbox[t][][t]{2cm}{\normalfont suche\\suchst\\sucht\\suchen\\sucht\\suchen} &
\parbox[t][][t]{2cm}{\normalfont suchte\\suchtest\\suchte\\suchten\\suchtet\\suchten}\\
\end{tabular}
\begin{tabular}{l}
\parbox[t][][t]{8cm}{}\\
\parbox[t][][t]{8cm}{\normalfont \footnotesize 
jdn/etw suchen - 
to look for sb/sth - 
jdn/etw suchen - 
(intensiver, auch vom Computer aus) - 
to search for sb/sth - 
sich dat jdn/etw suchen - 
to look for sb/sth - 
irgendwo nichts zu suchen haben - 
to have no business to be somewhere - 
du hast hier nichts zu suchen! 
}\\
\end{tabular}
}
%===tanzen===
\card{\normalfont \Huge tanzen}{
\begin{tabular}{lll}
\parbox[t][][t]{2.0 cm}{\normalfont \raggedleft ich\\du\\er/sie/es\\wir\\ihr\\sie} &    
\parbox[t][][t]{2cm}{\normalfont tanze\\tanzt\\tanzt\\tanzen\\tanzt\\tanzen} &
\parbox[t][][t]{2cm}{\normalfont tanzte\\tanztest\\tanzte\\tanzten\\tanztet\\tanzten}\\
\end{tabular}
\begin{tabular}{l}
\parbox[t][][t]{8cm}{}\\
\parbox[t][][t]{8cm}{\normalfont \footnotesize 
tanzen - 
to dance - 
wollen wir tanzen? - 
shall we dance? - 
tanzen - 
(o. zum Tanzen - 
) gehen - 
to go dancing - 
tanzen - 
to dance 
}\\
\end{tabular}
}
%===teilen===
\card{\normalfont \Huge teilen}{
\begin{tabular}{lll}
\parbox[t][][t]{2.0 cm}{\normalfont \raggedleft ich\\du\\er/sie/es\\wir\\ihr\\sie} &    
\parbox[t][][t]{2cm}{\normalfont teile\\teilst\\teilt\\teilen\\teilt\\teilen} &
\parbox[t][][t]{2cm}{\normalfont teilte\\teiltest\\teilte\\teilten\\teiltet\\teilten}\\
\end{tabular}
\begin{tabular}{l}
\parbox[t][][t]{8cm}{}\\
\parbox[t][][t]{8cm}{\normalfont \footnotesize 
etw (mit jdm) teilen - 
to share sth (with sb) - 
etw (durch etw akk) teilen - 
to divide sth (by sth) - 
etw (mit jdm) teilen - 
to share sth (with sb) - 
wir teilen Ihre Trauer - 
we share your grief - 
jds Schicksal teilen - 
to share sb's fate 
}\\
\end{tabular}
}
%===teilnehmen===
\card{\normalfont \Huge teilnehmen}{
\begin{tabular}{lll}
\parbox[t][][t]{2.0 cm}{\normalfont \raggedleft ich\\du\\er/sie/es\\wir\\ihr\\sie} &    
\parbox[t][][t]{2cm}{\normalfont nehme teil\\nimmst teil\\nimmt teil\\nehmen teil\\nehmt teil\\nehmen teil} &
\parbox[t][][t]{2cm}{\normalfont nahm teil\\nahmst teil\\nahm teil\\nahmen teil\\nahmt teil\\nahmen teil}\\
\end{tabular}
\begin{tabular}{l}
\parbox[t][][t]{8cm}{}\\
\parbox[t][][t]{8cm}{\normalfont \footnotesize 
(an etw dat) teilnehmen - 
to attend (sth) - 
am Gottesdienst teilnehmen - 
to attend a service - 
(an etw dat) teilnehmen - 
an einem Wettbewerb teilnehmen - 
to participate (or take part) in a contest - 
an einem Kurs (o. Unterricht) - 
teilnehmen - 
to attend a class (or lessons) 
}\\
\end{tabular}
}
%===töten===
\card{\normalfont \Huge töten}{
\begin{tabular}{lll}
\parbox[t][][t]{2.0 cm}{\normalfont \raggedleft ich\\du\\er/sie/es\\wir\\ihr\\sie} &    
\parbox[t][][t]{2cm}{\normalfont töte\\tötest\\tötet\\töten\\tötet\\töten} &
\parbox[t][][t]{2cm}{\normalfont tötete\\tötetest\\tötete\\töteten\\tötetet\\töteten}\\
\end{tabular}
\begin{tabular}{l}
\parbox[t][][t]{8cm}{}\\
\parbox[t][][t]{8cm}{\normalfont \footnotesize 
jdn/etw töten - 
to kill sb/sth - 
Nerv - 
nerve - 
Nerv - 
vein - 
die Nerven behalten - 
to keep calm - 
Nerven wie Drahtseile haben fam - 
to have nerves of steel 
}\\
\end{tabular}
}
%===tragen===
\card{\normalfont \Huge tragen}{
\begin{tabular}{lll}
\parbox[t][][t]{2.0 cm}{\normalfont \raggedleft ich\\du\\er/sie/es\\wir\\ihr\\sie} &    
\parbox[t][][t]{2cm}{\normalfont trage\\trägst\\trägt\\tragen\\tragt\\tragen} &
\parbox[t][][t]{2cm}{\normalfont trug\\trugst\\trug\\trugen\\trugt\\trugen}\\
\end{tabular}
\begin{tabular}{l}
\parbox[t][][t]{8cm}{}\\
\parbox[t][][t]{8cm}{\normalfont \footnotesize 
jdn/etw tragen - 
to carry (or take) sb/sth - 
einen Brief zur Post tragen - 
to take a letter to the post office - 
fig das Auto wurde aus der Kurve getragen - 
the car went off the bend - 
vom Wasser/Wind getragen - 
carried by water/(the) wind - 
etw tragen - 
to hold sth 
}\\
\end{tabular}
}
%===trauen===
\card{\normalfont \Huge trauen}{
\begin{tabular}{lll}
\parbox[t][][t]{2.0 cm}{\normalfont \raggedleft ich\\du\\er/sie/es\\wir\\ihr\\sie} &    
\parbox[t][][t]{2cm}{\normalfont traue\\traust\\traut\\trauen\\traut\\trauen} &
\parbox[t][][t]{2cm}{\normalfont traute\\trautest\\traute\\trauten\\trautet\\trauten}\\
\end{tabular}
\begin{tabular}{l}
\parbox[t][][t]{8cm}{}\\
\parbox[t][][t]{8cm}{\normalfont \footnotesize 
jdn trauen - 
to marry sb - 
jdn trauen - 
to join sb in marriage - 
sich akk - 
trauen lassen - 
to get married - 
sich akk - 
trauen lassen - 
to marry 
}\\
\end{tabular}
}
%===träumen===
\card{\normalfont \Huge träumen}{
\begin{tabular}{lll}
\parbox[t][][t]{2.0 cm}{\normalfont \raggedleft ich\\du\\er/sie/es\\wir\\ihr\\sie} &    
\parbox[t][][t]{2cm}{\normalfont träume\\träumst\\träumt\\träumen\\träumt\\träumen} &
\parbox[t][][t]{2cm}{\normalfont träumte\\träumtest\\träumte\\träumten\\träumtet\\träumten}\\
\end{tabular}
\begin{tabular}{l}
\parbox[t][][t]{8cm}{}\\
\parbox[t][][t]{8cm}{\normalfont \footnotesize 
träumen - 
to dream - 
träumen, dass jd etw tut/dass etw geschieht - 
to dream that sb does sth/that sth happens - 
schlecht träumen - 
to have bad dreams (or nightmares) - 
von jdm/etw träumen - 
to dream about sb/sth - 
sie hat immer davon geträumt, Ärztin zu werden - 
she had always dreamt of becoming a doctor 
}\\
\end{tabular}
}
%===treffen===
\card{\normalfont \Huge treffen}{
\begin{tabular}{lll}
\parbox[t][][t]{2.0 cm}{\normalfont \raggedleft ich\\du\\er/sie/es\\wir\\ihr\\sie} &    
\parbox[t][][t]{2cm}{\normalfont treffe\\triffst\\trifft\\treffen\\trefft\\treffen} &
\parbox[t][][t]{2cm}{\normalfont traf\\trafst\\traf\\trafen\\traft\\trafen}\\
\end{tabular}
\begin{tabular}{l}
\parbox[t][][t]{8cm}{}\\
\parbox[t][][t]{8cm}{\normalfont \footnotesize 
jdn treffen - 
to meet (up with) sb - 
wir haben uns dann später noch auf einen Drink getroffen - 
we met up again later for a drink - 
jdn zum Mittagessen treffen - 
to meet sb for lunch - 
jdn treffen - 
to run (or - 
fam bump) into sb - 
rate mal, wen ich heute getroffen habe! 
}\\
\end{tabular}
}
%===treiben===
\card{\normalfont \Huge treiben}{
\begin{tabular}{lll}
\parbox[t][][t]{2.0 cm}{\normalfont \raggedleft ich\\du\\er/sie/es\\wir\\ihr\\sie} &    
\parbox[t][][t]{2cm}{\normalfont treibe\\treibst\\treibt\\treiben\\treibt\\treiben} &
\parbox[t][][t]{2cm}{\normalfont trieb\\triebst\\trieb\\trieben\\triebt\\trieben}\\
\end{tabular}
\begin{tabular}{l}
\parbox[t][][t]{8cm}{}\\
\parbox[t][][t]{8cm}{\normalfont \footnotesize 
jdn/Tiere (irgendwohin) treiben - 
to drive sb/animals (somewhere) - 
Hasen/Wild treiben - 
JAGD - 
to beat hares/game - 
jdn/Tiere aus/von etw dat - 
treiben - 
to drive sb/animals out of/from sth - 
jdn/etw (irgendwohin) treiben - 
(durch Wasser) 
}\\
\end{tabular}
}
%===trennen===
\card{\normalfont \Huge trennen}{
\begin{tabular}{lll}
\parbox[t][][t]{2.0 cm}{\normalfont \raggedleft ich\\du\\er/sie/es\\wir\\ihr\\sie} &    
\parbox[t][][t]{2cm}{\normalfont trenne\\trennst\\trennt\\trennen\\trennt\\trennen} &
\parbox[t][][t]{2cm}{\normalfont trennte\\trenntest\\trennte\\trennten\\trenntet\\trennten}\\
\end{tabular}
\begin{tabular}{l}
\parbox[t][][t]{8cm}{}\\
\parbox[t][][t]{8cm}{\normalfont \footnotesize 
etw von etw dat - 
trennen - 
to separate sth from sth - 
etw von etw dat - 
trennen - 
(mit scharfem Gegenstand) - 
to cut sth off sth - 
etw von etw dat - 
trennen - 
(Körperteil bei einem Unfall) 
}\\
\end{tabular}
}
%===treten===
\card{\normalfont \Huge treten}{
\begin{tabular}{lll}
\parbox[t][][t]{2.0 cm}{\normalfont \raggedleft ich\\du\\er/sie/es\\wir\\ihr\\sie} &    
\parbox[t][][t]{2cm}{\normalfont trete\\trittst\\tritt\\treten\\tretet\\treten} &
\parbox[t][][t]{2cm}{\normalfont trat\\tratst \\trat\\traten\\tratet\\traten}\\
\end{tabular}
\begin{tabular}{l}
\parbox[t][][t]{8cm}{}\\
\parbox[t][][t]{8cm}{\normalfont \footnotesize 
irgendwohin treten - 
to step somewhere - 
irgendwohin treten - 
(hineingehen a.) - 
to go somewhere - 
irgendwohin treten - 
(hereinkommen a.) - 
to come somewhere - 
bitte treten Sie näher! - 
please come in! 
}\\
\end{tabular}
}
%===triefen===
\card{\normalfont \Huge triefen}{
\begin{tabular}{lll}
\parbox[t][][t]{2.0 cm}{\normalfont \raggedleft ich\\du\\er/sie/es\\wir\\ihr\\sie} &    
\parbox[t][][t]{2cm}{\normalfont triefe\\triefest\\triefet\\triefen\\triefet\\triefen} &
\parbox[t][][t]{2cm}{\normalfont troff\\troffest\\troff\\troffen\\troffet\\troffen}\\
\end{tabular}
\begin{tabular}{l}
\parbox[t][][t]{8cm}{}\\
\parbox[t][][t]{8cm}{\normalfont \footnotesize 
triefen - 
to run - 
triefen (Auge) - 
to water - 
ich habe Schnupfen, meine Nase trieft nur so! - 
I've got a cold, and it's given me such a runny nose! - 
aus (o. von) etw dat - 
triefen - 
to pour from sth - 
(von etw dat) triefen 
}\\
\end{tabular}
}
%===trinken===
\card{\normalfont \Huge trinken}{
\begin{tabular}{lll}
\parbox[t][][t]{2.0 cm}{\normalfont \raggedleft ich\\du\\er/sie/es\\wir\\ihr\\sie} &    
\parbox[t][][t]{2cm}{\normalfont trinke\\trinkst\\trinkt\\trinken\\trinkt\\trinken} &
\parbox[t][][t]{2cm}{\normalfont trank\\trankst\\trank\\tranken\\trankt\\tranken}\\
\end{tabular}
\begin{tabular}{l}
\parbox[t][][t]{8cm}{}\\
\parbox[t][][t]{8cm}{\normalfont \footnotesize 
etw trinken - 
to drink sth - 
Wasser trinken - 
to drink (or have) some water - 
kann ich bei Ihnen wohl ein Glas Wasser trinken? - 
could you give (or spare) me a glass of water (to drink)? - 
möchten Sie lieber Kaffee oder Tee trinken? - 
would you prefer coffee or tea (to drink)? - 
ich trinke gerne Orangensaft - 
I like drinking orange juice 
}\\
\end{tabular}
}
%===trösten===
\card{\normalfont \Huge trösten}{
\begin{tabular}{lll}
\parbox[t][][t]{2.0 cm}{\normalfont \raggedleft ich\\du\\er/sie/es\\wir\\ihr\\sie} &    
\parbox[t][][t]{2cm}{\normalfont tröste\\tröstest\\tröstet\\trösten\\tröstet\\trösten} &
\parbox[t][][t]{2cm}{\normalfont tröstete\\tröstetest\\tröstete\\trösteten\\tröstetet\\trösteten}\\
\end{tabular}
\begin{tabular}{l}
\parbox[t][][t]{8cm}{}\\
\parbox[t][][t]{8cm}{\normalfont \footnotesize 
jdn trösten - 
to comfort (or console) sb - 
sie war von nichts und niemandem zu trösten - 
she was utterly inconsolable - 
etw tröstet jdn - 
sth is of consolation to sb - 
sich akk (mit jdm/etw) trösten - 
to find consolation (with sb)/console oneself (with sth) - 
sich akk (mit jdm/etw) trösten - 
to find solace (in sth) form 
}\\
\end{tabular}
}
%===trügen===
\card{\normalfont \Huge trügen}{
\begin{tabular}{lll}
\parbox[t][][t]{2.0 cm}{\normalfont \raggedleft ich\\du\\er/sie/es\\wir\\ihr\\sie} &    
\parbox[t][][t]{2cm}{\normalfont trüge\\trügst\\trügt\\trügen\\trügt\\trügen} &
\parbox[t][][t]{2cm}{\normalfont trog\\trogst\\trog\\trogen\\trogt\\trogen}\\
\end{tabular}
\begin{tabular}{l}
\parbox[t][][t]{8cm}{}\\
\parbox[t][][t]{8cm}{\normalfont \footnotesize 
jdn trügen - 
geh - 
to deceive sb - 
wenn mich nicht alles trügt - 
unless I'm very much mistaken - 
trügen - 
to be deceptive - 
jdn/etw tragen - 
to carry (or take) sb/sth - 
einen Brief zur Post tragen 
}\\
\end{tabular}
}
%===tun===
\card{\normalfont \Huge tun}{
\begin{tabular}{lll}
\parbox[t][][t]{2.0 cm}{\normalfont \raggedleft ich\\du\\er/sie/es\\wir\\ihr\\sie} &    
\parbox[t][][t]{2cm}{\normalfont tue\\tust\\tut\\tun\\tut\\tun} &
\parbox[t][][t]{2cm}{\normalfont tat\\tatest\\tat\\taten\\tatet\\taten}\\
\end{tabular}
\begin{tabular}{l}
\parbox[t][][t]{8cm}{}\\
\parbox[t][][t]{8cm}{\normalfont \footnotesize 
tun (vat) - 
Fass nt - 
Add to my favourites Preselect for export to vocabulary trainer View selected vocabulary - 
tun (of metal) - 
Tonne f - 
Add to my favourites Preselect for export to vocabulary trainer View selected vocabulary - 
tun (for brewing) - 
Gärfass nt - 
Add to my favourites Preselect for export to vocabulary trainer View selected vocabulary - 
tun 
}\\
\end{tabular}
}
%===überholen===
\card{\normalfont \Huge überholen}{
\begin{tabular}{lll}
\parbox[t][][t]{2.0 cm}{\normalfont \raggedleft ich\\du\\er/sie/es\\wir\\ihr\\sie} &    
\parbox[t][][t]{2cm}{\normalfont überhole\\überholst\\überholt\\überholen\\überholt\\überholen} &
\parbox[t][][t]{2cm}{\normalfont überholte\\überholtest\\überholte\\überholten\\überholtet\\überholten}\\
\end{tabular}
\begin{tabular}{l}
\parbox[t][][t]{8cm}{}\\
\parbox[t][][t]{8cm}{\normalfont \footnotesize 
jdn/etw überholen - 
to pass (or - 
Brit overtake) sb/sth - 
jdn/etw überholen - 
to outstrip (or surpass) sb/sth - 
überholen - 
to pass - 
überholen - 
to overtake Brit - 
etw überholen 
}\\
\end{tabular}
}
%===überlegen===
\card{\normalfont \Huge überlegen}{
\begin{tabular}{lll}
\parbox[t][][t]{2.0 cm}{\normalfont \raggedleft ich\\du\\er/sie/es\\wir\\ihr\\sie} &    
\parbox[t][][t]{2cm}{\normalfont lege über\\legst über\\legt über\\legen über\\legt über\\legen über} &
\parbox[t][][t]{2cm}{\normalfont legte über\\legtest über\\legte über\\legten über\\legtet über\\legten über}\\
\end{tabular}
\begin{tabular}{l}
\parbox[t][][t]{8cm}{}\\
\parbox[t][][t]{8cm}{\normalfont \footnotesize 
überlegen - 
to think (about it) - 
nach kurzem/langem Überlegen - 
after a short time of thinking/after long deliberation - 
was gibt es denn da zu überlegen? - 
what's there to think about? - 
(sich dat) überlegen, dass ... - 
to think that ... - 
ohne zu überlegen - 
without thinking 
}\\
\end{tabular}
}
%===übernehmen===
\card{\normalfont \Huge übernehmen}{
\begin{tabular}{lll}
\parbox[t][][t]{2.0 cm}{\normalfont \raggedleft ich\\du\\er/sie/es\\wir\\ihr\\sie} &    
\parbox[t][][t]{2cm}{\normalfont übernehme\\übernimmst\\übernimmt\\übernehmen\\übernehmt\\übernehmen} &
\parbox[t][][t]{2cm}{\normalfont übernahm\\übernahmst\\übernahm\\übernahmen\\übernahmt\\übernahmen}\\
\end{tabular}
\begin{tabular}{l}
\parbox[t][][t]{8cm}{}\\
\parbox[t][][t]{8cm}{\normalfont \footnotesize 
etw übernehmen - 
to take (possession of form ) sth - 
etw übernehmen - 
(kaufen) - 
to buy sth - 
enteigneten Besitz/ein Geschäft übernehmen - 
to take over expropriated property/a business - 
etw übernehmen - 
to accept sth - 
lassen Sie es, das übernehme ich 
}\\
\end{tabular}
}
%===überraschen===
\card{\normalfont \Huge überraschen}{
\begin{tabular}{lll}
\parbox[t][][t]{2.0 cm}{\normalfont \raggedleft ich\\du\\er/sie/es\\wir\\ihr\\sie} &    
\parbox[t][][t]{2cm}{\normalfont überrasche\\überraschst\\überrascht\\überraschen\\überrascht\\überraschen} &
\parbox[t][][t]{2cm}{\normalfont überraschte\\überraschtest\\überraschte\\überraschten\\überraschtet\\überraschten}\\
\end{tabular}
\begin{tabular}{l}
\parbox[t][][t]{8cm}{}\\
\parbox[t][][t]{8cm}{\normalfont \footnotesize 
jdn überraschen - 
to surprise sb - 
jdn mit einem Besuch überraschen - 
to surprise sb with a visit - 
jdn mit einem Besuch überraschen - 
to give sb a surprise visit - 
jdn bei etw dat - 
überraschen - 
to surprise (or catch) sb doing sth - 
jdn dabei überraschen, wie jd etw tut 
}\\
\end{tabular}
}
%===übersetzen===
\card{\normalfont \Huge übersetzen}{
\begin{tabular}{lll}
\parbox[t][][t]{2.0 cm}{\normalfont \raggedleft ich\\du\\er/sie/es\\wir\\ihr\\sie} &    
\parbox[t][][t]{2cm}{\normalfont übersetze\\übersetzt\\übersetzt\\übersetzen\\übersetzt\\übersetzen} &
\parbox[t][][t]{2cm}{\normalfont übersetzte\\übersetztest\\übersetzte\\übersetzten\\übersetztet\\übersetzten}\\
\end{tabular}
\begin{tabular}{l}
\parbox[t][][t]{8cm}{}\\
\parbox[t][][t]{8cm}{\normalfont \footnotesize 
etw übersetzen - 
to translate sth - 
etw nur schwer/annähernd übersetzen - 
to translate sth only with difficulty/to do (or - 
form render) an approximate translation of sth - 
etw (aus dem Polnischen) (ins Französische) übersetzen - 
to translate sth (from Polish) (into French) - 
etw (aus dem Polnischen) (ins Französische) übersetzen - 
to render sth (into French) (from Polish) form - 
(aus etw dat) (in etw akk) übersetzen 
}\\
\end{tabular}
}
%===überweisen===
\card{\normalfont \Huge überweisen}{
\begin{tabular}{lll}
\parbox[t][][t]{2.0 cm}{\normalfont \raggedleft ich\\du\\er/sie/es\\wir\\ihr\\sie} &    
\parbox[t][][t]{2cm}{\normalfont überweise\\überweist\\überweist\\überweisen\\überweist\\überweisen} &
\parbox[t][][t]{2cm}{\normalfont überwies\\überwiesest\\überwies\\überwiesen\\überwiest\\überwiesen}\\
\end{tabular}
\begin{tabular}{l}
\parbox[t][][t]{8cm}{}\\
\parbox[t][][t]{8cm}{\normalfont \footnotesize 
(jdm) etw (auf etw akk) überweisen - 
to transfer sth (to (sb's) sth) - 
jdn (an jdn/zu jdm/in etw akk) überweisen - 
to refer sb (to sb/sth) - 
jdn zum Neurologen überweisen - 
to refer sb to a neurologist - 
einen Patienten zur neurologischen Abklärung überweisen - 
to refer a patient for neurologic evaluation - 
Geld überweisen - 
to transfer money 
}\\
\end{tabular}
}
%===überwinden===
\card{\normalfont \Huge überwinden}{
\begin{tabular}{lll}
\parbox[t][][t]{2.0 cm}{\normalfont \raggedleft ich\\du\\er/sie/es\\wir\\ihr\\sie} &    
\parbox[t][][t]{2cm}{\normalfont überwinde\\überwindest\\überwindet\\überwinden\\überwindet\\überwinden} &
\parbox[t][][t]{2cm}{\normalfont überwand\\überwandest \\überwand\\überwanden\\überwandet\\überwanden}\\
\end{tabular}
\begin{tabular}{l}
\parbox[t][][t]{8cm}{}\\
\parbox[t][][t]{8cm}{\normalfont \footnotesize 
etw überwinden - 
to overcome sth - 
ein Vorurteil überwinden - 
to outgrow a prejudice - 
jdn überwinden - 
to defeat sb - 
etw überwinden - 
to get over (or surmount) sth - 
sich akk überwinden - 
to overcome one's feelings/inclinations etc. 
}\\
\end{tabular}
}
%===überzeugen===
\card{\normalfont \Huge überzeugen}{
\begin{tabular}{lll}
\parbox[t][][t]{2.0 cm}{\normalfont \raggedleft ich\\du\\er/sie/es\\wir\\ihr\\sie} &    
\parbox[t][][t]{2cm}{\normalfont überzeuge\\überzeugst\\überzeugt\\überzeugen\\überzeugt\\überzeugen} &
\parbox[t][][t]{2cm}{\normalfont überzeugte\\überzeugtest\\überzeugte\\überzeugten\\überzeugtet\\überzeugten}\\
\end{tabular}
\begin{tabular}{l}
\parbox[t][][t]{8cm}{}\\
\parbox[t][][t]{8cm}{\normalfont \footnotesize 
jdn überzeugen - 
to convince sb - 
jdn überzeugen - 
(umstimmen a.) - 
to persuade sb - 
den Richter überzeugen - 
to satisfy the judge - 
jdn von etw dat - 
überzeugen - 
to convince sb of sth 
}\\
\end{tabular}
}
%===unterbrechen===
\card{\normalfont \Huge unterbrechen}{
\begin{tabular}{lll}
\parbox[t][][t]{2.0 cm}{\normalfont \raggedleft ich\\du\\er/sie/es\\wir\\ihr\\sie} &    
\parbox[t][][t]{2cm}{\normalfont unterbreche\\unterbrichst\\unterbricht\\unterbrechen\\unterbrecht\\unterbrechen} &
\parbox[t][][t]{2cm}{\normalfont unterbrach\\unterbrachst\\unterbrach\\unterbrachen\\unterbracht\\unterbrachen}\\
\end{tabular}
\begin{tabular}{l}
\parbox[t][][t]{8cm}{}\\
\parbox[t][][t]{8cm}{\normalfont \footnotesize 
etw unterbrechen - 
to interrupt sth - 
seine Arbeit unterbrechen - 
to interrupt one's work - 
eine Reise unterbrechen - 
to break a journey - 
eine Schwangerschaft unterbrechen - 
to terminate a pregnancy - 
jdn unterbrechen - 
to interrupt sb 
}\\
\end{tabular}
}
%===unterhalten===
\card{\normalfont \Huge unterhalten}{
\begin{tabular}{lll}
\parbox[t][][t]{2.0 cm}{\normalfont \raggedleft ich\\du\\er/sie/es\\wir\\ihr\\sie} &    
\parbox[t][][t]{2cm}{\normalfont unterhalte\\unterhaltest\\unterhaltet\\unterhalten\\unterhaltet\\unterhalten} &
\parbox[t][][t]{2cm}{\normalfont unterhaltete\\unterhaltetest\\unterhaltete\\unterhalteten\\unterhaltetet\\unterhalteten}\\
\end{tabular}
\begin{tabular}{l}
\parbox[t][][t]{8cm}{}\\
\parbox[t][][t]{8cm}{\normalfont \footnotesize 
jdn unterhalten - 
to support sb - 
er muss vier Kinder unterhalten - 
he has to support four children - 
etw unterhalten - 
to maintain sth - 
etw unterhalten - 
to run sth - 
etw unterhalten - 
to have sth 
}\\
\end{tabular}
}
%===unternehmen===
\card{\normalfont \Huge unternehmen}{
\begin{tabular}{lll}
\parbox[t][][t]{2.0 cm}{\normalfont \raggedleft ich\\du\\er/sie/es\\wir\\ihr\\sie} &    
\parbox[t][][t]{2cm}{\normalfont unternehme\\unternimmst\\unternimmt\\unternehmen\\unternehmt\\unternehmen} &
\parbox[t][][t]{2cm}{\normalfont unternahm\\unternahmst\\unternahm\\unternahmen\\unternahmt\\unternahmen}\\
\end{tabular}
\begin{tabular}{l}
\parbox[t][][t]{8cm}{}\\
\parbox[t][][t]{8cm}{\normalfont \footnotesize 
etw/nichts (gegen jdn/etw) unternehmen - 
to take action/no action (against sb/sth) - 
Schritte gegen etw akk - 
unternehmen - 
to take steps against sth - 
etw (mit jdm) unternehmen - 
to do sth (with sb) - 
wollen wir nicht etwas zusammen unternehmen? - 
why don't we do something together? - 
etw unternehmen 
}\\
\end{tabular}
}
%===unterrichten===
\card{\normalfont \Huge unterrichten}{
\begin{tabular}{lll}
\parbox[t][][t]{2.0 cm}{\normalfont \raggedleft ich\\du\\er/sie/es\\wir\\ihr\\sie} &    
\parbox[t][][t]{2cm}{\normalfont unterrichte\\unterrichtest\\unterrichtet\\unterrichten\\unterrichtet\\unterrichten} &
\parbox[t][][t]{2cm}{\normalfont unterrichtete\\unterrichtetest\\unterrichtete\\unterrichteten\\unterrichtetet\\unterrichteten}\\
\end{tabular}
\begin{tabular}{l}
\parbox[t][][t]{8cm}{}\\
\parbox[t][][t]{8cm}{\normalfont \footnotesize 
jdn/etw (in etw dat) unterrichten - 
to teach sb/sth (sth) - 
eine Klasse in Französisch unterrichten - 
to teach a class French - 
ich habe ihn früher in Mathematik unterrichtet - 
I used to teach him mathematics - 
etw unterrichten - 
to teach sth - 
Chemie unterrichten - 
to teach Chemistry 
}\\
\end{tabular}
}
%===unterscheiden===
\card{\normalfont \Huge unterscheiden}{
\begin{tabular}{lll}
\parbox[t][][t]{2.0 cm}{\normalfont \raggedleft ich\\du\\er/sie/es\\wir\\ihr\\sie} &    
\parbox[t][][t]{2cm}{\normalfont unterscheide\\unterscheidest\\unterscheidet\\unterscheiden\\unterscheidet\\unterscheiden} &
\parbox[t][][t]{2cm}{\normalfont unterschied\\unterschiedest \\unterschied\\unterschieden\\unterschiedet\\unterschieden}\\
\end{tabular}
\begin{tabular}{l}
\parbox[t][][t]{8cm}{}\\
\parbox[t][][t]{8cm}{\normalfont \footnotesize 
etw unterscheiden - 
to distinguish (or make a distinction) between sth - 
der Botaniker unterscheidet Fichten und Kiefern - 
the botanist makes a distinction between firs and pines - 
etw (von etw dat) unterscheiden - 
to tell sth from sth - 
etw voneinander (an etw dat) unterscheiden - 
to tell the difference between things (or to tell things apart)  (by sth) - 
ich kann die beiden nie unterscheiden - 
I can never tell the difference between the two 
}\\
\end{tabular}
}
%===unterschreiben===
\card{\normalfont \Huge unterschreiben}{
\begin{tabular}{lll}
\parbox[t][][t]{2.0 cm}{\normalfont \raggedleft ich\\du\\er/sie/es\\wir\\ihr\\sie} &    
\parbox[t][][t]{2cm}{\normalfont unterschreibe\\unterschreibst\\unterschreibt\\unterschreiben\\unterschreibt\\unterschreiben} &
\parbox[t][][t]{2cm}{\normalfont unterschrieb\\unterschriebst\\unterschrieb\\unterschrieben\\unterschriebt\\unterschrieben}\\
\end{tabular}
\begin{tabular}{l}
\parbox[t][][t]{8cm}{}\\
\parbox[t][][t]{8cm}{\normalfont \footnotesize 
(jdm) etw unterschreiben - 
to sign sth (for sb) - 
eine Meinung/Ansicht unterschreiben können fig - 
to be able to subscribe to an opinion/point of view fig - 
(auf etw dat) unterschreiben - 
to sign (sth) - 
eigenhändig unterschreiben - 
to sign in one's own hand 
}\\
\end{tabular}
}
%===unterstützen===
\card{\normalfont \Huge unterstützen}{
\begin{tabular}{lll}
\parbox[t][][t]{2.0 cm}{\normalfont \raggedleft ich\\du\\er/sie/es\\wir\\ihr\\sie} &    
\parbox[t][][t]{2cm}{\normalfont unterstütze\\unterstützt\\unterstützt\\unterstützen\\unterstützt\\unterstützen} &
\parbox[t][][t]{2cm}{\normalfont unterstützte\\unterstütztest\\unterstützte\\unterstützten\\unterstütztet\\unterstützten}\\
\end{tabular}
\begin{tabular}{l}
\parbox[t][][t]{8cm}{}\\
\parbox[t][][t]{8cm}{\normalfont \footnotesize 
jdn (bei etw dat/in etw dat) unterstützen - 
to support sb (in sth) - 
die Heilung unterstützen - 
to assist sb's recovery - 
jdn/etw (mit etw dat) unterstützen - 
to support sb/sth (with sth) - 
wirst du noch von deinen Eltern finanziell unterstützt? - 
do your parents still financially support you? - 
etw unterstützen - 
to back (or support) sth 
}\\
\end{tabular}
}
%===verabschieden===
\card{\normalfont \Huge verabschieden}{
\begin{tabular}{lll}
\parbox[t][][t]{2.0 cm}{\normalfont \raggedleft ich\\du\\er/sie/es\\wir\\ihr\\sie} &    
\parbox[t][][t]{2cm}{\normalfont verabschiede mich\\verabschiedest dich\\verabschiedet sich\\verabschieden uns\\verabschiedet euch\\verabschieden sich} &
\parbox[t][][t]{2cm}{\normalfont verabschiedete mich\\verabschiedetest dich\\verabschiedete sich\\verabschiedeten uns\\verabschiedetet euch\\verabschiedeten sich}\\
\end{tabular}
\begin{tabular}{l}
\parbox[t][][t]{8cm}{}\\
\parbox[t][][t]{8cm}{\normalfont \footnotesize 
sich akk (von jdm) verabschieden - 
to say goodbye (to sb) - 
sich akk (aus etw dat) verabschieden - 
to dissociate oneself from sth - 
etw verabschieden - 
to pass sth - 
einen Haushalt verabschieden - 
to adopt a budget - 
jdn verabschieden - 
to take one's leave of sb 
}\\
\end{tabular}
}
%===verändern===
\card{\normalfont \Huge verändern}{
\begin{tabular}{lll}
\parbox[t][][t]{2.0 cm}{\normalfont \raggedleft ich\\du\\er/sie/es\\wir\\ihr\\sie} &    
\parbox[t][][t]{2cm}{\normalfont verändere\\veränderst\\verändert\\verändern\\verändert\\verändern} &
\parbox[t][][t]{2cm}{\normalfont veränderte\\verändertest\\veränderte\\veränderten\\verändertet\\veränderten}\\
\end{tabular}
\begin{tabular}{l}
\parbox[t][][t]{8cm}{}\\
\parbox[t][][t]{8cm}{\normalfont \footnotesize 
etw verändern - 
to change sth - 
jdn verändern - 
(im Wesen) - 
to change sb - 
jdn/etw verändern - 
to make sb/sth look different/change sb's sth - 
sich akk - 
verändern - 
to change 
}\\
\end{tabular}
}
%===verbessern===
\card{\normalfont \Huge verbessern}{
\begin{tabular}{lll}
\parbox[t][][t]{2.0 cm}{\normalfont \raggedleft ich\\du\\er/sie/es\\wir\\ihr\\sie} &    
\parbox[t][][t]{2cm}{\normalfont verbessere\\verbesserst\\verbessert\\verbessern\\verbessert\\verbessern} &
\parbox[t][][t]{2cm}{\normalfont verbesserte\\verbessertest\\verbesserte\\verbesserten\\verbessertet\\verbesserten}\\
\end{tabular}
\begin{tabular}{l}
\parbox[t][][t]{8cm}{}\\
\parbox[t][][t]{8cm}{\normalfont \footnotesize 
etw verbessern - 
to improve sth - 
etw verbessern - 
to improve (up)on (or better) sth - 
einen Rekord verbessern - 
to break a record - 
etw verbessern - 
to correct sth - 
jdn verbessern - 
to correct sb 
}\\
\end{tabular}
}
%===verbieten===
\card{\normalfont \Huge verbieten}{
\begin{tabular}{lll}
\parbox[t][][t]{2.0 cm}{\normalfont \raggedleft ich\\du\\er/sie/es\\wir\\ihr\\sie} &    
\parbox[t][][t]{2cm}{\normalfont verbiete\\verbietest\\verbietet\\verbieten\\verbietet\\verbieten} &
\parbox[t][][t]{2cm}{\normalfont verbot\\verbotest\\verbot\\verboten\\verbotet\\verboten}\\
\end{tabular}
\begin{tabular}{l}
\parbox[t][][t]{8cm}{}\\
\parbox[t][][t]{8cm}{\normalfont \footnotesize 
etw verbieten - 
to ban sth - 
eine Organisation/Partei/Publikation verbieten - 
to ban (or outlaw) an organization/a party/publication - 
(jdm) etw verbieten - 
to forbid sth (or sb to do sth) - 
etw ist (jdm) verboten - 
sth is forbidden (as far as sb is concerned) - 
jdm verbieten, etw zu tun - 
to forbid sb to do sth 
}\\
\end{tabular}
}
%===verbinden===
\card{\normalfont \Huge verbinden}{
\begin{tabular}{lll}
\parbox[t][][t]{2.0 cm}{\normalfont \raggedleft ich\\du\\er/sie/es\\wir\\ihr\\sie} &    
\parbox[t][][t]{2cm}{\normalfont verbinde\\verbindest\\verbindet\\verbinden\\verbindet\\verbinden} &
\parbox[t][][t]{2cm}{\normalfont verband\\verbandest\\verband\\verbanden\\verbandet\\verbanden}\\
\end{tabular}
\begin{tabular}{l}
\parbox[t][][t]{8cm}{}\\
\parbox[t][][t]{8cm}{\normalfont \footnotesize 
jdn verbinden - 
to dress sb's wound(s) - 
(jdm/sich) etw verbinden - 
to dress (sb's/one's) sth - 
etw (miteinander) verbinden - 
to join (up sep ) sth - 
etw (mit etw dat) verbinden - 
to join sth (to sth) - 
jdn (mit jdm) verbinden - 
to put sb through (or connect sb)  (to sb) 
}\\
\end{tabular}
}
%===verbrauchen===
\card{\normalfont \Huge verbrauchen}{
\begin{tabular}{lll}
\parbox[t][][t]{2.0 cm}{\normalfont \raggedleft ich\\du\\er/sie/es\\wir\\ihr\\sie} &    
\parbox[t][][t]{2cm}{\normalfont verbrauche\\verbrauchst\\verbraucht\\verbrauchen\\verbraucht\\verbrauchen} &
\parbox[t][][t]{2cm}{\normalfont verbrauchte\\verbrauchtest\\verbrauchte\\verbrauchten\\verbrauchtet\\verbrauchten}\\
\end{tabular}
\begin{tabular}{l}
\parbox[t][][t]{8cm}{}\\
\parbox[t][][t]{8cm}{\normalfont \footnotesize 
etw verbrauchen - 
to use up sth sep - 
Lebensmittel verbrauchen - 
to eat (or consume) food (or - 
Brit foodstuffs) - 
Vorräte verbrauchen - 
to use up (one's) provisions - 
etw verbrauchen - 
to spend sth - 
etw verbrauchen 
}\\
\end{tabular}
}
%===verderben===
\card{\normalfont \Huge verderben}{
\begin{tabular}{lll}
\parbox[t][][t]{2.0 cm}{\normalfont \raggedleft ich\\du\\er/sie/es\\wir\\ihr\\sie} &    
\parbox[t][][t]{2cm}{\normalfont verderbe\\verdirbst\\verdirbt\\verderben\\verderbt\\verderben} &
\parbox[t][][t]{2cm}{\normalfont verdarb\\verdarbst\\verdarb\\verdarben\\verdarbt\\verdarben}\\
\end{tabular}
\begin{tabular}{l}
\parbox[t][][t]{8cm}{}\\
\parbox[t][][t]{8cm}{\normalfont \footnotesize 
jdn/etw verderben - 
to corrupt sb/sth - 
(jdm) etw verderben - 
to ruin (sb's) sth - 
jdn verderben - 
to ruin sb - 
(jdm) etw verderben - 
to spoil (or ruin)  (sb's) sth - 
es sich dat (mit jdm) verderben - 
to fall out (with sb) 
}\\
\end{tabular}
}
%===verdienen===
\card{\normalfont \Huge verdienen}{
\begin{tabular}{lll}
\parbox[t][][t]{2.0 cm}{\normalfont \raggedleft ich\\du\\er/sie/es\\wir\\ihr\\sie} &    
\parbox[t][][t]{2cm}{\normalfont verdiene\\verdienst\\verdient\\verdienen\\verdient\\verdienen} &
\parbox[t][][t]{2cm}{\normalfont verdiente\\verdientest\\verdiente\\verdienten\\verdientet\\verdienten}\\
\end{tabular}
\begin{tabular}{l}
\parbox[t][][t]{8cm}{}\\
\parbox[t][][t]{8cm}{\normalfont \footnotesize 
etw verdienen - 
to earn sth - 
er verdient nur 1.000 Euro im Monat - 
he only earns 1,000 euros a month - 
etw (an etw dat) verdienen - 
to make sth (on sth) - 
ich verdiene kaum 300 Euro am Wagen - 
I'm scarcely making 300 euros on the car - 
(sich dat) etw verdienen - 
to earn the money for sth 
}\\
\end{tabular}
}
%===verdrießen===
\card{\normalfont \Huge verdrießen}{
\begin{tabular}{lll}
\parbox[t][][t]{2.0 cm}{\normalfont \raggedleft ich\\du\\er/sie/es\\wir\\ihr\\sie} &    
\parbox[t][][t]{2cm}{\normalfont verdrieße\\verdrießt\\verdrießt\\verdrießen\\verdrießt\\verdrießen} &
\parbox[t][][t]{2cm}{\normalfont verdross\\verdrossest\\verdross\\verdrossen\\verdrosst\\verdrossen}\\
\end{tabular}
\begin{tabular}{l}
\parbox[t][][t]{8cm}{}\\
\parbox[t][][t]{8cm}{\normalfont \footnotesize 
jdn verdrießen - 
to irritate (or annoy) sb - 
es sich dat nicht verdrießen lassen - 
to not be put off 
}\\
\end{tabular}
}
%===vergessen===
\card{\normalfont \Huge vergessen}{
\begin{tabular}{lll}
\parbox[t][][t]{2.0 cm}{\normalfont \raggedleft ich\\du\\er/sie/es\\wir\\ihr\\sie} &    
\parbox[t][][t]{2cm}{\normalfont vergesse\\vergisst\\vergisst\\vergessen\\vergesst\\vergessen} &
\parbox[t][][t]{2cm}{\normalfont vergaß\\vergaßt\\vergaß\\vergaßen\\vergaßt\\vergaßen}\\
\end{tabular}
\begin{tabular}{l}
\parbox[t][][t]{8cm}{}\\
\parbox[t][][t]{8cm}{\normalfont \footnotesize 
etw/jdn vergessen - 
to forget sth/sb - 
jd wird jdm etw nie (o. nicht) - 
vergessen - 
sb will never (or not) forget sb's sth - 
jd wird jdm etw nie (o. nicht) - 
vergessen - 
sb will never forget what sb did - 
das werde ich ihm nicht vergessen, das zahle ich ihm heim - 
I won't forget what he did, I'll pay him back for that 
}\\
\end{tabular}
}
%===vergleichen===
\card{\normalfont \Huge vergleichen}{
\begin{tabular}{lll}
\parbox[t][][t]{2.0 cm}{\normalfont \raggedleft ich\\du\\er/sie/es\\wir\\ihr\\sie} &    
\parbox[t][][t]{2cm}{\normalfont vergleiche\\vergleichst\\vergleicht\\vergleichen\\vergleicht\\vergleichen} &
\parbox[t][][t]{2cm}{\normalfont verglich\\verglichst\\verglich\\verglichen\\verglicht\\verglichen}\\
\end{tabular}
\begin{tabular}{l}
\parbox[t][][t]{8cm}{}\\
\parbox[t][][t]{8cm}{\normalfont \footnotesize 
(miteinander) vergleichen - 
to compare things (with each other) - 
ich vergleiche die Preise immer genau - 
I always compare prices very carefully - 
jdn (mit jdm) vergleichen - 
to compare sb with sb - 
etw (mit etw dat) vergleichen - 
to compare sth (with sth) - 
verglichen mit - 
compared with (or to) 
}\\
\end{tabular}
}
%===verhaften===
\card{\normalfont \Huge verhaften}{
\begin{tabular}{lll}
\parbox[t][][t]{2.0 cm}{\normalfont \raggedleft ich\\du\\er/sie/es\\wir\\ihr\\sie} &    
\parbox[t][][t]{2cm}{\normalfont verhafte\\verhaftest\\verhaftet\\verhaften\\verhaftet\\verhaften} &
\parbox[t][][t]{2cm}{\normalfont verhaftete\\verhaftetest\\verhaftete\\verhafteten\\verhaftetet\\verhafteten}\\
\end{tabular}
\begin{tabular}{l}
\parbox[t][][t]{8cm}{}\\
\parbox[t][][t]{8cm}{\normalfont \footnotesize 
jdn verhaften - 
to arrest sb - 
Sie sind verhaftet! - 
you are under arrest! - 
Sie sind verhaftet! - 
you're nicked! (or busted!) sl 
}\\
\end{tabular}
}
%===verhalten===
\card{\normalfont \Huge verhalten}{
\begin{tabular}{lll}
\parbox[t][][t]{2.0 cm}{\normalfont \raggedleft ich\\du\\er/sie/es\\wir\\ihr\\sie} &    
\parbox[t][][t]{2cm}{\normalfont verhalte mich\\verhältst dich\\verhält sich\\verhalten uns\\verhaltet euch\\verhalten sich} &
\parbox[t][][t]{2cm}{\normalfont verhielt mich\\verhieltest \\verhielt sich\\verhielten uns\\verhieltet euch\\verhielten sich}\\
\end{tabular}
\begin{tabular}{l}
\parbox[t][][t]{8cm}{}\\
\parbox[t][][t]{8cm}{\normalfont \footnotesize 
sich akk (jdm gegenüber) (irgendwie) verhalten - 
to behave (in a certain manner) (towards sb) - 
sich akk (irgendwie) verhalten - 
to be (a certain way) - 
die Sache verhält sich anders, als du denkst - 
the matter is not as you think - 
sich akk (irgendwie) verhalten - 
to react (in a certain way) - 
die neue Verbindung verhält sich äußerst stabil - 
the new compound reacts extremely stably 
}\\
\end{tabular}
}
%===verhandeln===
\card{\normalfont \Huge verhandeln}{
\begin{tabular}{lll}
\parbox[t][][t]{2.0 cm}{\normalfont \raggedleft ich\\du\\er/sie/es\\wir\\ihr\\sie} &    
\parbox[t][][t]{2cm}{\normalfont verhandle \\verhandelst\\verhandelt\\verhandeln\\verhandelt\\verhandeln} &
\parbox[t][][t]{2cm}{\normalfont verhandelte\\verhandeltest\\verhandelte\\verhandelten\\verhandeltet\\verhandelten}\\
\end{tabular}
\begin{tabular}{l}
\parbox[t][][t]{8cm}{}\\
\parbox[t][][t]{8cm}{\normalfont \footnotesize 
(mit jdm) (über etw akk) verhandeln - 
to negotiate (with sb) (about sth) - 
(gegen jdn) (in etw dat) verhandeln - 
to try sb (in sth) - 
etw verhandeln - 
to negotiate sth - 
etw verhandeln - 
to hear sth - 
das Gericht wird diesen Fall wohl erst nach der Sommerpause verhandeln - 
the court will probably hear this case after the summer break 
}\\
\end{tabular}
}
%===verhindern===
\card{\normalfont \Huge verhindern}{
\begin{tabular}{lll}
\parbox[t][][t]{2.0 cm}{\normalfont \raggedleft ich\\du\\er/sie/es\\wir\\ihr\\sie} &    
\parbox[t][][t]{2cm}{\normalfont verhindere\\verhinderst\\verhindert\\verhindern\\verhindert\\verhindern} &
\parbox[t][][t]{2cm}{\normalfont verhinderte\\verhindertest\\verhinderte\\verhinderten\\verhindertet\\verhinderten}\\
\end{tabular}
\begin{tabular}{l}
\parbox[t][][t]{8cm}{}\\
\parbox[t][][t]{8cm}{\normalfont \footnotesize 
etw verhindern - 
to prevent (or stop) sth - 
verhindern, dass jd etw tut - 
to prevent (or stop) sb from doing sth - 
verhindern, dass etw geschieht - 
to prevent (or stop) sth from happening 
}\\
\end{tabular}
}
%===verkaufen===
\card{\normalfont \Huge verkaufen}{
\begin{tabular}{lll}
\parbox[t][][t]{2.0 cm}{\normalfont \raggedleft ich\\du\\er/sie/es\\wir\\ihr\\sie} &    
\parbox[t][][t]{2cm}{\normalfont verkaufe\\verkaufst\\verkauft\\verkaufen\\verkauft\\verkaufen} &
\parbox[t][][t]{2cm}{\normalfont verkaufte\\verkauftest\\verkaufte\\verkauften\\verkauftet\\verkauften}\\
\end{tabular}
\begin{tabular}{l}
\parbox[t][][t]{8cm}{}\\
\parbox[t][][t]{8cm}{\normalfont \footnotesize 
(jdm) etw (für etw akk) verkaufen - 
to sell (sb) sth (for sth) - 
etw (an jdn) verkaufen - 
to sell sth (to sb) - 
zu verkaufen sein - 
to be for sale - 
„zu verkaufen“ - 
“for sale” - 
meistbietend verkaufen - 
HANDEL 
}\\
\end{tabular}
}
%===verlangen===
\card{\normalfont \Huge verlangen}{
\begin{tabular}{lll}
\parbox[t][][t]{2.0 cm}{\normalfont \raggedleft ich\\du\\er/sie/es\\wir\\ihr\\sie} &    
\parbox[t][][t]{2cm}{\normalfont verlange\\verlangst\\verlangt\\verlangen\\verlangt\\verlangen} &
\parbox[t][][t]{2cm}{\normalfont verlangte\\verlangtest\\verlangte\\verlangten\\verlangtet\\verlangten}\\
\end{tabular}
\begin{tabular}{l}
\parbox[t][][t]{8cm}{}\\
\parbox[t][][t]{8cm}{\normalfont \footnotesize 
etw (von jdm) verlangen - 
to demand sth (from sb) - 
einen Preis verlangen - 
to ask (or charge)  a price - 
eine Bestrafung/das Eingreifen/eine Untersuchung verlangen - 
to demand (or call for) punishment/intervention/an investigation - 
Maßnahmen verlangen - 
to demand that steps (or measures) be taken - 
verlangen, dass jd etw tut/etw geschieht - 
to demand that sb does sth/sth be done 
}\\
\end{tabular}
}
%===verlassen===
\card{\normalfont \Huge verlassen}{
\begin{tabular}{lll}
\parbox[t][][t]{2.0 cm}{\normalfont \raggedleft ich\\du\\er/sie/es\\wir\\ihr\\sie} &    
\parbox[t][][t]{2cm}{\normalfont verlasse\\verlässt\\verlässt\\verlassen\\verlasst\\verlassen} &
\parbox[t][][t]{2cm}{\normalfont verließ\\verließt\\verließ\\verließen\\verließt\\verließen}\\
\end{tabular}
\begin{tabular}{l}
\parbox[t][][t]{8cm}{}\\
\parbox[t][][t]{8cm}{\normalfont \footnotesize 
jdn verlassen - 
to abandon (or leave) - 
(or desert) sb - 
etw verlassen - 
to leave sth - 
jdn verlassen - 
to pass away (or on) - 
jdn verlassen - 
to desert sb - 
der Mut verließ ihn 
}\\
\end{tabular}
}
%===verletzen===
\card{\normalfont \Huge verletzen}{
\begin{tabular}{lll}
\parbox[t][][t]{2.0 cm}{\normalfont \raggedleft ich\\du\\er/sie/es\\wir\\ihr\\sie} &    
\parbox[t][][t]{2cm}{\normalfont verletze\\verletzt\\verletzt\\verletzen\\verletzt\\verletzen} &
\parbox[t][][t]{2cm}{\normalfont verletzte\\verletztest\\verletzte\\verletzten\\verletztet\\verletzten}\\
\end{tabular}
\begin{tabular}{l}
\parbox[t][][t]{8cm}{}\\
\parbox[t][][t]{8cm}{\normalfont \footnotesize 
jdm/sich etw verletzen - 
to injure (or hurt) sb's/one's sth - 
jdn (an etw dat) verletzen - 
to injure (or hurt) sb('s sth) - 
sich akk - 
verletzen - 
to injure (or hurt) oneself - 
sich akk beim Schneiden verletzen - 
to cut oneself - 
sich akk etw (o. an etw dat 
}\\
\end{tabular}
}
%===vermuten===
\card{\normalfont \Huge vermuten}{
\begin{tabular}{lll}
\parbox[t][][t]{2.0 cm}{\normalfont \raggedleft ich\\du\\er/sie/es\\wir\\ihr\\sie} &    
\parbox[t][][t]{2cm}{\normalfont vermute\\vermutest\\vermutet\\vermuten\\vermutet\\vermuten} &
\parbox[t][][t]{2cm}{\normalfont vermutete\\vermutetest\\vermutete\\vermuteten\\vermutetet\\vermuteten}\\
\end{tabular}
\begin{tabular}{l}
\parbox[t][][t]{8cm}{}\\
\parbox[t][][t]{8cm}{\normalfont \footnotesize 
etw (hinter etw dat) vermuten - 
to suspect sth ((is) behind sth) - 
vermuten, (dass) ... - 
to suspect (that) ... - 
vermuten lassen, dass ... - 
to give rise to the suspicion (or supposition) that ... - 
jdn irgendwo vermuten - 
to think that sb is (or to suppose sb to be) somewhere - 
vermuten - 
to guess 
}\\
\end{tabular}
}
%===versichern===
\card{\normalfont \Huge versichern}{
\begin{tabular}{lll}
\parbox[t][][t]{2.0 cm}{\normalfont \raggedleft ich\\du\\er/sie/es\\wir\\ihr\\sie} &    
\parbox[t][][t]{2cm}{\normalfont versichere\\versicherst\\versichert\\versichern\\versichert\\versichern} &
\parbox[t][][t]{2cm}{\normalfont versicherte\\versichertest\\versicherte\\versicherten\\versichertet\\versicherten}\\
\end{tabular}
\begin{tabular}{l}
\parbox[t][][t]{8cm}{}\\
\parbox[t][][t]{8cm}{\normalfont \footnotesize 
jdn/etw (gegen etw akk) versichern - 
to insure sb/sth (against sth) - 
(gegen etw akk) versichert sein - 
to be insured (against sth) - 
jdm versichern, (dass) ... - 
to assure sb (that) ... - 
jdn einer S. gen - 
versichern - 
to assure sb of sth - 
jdn seiner Freundschaft versichern 
}\\
\end{tabular}
}
%===versprechen===
\card{\normalfont \Huge versprechen}{
\begin{tabular}{lll}
\parbox[t][][t]{2.0 cm}{\normalfont \raggedleft ich\\du\\er/sie/es\\wir\\ihr\\sie} &    
\parbox[t][][t]{2cm}{\normalfont verspreche\\versprichst\\verspricht\\versprechen\\versprecht\\versprechen} &
\parbox[t][][t]{2cm}{\normalfont versprach\\versprachst\\versprach\\versprachen\\verspracht\\versprachen}\\
\end{tabular}
\begin{tabular}{l}
\parbox[t][][t]{8cm}{}\\
\parbox[t][][t]{8cm}{\normalfont \footnotesize 
(jdm) etw versprechen - 
to promise (sb) sth (or sth to sb) - 
(jdm) versprechen, etw zu tun - 
to promise to do sth - 
(jdm) versprechen, etw zu tun - 
to promise sb (that) one will do sth - 
(jdm) versprechen, dass etw geschieht - 
to promise (sb) (that) sth will happen - 
ich kann nicht versprechen, dass es klappt - 
I can't promise it will work 
}\\
\end{tabular}
}
%===verstehen===
\card{\normalfont \Huge verstehen}{
\begin{tabular}{lll}
\parbox[t][][t]{2.0 cm}{\normalfont \raggedleft ich\\du\\er/sie/es\\wir\\ihr\\sie} &    
\parbox[t][][t]{2cm}{\normalfont verstehe\\verstehst\\versteht\\verstehen\\versteht\\verstehen} &
\parbox[t][][t]{2cm}{\normalfont verstand\\verstandest \\verstand\\verstanden\\verstandet\\verstanden}\\
\end{tabular}
\begin{tabular}{l}
\parbox[t][][t]{8cm}{}\\
\parbox[t][][t]{8cm}{\normalfont \footnotesize 
jdn/etw verstehen - 
to hear (or understand) sb/sth - 
ich verstehe nicht, was da gesagt wird - 
I can't make out what's being said - 
verstehen Sie mich (o. können Sie mich verstehen?)? - 
can you hear me? - 
verstehen Sie mich (o. können Sie mich verstehen?)? (Funk) - 
can you read me? - 
ich kann Sie nicht (gut) verstehen - 
I don't understand (very well) what you're saying 
}\\
\end{tabular}
}
%===versuchen===
\card{\normalfont \Huge versuchen}{
\begin{tabular}{lll}
\parbox[t][][t]{2.0 cm}{\normalfont \raggedleft ich\\du\\er/sie/es\\wir\\ihr\\sie} &    
\parbox[t][][t]{2cm}{\normalfont versuche\\versuchst\\versucht\\versuchen\\versucht\\versuchen} &
\parbox[t][][t]{2cm}{\normalfont versuchte\\versuchtest\\versuchte\\versuchten\\versuchtet\\versuchten}\\
\end{tabular}
\begin{tabular}{l}
\parbox[t][][t]{8cm}{}\\
\parbox[t][][t]{8cm}{\normalfont \footnotesize 
etw versuchen - 
to try (or attempt) sth - 
es mit jdm/etw versuchen - 
to give sb/sth a try - 
es mit jdm/etw versuchen - 
to try sb/sth - 
etw versuchen - 
to try (or taste) sth - 
jdn versuchen - 
to tempt sb 
}\\
\end{tabular}
}
%===verteilen===
\card{\normalfont \Huge verteilen}{
\begin{tabular}{lll}
\parbox[t][][t]{2.0 cm}{\normalfont \raggedleft ich\\du\\er/sie/es\\wir\\ihr\\sie} &    
\parbox[t][][t]{2cm}{\normalfont verteile\\verteilst\\verteilt\\verteilen\\verteilt\\verteilen} &
\parbox[t][][t]{2cm}{\normalfont verteilte\\verteiltest\\verteilte\\verteilten\\verteiltet\\verteilten}\\
\end{tabular}
\begin{tabular}{l}
\parbox[t][][t]{8cm}{}\\
\parbox[t][][t]{8cm}{\normalfont \footnotesize 
etw (an jdn) verteilen - 
to distribute sth (to sb) - 
Geschenke/Flugblätter verteilen - 
to distribute (or - 
sep hand out) presents/leaflets - 
Auszeichnungen/Orden verteilen - 
to give (or hand) - 
(or - 
fam dish) out decorations/medals sep - 
etw neu verteilen 
}\\
\end{tabular}
}
%===vertrauen===
\card{\normalfont \Huge vertrauen}{
\begin{tabular}{lll}
\parbox[t][][t]{2.0 cm}{\normalfont \raggedleft ich\\du\\er/sie/es\\wir\\ihr\\sie} &    
\parbox[t][][t]{2cm}{\normalfont vertraue\\vertraust\\vertraut\\vertrauen\\vertraut\\vertrauen} &
\parbox[t][][t]{2cm}{\normalfont vertraute\\vertrautest\\vertraute\\vertrauten\\vertrautet\\vertrauten}\\
\end{tabular}
\begin{tabular}{l}
\parbox[t][][t]{8cm}{}\\
\parbox[t][][t]{8cm}{\normalfont \footnotesize 
jdm vertrauen - 
to trust sb - 
auf jdn vertrauen - 
to trust in sb - 
auf etw akk - 
vertrauen - 
to trust in sth - 
auf sein Glück vertrauen - 
to trust to luck - 
auf Gott vertrauen 
}\\
\end{tabular}
}
%===verwenden===
\card{\normalfont \Huge verwenden}{
\begin{tabular}{lll}
\parbox[t][][t]{2.0 cm}{\normalfont \raggedleft ich\\du\\er/sie/es\\wir\\ihr\\sie} &    
\parbox[t][][t]{2cm}{\normalfont verwende\\verwendest\\verwendet\\verwenden\\verwendet\\verwenden} &
\parbox[t][][t]{2cm}{\normalfont verwendete \\verwendetest\\verwendete \\verwendeten \\verwendetet \\verwendeten }\\
\end{tabular}
\begin{tabular}{l}
\parbox[t][][t]{8cm}{}\\
\parbox[t][][t]{8cm}{\normalfont \footnotesize 
etw (für etw akk) verwenden - 
to use sth (for sth) - 
etw ist noch zu verwenden - 
sth can still be used (or is still usable) - 
verwendet - 
applied - 
verwendet oder verbraucht - 
applied or used - 
nicht mehr verwendet - 
disused 
}\\
\end{tabular}
}
%===verwirren===
\card{\normalfont \Huge verwirren}{
\begin{tabular}{lll}
\parbox[t][][t]{2.0 cm}{\normalfont \raggedleft ich\\du\\er/sie/es\\wir\\ihr\\sie} &    
\parbox[t][][t]{2cm}{\normalfont verwirre\\verwirrst\\verwirrt\\verwirren\\verwirrt\\verwirren} &
\parbox[t][][t]{2cm}{\normalfont verwirrte\\verwirrtest\\verwirrte\\verwirrten\\verwirrtet\\verwirrten}\\
\end{tabular}
\begin{tabular}{l}
\parbox[t][][t]{8cm}{}\\
\parbox[t][][t]{8cm}{\normalfont \footnotesize 
jdn (mit etw dat) verwirren - 
to confuse sb (with sth) - 
jdn (mit etw dat) verwirren - 
to bewilder sb - 
jdn verwirren - 
to discombobulate sb Am fam 
}\\
\end{tabular}
}
%===verzeihen===
\card{\normalfont \Huge verzeihen}{
\begin{tabular}{lll}
\parbox[t][][t]{2.0 cm}{\normalfont \raggedleft ich\\du\\er/sie/es\\wir\\ihr\\sie} &    
\parbox[t][][t]{2cm}{\normalfont verzeihe\\verzeihst\\verzeiht\\verzeihen\\verzeiht\\verzeihen} &
\parbox[t][][t]{2cm}{\normalfont verzieh\\verziehst\\verzieh\\verziehen\\verzieht\\verziehen}\\
\end{tabular}
\begin{tabular}{l}
\parbox[t][][t]{8cm}{}\\
\parbox[t][][t]{8cm}{\normalfont \footnotesize 
etw verzeihen - 
to excuse sth - 
ein Unrecht/eine Sünde verzeihen - 
to forgive an injustice/a sin - 
jdm etw verzeihen - 
to forgive sb sth - 
jdm etw verzeihen - 
to excuse (or pardon) sb for sth - 
verzeihen - 
to forgive sb 
}\\
\end{tabular}
}
%===vorbereiten===
\card{\normalfont \Huge vorbereiten}{
\begin{tabular}{lll}
\parbox[t][][t]{2.0 cm}{\normalfont \raggedleft ich\\du\\er/sie/es\\wir\\ihr\\sie} &    
\parbox[t][][t]{2cm}{\normalfont bereite vor\\bereitest vor\\bereitet vor\\bereiten vor\\bereitet vor\\bereiten vor} &
\parbox[t][][t]{2cm}{\normalfont bereitete vor\\bereitetest vor\\bereitete vor\\bereiteten vor\\bereitetet vor\\bereiteten vor}\\
\end{tabular}
\begin{tabular}{l}
\parbox[t][][t]{8cm}{}\\
\parbox[t][][t]{8cm}{\normalfont \footnotesize 
etw (für etw akk) vorbereiten - 
to prepare sth (for sth) - 
jdn (auf etw akk) vorbereiten - 
to prepare sb (for sth) - 
sich akk (auf etw akk) vorbereiten - 
to prepare oneself (for sth) - 
ich möchte dich auf eine unangenehme Nachricht vorbereiten! - 
I would like you to prepare yourself for some bad news! - 
wir bereiten uns auf ihre Ankunft vor - 
we're preparing for her arrival 
}\\
\end{tabular}
}
%===vorkommen===
\card{\normalfont \Huge vorkommen}{
\begin{tabular}{lll}
\parbox[t][][t]{2.0 cm}{\normalfont \raggedleft ich\\du\\er/sie/es\\wir\\ihr\\sie} &    
\parbox[t][][t]{2cm}{\normalfont komme vor\\kommst vor\\kommt vor\\kommen vor\\kommt vor\\kommen vor} &
\parbox[t][][t]{2cm}{\normalfont kam vor\\kamst vor\\kam vor\\kamen vor\\kamt vor\\kamen vor}\\
\end{tabular}
\begin{tabular}{l}
\parbox[t][][t]{8cm}{}\\
\parbox[t][][t]{8cm}{\normalfont \footnotesize 
vorkommen - 
to happen - 
es kommt vor, dass ... - 
it can happen that ... - 
es kommt selten vor, dass ich mal etwas vergesse - 
I rarely forget anything - 
das kann (schon mal) vorkommen - 
it happens - 
das kann (schon mal) vorkommen - 
these things (can) happen 
}\\
\end{tabular}
}
%===vorschlagen===
\card{\normalfont \Huge vorschlagen}{
\begin{tabular}{lll}
\parbox[t][][t]{2.0 cm}{\normalfont \raggedleft ich\\du\\er/sie/es\\wir\\ihr\\sie} &    
\parbox[t][][t]{2cm}{\normalfont schlage vor\\schlägst vor\\schlägt vor\\schlagen vor\\schlagt vor\\schlagen vor} &
\parbox[t][][t]{2cm}{\normalfont schlug vor\\schlugst vor\\schlug vor\\schlugen vor\\schlugt vor\\schlugen vor}\\
\end{tabular}
\begin{tabular}{l}
\parbox[t][][t]{8cm}{}\\
\parbox[t][][t]{8cm}{\normalfont \footnotesize 
(jdm) etw vorschlagen - 
to propose (or suggest) sth (to sb) - 
jdm vorschlagen, etw zu tun - 
to suggest to sb that he/she do sth - 
jdm vorschlagen, etw zu tun - 
to suggest that sb do sth - 
jdn (als jdn/für etw akk) vorschlagen - 
to recommend sb (as sb/for sth) 
}\\
\end{tabular}
}
%===vorstellen===
\card{\normalfont \Huge vorstellen}{
\begin{tabular}{lll}
\parbox[t][][t]{2.0 cm}{\normalfont \raggedleft ich\\du\\er/sie/es\\wir\\ihr\\sie} &    
\parbox[t][][t]{2cm}{\normalfont stelle vor\\stellst vor\\stellt vor\\stellen vor\\stellt vor\\stellen vor} &
\parbox[t][][t]{2cm}{\normalfont stellte vor\\stelltest vor\\stellte vor\\stellten vor\\stelltet vor\\stellten vor}\\
\end{tabular}
\begin{tabular}{l}
\parbox[t][][t]{8cm}{}\\
\parbox[t][][t]{8cm}{\normalfont \footnotesize 
sich dat etw vorstellen - 
to imagine sth - 
das muss man sich mal vorstellen! - 
just imagine (it)! - 
sich dat - 
vorstellen, dass/wie ... - 
to think (or imagine) that/how ... - 
sich dat etw vorstellen - 
to have sth in mind - 
sich dat etw vorstellen 
}\\
\end{tabular}
}
%===wachsen===
\card{\normalfont \Huge wachsen}{
\begin{tabular}{lll}
\parbox[t][][t]{2.0 cm}{\normalfont \raggedleft ich\\du\\er/sie/es\\wir\\ihr\\sie} &    
\parbox[t][][t]{2cm}{\normalfont wachse\\wächst\\wächst\\wachsen\\wachst\\wachsen} &
\parbox[t][][t]{2cm}{\normalfont wuchs\\wuchsest\\wuchs\\wuchsen\\wuchst\\wuchsen}\\
\end{tabular}
\begin{tabular}{l}
\parbox[t][][t]{8cm}{}\\
\parbox[t][][t]{8cm}{\normalfont \footnotesize 
wachsen - 
to grow - 
in die Breite wachsen - 
to grow broader (or to broaden (out)) - 
in die Höhe wachsen - 
to grow taller - 
wachsendes Defizit - 
growing deficit - 
wachsen - 
to grow 
}\\
\end{tabular}
}
%===wagen===
\card{\normalfont \Huge wagen}{
\begin{tabular}{lll}
\parbox[t][][t]{2.0 cm}{\normalfont \raggedleft ich\\du\\er/sie/es\\wir\\ihr\\sie} &    
\parbox[t][][t]{2cm}{\normalfont wage\\wagst\\wagt\\wagen\\wagt\\wagen} &
\parbox[t][][t]{2cm}{\normalfont wagte\\wagtest\\wagte\\wagten\\wagtet\\wagten}\\
\end{tabular}
\begin{tabular}{l}
\parbox[t][][t]{8cm}{}\\
\parbox[t][][t]{8cm}{\normalfont \footnotesize 
etw wagen - 
to risk sth - 
es wagen, etw zu tun - 
to dare (to) do sth - 
wer nicht wagt, der nicht gewinnt prov - 
nothing ventured, nothing gained prov - 
sich akk an etw akk - 
wagen - 
to venture to tackle sth - 
sich akk irgendwohin/irgendwoher wagen 
}\\
\end{tabular}
}
%===wägen===
\card{\normalfont \Huge wägen}{
\begin{tabular}{lll}
\parbox[t][][t]{2.0 cm}{\normalfont \raggedleft ich\\du\\er/sie/es\\wir\\ihr\\sie} &    
\parbox[t][][t]{2cm}{\normalfont wäge\\wägst\\wägt\\wägen\\wägt\\wägen} &
\parbox[t][][t]{2cm}{\normalfont wog\\wogst\\wog\\wogen\\wogt\\wogen}\\
\end{tabular}
\begin{tabular}{l}
\parbox[t][][t]{8cm}{}\\
\parbox[t][][t]{8cm}{\normalfont \footnotesize 
etw wägen - 
to weigh sth - 
etw wägen - 
to weigh sth up - 
etw wagen - 
to risk sth - 
es wagen, etw zu tun - 
to dare (to) do sth - 
wer nicht wagt, der nicht gewinnt prov - 
nothing ventured, nothing gained prov 
}\\
\end{tabular}
}
%===wählen===
\card{\normalfont \Huge wählen}{
\begin{tabular}{lll}
\parbox[t][][t]{2.0 cm}{\normalfont \raggedleft ich\\du\\er/sie/es\\wir\\ihr\\sie} &    
\parbox[t][][t]{2cm}{\normalfont wähle\\wählst\\wählt\\wählen\\wählt\\wählen} &
\parbox[t][][t]{2cm}{\normalfont wählte\\wähltest\\wählte\\wählten\\wähltet\\wählten}\\
\end{tabular}
\begin{tabular}{l}
\parbox[t][][t]{8cm}{}\\
\parbox[t][][t]{8cm}{\normalfont \footnotesize 
jdn/etw wählen - 
to vote for sb/sth - 
jdn in etw akk/zu etw dat - 
wählen - 
to elect sb to sth/as sth - 
etw wählen - 
to dial sth - 
ich glaube, Sie haben die falsche Nummer gewählt - 
I think you've dialled the wrong number (or misdialled) - 
wählen 
}\\
\end{tabular}
}
%===wandern===
\card{\normalfont \Huge wandern}{
\begin{tabular}{lll}
\parbox[t][][t]{2.0 cm}{\normalfont \raggedleft ich\\du\\er/sie/es\\wir\\ihr\\sie} &    
\parbox[t][][t]{2cm}{\normalfont wandere\\wanderst\\wandert\\wandern\\wandert\\wandern} &
\parbox[t][][t]{2cm}{\normalfont wanderte\\wandertest\\wanderte\\wanderten\\wandertet\\wanderten}\\
\end{tabular}
\begin{tabular}{l}
\parbox[t][][t]{8cm}{}\\
\parbox[t][][t]{8cm}{\normalfont \footnotesize 
wandern - 
to hike - 
wandern - 
to go rambling - 
wandern - 
to go on a hike - 
irgendwoher/irgendwohin wandern - 
to hike from somewhere/to somewhere - 
am Wochenende wandern wir gerne um den See - 
at the weekend we like to go on a ramble around the lake 
}\\
\end{tabular}
}
%===warnen===
\card{\normalfont \Huge warnen}{
\begin{tabular}{lll}
\parbox[t][][t]{2.0 cm}{\normalfont \raggedleft ich\\du\\er/sie/es\\wir\\ihr\\sie} &    
\parbox[t][][t]{2cm}{\normalfont warne\\warnst\\warnt\\warnen\\warnt\\warnen} &
\parbox[t][][t]{2cm}{\normalfont warnte\\warntest\\warnte\\warnten\\warntet\\warnten}\\
\end{tabular}
\begin{tabular}{l}
\parbox[t][][t]{8cm}{}\\
\parbox[t][][t]{8cm}{\normalfont \footnotesize 
jdn warnen - 
to warn sb - 
jdn vor jdm/etw warnen - 
to warn sb about sb/sth - 
ich muss dich vor ihm warnen - 
I must warn you about him - 
jdn (davor) warnen, etw zu tun - 
to warn sb about doing sth - 
ich warne dich/Sie! - 
I'm warning you! 
}\\
\end{tabular}
}
%===warten===
\card{\normalfont \Huge warten}{
\begin{tabular}{lll}
\parbox[t][][t]{2.0 cm}{\normalfont \raggedleft ich\\du\\er/sie/es\\wir\\ihr\\sie} &    
\parbox[t][][t]{2cm}{\normalfont warte\\wartest\\wartet\\warten\\wartet\\warten} &
\parbox[t][][t]{2cm}{\normalfont wartete\\wartetest\\wartete\\warteten\\wartetet\\warteten}\\
\end{tabular}
\begin{tabular}{l}
\parbox[t][][t]{8cm}{}\\
\parbox[t][][t]{8cm}{\normalfont \footnotesize 
warten - 
to wait - 
auf jdn/etw warten - 
to wait for sb/sth - 
mit etw dat (auf jdn) warten - 
to wait (for sb) before doing sth - 
jdn/etw kann warten - 
sb/sth can (or has to) wait - 
auf sich akk - 
warten lassen 
}\\
\end{tabular}
}
%===waschen===
\card{\normalfont \Huge waschen}{
\begin{tabular}{lll}
\parbox[t][][t]{2.0 cm}{\normalfont \raggedleft ich\\du\\er/sie/es\\wir\\ihr\\sie} &    
\parbox[t][][t]{2cm}{\normalfont wasche\\wäschst\\wäscht\\waschen\\wascht\\waschen} &
\parbox[t][][t]{2cm}{\normalfont wusch\\wuschest\\wusch\\wuschen\\wuscht\\wuschen}\\
\end{tabular}
\begin{tabular}{l}
\parbox[t][][t]{8cm}{}\\
\parbox[t][][t]{8cm}{\normalfont \footnotesize 
jdn/etw waschen - 
to wash sb/sth - 
sich akk - 
waschen - 
to wash (oneself) - 
sich akk kalt/warm waschen - 
to wash (oneself) in cold/hot water - 
(sich dat) etw waschen - 
to wash (one's) sth - 
etw waschen 
}\\
\end{tabular}
}
%===wechseln===
\card{\normalfont \Huge wechseln}{
\begin{tabular}{lll}
\parbox[t][][t]{2.0 cm}{\normalfont \raggedleft ich\\du\\er/sie/es\\wir\\ihr\\sie} &    
\parbox[t][][t]{2cm}{\normalfont wechsle\\wechselst\\wechselt\\wechseln\\wechselt\\wechseln} &
\parbox[t][][t]{2cm}{\normalfont wechselte\\wechseltest\\wechselte\\wechselten\\wechseltet\\wechselten}\\
\end{tabular}
\begin{tabular}{l}
\parbox[t][][t]{8cm}{}\\
\parbox[t][][t]{8cm}{\normalfont \footnotesize 
etw wechseln - 
Adresse, Thema - 
to change sth - 
den Arzt/die Schule wechseln - 
to change doctors/schools - 
die Fahrspur wechseln - 
to change (or switch) lanes - 
die Straßenseite wechseln - 
to cross over to the other side of the street - 
etw wechseln Bettwäsche, Handtücher, Kleidung 
}\\
\end{tabular}
}
%===wecken===
\card{\normalfont \Huge wecken}{
\begin{tabular}{lll}
\parbox[t][][t]{2.0 cm}{\normalfont \raggedleft ich\\du\\er/sie/es\\wir\\ihr\\sie} &    
\parbox[t][][t]{2cm}{\normalfont wecke\\weckst\\weckt\\wecken\\weckt\\wecken} &
\parbox[t][][t]{2cm}{\normalfont weckte\\wecktest\\weckte\\weckten\\wecktet\\weckten}\\
\end{tabular}
\begin{tabular}{l}
\parbox[t][][t]{8cm}{}\\
\parbox[t][][t]{8cm}{\normalfont \footnotesize 
jdn wecken - 
to wake sb (up) - 
von Lärm geweckt werden - 
to be woken by noise - 
sich akk (von jdm/etw) wecken lassen - 
to have sb/sth wake one up - 
das Wecken - 
MILIT - 
reveille no pl - 
eine Stunde nach dem Wecken 
}\\
\end{tabular}
}
%===wehren===
\card{\normalfont \Huge wehren}{
\begin{tabular}{lll}
\parbox[t][][t]{2.0 cm}{\normalfont \raggedleft ich\\du\\er/sie/es\\wir\\ihr\\sie} &    
\parbox[t][][t]{2cm}{\normalfont wehre\\wehrst\\wehrt\\wehren\\wehrt\\wehren} &
\parbox[t][][t]{2cm}{\normalfont wehrte\\wehrtest\\wehrte\\wehrten\\wehrtet\\wehrten}\\
\end{tabular}
\begin{tabular}{l}
\parbox[t][][t]{8cm}{}\\
\parbox[t][][t]{8cm}{\normalfont \footnotesize 
sich akk (gegen jdn/etw) wehren - 
to defend oneself (against sb/sth) - 
sich akk gegen etw akk - 
wehren - 
to fight against sth - 
sich akk dagegen wehren, etw zu tun - 
to resist doing sth - 
etw dat - 
wehren - 
to prevent a thing spreading 
}\\
\end{tabular}
}
%===weichen===
\card{\normalfont \Huge weichen}{
\begin{tabular}{lll}
\parbox[t][][t]{2.0 cm}{\normalfont \raggedleft ich\\du\\er/sie/es\\wir\\ihr\\sie} &    
\parbox[t][][t]{2cm}{\normalfont weiche\\weichst\\weicht\\weichen\\weicht\\weichen} &
\parbox[t][][t]{2cm}{\normalfont wich\\wichst\\wich\\wichen\\wicht\\wichen}\\
\end{tabular}
\begin{tabular}{l}
\parbox[t][][t]{8cm}{}\\
\parbox[t][][t]{8cm}{\normalfont \footnotesize 
etw dat - 
weichen - 
to give way to sth - 
weichen - 
to subside - 
weichen - 
to go - 
er wich nicht von der Stelle - 
he didn't budge from the spot - 
als ihre Angst gewichen war, ... geh 
}\\
\end{tabular}
}
%===weinen===
\card{\normalfont \Huge weinen}{
\begin{tabular}{lll}
\parbox[t][][t]{2.0 cm}{\normalfont \raggedleft ich\\du\\er/sie/es\\wir\\ihr\\sie} &    
\parbox[t][][t]{2cm}{\normalfont weine\\weinst\\weint\\weinen\\weint\\weinen} &
\parbox[t][][t]{2cm}{\normalfont weinte\\weintest\\weinte\\weinten\\weintet\\weinten}\\
\end{tabular}
\begin{tabular}{l}
\parbox[t][][t]{8cm}{}\\
\parbox[t][][t]{8cm}{\normalfont \footnotesize 
weinen - 
to cry - 
vor Freude weinen - 
to cry with joy - 
um jdn/etw weinen - 
to cry for sb/sth - 
es ist zum Weinen! - 
it's enough to make you weep - 
etw weinen - 
to cry sth 
}\\
\end{tabular}
}
%===weisen===
\card{\normalfont \Huge weisen}{
\begin{tabular}{lll}
\parbox[t][][t]{2.0 cm}{\normalfont \raggedleft ich\\du\\er/sie/es\\wir\\ihr\\sie} &    
\parbox[t][][t]{2cm}{\normalfont weise\\weist\\weist\\weisen\\weist\\weisen} &
\parbox[t][][t]{2cm}{\normalfont wies\\wiesest\\wies\\wiesen\\wiest\\wiesen}\\
\end{tabular}
\begin{tabular}{l}
\parbox[t][][t]{8cm}{}\\
\parbox[t][][t]{8cm}{\normalfont \footnotesize 
jdn aus etw dat/von etw dat - 
weisen - 
to expel sb from sth - 
etw (weit) von sich dat - 
weisen - 
to reject sth (emphatically) - 
irgendwohin weisen - 
to point somewhere - 
weise - 
wise 
}\\
\end{tabular}
}
%===wenden===
\card{\normalfont \Huge wenden}{
\begin{tabular}{lll}
\parbox[t][][t]{2.0 cm}{\normalfont \raggedleft ich\\du\\er/sie/es\\wir\\ihr\\sie} &    
\parbox[t][][t]{2cm}{\normalfont wende\\wendest\\wendet\\wenden\\wendet\\wenden} &
\parbox[t][][t]{2cm}{\normalfont wandte \\wandtest \\wandte\\wandten\\wandtet \\wandten }\\
\end{tabular}
\begin{tabular}{l}
\parbox[t][][t]{8cm}{}\\
\parbox[t][][t]{8cm}{\normalfont \footnotesize 
etw wenden - 
Blatt, Buchseite - 
to turn over sth sep - 
bitte wenden! - 
please turn over - 
etw wenden Braten - 
to turn over sth sep - 
etw wenden Pfannkuchen - 
to flip over sth sep - 
etw in Mehl wenden 
}\\
\end{tabular}
}
%===werben===
\card{\normalfont \Huge werben}{
\begin{tabular}{lll}
\parbox[t][][t]{2.0 cm}{\normalfont \raggedleft ich\\du\\er/sie/es\\wir\\ihr\\sie} &    
\parbox[t][][t]{2cm}{\normalfont werbe\\wirbst\\wirbt\\werben\\werbt\\werben} &
\parbox[t][][t]{2cm}{\normalfont warb\\warbst\\warb\\warben\\warbt\\warben}\\
\end{tabular}
\begin{tabular}{l}
\parbox[t][][t]{8cm}{}\\
\parbox[t][][t]{8cm}{\normalfont \footnotesize 
jdn (für etw akk) werben - 
to recruit sb (for sth) - 
für etw akk - 
werben - 
to advertise (or promote) sth - 
für eine Partei werben - 
to try to win support for a party - 
um eine Frau/einen Mann werben - 
to woo a woman/pursue a man - 
um Unterstützung werben 
}\\
\end{tabular}
}
%===werden===
\card{\normalfont \Huge werden}{
\begin{tabular}{lll}
\parbox[t][][t]{2.0 cm}{\normalfont \raggedleft ich\\du\\er/sie/es\\wir\\ihr\\sie} &    
\parbox[t][][t]{2cm}{\normalfont werde\\wirst\\wird\\werden\\werdet\\werden} &
\parbox[t][][t]{2cm}{\normalfont wurde\\wurdest\\wurde \\wurden\\wurdet\\wurden}\\
\end{tabular}
\begin{tabular}{l}
\parbox[t][][t]{8cm}{}\\
\parbox[t][][t]{8cm}{\normalfont \footnotesize 
er ist (gerade) 30 geworden - 
he's (just) turned thirty (or had his 30th birthday) - 
sie wird morgen 80 - 
she'll be 80 tomorrow - 
alt/älter werden - 
to get (or grow) - 
(or become) old(er)/older - 
alt/älter werden - 
to be getting on fam - 
anders werden 
}\\
\end{tabular}
}
%===werfen===
\card{\normalfont \Huge werfen}{
\begin{tabular}{lll}
\parbox[t][][t]{2.0 cm}{\normalfont \raggedleft ich\\du\\er/sie/es\\wir\\ihr\\sie} &    
\parbox[t][][t]{2cm}{\normalfont werfe\\wirfst\\wirft\\werfen\\werft\\werfen} &
\parbox[t][][t]{2cm}{\normalfont warf\\warfst\\warf\\warfen\\warft\\warfen}\\
\end{tabular}
\begin{tabular}{l}
\parbox[t][][t]{8cm}{}\\
\parbox[t][][t]{8cm}{\normalfont \footnotesize 
etw irgendwohin werfen - 
to throw sth somewhere - 
etw auf jdn/etw werfen - 
to throw sth at sb/sth - 
hör auf, Steine ans Fenster zu werfen! - 
stop throwing stones at the window! - 
das Boot wurde gegen die Felsen geworfen - 
the boat was thrown onto the rocks - 
etw auf den Boden werfen - 
to throw sth to the ground 
}\\
\end{tabular}
}
%===widersprechen===
\card{\normalfont \Huge widersprechen}{
\begin{tabular}{lll}
\parbox[t][][t]{2.0 cm}{\normalfont \raggedleft ich\\du\\er/sie/es\\wir\\ihr\\sie} &    
\parbox[t][][t]{2cm}{\normalfont widerspreche\\widersprichst\\widerspricht\\widersprechen\\widersprecht\\widersprechen} &
\parbox[t][][t]{2cm}{\normalfont widersprach\\widersprachst\\widersprach\\widersprachen\\widerspracht\\widersprachen}\\
\end{tabular}
\begin{tabular}{l}
\parbox[t][][t]{8cm}{}\\
\parbox[t][][t]{8cm}{\normalfont \footnotesize 
(jdm/etw/einander) widersprechen - 
to contradict (sb/sth/each other) - 
etw dat - 
widersprechen - 
to contradict sth - 
etw dat - 
widersprechen - 
to be inconsistent with sth - 
sich dat - 
widersprechen 
}\\
\end{tabular}
}
%===widmen===
\card{\normalfont \Huge widmen}{
\begin{tabular}{lll}
\parbox[t][][t]{2.0 cm}{\normalfont \raggedleft ich\\du\\er/sie/es\\wir\\ihr\\sie} &    
\parbox[t][][t]{2cm}{\normalfont widme\\widmest\\widmet\\widmen\\widmet\\widmen} &
\parbox[t][][t]{2cm}{\normalfont widmete\\widmetest\\widmete\\widmeten\\widmetet\\widmeten}\\
\end{tabular}
\begin{tabular}{l}
\parbox[t][][t]{8cm}{}\\
\parbox[t][][t]{8cm}{\normalfont \footnotesize 
jdm etw widmen - 
to dedicate sth to sb - 
etw etw dat - 
widmen - 
to dedicate (or devote) sth to a thing - 
etw etw dat - 
widmen - 
to open sth officially to a thing - 
sich akk jdm widmen - 
to attend to sb 
}\\
\end{tabular}
}
%===wiederholen===
\card{\normalfont \Huge wiederholen}{
\begin{tabular}{lll}
\parbox[t][][t]{2.0 cm}{\normalfont \raggedleft ich\\du\\er/sie/es\\wir\\ihr\\sie} &    
\parbox[t][][t]{2cm}{\normalfont hole wieder\\holst wieder\\holt wieder\\holen wieder\\holt wieder\\holen wieder} &
\parbox[t][][t]{2cm}{\normalfont holte wieder\\holtest wieder\\holte wieder\\holten wieder\\holtet wieder\\holten wieder}\\
\end{tabular}
\begin{tabular}{l}
\parbox[t][][t]{8cm}{}\\
\parbox[t][][t]{8cm}{\normalfont \footnotesize 
etw wiederholen - 
to repeat sth - 
etw wiederholen - 
to repeat sth - 
etw wiederholen - 
to revise sth - 
etw wiederholen - 
to retake - 
etw wiederholen - 
Brit a. to resit 
}\\
\end{tabular}
}
%===wiegen===
\card{\normalfont \Huge wiegen}{
\begin{tabular}{lll}
\parbox[t][][t]{2.0 cm}{\normalfont \raggedleft ich\\du\\er/sie/es\\wir\\ihr\\sie} &    
\parbox[t][][t]{2cm}{\normalfont wiege\\wiegst\\wiegt\\wiegen\\wiegt\\wiegen} &
\parbox[t][][t]{2cm}{\normalfont wog\\wogst\\wog\\wogen\\wogt\\wogen}\\
\end{tabular}
\begin{tabular}{l}
\parbox[t][][t]{8cm}{}\\
\parbox[t][][t]{8cm}{\normalfont \footnotesize 
jdn/etw wiegen - 
to weigh sb/sth - 
sich akk - 
wiegen - 
to weigh oneself - 
wiegen - 
to weigh - 
viel/wenig/eine bestimmte Anzahl von Kilo wiegen - 
to weigh a lot/not to weigh much/to weigh a certain number of kilos - 
jdn/etw wiegen 
}\\
\end{tabular}
}
%===winden===
\card{\normalfont \Huge winden}{
\begin{tabular}{lll}
\parbox[t][][t]{2.0 cm}{\normalfont \raggedleft ich\\du\\er/sie/es\\wir\\ihr\\sie} &    
\parbox[t][][t]{2cm}{\normalfont winde\\windest\\windet\\winden\\windet\\winden} &
\parbox[t][][t]{2cm}{\normalfont wand\\wandest\\wand\\wanden\\wandet\\wanden}\\
\end{tabular}
\begin{tabular}{l}
\parbox[t][][t]{8cm}{}\\
\parbox[t][][t]{8cm}{\normalfont \footnotesize 
sich akk - 
winden - 
to attempt to wriggle out of sth - 
sich akk (in etw dat/vor etw dat) winden - 
to writhe (in sth) - 
sich akk vor (o. in) Schmerzen winden - 
to writhe with (or in) pain - 
sich akk vor Scham winden - 
to squirm with (or in) shame - 
sich akk irgendwohin winden 
}\\
\end{tabular}
}
%===wirken===
\card{\normalfont \Huge wirken}{
\begin{tabular}{lll}
\parbox[t][][t]{2.0 cm}{\normalfont \raggedleft ich\\du\\er/sie/es\\wir\\ihr\\sie} &    
\parbox[t][][t]{2cm}{\normalfont wirke\\wirkst\\wirkt\\wirken\\wirkt\\wirken} &
\parbox[t][][t]{2cm}{\normalfont wirkte\\wirktest\\wirkte\\wirkten\\wirktet\\wirkten}\\
\end{tabular}
\begin{tabular}{l}
\parbox[t][][t]{8cm}{}\\
\parbox[t][][t]{8cm}{\normalfont \footnotesize 
wirken (Wirkung haben) - 
to have an effect - 
wirken (beabsichtigten Effekt haben) - 
to work - 
dieses Medikament wirkt sofort - 
this medicine takes effect immediately - 
etw auf sich akk - 
wirken lassen - 
to take sth in - 
ich lasse die Musik auf mich wirken 
}\\
\end{tabular}
}
%===wissen===
\card{\normalfont \Huge wissen}{
\begin{tabular}{lll}
\parbox[t][][t]{2.0 cm}{\normalfont \raggedleft ich\\du\\er/sie/es\\wir\\ihr\\sie} &    
\parbox[t][][t]{2cm}{\normalfont weiß\\weißt\\weiß\\wissen\\wisst\\wissen} &
\parbox[t][][t]{2cm}{\normalfont wusste\\wusstest\\wusste\\wussten\\wusstet\\wussten}\\
\end{tabular}
\begin{tabular}{l}
\parbox[t][][t]{8cm}{}\\
\parbox[t][][t]{8cm}{\normalfont \footnotesize 
etw (über jdn/etw) wissen - 
to know sth (about sb/sth) - 
wenn ich das gewusst hätte! - 
if only I had known (that)! - 
dass du es (nur) (gleich) weißt - 
just so you know - 
er weiß immer alles besser - 
he always knows better - 
woher soll ich das wissen? - 
how should I know that? 
}\\
\end{tabular}
}
%===wohnen===
\card{\normalfont \Huge wohnen}{
\begin{tabular}{lll}
\parbox[t][][t]{2.0 cm}{\normalfont \raggedleft ich\\du\\er/sie/es\\wir\\ihr\\sie} &    
\parbox[t][][t]{2cm}{\normalfont wohne\\wohnst\\wohnt\\wohnen\\wohnt\\wohnen} &
\parbox[t][][t]{2cm}{\normalfont wohnte\\wohntest\\wohnte\\wohnten\\wohntet\\wohnten}\\
\end{tabular}
\begin{tabular}{l}
\parbox[t][][t]{8cm}{}\\
\parbox[t][][t]{8cm}{\normalfont \footnotesize 
irgendwo wohnen - 
to live somewhere - 
ich wohne im Hotel - 
I am staying at the hotel - 
irgendwie wohnen - 
to live somehow - 
in diesem Viertel wohnt man sehr schön - 
this area is a nice place to live 
}\\
\end{tabular}
}
%===wollen===
\card{\normalfont \Huge wollen}{
\begin{tabular}{lll}
\parbox[t][][t]{2.0 cm}{\normalfont \raggedleft ich\\du\\er/sie/es\\wir\\ihr\\sie} &    
\parbox[t][][t]{2cm}{\normalfont will\\willst\\will\\wollen\\wollt\\wollen} &
\parbox[t][][t]{2cm}{\normalfont wollte\\wolltest\\wollte\\wollten\\wolltet\\wollten}\\
\end{tabular}
\begin{tabular}{l}
\parbox[t][][t]{8cm}{}\\
\parbox[t][][t]{8cm}{\normalfont \footnotesize 
wollen - 
woollen - 
etw tun wollen - 
to want to do sth - 
etw tun wollen - 
(fest) - 
to be going to do sth - 
ihr wollt schon gehen? - 
are you leaving already? - 
es sieht aus, als wolle es gleich regnen 
}\\
\end{tabular}
}
%===wühlen===
\card{\normalfont \Huge wühlen}{
\begin{tabular}{lll}
\parbox[t][][t]{2.0 cm}{\normalfont \raggedleft ich\\du\\er/sie/es\\wir\\ihr\\sie} &    
\parbox[t][][t]{2cm}{\normalfont wühle\\wühlst\\wühlt\\wühlen\\wühlt\\wühlen} &
\parbox[t][][t]{2cm}{\normalfont wühlte\\wühltest\\wühlte\\wühlten\\wühltet\\wühlten}\\
\end{tabular}
\begin{tabular}{l}
\parbox[t][][t]{8cm}{}\\
\parbox[t][][t]{8cm}{\normalfont \footnotesize 
in etw dat (nach etw dat) wühlen - 
to rummage (or root) through sth (for sth) - 
wonach wühlst du denn in der alten Truhe? - 
what are you rummaging for in that old chest? - 
einen Schlüssel aus der Tasche wühlen - 
to root (or dig)  a key out of a bag - 
in etw dat (nach etw dat) wühlen - 
to root through sth (for sth) - 
bei uns im Garten wühlen wieder Maulwürfe - 
we've got moles in our garden again 
}\\
\end{tabular}
}
%===wundern===
\card{\normalfont \Huge wundern}{
\begin{tabular}{lll}
\parbox[t][][t]{2.0 cm}{\normalfont \raggedleft ich\\du\\er/sie/es\\wir\\ihr\\sie} &    
\parbox[t][][t]{2cm}{\normalfont wundere\\wunderst\\wundert\\wundern\\wundert\\wundern} &
\parbox[t][][t]{2cm}{\normalfont wunderte\\wundertest\\wunderte\\wunderten\\wundertet\\wunderten}\\
\end{tabular}
\begin{tabular}{l}
\parbox[t][][t]{8cm}{}\\
\parbox[t][][t]{8cm}{\normalfont \footnotesize 
jdn wundern - 
to surprise sb - 
das wundert mich (nicht) - 
I'm (not) surprised at that - 
das hätte uns eigentlich nicht wundern dürfen - 
that shouldn't have come as a surprise to us - 
es wundert mich, dass ... - 
I am surprised that ... - 
es wundert mich, dass ... - 
it surprises me that ... 
}\\
\end{tabular}
}
%===wünschen===
\card{\normalfont \Huge wünschen}{
\begin{tabular}{lll}
\parbox[t][][t]{2.0 cm}{\normalfont \raggedleft ich\\du\\er/sie/es\\wir\\ihr\\sie} &    
\parbox[t][][t]{2cm}{\normalfont wünsche\\wünschst\\wünscht\\wünschen\\wünscht\\wünschen} &
\parbox[t][][t]{2cm}{\normalfont wünschte\\wünschtest\\wünschte\\wünschten\\wünschtet\\wünschten}\\
\end{tabular}
\begin{tabular}{l}
\parbox[t][][t]{8cm}{}\\
\parbox[t][][t]{8cm}{\normalfont \footnotesize 
jdm etw wünschen - 
to wish sb sth - 
ich wünsche dir gute Besserung - 
get well soon! - 
ich wünsche dir alles Glück dieser Welt! - 
I wish you all the luck in the world! - 
ich wünsche dir alles Glück dieser Welt! - 
I hope you get everything you could possibly wish for! - 
ich wünsche dir gutes Gelingen - 
I wish you every success 
}\\
\end{tabular}
}
%===zahlen===
\card{\normalfont \Huge zahlen}{
\begin{tabular}{lll}
\parbox[t][][t]{2.0 cm}{\normalfont \raggedleft ich\\du\\er/sie/es\\wir\\ihr\\sie} &    
\parbox[t][][t]{2cm}{\normalfont zahle\\zahlst\\zahlt\\zahlen\\zahlt\\zahlen} &
\parbox[t][][t]{2cm}{\normalfont zahlte\\zahltest\\zahlte\\zahlten\\zahltet\\zahlten}\\
\end{tabular}
\begin{tabular}{l}
\parbox[t][][t]{8cm}{}\\
\parbox[t][][t]{8cm}{\normalfont \footnotesize 
(jdm) etw (für etw akk) zahlen - 
to pay (sb) sth (for sth) - 
seine Miete/Schulden zahlen - 
to pay one's rent/debts - 
das Hotelzimmer/Taxi zahlen - 
fam - 
to pay for a hotel room/taxi - 
(jdm) etw zahlen - 
to pay (sb) sth - 
(gut/besser/schlecht) zahlen 
}\\
\end{tabular}
}
%===zählen===
\card{\normalfont \Huge zählen}{
\begin{tabular}{lll}
\parbox[t][][t]{2.0 cm}{\normalfont \raggedleft ich\\du\\er/sie/es\\wir\\ihr\\sie} &    
\parbox[t][][t]{2cm}{\normalfont zähle\\zählst\\zählt\\zählen\\zählt\\zählen} &
\parbox[t][][t]{2cm}{\normalfont zählte\\zähltest\\zählte\\zählten\\zähltet\\zählten}\\
\end{tabular}
\begin{tabular}{l}
\parbox[t][][t]{8cm}{}\\
\parbox[t][][t]{8cm}{\normalfont \footnotesize 
etw zählen - 
to count sth - 
das Geld auf den Tisch zählen - 
to count the money on the table - 
etw zählen - 
to number sth form - 
etw zählen - 
to have sth - 
der Verein zählt 59 Mitglieder - 
the club has (or numbers) 59 members 
}\\
\end{tabular}
}
%===zeichnen===
\card{\normalfont \Huge zeichnen}{
\begin{tabular}{lll}
\parbox[t][][t]{2.0 cm}{\normalfont \raggedleft ich\\du\\er/sie/es\\wir\\ihr\\sie} &    
\parbox[t][][t]{2cm}{\normalfont zeichne\\zeichnest\\zeichnet\\zeichnen\\zeichnet\\zeichnen} &
\parbox[t][][t]{2cm}{\normalfont zeichnete\\zeichnetest\\zeichnete\\zeichneten\\zeichnetet\\zeichneten}\\
\end{tabular}
\begin{tabular}{l}
\parbox[t][][t]{8cm}{}\\
\parbox[t][][t]{8cm}{\normalfont \footnotesize 
jdn/etw zeichnen - 
to draw sb/sth - 
eine Landschaft zeichnen - 
to draw a landscape - 
einen Akt zeichnen - 
to draw a nude - 
einen Grundriss zeichnen - 
to draw an outline - 
etw zeichnen - 
to subscribe for sth 
}\\
\end{tabular}
}
%===zeigen===
\card{\normalfont \Huge zeigen}{
\begin{tabular}{lll}
\parbox[t][][t]{2.0 cm}{\normalfont \raggedleft ich\\du\\er/sie/es\\wir\\ihr\\sie} &    
\parbox[t][][t]{2cm}{\normalfont zeige\\zeigst\\zeigt\\zeigen\\zeigt\\zeigen} &
\parbox[t][][t]{2cm}{\normalfont zeigte\\zeigtest\\zeigte\\zeigten\\zeigtet\\zeigten}\\
\end{tabular}
\begin{tabular}{l}
\parbox[t][][t]{8cm}{}\\
\parbox[t][][t]{8cm}{\normalfont \footnotesize 
jdm etw zeigen - 
to show sb sth - 
jdm die Richtung/den Weg zeigen - 
to show sb the way - 
(jdm) jdn/etw zeigen - 
to show (sb) sb/sth - 
sich dat von jdm zeigen lassen, wie etw gemacht wird - 
to get sb to show one how to do sth - 
sich dat sein Zimmer zeigen lassen - 
to be shown one's room 
}\\
\end{tabular}
}
%===zerstieben===
\card{\normalfont \Huge zerstieben}{
\begin{tabular}{lll}
\parbox[t][][t]{2.0 cm}{\normalfont \raggedleft ich\\du\\er/sie/es\\wir\\ihr\\sie} &    
\parbox[t][][t]{2cm}{\normalfont zerstiebe\\zerstiebst\\zerstiebt\\zerstieben\\zerstiebt\\zerstieben} &
\parbox[t][][t]{2cm}{\normalfont zerstob \\zerstobst \\zerstob \\zerstoben \\zerstobt \\zerstobten }\\
\end{tabular}
\begin{tabular}{l}
\parbox[t][][t]{8cm}{}\\
\parbox[t][][t]{8cm}{\normalfont \footnotesize 
zerstieben - 
to scatter - 
zerstieben Wasser - 
to spray 
}\\
\end{tabular}
}
%===zerstören===
\card{\normalfont \Huge zerstören}{
\begin{tabular}{lll}
\parbox[t][][t]{2.0 cm}{\normalfont \raggedleft ich\\du\\er/sie/es\\wir\\ihr\\sie} &    
\parbox[t][][t]{2cm}{\normalfont zerstöre\\zerstörst\\zerstört\\zerstören\\zerstört\\zerstören} &
\parbox[t][][t]{2cm}{\normalfont zerstörte\\zerstörtest\\zerstörte\\zerstörten\\zerstörtet\\zerstörten}\\
\end{tabular}
\begin{tabular}{l}
\parbox[t][][t]{8cm}{}\\
\parbox[t][][t]{8cm}{\normalfont \footnotesize 
etw zerstören - 
to destroy sth - 
etw zerstören - 
to ruin sth - 
eine Ehe/die Gesundheit zerstören - 
to ruin (or wreck)  a marriage/one's health 
}\\
\end{tabular}
}
%===ziehen===
\card{\normalfont \Huge ziehen}{
\begin{tabular}{lll}
\parbox[t][][t]{2.0 cm}{\normalfont \raggedleft ich\\du\\er/sie/es\\wir\\ihr\\sie} &    
\parbox[t][][t]{2cm}{\normalfont ziehe\\ziehst\\zieht\\ziehen\\zieht\\ziehen} &
\parbox[t][][t]{2cm}{\normalfont zog\\zogst\\zog\\zogen\\zogt\\zogen}\\
\end{tabular}
\begin{tabular}{l}
\parbox[t][][t]{8cm}{}\\
\parbox[t][][t]{8cm}{\normalfont \footnotesize 
etw ziehen - 
to pull sth - 
die Kutsche wurde von vier Pferden gezogen - 
the coach was drawn by four horses - 
etw ziehen - 
to pull sth - 
kannst du nicht die Wasserspülung ziehen? - 
can't you flush the toilet? - 
den Choke/Starter ziehen - 
to pull out the choke/starter 
}\\
\end{tabular}
}
%===zuhören===
\card{\normalfont \Huge zuhören}{
\begin{tabular}{lll}
\parbox[t][][t]{2.0 cm}{\normalfont \raggedleft ich\\du\\er/sie/es\\wir\\ihr\\sie} &    
\parbox[t][][t]{2cm}{\normalfont höre zu\\hörst zu\\hört zu\\hören zu\\hört zu\\hören zu} &
\parbox[t][][t]{2cm}{\normalfont hörte zu\\hörtest zu\\hörte zu\\hörten zu\\hörtet zu\\hörten zu}\\
\end{tabular}
\begin{tabular}{l}
\parbox[t][][t]{8cm}{}\\
\parbox[t][][t]{8cm}{\normalfont \footnotesize 
zuhören - 
to listen - 
jdm/etw zuhören - 
to listen to sb/sth - 
nun hör mir doch mal richtig zu! - 
now listen carefully to me! 
}\\
\end{tabular}
}
%===zunehmen===
\card{\normalfont \Huge zunehmen}{
\begin{tabular}{lll}
\parbox[t][][t]{2.0 cm}{\normalfont \raggedleft ich\\du\\er/sie/es\\wir\\ihr\\sie} &    
\parbox[t][][t]{2cm}{\normalfont nehme zu\\nimmst zu\\nimmt zu\\nehmen zu\\nehmt zu\\nehmen zu} &
\parbox[t][][t]{2cm}{\normalfont nahm zu\\nahmst zu\\nahm zu\\nahmen zu\\nahmt zu\\nahmen zu}\\
\end{tabular}
\begin{tabular}{l}
\parbox[t][][t]{8cm}{}\\
\parbox[t][][t]{8cm}{\normalfont \footnotesize 
zunehmen - 
to gain (or put on) weight - 
an Gewicht zunehmen - 
to gain (or put on) weight - 
(an etw dat) zunehmen - 
to increase (in sth) - 
zunehmen - 
to increase - 
zunehmen Schmerzen - 
to intensify 
}\\
\end{tabular}
}
%===zusammenarbeiten===
\card{\normalfont \Huge zusammenarbeiten}{
\begin{tabular}{lll}
\parbox[t][][t]{1.5 cm}{\normalfont \raggedleft ich\\du\\er/sie/es\\wir\\ihr\\sie} &    
\parbox[t][][t]{3cm}{\normalfont arbeite zusammen\\arbeitest zusammen\\arbeitet zusammen\\arbeiten zusammen\\arbeitet zusammen\\arbeiten zusammen} &
\parbox[t][][t]{3cm}{\normalfont arbeitete zusammen\\arbeitetest zusammen\\arbeitete zusammen\\arbeiteten zusammen\\arbeitetet zusammen\\arbeiteten zusammen}\\
\end{tabular}
\begin{tabular}{l}
\parbox[t][][t]{8cm}{}\\
\parbox[t][][t]{8cm}{\normalfont \footnotesize 
mit jdm zusammenarbeiten - 
to work (together) with sb - 
mit jdm zusammenarbeiten - 
(kooperieren) - 
to cooperate with sb - 
Zusammenarbeit - 
cooperation no art, no pl - 
in Zusammenarbeit mit jdm - 
in cooperation with sb - 
Zusammenarbeit 
}\\
\end{tabular}
}
%===zusammenfassen===
\card{\normalfont \Huge zusammenfassen}{
\begin{tabular}{lll}
\parbox[t][][t]{1.5 cm}{\normalfont \raggedleft ich\\du\\er/sie/es\\wir\\ihr\\sie} &    
\parbox[t][][t]{3cm}{\normalfont fasse zusammen\\fasst zusammen\\fasst zusammen\\fassen zusammen\\fasst zusammen\\fassen zusammen} &
\parbox[t][][t]{3cm}{\normalfont fasste zusammen\\fasstest zusammen\\fasste zusammen\\fassten zusammen\\fasstet zusammen\\fassten zusammen}\\
\end{tabular}
\begin{tabular}{l}
\parbox[t][][t]{8cm}{}\\
\parbox[t][][t]{8cm}{\normalfont \footnotesize 
etw zusammenfassen - 
to summarize sth - 
etw in wenigen Worten zusammenfassen - 
to put sth in a nutshell - 
die Bewerber in Gruppen zusammenfassen - 
to divide the applicants into groups - 
Truppen zusammenfassen - 
to concentrate troops - 
jdn/etw in etw dat - 
zusammenfassen 
}\\
\end{tabular}
}
%===zweifeln===
\card{\normalfont \Huge zweifeln}{
\begin{tabular}{lll}
\parbox[t][][t]{2.0 cm}{\normalfont \raggedleft ich\\du\\er/sie/es\\wir\\ihr\\sie} &    
\parbox[t][][t]{2cm}{\normalfont zweifle\\zweifelst\\zweifelt\\zweifeln\\zweifelt\\zweifeln} &
\parbox[t][][t]{2cm}{\normalfont zweifelte\\zweifeltest\\zweifelte\\zweifelten\\zweifeltet\\zweifelten}\\
\end{tabular}
\begin{tabular}{l}
\parbox[t][][t]{8cm}{}\\
\parbox[t][][t]{8cm}{\normalfont \footnotesize 
an jdm/etw zweifeln - 
to doubt (or have one's doubts about) sb/sth - 
an jdm/etw zweifeln - 
(skeptisch sein a.) - 
to be sceptical (or - 
Am skeptical) about sb/sth - 
(daran) zweifeln, ob ... - 
to doubt (or have doubts (about (or as to))) whether ... - 
nicht (daran) zweifeln, dass ... - 
to not (or have no) doubt that ... 
}\\
\end{tabular}
}
%===zwingen===
\card{\normalfont \Huge zwingen}{
\begin{tabular}{lll}
\parbox[t][][t]{2.0 cm}{\normalfont \raggedleft ich\\du\\er/sie/es\\wir\\ihr\\sie} &    
\parbox[t][][t]{2cm}{\normalfont zwinge\\zwingst\\zwingt\\zwingen\\zwingt\\zwingen} &
\parbox[t][][t]{2cm}{\normalfont zwang\\zwangst\\zwang\\zwangen\\zwangt\\zwangen}\\
\end{tabular}
\begin{tabular}{l}
\parbox[t][][t]{8cm}{}\\
\parbox[t][][t]{8cm}{\normalfont \footnotesize 
jdn zwingen - 
to force (or compel) sb - 
du musst noch nicht gehen, es zwingt dich niemand! - 
you don't have to go yet, nobody's forcing you! - 
ich lasse mich nicht zwingen - 
I won't give in to force - 
jdn zwingen, etw zu tun - 
to force sb into doing (or to do) sth - 
jdn zwingen, etw zu tun - 
to make sb do sth 
}\\
\end{tabular}
}
\end{document}
